\subsection{Imaginary-time formalism}
\label{imaginary-time formalism}
In classical mechanics, a thermal system at temperature $T = 1 / \beta$ is described as an ensemble state, which have a probability $P_n$ of being in state $n$, with energy $E_n$.
In the canonical ensemble, the probability is proportional to $e^{-\beta E_n}$.
The expectation value of some quantity $A$, with value $A_n$ in state $n$ is
\begin{equation*}
    \ex{A} 
    = \sum_n A_n P_n = \frac{1}{Z} \sum_n A_n e^{- \beta E_n}, \quad 
    Z  = \sum_n e^{-\beta A_n}.
\end{equation*}
$Z$ is the partition function.

In quantum mechanics, an ensemble configuration is described by a non-pure density operator,
\begin{equation*}
    \hat \rho = C \sum_n P_n \ketbra{n}{n},
\end{equation*}
where $\ket{n}$ is some basis for the relevant Hilbert space and $C$ is a constant. Assuming $\ket{n}$ are energy eigenvectors, i.e. $\hat H \ket{n} = E_n \ket{n}$, the density operator for the canonical ensemble is
\begin{equation*}
    \hat \rho 
    = C \sum_n e^{-\beta E_n} \ketbra{n}{n} 
    = C e^{-\beta \hat H} \sum_n \ketbra{n}{n} 
    = C e^{-\beta \hat H}.
\end{equation*}
The expectation value in the ensemble state of a quantity corresponding to the operator $\hat A$ is given by
\begin{align}
    \ex{A} = \frac{ \Tr{\hat \rho \hat A} }{\Tr{\hat \rho }}
    = \frac{1}{Z} \Tr{\hat A e^{-\beta \hat H}}
\end{align}
The partition function $Z$ ensures that the probabilities adds up to 1, and is defined as
\begin{equation}
    Z = \Tr{e^{-\beta \hat H}}.
\end{equation}

The grand canonical ensemble takes into account the conserved charges of the system.
Conserved charges are a result of Nöther's theorem.
Assume we have a set of fields $\varphi_\alpha$. Nöther's theorem tells us that if the Lagrangian $\Ell[\varphi_\alpha]$ has a \emph{continuous symmetry}, then there is a corresponding conserved current~\cite{Peskin:IntroQFT,Carroll:spacetime}.
To define a continuous symmetry of the Lagrangian, we need a one-parameter family of transformations,
\begin{equation*}
    \varphi_\alpha(x) \longrightarrow \varphi_\alpha'(x; \epsilon)
    \sim \varphi_\alpha(x) + \varepsilon \eta_\alpha(x), \, \varepsilon \rightarrow 0.
\end{equation*}
Here, $\eta_\alpha(x)$ is some arbitrary function which define the transformation as $\varepsilon \rightarrow 0$.
Applying this transformation to the Lagrangian will in general change its form,
\begin{equation*}
    \Ell[\varphi_\alpha] \rightarrow \Ell[\varphi_\alpha']
    \sim \Ell[\varphi_\alpha] + \varepsilon \delta \Ell, \,
    \varepsilon \rightarrow 0.
\end{equation*}
If the change in the Lagrangian can be written as a total derivative, i.e.
\begin{equation*}
    \delta \Ell = \partial_\mu K^\mu(x),
\end{equation*}
we say that the Lagrangian has a continuous symmetry.
This is because a term of this form will result in a boundary term in the action integral, which does not contribute to the variation of the action.
Nöther's theorem states more precisely that the current
\begin{equation}
    j^\mu = \pdv{\Ell}{(\partial_\mu \varphi_\alpha)} \eta_\alpha - K^\mu
\end{equation}
obeys the conservation law
\begin{equation}
    \partial_\mu j^\mu = 0.
\end{equation}
The flux of current through some space-like surface $V$, i.e. a surface with a time-like normal vector, defines a conserved charge. This surface is most commonly a surface of constant time in some reference frame. 
The charge is defined as
\begin{equation}
    Q = \int_V \dd^3 x\, n_\mu j^\mu, \quad \, \pdv{t} Q = \nabla \cdot \vec j,
\end{equation}
where $n^\mu$ is the normal vector of $V$.
For a surface of constant time, $n^\mu = (1, 0, 0, 0)$, and the conserved charged is
\begin{equation}
    Q = \int_V \dd x \, j^0.
\end{equation}
In the grand canonical ensemble, a system with $n$ conserved charges $Q_i$ has probability proportional to $e^{-\beta (H - \mu_i Q_i)}$.
$\mu_i$ are the chemical potentials corresponding to conserved charge $Q_i$.
This leads to the partition function
\begin{equation}
    Z = \Tr{e^{-\beta(\hat H - \mu_i \hat Q_i)}}.
\end{equation}

The partition function may be calculated in a similar way to the path integral approach, in what is called the imaginary-time formalism. 
This formalism is restricted to time independent problems, and is used to study fields in a volume $V$.
This volume is taken to infinity in the thermodynamic limit.
As an example, take a scalar quantum field theory with the Hamiltonian
\begin{equation}
    \hat H
    = \int_V \dd^3 x \, \hat \He\left[\hat \varphi(\vec x), \hat \pi(\vec x)\right],
\end{equation}
where $\hat \varphi(\vec x)$ is the field operator, and $\hat \pi(\vec x)$ is the corresponding canonical momentum operator.
These field operators have time independent eigenvectors, $\ket{\varphi}$ and $\ket \pi$, defined by
\begin{equation}
    \hat \varphi(\vec x) \ket{\varphi} = \varphi(\vec x) \ket{\varphi}, \quad
    \hat \pi(\vec x) \ket{\pi} = \pi(\vec x) \ket{\pi}.
\end{equation}
In analogy with regular quantum mechanics, they obey the relations~\cite{Kapusta:finiteTemp}
\footnote{Some authors write $\D\pi/2 \pi$. This extra factor $2\pi$ is a convention which in this text is left out for notational clarity.}
\begin{gather}
    \label{functional completness}
    \one
    = \int \D \varphi(\vec x) \ketbra{\varphi}{\varphi} 
    = \int \D\pi(\vec x) \ketbra{\pi}{\pi}, \\
     \braket{\varphi}{\pi} 
    = \exp(i \int_V \dd x \, \varphi(\vec x) \pi(\vec x)), \\
    \braket{\pi_a}{\pi_b}
    =  \delta(\phi_a - \phi_b), \quad
    \braket{\varphi_a}{\varphi_b} 
    = \delta(\varphi_a - \varphi_b).
\end{gather}
The functional integral is defined by starting with $M$ degrees of freedom, $\{\varphi_m\}_{m=1}^M$ located at a finite grid $\{\vec x_m\}_{m=1}^M \subset V$.
The integral is then the limit of the integral over all degrees of freedom, as $M \rightarrow \infty$:
\begin{equation*}
    \int \D \varphi(\vec x) = \lim_{M \rightarrow \infty} \int \left(\prod_{m=1}^M \dd \varphi_m\right).
\end{equation*}
The functional Dirac-delta $\delta(f) = \prod_x\delta(f(x))$ is generalization of the familiar Dirac delta function.
Given a functional $\mathcal{F}[f]$, it is defined by the relation
\begin{equation}
    \int \D f(x)\, \mathcal{F}[f] \delta(f - g) = \mathcal{F}[g].
\end{equation}
The Hamiltonian is the limit of a sum of Hamiltonians $\hat H_m$ for each point $\vec x_m$
\begin{equation*}
    \hat H
    = \lim_{M \rightarrow \infty} \sum_{m=1}^M 
    \frac{V}{M} \hat H_m(\{\hat \varphi_m\}, \{\hat \pi_m\}).
\end{equation*}
$H_m$ may depend on the local degrees of freedom $\hat \varphi_m, \, \hat \pi_m$ as well as those at neighboring points.
By inserting the completeness relations \autoref{functional completness} $N$ times into the definition of the partition function, it may be written as
\begin{align*}
    Z& 
    = \int \D\varphi(\vec x) \, \inner{\varphi}{e^{- \beta \hat H}}{\varphi}
    = 
    \prod_{n=1}^N  
    \left(
        \int \D \varphi_n (\vec x) \int \D \pi_n(\vec x)
    \right) 
    \prod_{n=1}^N  \braket{\varphi_1}{\pi_1} 
    \inner{\pi_n}{e^{- \epsilon \hat H}}{\varphi_{n+1}},
\end{align*}
where $\epsilon = \beta / N$ and $\varphi_1 = \varphi_{n+1}$.
Strictly speaking, we only need to require $\varphi_1 = e^{i\theta}\varphi_{n+1}$, as the partition function is only defined up to a constant.
As will be shown later, bosons such as the scalar field $\varphi$, follow the periodic boundary condition $\varphi(0, \vec x) = \varphi(\beta, \vec x)$, i.e. $e^{i\theta} = 1$, while fermions follow the anti-periodic boundary condition $\psi(0, \vec x) = -\psi(\beta, \vec x)$, i.e. $e^{i\theta} = -1$.
We now want to exploit the fact that $\ket{\pi}$ and $\ket{\varphi}$ are the eigenvectors of the operators that define the Hamiltonian.
In our case, as the Hamiltonian density $\He$ can be written as a sum of functions of $\varphi$ and $\pi$ separately, $\He[\varphi(\vec x), \pi(\vec x)] = \mathcal{F}_1[\varphi(\vec x)] + \mathcal{F}_2[\pi(\vec x)]$ we may evaluate it as $\inner{\pi_n}{\He[\hat \varphi(x), \hat \pi(x)]}{\varphi_{n+1}} = \He[\varphi_{n+1}(x), \pi_n(x)] \braket{\pi_n}{\varphi_{n+1}}$.
This relationship does not, however, hold for more general functions of the field operators.
In that case, one has to be more careful about the ordering of the operators, for example by using \emph{Weyl ordering}~\cite{Peskin:IntroQFT}.
By series expanding $e^{-\epsilon \hat H}$ and exploiting this relationship, the partition function can be written as, to second order in $\epsilon$,
\begin{align*}
    Z = 
    \prod_{n=1}^N  
    \left(
        \int \D \varphi_n (\vec x) \int \D \pi_n(\vec x)
    \right)
    \exp[-\epsilon \sum_{n=1}^N \int_V \dd^3x \,
    \left(
        \He(\varphi_n(\vec x), \pi_n(\vec x)) - i \pi_n(\vec x) \frac{\varphi_n(\vec x) - \varphi_{n+1}(\vec x)}{\epsilon}
    \right)
    ].
\end{align*}
We denote $\varphi_n(\vec x) = \varphi(\tau_n, \vec x) $, $\tau \in [0, \beta]$ and likewise with $\pi_n(\vec x)$. 
In the limit $N \rightarrow \infty$, the expression for the partition function becomes
\begin{align}
    \label{Thermal partition function}
    Z = \int_S \D \varphi(\tau, \vec x)
    \int \D \pi(\tau, \vec x)
    \exp{
        - \int_0^\beta \dd \tau \int_V \dd \vec x \, 
        \left\{
            \He[\varphi(\tau, \vec x), \pi(\tau, \vec x)]
            - i \pi(\tau, \vec x) \dot \varphi(\tau, \vec x)
        \right\}
        },
\end{align}
where $S$ is the set of field configurations $\varphi$ such that  $\varphi(\beta, \vec x) = \varphi(0, \vec x)$.
With a Hamiltonian density of the form $\He = \frac{1}{2} \pi^2 + \frac{1}{2} (\nabla \varphi)^2 + \Ve(\varphi)$, we can evaluate the integral over the canonical momentum $\pi$ by discretizing $\pi(\tau_n, \vec x_m) = \pi_{n,m}$,
\begin{align*}
    & \int \D \pi \exp{-  \int_0^\beta\dd \tau \int_V \dd^3x 
    \left(
        \frac{1}{2} \pi^2 - i \pi \dot \varphi 
    \right)} \\
    & = \lim_{M,N \rightarrow \infty} \int \left(\prod_{m, n = 1}^{M, N} \frac{\dd \pi_{m, n}}{2 \pi}\right)
    \exp{
        - \sum_{m, n} \frac{V\beta}{MN}
        \left[
            \frac{1}{2}  (\pi_{m, n} - i \dot \varphi_{m, n})^2
            + \frac{1}{2} \dot \varphi_{m, n}^2
        \right]
    } \\
    & = \lim_{M,N \rightarrow \infty} \left( \frac{M N }{2 \pi V \beta} \right)^{MN/2}
    \exp{- \int_0^\beta\dd \tau \int_V \dd^3x \, \frac{1}{2}\dot \varphi^2},
\end{align*}
where $\dot \varphi_{m, n} = (\varphi_{m, n+1} - \varphi_{m, n})/\epsilon$.
The partition function is then, 
\begin{equation}
    Z = C \int \D \varphi
    \exp{
        - \int_0^\beta \dd \tau \int_V \dd^3 x
        \left[
            \frac{1}{2} \left(\dot \varphi^2 + \nabla \varphi^2\right) 
            + \Ve(\varphi)
        \right]
    }.
\end{equation}
Here, $C$ is the divergent constant that results from the $\pi$-integral.
In the last line, we exploited the fact that the variable of integration $\pi_{n,m}$ may be shifted by a constant without changing the integral, and used the Gaussian integral
\begin{equation*}
    \int_{-\infty}^{\infty} \dd x \, e^{-a x^2/2} = \sqrt{\frac{2 \pi}{a}}.
\end{equation*}

The partition function resulting from this procedure may also be obtained by starting with the ground state path integral
\begin{equation*}
    Z_g
    =\int \D\varphi\D\pi
    \exp{i \int_{\Omega'} \dd^4 x \, \left(\pi\dot\varphi - \He[\varphi, \pi]\right)}
    = C' \int \D \varphi(x)
    \exp{i \int_{\Omega'} \dd^4 x \, \Ell[\varphi, \partial_\mu \varphi]},
\end{equation*}
and follow a formal procedure.
The domain of the functional integral $\int \D \varphi$ is restricted from \emph{all} (smooth enough) field configurations $\varphi(t, \vec x)$, to only those that obey $\varphi(\beta, \vec x) = e^{i\theta} \varphi(0, \vec x) $, which is denoted $S$.
The action integral is modified by performing a Wick-rotation of the time coordinate $t$. 
This involves changing the domain of $t$ from the real line to the imaginary line by closing the contour at infinity, and making the change of variable $it \rightarrow \tau$.
The new variable is then restricted to the interval $\tau\in [0, \beta]$.
This procedure motivates the introduction of  the Euclidean Lagrange density, 
$\Ell_E(\tau, \vec x) = -\Ell(-i \tau, \vec x)$, as well as the name ``imaginary-time formalism''.
The result is the same partition function as before,
\begin{align*}
    Z = C \int_S \D \varphi \int \D \pi
    \exp{
        \int_0^\beta \dd \tau \int_V \dd^3x \, 
        \left[
            i\dot \varphi \pi
            - \He(\varphi, \pi)
        \right]
    } 
    =
    C' \int_S \D \varphi
    \exp{- \int_0^\beta \dd \tau \int_V \dd^3x \, \Ell_E(\varphi, \pi)}.
\end{align*}

