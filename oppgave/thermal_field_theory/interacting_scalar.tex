\subsection*{Interacting scalar}

We now study a scalar field with a $\lambda \varphi^4$ interaction term.
We write the Lagrangian in the form
\begin{equation*}
    \Ell = \Ell^{(0)} + \Ell^{(I)}, \quad 
    \Ell^{(0)} = 
    \frac{1}{2} \partial_\mu \varphi \partial^\mu \varphi  + m^2 \varphi^2 , \quad
    \Ell^{(I)} = - \frac{\lambda}{4!} \varphi^4
\end{equation*}
$\Ell^{(I)}$ is called the interaction term, and makes it impossible to exactly solve for the partition function.
Instead, we turn to perturbation theory.
The grand canonical partition function in this theory
\begin{equation}
    Z = \Tr{e^{- \beta \hat H}}
    = \int_S \D \varphi \, \exp{
        - \int_\Omega \dd X \left(\Ell_E^{(0)} + \Ell_E^{(I)}\right)
    }
    = \int_S \D \varphi \, e^{S_0} e^{S_I}.
\end{equation}
Here, $S_0$ and $S_I$ denote the Euclidean action due to the free and interacting Lagrangian, respectively.
The domain of integration $S$ is again periodic field configurations $\varphi(\beta, \vec x) = \varphi(0, \vec x)$.
We may write the free energy as
\begin{equation*}
    - \beta F = \ln
    \left[
        \int_S \D \varphi \, e^{S_0} \sum_n \frac{1}{n!} {S_I}^n
    \right]
    = \ln[Z_0] 
    + \ln
    \left[
        Z_I
        % \sum_n \frac{1}{n!}  
        % \frac{
        %     \int_S \D \varphi \, e^{S_0} {S_I}^n}
        % {\int_S \D \varphi \, e^{S_0}}
    \right],
\end{equation*}
where $Z_0$ is the partition function from the free theory.
The correction to the partition function is thus given by
\begin{equation}
    Z_1 = \sum_{n=0}^\infty \frac{1}{n!} \ex{{S_I}^n}_0,
\end{equation}
where
\begin{equation}
    \ex{A}_0 = \frac{
        \int_S \D \varphi \, A \, e^{S_0} }
    {\int_S \D \varphi \, e^{S_0}}.
\end{equation}
% Notice that the constant factor from the Jacobian due to the change of variable $\varphi \rightarrow \tilde \varphi$ does not affect the expectation value, as the same factor is in both the numerator and denominator.
% If the quantity $A$ is a function of the momentum-space fields, $A = A[\tilde \varphi(K)]$, then this expectation value takes the form
% \begin{equation}
%     \ex{A}_0 = 
%     \frac{
%         \int_{\tilde S} \D \tilde \varphi(K) \, f[\tilde\varphi(K)] \, 
%         \exp{- \frac{1}{2} \langle \tilde \varphi^*, D \tilde \varphi \rangle}
%         }
%     {
%         \int_{\tilde S} \D \tilde \varphi(K) \,
%         \exp{
%             - \frac{1}{2} \langle \tilde \varphi^*, D \tilde \varphi \rangle
%             }
%     }.
% \end{equation}
% where, as before, 
% \begin{equation}
%     \langle \tilde \varphi^*, D \tilde \varphi \rangle
%     = 
%     \int_\Omega 
%             \tilde \dd K\, [\beta^2(\omega_n^2 + \omega_n^2)] |\tilde \varphi(K)|^2
% \end{equation}

To evaluate expectation values of the form $\ex{\varphi(X_1) ... }_0$, we introduce the the partition function with a source term
\begin{align}
    Z[J] = \int_S \D \varphi \, \exp{
        - \frac{1}{2} \int_\Omega \dd X \, \varphi (-\partial_E^2 + m^2) \varphi
        + \int_\Omega \dd X \, J \varphi
    }.
\end{align}
Using the thermal Greens function $D(X, Y)$, as defined in \autoref{Conventions and notation}, we may complete the square to write
\begin{align}
    Z[J] = Z[0]\exp{\frac{1}{2} \int_{\Omega} \dd X \dd Y J(X) D_0(X, Y) J(Y)}
    = Z[0] \exp(W[J])
\end{align}
We can now write
\begin{equation}
    \ex{\varphi(X)\varphi(Y)}_0 
    = \frac{1}{Z[0]}
    \frac{\delta}{\delta J(X)} \frac{\delta}{\delta J(Y)} 
    Z[J] \Big|_{J=0} 
    = D(X, Y),
\end{equation}
where $D(X, Y)$ is the thermal propagator, as defined in \autoref{Conventions and notation}
This generalizes to higher order expectation values,
\begin{equation}
    \ex{\varphi(X_i) \dots \varphi(X_n)}_0
    = \left(\prod_{i=1}^n \frac{\delta}{\delta J(X_i)}\right) 
    Z[J] \Big|_{J=0},
\end{equation}
The exponential form of $Z[J]$ leads straight forwardly to Wick's theorem, which states that an expectation value of $2n$ fields is a sum of \emph{all possible, distinct} combination of $n$ propagators.
To write this in a formal way, we define the functions $a$ and $b$, which define a way to pair up $2m$ elements.
The domain of the functions are the integers between $1$ and $m$, the image a subset of the integers between $1$ and $2m$ of size $m$.
A valid pairing is a set $\{(a(1), b(1)), \dots (a(m), b(m))\}$, where all elements $a(i)$ and $b(j)$ are different, such all integers up to and including $2m$ are featured.
A pair is not directed, so $(a(i), b(i))$ is the same pair as $(b(i), a(i))$.
Wick theorem states that,
\begin{equation}
    \ex{\prod_{i=1}^{2m} \varphi(X_i)  }_0
    = \sum_{\{(a, b)\}} \ex{\varphi(X_{a(i)}) \varphi(X_{b(i)})}.
\end{equation}
where the  sum is over all tuples $(a, b)$ that define a valid and unique pairing.
Using Wick's theorem, the expectation values we are evaluating can be written
\begin{align*}
    \ex{{S_I}^m} & 
    = \left(- \frac{\lambda }{4!}\right)^m 
    \int_{\Omega} \dd X_1 \dots \dd X_m
    \ex{\varphi(X_1)^4 \dots \varphi(X_m)^4} \\ 
    & \quad
    = \left(- \frac{\lambda }{4!}\right)^m 
    \int_{\Omega} \dd X_1 \dots \dd X_m \sum_{\{a, b\}}
    \ex{\varphi(X_{a(1)}) \varphi(X_{b(1)})} 
    \dots
    \ex{\varphi(X_{a(2m)}) \varphi(X_{b(2m)})}
\end{align*}
where $X_i$ for $i>m$ is defined as $X_j$, where $j = i \mod m$.
Inserting the fourier expandions of the field gives
\begin{align*}
    & \ex{{S_I}^m} \\ 
    &\quad 
    = \left(-\frac{\lambda }{4!}\right)^m 
    \int_{\Omega} \dd X_1 \dots \dd X_m
    (V \beta)^2 \int_{\tilde \Omega} \dd K_1 ... \dd K_{2m} \sum_{\{a, b\}} \\
    & \quad \quad \quad \quad\quad \quad \quad
    \ex{\varphi(K_{a(1)}) \varphi(K_{b(1)})} 
    \dots
    \ex{\varphi(K_{a(2m)}) \varphi(K_{b(2m)})}     
    \exp(i {\sum}_{i=1}^{m} X_i \cdot K_i)\\ 
    & \quad  
    = \left(-\frac{\lambda }{4!}\right)^m 
    \frac{(V \beta)^{2m} \beta^m}{(V \beta^2)^{2m}}
    \int_{\tilde \Omega} \dd K_1 ... \dd K_{2m} \sum_{\{a, b\}} \\
    & \quad \quad \quad \quad \quad \quad \quad \quad \quad
    \tilde D(K_{a(1)}) \delta(K_{a(1)} + K_{b(1)}) \dots 
    \tilde D(K_{a(2m)}) \delta(K_{a(2m)} + K_{b(2m)})
    \prod_{i=1}^m \delta\left(X_i \cdot {\sum}_{j=0}^3 K_{i + jm}\right) \\
    & \quad 
    = \left(-\frac{\lambda \beta}{4!}\right)^m 
    \prod_{i=1}^{2m} \int_{\tilde \Omega} 
    \left( \dd K_i \frac{1}{\beta} \tilde D(K_i)  \right) 
    \prod_{i=1}^m \delta\left(X_i \cdot {\sum}_{j=0}^3 K_{i + jm}\right)
    \sum_{\{a, b\}} 
    \prod_{n=1}^{2m}\delta(K_{a(k)} + K_{b(k)})
\end{align*}
Here we have used that $V \beta^2 \tilde D_0(K, P) = \tilde D_0(K) \delta(P + K)$, and $\tilde D(K)$ is the thermal propagator for the free field, as defined in \autoref{Conventions and notation}.
In this case, the thermal propagator of the free field is
\begin{equation}
    D_0(K) = D_0(\omega_n, \vec k) = \frac{1}{\omega_k^2 + \omega_n^2}.
\end{equation}
This expectation value can be represented graphically using Feynman diagrams.
The thermal $\lambda \varphi^2$-theory gets the prescription

\begin{align}
    \feynmandiagram [inline=(a.base), small, horizontal=i1 to f2]
    {
    {i1} -- [fermion, edge label'=$K_1$] a[dot] 
    -- [anti fermion, edge label'=$K_3$] {f1},
    {i2} -- [fermion, edge label'=$K_2$] a -- [anti fermion, edge label'=$K_4$] {f2},
    };
    & = -\lambda \beta
    \delta \left({\sum}_i K_i \right), \\ \nonumber \\
    \feynmandiagram[horizontal= i to f]{
        i[particle=$K$] -- [fermion] f,
    };
    & = \frac{1}{\beta} \int_{\tilde \Omega} \dd K \,  D_0(K).
\end{align}
The factor $1/4!$ is removed as a general Feynman diagram represent all diagrams with the same form, but different pairing of the momenta.
Some diagrams are more symmetric, such that an exchange of momenta still gives \emph{the same pairing}. 
This is dealt with by dividing with a symmetry factor $s$, which is described in detail in~\cite{Peskin:IntroQFT}.

Calculating $\ex{{S_I}^n}_0$ boils down to the sum of all possible Feynman diagrams with $m$ vertices.
The first example is 
\begin{align}
    \ex{S_I} = 
    \feynmandiagram[small, horizontal=a to b, inline=(b.base)]
    {
        b[dot] --[fermion, half left, looseness=1.5, edge label'=$K_1$] a 
        --[fermion, half left, looseness=1.5, edge label'=$K_2$] b,
        b --[fermion, half right, looseness=1.5, edge label'=$K_3$] c 
        --[fermion, half right, looseness=1.5, edge label'=$K_4$] b,
    }; 
    +
    \feynmandiagram[small, horizontal=a to b, inline=(b.base)]
    {
        b[dot] --[fermion, half left, looseness=1.5, edge label'=$K_1$] a 
        --[fermion, half left, looseness=1.5, edge label'=$K_4$] b,
        b --[fermion, half right, looseness=1.5, edge label'=$K_2$] c 
        --[fermion, half right, looseness=1.5, edge label'=$K_4$] b,
    };
    +
    \feynmandiagram[small, horizontal=a to b, inline=(b.base)]
    {
        b[dot] --[fermion, half left, looseness=1.5, edge label'=$K_1$] a 
        --[fermion, half left, looseness=1.5, edge label'=$K_4$] b,
        b --[fermion, half right, looseness=1.5, edge label'=$K_2$] c 
        --[fermion, half right, looseness=1.5, edge label'=$K_3$] b,
    };
    = 
    \frac{3}{4!} \times 
    \feynmandiagram[small, horizontal=a to b, inline=(b.base)]
    {
        b[dot] --[fermion, half left, looseness=1.5] a 
        --[fermion, half left, looseness=1.5] b,
        b --[fermion, half right] c 
        --[fermion, half right] b,
    };.
\end{align}

For higher order, one gets both connected and disconnected diagrams.
Let $\ex{{S_I}^n}_{0, c}$ be only the connected diagrams, i.e. those diagrams in which it is possible to move between all vertices along a series of edges.

A general diagram contain $n_i$ copies of a connected diagram with the value $V_i$.
The value of the total diagram is then the product of the value of all its disconnected pieces, but with the caveat that each diagram has a symmetry factor of $n_i!$.
The sum of all diagrams is thus 
\begin{equation}
    Z_I = \sum_n \frac{1}{n!} \ex{{S_I}^n} 
    = \sum_{ all\,sets\,\{n_i\}} \prod_i \frac{1}{n_i!}V_i^{n_i}
    = \prod_i \sum_{n_i} \frac{1}{n_i!}V_i^{n_i} = \exp({\sum}_i V_i).
\end{equation}
Thus, the correction to the free energy is given by the sum of all connected diagrams,
\begin{equation}
    - \beta F = \ln(Z_0) + \sum_n \ex{{S_I}^n}_{0, c}.
\end{equation}

\subsection*{Rester}


Expectation values of field configurations with different momenta vanish,
\begin{equation}
    K' \neq K \implies \ex{\varphi(K) \varphi(K')}_0 = 0.
\end{equation}
This is due to the symmetry of the functional integrand.
This can be seen by going back to the discrete version of the integral.
The integral can then be factored into integrals over each degree of freedom, $\varphi_{n,m}$.
The integrall over the degrees of freedom corresponding to $\varphi(K)$ and $\varphi(K')$ are of the form
\begin{equation}
    \int \dd \varphi_{n, m} \, \varphi_{n, m} \exp(- S^{(0)}_{n, m}) = 0,
\end{equation}
as the integral is over the entier real line, and the integrand is anti-symmetric in $\varphi_{n, m}$.
The generalization of this rule is straight forward. 
Assuming $f[\tilde \varphi]$ \emph{does not} depend on $\tilde \varphi(K)$, but possibly on $\tilde \varphi(K'), \, K \neq K'$, then
\begin{equation}
    \ex{f[\tilde \varphi](\tilde \varphi(K))^{(2n+1)}}_0 = 0, \, n \in \mathbb{Z}.
\end{equation}
Furthermore, as the field at different values of $K$ are independent, we get the familiar formula for expectation values of independent variables, that is
\begin{equation}
    \ex{f[\varphi(K)]g[\varphi(K')]}_0 = \ex{f[\varphi(K)]}_0 \ex{g[\varphi(K')]}_0, \quad
    K \neq K'.
\end{equation}
This leads to a version of Wick's theorem for thermal systems, as it lays constrains on integrals of the form
\begin{equation}
    \int_{\tilde \Omega} \dd K_1 \dots \dd K_n \,
    \ex{\tilde \varphi(K_1) ... \tilde \varphi(K_n)}_0 \delta\left({\sum}_i K_i\right)
    = 
\end{equation}
