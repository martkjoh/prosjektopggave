\subsection{Regulating the free energy}
\label{section: regualting free energy}
Using the result from \autoref{section:thermal sum} on the result for the free energy density of the free scalar field, \autoref{free scalar result 2}, we get
\begin{equation}
    \label{free scalar free enrgy}
    \Ef = \frac{\ln(Z)}{\beta V}
    = \frac{1}{2} \int_{\tilde V} \frac{\dd^3 k}{(2 \pi)^3}
    \left[
        \omega_k + \frac{2}{\beta }\ln(1 - e^{-\beta\omega_k})
    \right].
\end{equation}
The first part of this integral is a temperature independent vacuum energy, while the second part encodes all the termperature dependence of the free energy density
This free energy has two parts, the first part is dependent on temperature, the other is temperature independent vacuum contribution.
Noticing that the integral is spherically symmetric, we may write the two contributions as
\begin{equation}
    J_0 = \frac{1}{2} \frac{1}{2 \pi^2}\int_{\R} \dd k \, k^2 \sqrt{k^2 + m^2}, \quad
    J_T = \frac{T^4}{2 \pi^2}\int_\R \dd x \, x^2  \ln(1 - e^{-\sqrt{x^2 + (m/T)^2}}), 
\end{equation}
The temperature-independent part, $J_0$, is clearly divergent, and we must therefore impose a regulator, and then add counter-terms.
$J_T$, however, is convergent. 
To see this, we use the series expansion $\ln(1 + \epsilon) \sim \epsilon + \Oh{\epsilon}$ to find the leading part of the integrand for large $k$'s, 
\begin{equation}
    x^2 \ln(1 - e^{-\sqrt{x^2 + (\beta m)^2}}) \sim - x^2 e^{-x}, 
\end{equation}
which is exponentially suppressed, making the integral convergent.
In the limit of $T \rightarrow \infty$, we get
\begin{equation}
    J_\infty \sim \frac{T^4}{2 \pi^2} \int_\R \dd x \, x^2 \ln(1 - e^{-x})
    = - \frac{T^4}{2 \pi^2} \sum_{n=1} \frac{1}{n} \diffp[2]{}{n} \int \dd x e^{-nx}
    = - \frac{T^4}{2 \pi^2} \sum_{n=1} \frac{2}{n^4}
    = - \frac{T^4}{\pi^2} \zeta(4)
    = - T^4 \frac{\pi^2}{90}.
    % = -\frac{3}{2} \sigma T^4,
\end{equation}
Where $\zeta$ is the Riemann-zeta function.

Returning to the temperature-independent part, we use dimensional regularization to see its singular behavior.
To that end, we define
\begin{equation}
    \label{def dimreg integral}
    \Phi_n(m, d, \alpha) = \mu^{n - d}\int_{\tilde \Omega} \frac{\dd^d k}{(2 \pi)^d} (k^2 + m^2)^{-\alpha},
\end{equation}
so that $J_0 = \Phi_3(m, 3, 1/2) / 2$.
The parameter $\mu$ has the dimensions of $k$, and is inserted to ensure that $\Phi_n$ does not change physical dimension for $d \neq n$.
Furthermore, as non-rational exponents are defined through the exponential functions, this parameter is needed to make the expression well-defined.
Dimensional regularization takes uses the formula for integration of spherically symmetric function in $d$-dimensions,
\begin{equation}
    \int_{\R^d} \dd^d x \, f(r) 
    = \frac{2 \pi^{d/2}}{\Gamma(d/2)} \int_\R \dd r \, r^{d-1}f(r),
\end{equation}
where $r = \sqrt{x_i x_i}$ is the radial distance, and $\Gamma$ is the Gamma function.
The factor in the front of the integral comes from the solid angle.
By extending this formula from integer-valued $d$ to real numbers, the function we defined becomes
\begin{equation}
    \Phi_n
    = \frac{2 \pi^{d/2} \mu^{n - d} }{\Gamma(d/2)} \int_\R \dd k \, 
    \frac{k^{d-1}}{(k^2 + m^2)^\alpha}
    = \frac{m^{n-2\alpha}}{(4 \pi)^{d / 2}\Gamma(d/2)} 
    \left(\frac{m}{\mu}\right)^{d-n} 
    2 \int_\R \dd z \, \frac{z^{d - 1}}{(1 + z)^\alpha}, 
\end{equation}
where we have made the change of variables $m z = k$.
We make one more change of variable to the integral,
\begin{equation}
    I = 2 \int_\R \dd z \, \frac{z^{d - 1}}{(1 + z)^\alpha}
\end{equation}
Let
\begin{equation}
    z^2 = \frac{1}{s} - 1 \implies 2 z \dd z = - \frac{\dd s}{s^2}
\end{equation}
Thus,
\begin{equation}
    I = \int_0^a \dd s \, s^{\alpha - d/2 - 1} (1 - z)^{d/2 - 1}.
\end{equation}
This is the beta function, which can be written in terms of Gamma functions~\cite{Peskin:IntroQFT}
\begin{equation}
    I = B\left(\alpha - \frac{d}{2}, \frac{d}{2}\right) 
    = \frac{\Gamma\left(\alpha - \frac{d}{2}\right) \Gamma\left(\frac{d}{2}\right)}{\Gamma(\alpha)}.
\end{equation}
Combining this gives
\begin{equation}
    \label{result dimreg}
    \Phi_n(m, d, \alpha) = \frac{m^{n - 2\alpha}}{(4 \pi)^{d / 2}}
    \frac{
        \Gamma \left(\alpha - \frac{d}{2} \right) 
    }
    {\Gamma(\alpha)}
    \left(\frac{m^2}{\mu^2}\right)^{\flatfrac{(d-n)}{2}} 
    .
\end{equation}
Inserting $n=3$, $d = 3 - 2\epsilon$ and $\alpha = -1/2$, we get
\begin{equation}
    \Phi_3(m, 3 - 2\epsilon, -1/2)
    =
    \frac{m^4}{(4 \pi)^{d/2}\Gamma(-1/2)} \Gamma(-2 + \epsilon) \left(\frac{m^2}{\mu^2}\right)^{-\epsilon}
    =
    - \frac{m^4}{(4 \pi)^{2}}
    \left(\frac{m^2}{4 \pi \mu^2}\right)^{- \epsilon}
    \frac{\Gamma(\epsilon)}{(\epsilon - 2)(\epsilon - 1)},
\end{equation}
where we have used the defining property $\Gamma(z + 1) = z\Gamma(z)$ and $\Gamma(1/2) = \sqrt \pi$.
Expanding around $\epsilon = 0$ gives
\begin{align}
    \left(\frac{m^2}{4 \pi \mu^2}\right)^{- \epsilon}
    &\sim 1 + \epsilon \ln\left(4 \pi \frac{\mu^2}{m^2}\right),\\
    \Gamma(\epsilon) 
    & \sim \frac{1}{\epsilon} - \gamma, \\
    \frac{1}{(\epsilon - 2)(\epsilon - 1)}
    &\sim \frac{1}{2}\left(1 + \frac{3}{2} \epsilon\right).
\end{align}
The singular behavior of the time-independent term is therefore
\begin{align}
    J_0 \sim
    % - \frac{m^4}{2^6 \pi^2} \mu^{-2 \epsilon}
    % \left[1 + \epsilon \ln\left(4 \pi \frac{\mu^2}{m^2}\right)\right]
    % \left(\frac{1}{\epsilon} - \gamma\right) \left(1 + \frac{3}{2} \epsilon\right)
    % \sim
    - \frac{1}{4}\frac{m^4}{(4 \pi)^2}
    \left[
        \frac{1}{\epsilon} 
        - \gamma + \frac{3}{2}
        + \ln\left(4 \pi \frac{\mu^2}{m^2}\right)
    \right].
\end{align}

With this regulator, one can then add counter-terms to cancel the $\frac{1}{\epsilon}$-divergence.
The exact form of the counter-term is convention.
One may also cancel the finite contribution due to the regulator.
The minimal subtraction, or $\mathrm{MS}$, scheme, is to only subtract the divergent term, as the name suggest.
We will use the modified minimal subtraction, or $\overline{ \mathrm{MS}}$, scheme.
In this scheme, one also removes the $-\gamma$ and $\ln(4 \pi)$ term,
which can be interpreted as changing the parameter $\mu$
\begin{equation}
    -\gamma + \ln(4\pi \frac{\mu^2}{m^2}) \rightarrow \ln(\frac{\mu^2}{m^2}),
\end{equation}
which leads to the expression
\begin{equation}
    J_0 \sim
    - \frac{1}{4}\frac{m^4}{(4 \pi)^2}
    \left[
        \frac{1}{\epsilon} 
        + \frac{3}{2}
        + \ln\left(\frac{\mu^2}{m^2}\right)
    \right].
\end{equation}

