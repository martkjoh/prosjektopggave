\subsection{Regulating the free energy}
The free energy integral we obtained from the energy for the free scalar \autoref{free scalar free enrgy}, has two parts,
Noticing that the integral is spherically symmetric, we may write
\begin{equation}
    J_0 = \frac{1}{2} \int_{\tilde \Omega} \frac{\dd^3 k}{(2 \pi)^3} \sqrt{k^2 + m^2}, \quad
    J_T = \frac{1}{\beta} \int_{\tilde \Omega} \frac{\dd^3 k}{(2 \pi)^3} 
    \ln\left[1 - \exp(- \beta \sqrt{k^2 + m^2})\right], 
\end{equation}
The temperature-independent part, $J_0$, is clearly divergent, and we must therefore impose a regulator, and then add counter-terms.
This differs from the zero-temperature formalism, where there were no need to renormalize the free theory.
The second part of the integral, is convergent. 
To see this, we first exploit that it is spherically symmetric to write
\begin{equation}
    J_T = \frac{T^4}{2 \pi^2}\int_\R \dd x \, x^2  \ln(1 - e^{-\sqrt{x^2 + (\beta m)^2}}).
\end{equation}
Using the series expansion $\ln(1 + \epsilon) \sim \epsilon + \Oh{\epsilon}$, we can find the leading part of the  integrand for large $k$'s, 
\begin{equation}
    x^2 \ln(1 - e^{-\sqrt{x^2 + (\beta m)^2}}) \sim - x^2 e^{-x}, 
\end{equation}
which is suppressed exponentially, making the integral convergent.
In the limit of $T \rightarrow \infty$, we get
\begin{equation}
    J_\infty \sim \frac{T^4}{2 \pi^2} \int_\R \dd x \, x^2 \ln(1 - e^{-x})
    = - T^4 \frac{\pi^2}{90} = -\frac{3}{2} \sigma T^4,
\end{equation}
where $\sigma$ is the Stefan-Boltzmann constant. 

We use dimensional regularization on the temperature independent term. 
To that end, we define
\begin{equation}
    \Phi(m, d, A) = \int_{\tilde \Omega} \frac{\dd^d k}{(2 \pi)^d} (k^2 + m^2)^{-A},
\end{equation}
so that $J_0 = \Phi(m, 3, 1/2) / 2$.
Dimensional regularization takes uses the formula for integration of spherically symmetric function in $d$-dimensions,
\begin{equation}
    \int_{\R^d} \dd^d x \, f(r) 
    = \frac{2 \pi^{d/2}}{\Gamma(d/2)} \int_\R \dd r \, r^{d-1}f(r),
\end{equation}
where $r = \sqrt{x_i x_i}$ is the radial distance, and $\Gamma$ is the gamma-function.
The factor in the front of the integral comes from the solid-angle.
By extending this formula from integer-valued $d$ to real numbers, the function we defined becomes
\begin{equation}
    \Phi 
    = \frac{2 \pi^{d/2}}{\Gamma(d/2)} \int_\R \dd k \, 
    \frac{k^{d-1}}{(k^2 + m^2)^A}
    = \frac{m^{d - 2A}}{(4 \pi)^{d / 2}\Gamma(d/2)} 
    2 \int_\R \dd z \, \frac{z^{d - 1}}{(1 + z)^A}, 
\end{equation}
where we have made the change of variables $m z = k$.
We make one more change of variable to the integral,
\begin{equation}
    I = 2 \int_\R \dd z \, \frac{z^{d - 1}}{(1 + z)^A}
\end{equation}
Let
\begin{equation}
    z^2 = \frac{1}{s} - 1 \implies 2 z \dd z = - \frac{\dd s}{s^2}
\end{equation}
Thus,
\begin{equation}
    I = \int_0^a \dd s \, s^{A - d/2 - 1} (1 - z)^{d/2 - 1}.
\end{equation}
This is the beta-function, which can be written in terms of gamma-funcitons~\cite{Peskin:IntroQFT}
\begin{equation}
    I = B\left(A - \frac{d}{2}, \frac{d}{2}\right) 
    = \frac{\Gamma\left(A - \frac{d}{2}\right) \Gamma\left(\frac{d}{2}\right)}{\Gamma(A)}.
\end{equation}
Putting it all together, this gives
\begin{equation}
    \Phi = 
    \frac{
        (m^2)^{d/2 - A} \Gamma \left(A - \frac{d}{2} \right) 
    }
    {
        (4 \pi)^{d / 2}\Gamma(A)
    }.
\end{equation}
Inserting $d = 3 - 2\epsilon$ and $A = -1/2$, we get
\begin{equation}
    \frac{
        (m^2)^{3/2 - \epsilon - A} \Gamma \left(-2 + \epsilon \right) 
    }
    {
        (4 \pi)^{3/2 - \epsilon}\Gamma(-1/2)
    }
    = 
    \left(\frac{m^2}{4 \pi}\right)^{3/2} 
    \frac{m}{- 2 \pi^{1/2}}
    \mu^{-2\epsilon}
    \left(\frac{m^2}{4 \pi \mu^2}\right)^{- \epsilon}
    \frac{\Gamma(\epsilon)}{(\epsilon - 2)(\epsilon - 1)},
\end{equation}
where we have used the defining property $\Gamma(z + 1) = z\Gamma(z)$, and inserted a parameter $\mu$ with the dimensions of $m$.
Expanding around $\epsilon = 0$ gives
\begin{align}
    \left(\frac{m^2}{4 \pi \mu^2}\right)^{- \epsilon}
    &\sim 1 + \epsilon \ln\left(4 \pi \frac{\mu^2}{m^2}\right),\\
    \Gamma(\epsilon) 
    & \sim \frac{1}{\epsilon} - \gamma, \\
    \frac{1}{(\epsilon - 2)(\epsilon - 1)}
    &\sim \frac{1}{2}\left(1 + \frac{3}{2} \epsilon\right).
\end{align}
Inserting this back into the function gives
\begin{align}
    \Phi = 
    - \frac{m^4}{2^5 \pi^2} \mu^{-2 \epsilon}
    \left[1 + \epsilon \ln\left(4 \pi \frac{\mu^2}{m^2}\right)\right]
    \left(\frac{1}{\epsilon} - \gamma\right) \left(1 + \frac{3}{2} \epsilon\right)
    \sim
    - \frac{m^4}{2^5 \pi^2} \mu^{-2 \epsilon}
    \left[
        \frac{1}{\epsilon} 
        - \gamma + \frac{3}{2}
        + \ln\left(4 \pi \frac{\mu^2}{m^2}\right)
    \right].
\end{align}
