\subsection*{Low temperature limit}

Using the result from \autoref{section:thermal sum} on the result for the free energy density of the free scalar field, \autoref{free scalar result 2}, we get
\begin{equation}
    \label{free scalar free enrgy}
    \Ef = \frac{\ln(Z)}{\beta V}
    = \frac{1}{2} \int_{\tilde V} \frac{\dd^3 k}{(2 \pi)^3}
    \left[
        \omega_k + \frac{2}{\beta }\ln(1 - e^{-\beta\omega_k})
    \right].
\end{equation}
The free energy thus has two parts, the first part is dependent on temperature, the other is a temperature independent vacuum contribution.
Noticing that the integral is spherically symmetric, we may write the two contributions as
\begin{equation}
    \Ef_0 = \frac{1}{2} \frac{1}{2 \pi^2}\int_{\R} \dd k \, k^2 \sqrt{k^2 + m^2}, \quad
    \Ef_T = \frac{T^4}{2 \pi^2}\int_\R \dd x \, x^2  \ln(1 - e^{-\sqrt{x^2 + (m/T)^2}}), 
\end{equation}
The temperature-independent part, $\Ef_0$, is clearly divergent, and we must therefore impose a regulator, and then add counter-terms.
$\Ef_T$, however, is convergent. 
To see this, we use the series expansion $\ln(1 + \epsilon) \sim \epsilon + \Oh{\epsilon}$ to find the leading part of the integrand for large $k$'s, 
\begin{equation}
    x^2 \ln(1 - e^{-\sqrt{x^2 + (\beta m)^2}}) \sim - x^2 e^{-x}, 
\end{equation}
which is exponentially suppressed, making the integral convergent.
In the limit of $T \rightarrow 0$, we get
\begin{align}
    \nonumber
    \Ef_T & \sim 
    \frac{T^4}{2 \pi^2} \int_\R \dd x \, x^2 \ln(1 - e^{-x})
    = - \frac{T^4}{2 \pi^2} \sum_{n=1} \frac{1}{n} \diffp[2]{}{n} \int \dd x e^{-nx}
    = - \frac{T^4}{2 \pi^2} \sum_{n=1} \frac{2}{n^4}
    = - \frac{T^4}{\pi^2} \zeta(4)
\end{align}
Where $\zeta$ is the Riemann-zeta function.
Using $\zeta(4) = \frac{\pi^4}{90}$, we get
\begin{equation}
    \Ef_T \sim - \frac{\pi^2}{90} T^4, \quad T \rightarrow 0.
\end{equation}

\subsection*{Regularization}
\label{section: regualting free energy}

Returning to the temperature-independent part, we use dimensional regularization to control its divergent behavior.
To that end, we define
\begin{equation}
    \label{def dimreg integral}
    \Phi_n(m, d, \alpha) 
    = \int_{\tilde \Omega} \frac{\dd^d k}{(2 \pi)^d} (k^2 + m^2)^{-\alpha},
\end{equation}
so that $\Ef_0 = \Phi_3(m, 3, -1/2) / 2$.
We will use the formula for integration of spherically symmetric function in $d$-dimensions,
\begin{equation}
    \int_{\R^d} \dd^d x \, f(r) 
    = \frac{2 \pi^{d/2}}{\Gamma(d/2)} \int_\R \dd r \, r^{d-1}f(r),
\end{equation}
where $r = \sqrt{x_i x_i}$ is the radial distance, and $\Gamma$ is the Gamma function.
The factor in the front of the integral comes from the solid angle.
By extending this formula from integer-valued $d$ to real numbers, the function we defined becomes
\begin{equation}
    \Phi_n
    = \frac{2 \pi^{d/2} }{\Gamma(d/2)} \int_\R \dd k \, 
    \frac{k^{d-1}}{(k^2 + m^2)^\alpha}
    = \frac{m^{n-2\alpha}m^{d-n} }{(4 \pi)^{d / 2}\Gamma(d/2)} 
    2 \int_\R \dd z \, \frac{z^{d - 1}}{(1 + z)^\alpha}, 
\end{equation}
where we have made the change of variables $m z = k$.
We make one more change of variable to the integral,
\begin{equation}
    I = 2 \int_\R \dd z \, \frac{z^{d - 1}}{(1 + z)^\alpha}
\end{equation}
Let
\begin{equation}
    z^2 = \frac{1}{s} - 1 \implies 2 z \dd z = - \frac{\dd s}{s^2}
\end{equation}
Thus,
\begin{equation}
    I = \int_0^a \dd s \, s^{\alpha - d/2 - 1} (1 - z)^{d/2 - 1}.
\end{equation}
This is the beta function, which can be written in terms of Gamma functions~\cite{Peskin:IntroQFT}
\begin{equation}
    I = B\left(\alpha - \frac{d}{2}, \frac{d}{2}\right) 
    = \frac{\Gamma\left(\alpha - \frac{d}{2}\right) \Gamma\left(\frac{d}{2}\right)}{\Gamma(\alpha)}.
\end{equation}
Combining this gives
\begin{equation}
    \label{result dimreg}
    \Phi_n(m, d, \alpha) 
    = \mu^{n-d} \frac{m^{n - 2\alpha}}{(4 \pi)^{d / 2}}
    \frac{
        \Gamma \left(\alpha - \frac{d}{2} \right) 
    }
    {\Gamma(\alpha)}
    \left(\frac{m^2}{\mu^2}\right)^{\flatfrac{(d-n)}{2}} 
    .
\end{equation}
In the last step, we have introduced a parameter $\mu$ with mass dimension 1, that is $[\mu] = [m]$.
This is done to be able to series expand around $d - n$ in a dimensionless variable. 
This parameter is arbitrary, and all physical quantities should thererfore be independent of it.
We will shortly justify this parameter further.


Inserting $n=3$, $d = 3 - 2\epsilon$ and $\alpha = -1/2$, we get
\begin{equation}
    \Phi_3(m, 3 - 2\epsilon, -1/2)
    =
    \frac{m^4 \mu^{-2\epsilon}}{(4 \pi)^{d/2}\Gamma(-1/2)} \Gamma(-2 + \epsilon) \left(\frac{m^2}{\mu^2}\right)^{-\epsilon}
    =
    - \mu^{-2\epsilon} \frac{m^4}{(4 \pi)^{2}}
    \left(\frac{m^2}{4 \pi \mu^2}\right)^{- \epsilon}
    \frac{\Gamma(\epsilon)}{(\epsilon - 2)(\epsilon - 1)},
\end{equation}
where we have used the defining property $\Gamma(z + 1) = z\Gamma(z)$ and $\Gamma(1/2) = \sqrt \pi$.
Expanding around $\epsilon = 0$ gives
\begin{align}
    \left(\frac{m^2}{4 \pi \mu^2}\right)^{- \epsilon}
    &\sim 1 + \epsilon \ln\left(4 \pi \frac{\mu^2}{m^2}\right),\\
    \Gamma(\epsilon) 
    & \sim \frac{1}{\epsilon} - \gamma, \\
    \frac{1}{(\epsilon - 2)(\epsilon - 1)}
    &\sim \frac{1}{2}\left(1 + \frac{3}{2} \epsilon\right).
\end{align}
The divergent behavior of the temperature-independent term is therefore
\begin{align}
    \Ef_0 \sim
    - \mu^{-2\epsilon} \frac{1}{4}\frac{m^4}{(4 \pi)^2}
    \left[
        \frac{1}{\epsilon} 
        - \gamma + \frac{3}{2}
        + \ln\left(4 \pi \frac{\mu^2}{m^2}\right)
    \right].
\end{align}

With this regulator, one can then add counter-terms to cancel the $\epsilon^{-1}$-divergence.
The exact form of the counter-term is convention.
One may also cancel the finite contribution due to the regulator.
The minimal subtraction, or $\mathrm{MS}$, scheme, is to only subtract the divergent term, as the name suggest.
We will use the modified minimal subtraction, or $\overline{ \mathrm{MS}}$, scheme.
In this scheme, one also removes the $-\gamma$ and $\ln(4 \pi)$ term, by defining a new mass scale $\tilde \mu$ by
\begin{equation}
    \label{definition mu tilde MS bar}
    -\gamma + \ln(4\pi \frac{\mu^2}{m^2}) = \ln(4\pi e^{-\gamma} \frac{\mu^2}{m^2}) = \ln(\frac{\tilde\mu^2}{m^2}),
\end{equation}
which leads to the expression
\begin{equation}
    \label{free field regularized energy}
    \Ef_0 \sim
    - \mu^{-2\epsilon} \frac{1}{4}\frac{m^4}{(4 \pi)^2}
    \left(
        \frac{1}{\epsilon} 
        + \frac{3}{2}
        + \ln \frac{\tilde\mu^2}{m^2}
    \right), \quad \epsilon \rightarrow 0.
\end{equation}

\subsection*{Renormalization}

Now that we have applied a regulator, we are able to handle the divergence in a well-defined way.
When $\epsilon \neq 0$, we can subtract terms which are proportional to $\epsilon^{-1}$, and be left with a term that is finite in the limit $\epsilon \rightarrow 0$.
Consider an arbitrary Lagrangian, 
\begin{equation}
    \Ell[\varphi] = \sum_n \lambda_n \mathcal{O}_n[\varphi].
\end{equation}
Here, $\mathcal{O}_n[\varphi]$ are operators consisting of $\varphi$ and $\partial_\mu \varphi$, and $\lambda_n$ are coupling constants.
In $d$ dimensions, the action integral is
\begin{equation}
    S[\varphi] = \sum_n \int \dd^d x \, \lambda_n \mathcal{O}_n[\varphi].
\end{equation}
The action has mass dimension $0$.
This means that all terms $\lambda_n \mathcal O_n$ must have mass dimension $d$, as $[\dd^d x] = -d$.
We are free to choose the coupling constant corresponding $\mathcal O_0 = \partial_\mu \varphi \partial^\mu \varphi$ to be of mass dimension 0, and therefore set $\lambda_0 = 1/2$ to get canonicall normalization.
This allows us to deduce the dimensionality of $\varphi$.
As $[\partial_\mu] = 1$, we have that $[\varphi] = (d-2)/2$.
Assume $\mathcal O_n$ consists of $k_n$ factors of $\varphi$, and $l_n$ factors of $\partial_\mu \varphi$.
We must then have
\begin{align}
    [\lambda_n] + [\mathcal{O}_n] - d &= [\lambda_n] + (k_n + l_n)(d - 2) / 2 + l_n - d = 0, \\
    \implies D_n := [\lambda_n] &= d - k_n \frac{d - 2}{2} - l_n \frac{d}{2}.
\end{align}
From this formula, we recover that $[\lambda_0] = 0$, and if $\lambda_1 \varphi^2$, then $[\lambda_1] = 2$, which we recognize as the mass term squared.
The mass dimensions of these coupling constants are independent of $d$.
However, the coupling constant for the interaction term
\begin{equation}
    - \frac{1}{4!} \lambda_3 \varphi^4
\end{equation}
has mass dimensions $[\lambda_3] = d -4(d-2)/2 = 4 - 2d$.
Our goal now is to exchange the bare coupling constants $\lambda_n$ with renormalized ones, $\lambda_n^r$, and remove the divergent terms proportional to $(d - 4)^{-m}$.
We can always define the renormalized coupling constants as dimensionless, i.e. $[\lambda_n^r] = 0$, by measuring them in units of a mass scale.
We therefore write 
\begin{align*}
    \lambda_n = \mu^{4 - D_n}
    \left[
        \lambda_n^r + \sum_{m=1} \frac{a_m(\lambda_n^r)}{(d - 4)^{m}}
    \right],
\end{align*}
where we have introduced the dimensionfull parameter $\mu$ to ensure that $\lambda_n$ has the correct mass dimension, so that the action integral stays dimensionless.
The functions $a_m$ are then determined to each order in perturbation theory by calculating Feynman diagrams.
As $\mu$ again is arbitrary, $\lambda_4'$ should not depend on this parameter.
In this case, we chose the same renormalization scale as we did when regulating the one-loop integral.
This is only for our own convenience.
This means that if we change $\mu \rightarrow \mu'$, then $\lambda_i^r$ and $a_m$ must adjust to compensate and keep $\lambda_n$ constant~\cite{'t_hoft_dim_reg}.

The vacuum energy term absorbs the divergence in the one loop contribution to the free energy density.
It is
\begin{equation}
    \label{scalar field static term}
    \lambda_4 \mathcal{O}_4 = \lambda_4 = m^4 \lambda_4'.
\end{equation}
Using the expansion in terms of renormalized coupling, we have, using $d = 4 - 2\epsilon$
\begin{equation}
    \lambda_4' = \mu^{- 2 \epsilon}\left[ \lambda_4^r + \frac{1}{2 \epsilon} a_1(\lambda_r^4) + ... \right].
\end{equation}

After adding \cref{scalar field static term} to the Lagrangian of the free scalar, the temperature independent free energy density becomes
\begin{equation}
    \Ef_0 \sim - \mu^{-2 \epsilon}  \frac{1}{4} \frac{m^4}{(4 \pi)^2}  
    \left[
        \frac{1}{\epsilon} + \frac{3}{2} + \ln{\frac{\tilde \mu^2}{m^2}}
        + (4 (4 \pi)^2) \left(\lambda_4^r + \frac{1}{2\epsilon} a_1(\lambda_4^r)\right)
    \right],
    \quad \epsilon \rightarrow 0.
\end{equation}
Thus, if we choose $a_1 = -8 (4\pi)^2 + \mathcal{O}((\lambda_4^r))$, and define $\lambda_4'^r = 4(4\pi)^2\lambda_4^r$, we are able to cancel the divergence, and may take the limit $\epsilon \rightarrow 0$ safely.
The free energy thus becomes
\begin{equation}
    \Ef = -\frac{1}{4} \frac{m^4}{(4 \pi)^2} 
    \left(
        \frac{3}{2} + \lambda_4'^r + \ln \frac{\tilde \mu^2}{m^2}
    \right)
    +
    \frac{T^4}{2\pi^2} \int \dd x \, x^2 \ln(1 - \exp{-\sqrt{x^2 + \beta^2 m^2}}).
\end{equation}
Notice that the choices we have made up until now, for example by defining $\lambda_4 = m^4 \lambda_4'$, and using the same renormalization scale $\mu$ has no impact on this result, as any other definition would just force us to define $\lambda_4^r$ and $a_4$ differently.

