\subsection{Fermions}
The phase factor $e^{i\theta}$ that was introduced in (REF IMAG FORM.) can be decided by studying the properties of the thermal greens function, which are defined as
\begin{equation*}
    G(X_1, X_2) = G(\vec x, \vec y, \tau_1, \tau_2) 
    = \ex{e^{-\beta \hat H} \T{ \varphi(X_1) \varphi(X_2) } }.
\end{equation*}
$\T{...}$ is the temperature ordering operator, analogous to the time ordering operator from scattering theory.
It is defined as
\begin{equation*}
    \T{\varphi(\tau_1)\varphi(\tau_2)}
    = \theta(\tau_1 - \tau_2) \varphi(\tau_1)\varphi(\tau_2)
    + \nu \theta(\tau_2 - \tau_1) \varphi(\tau_2)\varphi(\tau_1),
\end{equation*}
where $\nu = \pm 1$ for respectively bosons and fermions, and $\theta(\tau)$ is the Heaviside step function.
In the same way that $i \hat H$ generates the time-dependence of a quantum field operator through $\hat\varphi(x) = \hat\varphi(t, \vec x) = e^{it\hat H} \hat \varphi(\vec x) e^{-it\hat H} $, the imaginary-time formalism implies the relation
\begin{equation}
    \hat\varphi(X) = \hat\varphi(\tau, \vec x) 
    = e^{\tau\hat H} \hat \varphi(\vec x) e^{-\tau \hat H}.
\end{equation}
Using $\one = e^{\tau \hat H} e^{-\tau \hat H}$ and the cyclic property of the trace, we show that, assuming $\beta>\tau>0$,
\begin{align*}
    G(\vec x, \vec y, \tau, 0)
    & = \ex{e^{-\beta \hat H} \T{\varphi(\tau, \vec x) \varphi(0, \vec y)}} \\
    & = \frac{1}{Z} \Tr{
        e^{-\beta \hat H} \varphi(\tau, \vec x) \varphi(0, \vec y)
    } \\
    & = \frac{1}{Z} \Tr{
        \varphi(0, \vec y) e^{-\beta \hat H} \varphi(\tau, \vec x)
    } \\
    & = \frac{1}{Z} \Tr{
        e^{-\beta \hat H} e^{\beta \hat H} \varphi(0, \vec y) 
        e^{-\beta \hat H} \varphi(\tau, \vec x)
    } \\
    & = \frac{1}{Z} \Tr{
        e^{-\beta \hat H} \varphi(\vec y, \beta) \varphi(\tau, \vec x)
    } \\
    & = \nu \ex{
        e^{-\beta \hat H} \T{ \varphi(\tau, \vec x) \varphi(\vec y, \beta) }
    }.
\end{align*}
This implies that $\varphi(0, x) = \nu \varphi(\beta, \varphi)$, which show that bosons are periodic in time, as stated earlier, while fermions are anti-periodic.

The Lagrangian density of a free fermion is
\begin{equation}
    \Ell = \bar \psi \left( i \slashed{\partial} - m \right) \psi.
\end{equation}
This Lagrangian is invariant under the transformation $\psi \rightarrow e^{-i \alpha} \psi$, which by Nöther's theorem results in a conserved current
\begin{equation}
    j^\mu = \pdv{\Ell}{(\partial_\mu \psi)} \delta \psi=  \bar \psi \gamma^\mu \psi.
\end{equation}
The corresponding charge is 
\begin{equation}
    Q = \int_V \dd^3 x\, j^0 = \int_V \dd^3 x \, \psi^\dagger \psi.
\end{equation}
We can now use our earlier result for the thermal partition function, \autoref{Thermal partition function}, only with the substitution $\He \rightarrow \He - \mu \psi^\dagger \psi$, and integrate over anti-periodic $\psi$'s:
\begin{equation*}
    Z = \Tr{e^{-\beta(\hat H - \mu \hat Q)}}
    = \prod_{a b}\int \D \psi_a\D \pi_b \exp{
        \int_{\Omega} \dd X \, 
        \left(
            i\dot \psi \pi - \He(\psi, \pi) + \mu \psi^\dagger \psi
        \right)
    },
\end{equation*}
where $a, b$ are the spinor indices.
The canonical momentum corresponding to $\psi$ is
\begin{equation}
    \pi = \pdv{\Ell}{(\partial_0 \psi)} = i \psi^\dagger,
\end{equation}
and the Hamiltonian density is 
\begin{equation}
    \He = \pi \dot\psi - \Ell
    = \bar \psi ( - \gamma^i\partial_i + m) \psi
\end{equation}
which gives
\begin{equation}
    i \dot\psi \pi - \He(\psi, \pi) + \mu \psi^\dagger \psi
    = \bar\psi[-\gamma^0 (\partial_\tau - \mu) + i\gamma^i \partial_i - m] \psi,
\end{equation}
By using the Grassman-version of the Gaussian integral formula, the partition function can be written
\begin{align*}
    Z = \prod_{a b}\int \D \psi_a\D i\psi^\dagger_b 
    \exp{
        \int_\omega \dd X \, \bar \psi
        \left[
            -\gamma^0 (\partial_\tau -\mu)+ i \gamma^i \partial_i - m
        \right]
        \psi
    }
    = \prod_{a b}\int \D \psi_a\D i\psi^\dagger_b  e^{\langle i\psi^\dagger, D \psi\rangle} 
    = C \det(D),
\end{align*}
where 
\begin{equation}
    D = -i \gamma^0  [
        (\mu-\partial_\tau) + i \gamma^i \partial_i - m
        ].
\end{equation}
This results in
\begin{equation}
    -\beta F = \ln(Z) = \Tr[\ln(D)] + C.
\end{equation}
Assuming $\psi_a(X)$ is a orthonormal set, we can calculate the trace
\begin{align*}
    \Tr[\ln(D)] 
    & = \int_\Omega \dd X \, (i \psi_a(X)^\dagger) 
    \ln\{ -i \gamma^0 [\gamma^0 (\mu-\partial_\tau) + i \gamma^i\partial_i -  m]\}_{ab} 
    \psi_b(X) \\
    & =
    V \int_{\tilde \Omega} \dd K \,
    \ln\{ -i \beta\gamma^0  [\gamma^0 (\mu - i\omega_n) - \gamma^i p_i - m]\}_{aa} \\
    & =
    V \int_{\tilde \Omega} \dd K \,
    \ln\{ i \beta \gamma^0 [i\gamma^0 (\omega_n + i\mu) + i(-i\gamma^i) p_i + m]\}_{aa}\\
    & =
    V \int_{\tilde \Omega} \dd K \,
    \ln\{ i \beta \gamma^0 [i \tilde \gamma_a p_{n;a} + m]\}_{aa}.
\end{align*}
In the last line, we have introduced the notation $p_{n;a} = (\omega_n + i \mu, p_i)$ and use the Euclidean gamma matrices, as defined in \autoref{Conventions and notation}.
(JEG ER IKKE SIKKER PÅ DETTE) 
Furthermore, we use the fact that
\begin{equation*}
    \ln[i\tilde\gamma_a p_a - m]
    = \ln[i\tilde\gamma_a p_a - m] + \ln(\tilde\gamma_5 \tilde\gamma_5) 
    = \ln[\tilde\gamma_5(i\tilde\gamma_a p_a - m)\tilde\gamma_5]
    = \ln[-i\tilde\gamma_a p_a - m],
\end{equation*}
which allows us to calculate
\begin{align*}
    &\ln\{ i \beta \gamma^0[i\tilde\gamma^0 (\omega_n + i\mu) + i(-i\tilde\gamma^i) p_i + m]\}\\
    & = \ln[(i \tilde\gamma_a p_a + m)] + \frac{1}{4} \ln[(i \beta \gamma^0)^4]\\
    & = \ln[\beta] + \frac{1}{2} 
    \left\{
        \ln[i \tilde\gamma^a p^a + m] + \ln[- i \tilde\gamma^a p^a + m]
    \right\} \\
    & = \frac{1}{2} 
    \left\{
        \ln[\beta^2 (i \tilde\gamma^a p^a + m)( - i \tilde\gamma^a p^a + m)]
    \right\} \\
    & = \frac{1}{2} \one \ln\{ \beta^2[(\omega_n + i\mu)^2 + \omega^2]\} 
\end{align*}
which gives
\begin{equation}
    \Ef
    = -\frac{2}{\beta} \int_{\tilde \Omega} \dd X \,  \ln\{ \beta^2[(\omega_n + i\mu)^2 + \omega^2]\} .
\end{equation}
Using the fermionic version of the thermal sum, this gives the answer
\begin{equation}
    \Ef = - \frac{2}{\beta} \int\frac{\dd^3 p}{(2\pi)^3} \, 
    \left[
        \beta \omega 
        + \ln\left(1 + e^{-\beta(\omega+\mu)}\right)
        + \ln\left(1 + e^{-\beta(\omega-\mu)}\right)
    \right].
\end{equation}