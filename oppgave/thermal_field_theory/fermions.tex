\subsection{Fermions}
The phase factor $e^{i\theta}$ that was introduced in \autoref{imaginary-time formalism} can be determined by studying the properties of the thermal Greens function.
The thermal Greens function may be written 
\begin{equation*}
    D(X_1, X_2) = D(\vec x, \vec y, \tau_1, \tau_2) 
    = \ex{e^{-\beta \hat H} \T{ \hat \varphi(X_1) \hat \varphi(X_2) } }.
\end{equation*}
$\T{...}$ is time-ordering operator,
% the temperature ordering operator, analogous to the time ordering operator from scattering theory.
and is defined as
\begin{equation*}
    \T{\varphi(\tau_1)\varphi(\tau_2)}
    = \theta(\tau_1 - \tau_2) \varphi(\tau_1)\varphi(\tau_2)
    + \nu \theta(\tau_2 - \tau_1) \varphi(\tau_2)\varphi(\tau_1),
\end{equation*}
where $\nu = \pm 1$ for bosons and fermions respectively, and $\theta(\tau)$ is the Heaviside step function.
In the same way that $i \hat H$ generates the time translation of a quantum field operator through $\hat\varphi(x) = \hat\varphi(t, \vec x) = e^{it\hat H} \hat \varphi(0, \vec x) e^{-it\hat H} $, the imaginary-time formalism implies the relation
\begin{equation}
    \hat\varphi(X) = \hat\varphi(\tau, \vec x) 
    = e^{\tau\hat H} \hat \varphi(0, \vec x) e^{-\tau \hat H}.
\end{equation}
Using $\one = e^{\tau \hat H} e^{-\tau \hat H}$ and the cyclic property of the trace, we show that, assuming $\beta>\tau>0$,
\begin{align*}
    G(\vec x, \vec y, \tau, 0)
    & = \ex{e^{-\beta \hat H} \T{\varphi(\tau, \vec x) \varphi(0, \vec y)}} \\
    & = \frac{1}{Z} \Tr{
        e^{-\beta \hat H} \varphi(\tau, \vec x) \varphi(0, \vec y)
    } \\
    & = \frac{1}{Z} \Tr{
        \varphi(0, \vec y) e^{-\beta \hat H} \varphi(\tau, \vec x)
    } \\
    & = \frac{1}{Z} \Tr{
        e^{-\beta \hat H} e^{\beta \hat H} \varphi(0, \vec y) 
        e^{-\beta \hat H} \varphi(\tau, \vec x)
    } \\
    & = \frac{1}{Z} \Tr{
        e^{-\beta \hat H} \varphi(\vec y, \beta) \varphi(\tau, \vec x)
    } \\
    & = \nu \ex{
        e^{-\beta \hat H} \T{ \varphi(\tau, \vec x) \varphi( \beta, \vec y) }
    }.
\end{align*}
This implies that $\varphi(0, x) = \nu \varphi(\beta, \varphi)$, which shows that bosons are periodic in time, as stated earlier, while fermions are anti-periodic.

The Lagrangian density of a free fermion is
\begin{equation}
    \Ell = \bar \psi \left( i \slashed{\partial} - m \right) \psi.
\end{equation}
This Lagrangian is invariant under the transformation $\psi \rightarrow e^{-i \alpha} \psi$, which by Nöther's theorem results in a conserved current
\begin{equation}
    j^\mu = \pdv{\Ell}{(\partial_\mu \psi)} \delta \psi=  \bar \psi \gamma^\mu \psi.
\end{equation}
The corresponding conserved charge is 
\begin{equation}
    Q = \int_V \dd^3 x\, j^0 = \int_V \dd^3 x \, \bar \psi \gamma^0 \psi.
\end{equation}
We can now use our earlier result for the thermal partition function, \autoref{Thermal partition function}, only with the substitution $\He \rightarrow \He - \mu \bar \psi \gamma^0 \psi$, and integrate over anti-periodic $\psi$'s:
\begin{equation*}
    Z = \Tr{e^{-\beta(\hat H - \mu \hat Q)}}
    = \prod_{a b}\int \D \psi_a\D \pi_b \exp{
        \int_{\Omega} \dd X \, 
        \left(
            i\dot \psi \pi - \He(\psi, \pi) + \mu \bar \psi \gamma^0 \psi
        \right)
    },
\end{equation*}
where $a, b$ are the spinor indices.
The canonical momentum corresponding to $\psi$ is
\begin{equation}
    \pi = \pdv{\Ell}{(\partial_0 \psi)} = i \bar \psi \gamma^0,
\end{equation}
and the Hamiltonian density is 
\begin{equation}
    \He = \pi \dot\psi - \Ell
    = \bar \psi (-i\gamma^i\partial_i + m) \psi
\end{equation}
which gives
\begin{equation}
    \Ell_E = 
    - i \pi \dot\psi + \He(\psi, \pi) - \mu \bar \psi \gamma^0 \psi
    = \bar\psi[\gamma^0 (\partial_\tau - \mu) - i\gamma^i \partial_i + m] \psi,
\end{equation}
By using the Grassman-version of the Gaussian integral formula, the partition function can be written
\begin{align*}
    Z & = \prod_{a b}\int \D \psi_a\D \bar \psi_b 
    \exp{
        - \int_\Omega \dd X \, \bar \psi
        \left[
            \tilde\gamma_0(\partial_\tau -\mu) -  i \gamma^i \partial_i + m
        \right]
        \psi
    }\\
    & = C \prod_{a b}\int \D \tilde \psi_a\D \tilde {\bar \psi}_b 
    \exp{
        - \int_{\tilde \Omega} \dd K \, \tilde {\bar \psi}
        \left[
            i \tilde\gamma_0(\omega_n + i\mu) + i \gamma_i p_i + m
        \right]
        \tilde \psi
    } \\
    & = C \prod_{a b}\int \D \tilde \psi_a\D \tilde {\bar \psi}_b 
    e^{- \langle \tilde {\bar \psi}, D_0^{-1} \psi\rangle} 
    = \det(D_0^{-1}).
\end{align*}
In the second line, we have inserted the Fourier expansion of the field, as defined in \autoref{Conventions and notation}, and changed variable of integration, as we did for the scalar field.
The linear operator in this case is 
\begin{equation}
    D_0^{-1} = i \gamma^0 (-i\partial_\tau + i\mu) - (- i \gamma^i) \partial_i + m
    = 
    \beta [i \tilde \gamma_a p_a + m ].
\end{equation}
This equality must be understood as an equality between linear operators, which are represented in different bases.
We introduced the notation $p_{n;a} = (\omega_n + i \mu, p_i)$ and use the Euclidean gamma matrices, as defined in \autoref{Conventions and notation}.
We use the fact that 
\begin{equation*}
    \det(i\tilde\gamma_a p_a + m)
    = \det(\gamma^5 \gamma^5)
    \det(i\tilde\gamma_a p_a + m)
    = \det[\gamma^5 (i\tilde\gamma_a p_a + m) \gamma^5]
    = \det(-i\tilde\gamma_a p_a + m),
\end{equation*}
Let $\tilde D = -i\tilde\gamma_a p_a + m$, which means we can write
\begin{equation}
    Z = \sqrt{\det(D)\det(\tilde D)} = \sqrt{\det(D\tilde D)} = \det[\one(p_a p_a + m^2)]^{1/2},
\end{equation}
where we have used the anti-commutation rule for the Euclidean gamma-matrices, $\acom{\gamma_a}{\gamma_b} = 2 \delta_{ab}$.
It is important to keep in mind that the determinant here refers to linear operators on the space of spinor functions.
\begin{align}
    \nonumber
    \ln(Z) & = \ln\left\{\det[\one(p_a p_a + m^2)]^{1/2}\right\}
    = \frac{1}{2} \Tr{\ln[\one(p_a p_a + m^2)]} \\
    & = \frac{1}{2} \int_{\tilde \Omega} \dd K \ln[\one \beta^2 (p_a p_a + m^2)]_{aa}
\end{align}
As the matrix within the logarithm is diagonal, the matrix-part of the trace is trivial, and the free energy may be written
\begin{equation}
    \beta \Ef
    = - 2 \int_{\tilde \Omega} \dd X \,  \ln\{ \beta^2[(\omega_n + i\mu)^2 + \omega_k^2]\} .
\end{equation}
Using the fermionic version of the thermal sum from \autoref{section:thermal sum} gives the answer
\begin{equation}
    \Ef = 2 \int\frac{\dd^3 p}{(2\pi)^3} \, 
    \left[
        \beta \omega_k
        + \frac{1}{\beta} \ln\left(1 + e^{-\beta(\omega_k-\mu)}\right)
        + \frac{1}{\beta} \ln\left(1 + e^{-\beta(\omega_k+\mu)}\right)
    \right].
\end{equation}
We see again that the temperature-independent part of the integral diverges, and must be regulated.
There are two temperature-dependent terms, one from the particle and one from the anti-particle.
