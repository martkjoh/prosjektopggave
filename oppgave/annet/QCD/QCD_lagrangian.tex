\documentclass{article}

\usepackage[left=2.5cm, right=2.5cm, top=2cm, bottom=2cm]{geometry}

\usepackage{amsmath}
\usepackage{amsfonts}
\usepackage{amssymb}
\usepackage{physics}
\usepackage{slashed}
\usepackage{hyperref}
\usepackage[title]{appendix}
\usepackage[intlimits]{mathtools}
\usepackage[export]{adjustbox}
\usepackage{esvect}
\usepackage[capitalize]{cleveref}


\usepackage{pgfplots}
\pgfplotsset{compat=1.16}

\usepackage{tikz}
\usepackage[compat=1.1.0]{tikz-feynman}

\usepackage{caption}
\usepackage{subcaption}

\usepackage[ISO]{diffcoeff} % Get straight d's when differantiating

\usepackage{tocloft}    % Nice table of contents

\usepackage[section]{placeins} % Get floats right
\usepackage[super]{nth}


\setlength\cftparskip{1pt}
\setlength\cftbeforechapskip{0pt}

\setlength{\parindent}{0em}
\setlength{\parskip}{0.8em}

\def\equationautorefname~#1\null{Eq.~(#1)\null}

% Simple shortcuts
\newcommand{\Ell}{\mathcal{L}}      % Lagrangian L
\newcommand{\He}{\mathcal{H}}       % Hamiltonian H
\newcommand{\Ve}{\mathcal{V}}       % Potential V
\newcommand{\Em}{\mathcal{M}}       % Manifold M
\newcommand{\Ef}{\mathcal{F}}       % Fancy f
\newcommand{\R}{\mathbb{R}}         % Real numbers
\newcommand{\chpt}{$\chi$PT }       % Chiral pertubation theory
\newcommand{\SU}{\mathrm{SU}}       % SU(n)
\newcommand{\eps}{\varepsilon}      % nice epsilon
% \newcommand{\one}{\mathbb{1}}       % Identity
\newcommand{\hc}{\mathrm{h.c.}}     % Hermitian conjugate
\newcommand{\ex}[1]{\expectationvalue{#1}}
\newcommand{\D}{\mathcal{D}}

\newcommand{\one}{\text{\usefont{U}{bbold}{m}{n}1}}
\MakeRobust{\one}

% Big-O notation 
\newcommand{\Oh}[2][2]{\mathcal{O}\left(#2^{#1}\right)}

% (anti) commutator
\newcommand{\com}[2]{\left[#1, #2 \right]}
\newcommand{\acom}[2]{\left\{#1, #2 \right\}}

% Lie algebra
\newcommand{\liea}[2]{\mathfrak{#1}\left(#2\right)}
\newcommand{\lieg}[2]{\mathrm{#1}\left(#2\right)}

% Curly brackets
\newcommand{\curly}[1]{\left\{ #1 \right\}}

% Fourier Transform
\newcommand{\F}[1]{\mathcal{F}\curly{#1}}
\newcommand{\FInv}[1]{\mathcal{F}^{-1}\curly{#1}}

% operator in braket
\newcommand{\inner}[3]{\left\langle #1 {\left| #2 \right|} #3 \right\rangle}

\newcommand{\T}[1]{\textrm{T} \left\{ #1 \right\}}


\title{QCD Lagrangian}
\author{Martin Johnsrud}
\vspace{-8ex}
\date{}


\begin{document}
    \maketitle
    The QCD Lagrangian is, in compact notation
    \begin{equation}
        \Ell = \sum_{f} \bar q_{f} \left(i \slashed{D} - m_f\right) q_{f} - \frac{1}{4} G_{\mu \nu}^{a} G^{\mu \nu a}
    \end{equation}
    Including all indices gives
    \begin{align*}
        \mathrm{QCD}_{f\,c\,j}^{f'c'j}
        & = 
        \one_{ff'} \gamma^\mu_{jj'} 
        \left[\one_{cc'}\partial_\mu - i g \lambda_{cc'}^a A_{a \mu}\right]
        - m_{ff'} \one_{cc'} \one_{jj'} \\
        \Ell_1
        & = \bar q_{f\,c\,j} \, \mathrm{QCD}_{f\,c\,j}^{f'c'j} \,\, q_{f'c'j'}
    \end{align*}
    Here, $f \in \{ u, d, s, c, t, b \}$ are flavors, $c \in \{r, g, b\}$ are colors, $j \in \{0, 1, 2, 3\}$ are spinor indices, $\mu \in \{0, 1, 2, 3\}$ are space-time indices, $a \in \{1, ... 8\}$ are indices for the $\liea{su}{3}$ color algebra.

    The gluon field strength tensor is given by
    \begin{equation*}
        G^a_{\mu \nu} = \partial_\mu A_\nu^a - \partial_\nu A_\mu^a + g f^{abc}A_\mu^b A_\nu^c,
    \end{equation*}
    where $A_\mu^a$ is the gluon field potential, $g$ the field coupling strength and $f^{abc}$ is the structure constants of the $\lieg{SU}{3}$ gauge group of the gluon potential. The covariant derivative is given by
    \begin{equation*}
        D_\mu = \partial_\mu - i g \lambda^a A^a_\mu,
    \end{equation*}
    which ensures the Lagrangians is invariant under the gauge transformation
    \begin{align}
        q(x) \rightarrow U(x) q(x), \quad U(x) = \exp(i \lambda^a \chi^a(x)) \in \lieg{SU}{3}.
    \end{align}

    \subsection{Global symmetries}

    The light quarks are quarks with the flavors the subset $l \in \{u, d, s\}$. Approximating $m_{ll'} = m_u \delta_{ll'}$ gives the Lagrangian a global $\lieg{U}{3}$ symmetry. In the chiral limit $m_u \rightarrow 0$, (HVORFOR KAN EN GJØRE DETTE) the Lagrangian splits into two parts, left- and right-handed, which both have the global flavor symmetry. Thus, the classical global symmetry becomes $\lieg{U}{3}_L \times \lieg{U}{3}_R$

    \printbibliography
\end{document}
