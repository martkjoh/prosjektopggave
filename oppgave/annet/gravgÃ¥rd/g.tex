\documentclass{article}

\usepackage[left=2.5cm, right=2.5cm, top=2cm, bottom=2cm]{geometry}

% math mode additions
\usepackage{amsmath}
\usepackage{amsfonts}
\usepackage{amssymb}
\usepackage{physics}
\usepackage{slashed}
\usepackage{mathtools}
% nice 3-vectors
\usepackage{esvect}
% nice differentation
% Get straight d's when differantiating
\usepackage[ISO]{diffcoeff}


% appendix
\usepackage[title]{appendix}
% \usepackage[intlimits]{mathtools}

\usepackage[export]{adjustbox}

% in-document references
\usepackage{hyperref}
\usepackage[capitalize]{cleveref}

% create plots
\usepackage{tikz}
\usepackage[compat=1.1.0]{tikz-feynman}
\usepackage{pgfplots}
\pgfplotsset{compat=1.16}

% Nice table of contents
\usepackage{tocloft}

% Get floats right
\usepackage[section]{placeins}

% adjust fig captions
\usepackage{caption}
\usepackage{subcaption}
\captionsetup{width=.9\textwidth}

\usepackage{titlepic}
% \usepackage[textsize=footnotesize]{todonotes}
\usepackage[disable]{todonotes}

\usepackage[style=numeric-comp,sorting=none,sortcites=true,doi=true,url=false,giveninits=true,hyperref]{biblatex}

\setlength{\marginparwidth}{2.2cm}

\setlength\cftparskip{1pt}
\setlength\cftbeforechapskip{0pt}

\setlength{\parindent}{0em}
\setlength{\parskip}{0.8em}

\def\equationautorefname~#1\null{Eq.~(#1)\null}


% Simple shortcuts
\newcommand{\Ell}{\mathcal{L}}      % Lagrangian L
\newcommand{\He}{\mathcal{H}}       % Hamiltonian H
\newcommand{\Ve}{\mathcal{V}}       % Potential V
\newcommand{\Em}{\mathcal{M}}       % Manifold M
\newcommand{\Ef}{\mathcal{F}}       % Fancy f
\newcommand{\R}{\mathbb{R}}         % Real numbers
\newcommand{\chpt}{$\chi$PT }       % Chiral pertubation theory
\newcommand{\SU}{\mathrm{SU}}       % SU(n)
\newcommand{\eps}{\varepsilon}      % nice epsilon
% \newcommand{\one}{\mathbb{1}}       % Identity
\newcommand{\hc}{\mathrm{h.c.}}     % Hermitian conjugate
\newcommand{\ex}[1]{\expectationvalue{#1}}
\newcommand{\D}{\mathcal{D}}

\newcommand{\one}{\text{\usefont{U}{bbold}{m}{n}1}}
\MakeRobust{\one}

% Big-O notation 
\newcommand{\Oh}[2][2]{\mathcal{O}\left(#2^{#1}\right)}

% (anti) commutator
\newcommand{\com}[2]{\left[#1, #2 \right]}
\newcommand{\acom}[2]{\left\{#1, #2 \right\}}

% Lie algebra
\newcommand{\liea}[2]{\mathfrak{#1}\left(#2\right)}
\newcommand{\lieg}[2]{\mathrm{#1}\left(#2\right)}

% Curly brackets
\newcommand{\C}[1]{\left\{ #1 \right\}}

% Fourier Transform
\newcommand{\F}[1]{\mathcal{F}\C{#1}}
\newcommand{\FInv}[1]{\mathcal{F}^{-1}\C{#1}}

% operator in braket
\newcommand{\inner}[3]{\left\langle #1 {\left| #2 \right|} #3 \right\rangle}

\newcommand{\T}[1]{\textrm{T}_\tau \left\{ #1 \right\}}

\title{Gravgård}
% \author{Martin Johnsrud}
\vspace{-8ex}
\date{}

\begin{document}
    \maketitle
    Gravgården er hvor tekster går for å dø. Ting som er fjernet fra main, men som jeg ikke helt klarer å slette enda
    \section{path integral}
    By Wick's theorem, an $n$-point correlated is given by the sum of all Feynman diagrams with $n$ external vertices.
    The factor $Z[0]^{-1}$ divides out all \emph{vacuum bubbles}, that is diagrams without external vertices.
    We can show this by considering 

    Here we defined the \emph{generating functional for connected diagrams}, $W[J]$.
    The reason for the name will become apparent later. (HUSK Å REFFERE TILBAKE)
    The expectation value of some function of the field-configuration, $A = A[\varphi]$, in the precesense of the source $J$ is
    \begin{equation}
        \ex{A}_J = \frac{1}{Z[J]} A\left( -i  \fdv{J}\right) Z[J].
    \end{equation}
    (DEFINE FUNCTIONAL DERIVATIVE)
    The expectation value of the field defines a functional,
    \begin{equation}
        \label{calssical field functional}
        \varphi[J](x) = \ex{\varphi(x)}_J = 
        \frac{1}{Z[J]} \left( -i  \fdv{J}\right) Z[J]
        = \fdv{J(x)} W[J],
    \end{equation}
    and is sometimes called the \emph{classical field}.
    The notation $\Ef[f](x)$ means that $\Ef$ is a functional which takes in a function $f$, and returns the new function $(\Ef[f])(x)$.
    One example is the Lagrangian density, which takes in a field, and returns a function which has a value for each point in space-time.
    We can reverse this relationship, by defining the functional $J[\varphi](x)$ as \emph{the current which causes the classical field $\varphi$}.
    That is, if $\varphi[J_0](x) = \varphi_0(x)$ for some source $J_0$, then $J[\varphi_0] = J_0$

\section{free scalar}
    Comparing with the definitions of the thermal propagator in \autoref{Conventions and notation}, we can write the free energy compactly as
    \begin{equation}
        \label{free energy from propagator}
        \beta F = \frac{1}{2} \Tr{\ln[D_0^{-1}(K, K'))]} 
        = \frac{1}{2} \Tr{\ln[\beta^2 D_0^{-1}(K)]}.
    \end{equation}

\section{interacting scalar}
    Notice that the constant factor from the Jacobian due to the change of variable $\varphi \rightarrow \tilde \varphi$ does not affect the expectation value, as the same factor is in both the numerator and denominator.
    If the quantity $A$ is a function of the momentum-space fields, $A = A[\tilde \varphi(K)]$, then this expectation value takes the form
    \begin{equation}
        \ex{A}_0 = 
        \frac{
            \int_{\tilde S} \D \tilde \varphi(K) \, f[\tilde\varphi(K)] \, 
            \exp{- \frac{1}{2} \langle \tilde \varphi^*, D \tilde \varphi \rangle}
            }
        {
            \int_{\tilde S} \D \tilde \varphi(K) \,
            \exp{
                - \frac{1}{2} \langle \tilde \varphi^*, D \tilde \varphi \rangle
                }
        }.
    \end{equation}
    where, as before, 
    \begin{equation}
        \langle \tilde \varphi^*, D \tilde \varphi \rangle
        = 
        \int_\Omega 
                \tilde \dd K\, [\beta^2(\omega_n^2 + \omega_n^2)] |\tilde \varphi(K)|^2
    \end{equation}
    The exponential form of $Z[J]$ leads straight forwardly to Wick's theorem, which states that an expectation value of $2n$ fields is a sum of \emph{all possible, distinct} combination of $n$ propagators.
    To write this in a formal way, we define the functions $a$ and $b$, which define a way to pair up $2m$ elements.
    The domain of the functions are the integers between $1$ and $m$, the image a subset of the integers between $1$ and $2m$ of size $m$.
    A valid pairing is a set $\{(a(1), b(1)), \dots (a(m), b(m))\}$, where all elements $a(i)$ and $b(j)$ are different, such all integers up to and including $2m$ are featured.
    A pair is not directed, so $(a(i), b(i))$ is the same pair as $(b(i), a(i))$.
    Wick theorem states that,
    \begin{equation}
        \ex{\prod_{i=1}^{2m} \varphi(X_i)  }_0
        = \sum_{\{(a, b)\}} \ex{\varphi(X_{a(i)}) \varphi(X_{b(i)})},
    \end{equation}
    where the sum is over all tuples $(a, b)$ that define a valid and unique pairing.

    \begin{align}
        \feynmandiagram [inline=(a.base), small, horizontal=i1 to f2]
        {
        {i1} -- [fermion, edge label'=$K_1$] a[dot] 
        -- [anti fermion, edge label'=$K_3$] {f1},
        {i2} -- [fermion, edge label'=$K_2$] a -- [anti fermion, edge label'=$K_4$] {f2},
        };
        \feynmandiagram[horizontal= i to f]{
            i[particle=$K$] -- [fermion] f,
        };
    \end{align}
    The expression is the integrated over all \emph{internal} momenta.
    The factor $1/4!$ is removed as a general Feynman diagram represent all diagrams with the same form, but different pairing of the momenta.
    Some diagrams are more symmetric, such that an exchange of momenta still gives \emph{the same pairing}. 
    
\section{effective action}

In free theory, we may write
\begin{equation}
    W[J] = \frac{1}{2} \int \dd^4 x \dd^4y J(x) D_0(x - y) J(y),
\end{equation}
where $D_0$ is the free propagator.
We may reverse the relation \autoref{calssical field functional} to write the source in terms of the field,
\begin{equation}
    J = D_0^{-1} \varphi(x)
\end{equation}
This is the field equation for the free field with a source.
For the scalar Klein-Gordon field, $D_0^{-1} = \partial^2 + m^2$
Inserting these two relation into the definition of the effective action, and assuming we can do partial integration with $D_0^{-1}$, we get
\begin{equation}
    \Gamma[\varphi] = W[J] - \int \dd^4x J(x)\varphi(x)
    = 
    \int \dd x( 
        \frac{1}{2}\int \dd y (D_0^{-1} \varphi ) D_0 (D_0^{-1} \varphi ) 
        - (D_0^{-1} \varphi ) \varphi
        )
    = - \frac{1}{2} \int \dd^4 x \varphi(x) D_0^{-1} \varphi(x)
\end{equation}
This is the classical action.
Thus, the effective action $\Gamma$ and the classical action $S$ are the same to first order in perturbation theory.


Let $\varphi^*$ solve the quantum mechanical version of the equation of motion, i.e.
\begin{equation}
    \fdv{\Gamma[\varphi^*]}{\varphi} = 0.
\end{equation}
We can Taylor-expand the classical action around this point, by setting $\varphi(x) = \varphi^*(x) + \eta(x)$ for some function $\eta$.
The generating functional becomes
\begin{align}
    Z[J] 
    = \int \D (\varphi^* + \eta) \, 
    \exp{i S[\varphi^* + \eta] + i \int \dd^4 x J (\varphi^* + \eta) }
\end{align}
The functional version of a Taylor expansion is
\begin{equation}
    S[\varphi^* + \eta] = 
    S[\varphi^*]
    + \int \dd x \fdv{S[\varphi^*]}{\varphi(x)} \eta(x)
    + \frac{1}{2} \int \dd x \dd y\,  \frac{\delta^2 S[\varphi^*]}{\delta\varphi(x)\delta\varphi(y)} \eta(x) \eta(y)
    + \dots
\end{equation}
Inserting this into $Z[J]$, with $S_I$ to denote the derivatives of higher order than $2$, we get
\begin{align*}
    &Z[J] = \\ 
    &\int \D \eta  
    \exp{
        i \int \dd^4 x \left(  \Ell[\varphi^*] + J \varphi^*  \right)
        + i \int \dd x \left(  \fdv{S[\varphi^*]}{\varphi(x)} + J(x) \right) \eta(x)
        + i \frac{1}{2} \int \dd x \dd y\,  
        \frac{\delta^2 S[\varphi^*]}{\delta\varphi(x)\delta\varphi(y)} \eta(x) \eta(y) 
        + i S_I[\eta]
        }
\end{align*}
In the first term we used the definition of the classical action. This term is constant with respect to $\eta$, and may therefore be taken outside the path integral.
The next term is the classical equation of motion with a source, 
\begin{equation}
    \fdv{S[\varphi]}{\varphi(x)} = - J(x),
\end{equation} 
evaluated at $\varphi^*$.
\begin{equation}
    \fdv{S[\varphi^*]}{\varphi(x)} + J(x)
    = \fdv{\Gamma[\varphi^*]}{\varphi(x)} + J(x)
    + \left(\fdv{S[\varphi^*]}{\varphi(x)} - \fdv{\Gamma[\varphi^*]}{\varphi(x)} \right)
    = \left(\fdv{S[\varphi^*]}{\varphi(x)} - \fdv{\Gamma[\varphi^*]}{\varphi(x)} \right)
    := \delta J
\end{equation}
The second to last term is a Gaussian integral, and may be evaluated as described in \autoref{section:gaussian integrals},
\begin{equation}
    \int \D \eta \, \exp(
        i \frac{1}{2} \int \dd x \dd y\,  
        \frac{\delta^2 S[\varphi^*]}{\varphi(x)\varphi(y)} \eta(x) \eta(y)
        )
        = C \det\left( \frac{\delta^2 S[\varphi^*]}{\delta \varphi^2} \right)^{-1/2}
\end{equation}
This leaves us with 
\begin{align}
    \label{generating functional}
    W[J] 
    & = -i \ln(Z) \\
    & = 
    \int\dd^4 x \, \left(\Ell[\varphi^*] + J \varphi^*\right)
    - \frac{1}{2} \Tr{\ln\left( - \frac{\delta^2 S[\varphi^*]}{\delta \varphi^2} \right)}
    + \int \D \eta \, \exp{i \int \dd^4 x \delta J(x) \eta  }
    + \int \D \eta \, e^{iS_I}
\end{align}
$\delta J$ is ultimately dependent on our choice of $J$ to define $\varphi$.
It contributes to the expectation value of $\eta$, through tadpole diagrams
\begin{align}
    \ex{\eta}_{j=0} = 
    \feynmandiagram [horizontal=a to b]{
    a --[inline=(b.base), fermion] b[blob]
    }; 
\end{align}
This can be removed by using the renormalization condition
\begin{equation}
    \feynmandiagram [horizontal=a to b]{
    a --[inline=(b.base), fermion] b[blob]
    };
    = 0.
\end{equation}



\end{document}