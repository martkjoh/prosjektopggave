\documentclass{article}

\usepackage[left=2.5cm, right=2.5cm, top=2cm, bottom=2cm]{geometry}

% math mode additions
\usepackage{amsmath}
\usepackage{amsfonts}
\usepackage{amssymb}
\usepackage{physics}
\usepackage{slashed}
\usepackage{mathtools}
% nice 3-vectors
\usepackage{esvect}
% nice differentation
% Get straight d's when differantiating
\usepackage[ISO]{diffcoeff}


% appendix
\usepackage[title]{appendix}
% \usepackage[intlimits]{mathtools}

\usepackage[export]{adjustbox}

% in-document references
\usepackage{hyperref}
\usepackage[capitalize]{cleveref}

% create plots
\usepackage{tikz}
\usepackage[compat=1.1.0]{tikz-feynman}
\usepackage{pgfplots}
\pgfplotsset{compat=1.16}

% Nice table of contents
\usepackage{tocloft}

% Get floats right
\usepackage[section]{placeins}

% adjust fig captions
\usepackage{caption}
\usepackage{subcaption}
\captionsetup{width=.9\textwidth}

\usepackage{titlepic}
% \usepackage[textsize=footnotesize]{todonotes}
\usepackage[disable]{todonotes}

\usepackage[style=numeric-comp,sorting=none,sortcites=true,doi=true,url=false,giveninits=true,hyperref]{biblatex}

\setlength{\marginparwidth}{2.2cm}

\setlength\cftparskip{1pt}
\setlength\cftbeforechapskip{0pt}

\setlength{\parindent}{0em}
\setlength{\parskip}{0.8em}

\def\equationautorefname~#1\null{Eq.~(#1)\null}


% Simple shortcuts
\newcommand{\Ell}{\mathcal{L}}      % Lagrangian L
\newcommand{\He}{\mathcal{H}}       % Hamiltonian H
\newcommand{\Ve}{\mathcal{V}}       % Potential V
\newcommand{\Em}{\mathcal{M}}       % Manifold M
\newcommand{\Ef}{\mathcal{F}}       % Fancy f
\newcommand{\R}{\mathbb{R}}         % Real numbers
\newcommand{\chpt}{$\chi$PT }       % Chiral pertubation theory
\newcommand{\SU}{\mathrm{SU}}       % SU(n)
\newcommand{\eps}{\varepsilon}      % nice epsilon
% \newcommand{\one}{\mathbb{1}}       % Identity
\newcommand{\hc}{\mathrm{h.c.}}     % Hermitian conjugate
\newcommand{\ex}[1]{\expectationvalue{#1}}
\newcommand{\D}{\mathcal{D}}

\newcommand{\one}{\text{\usefont{U}{bbold}{m}{n}1}}
\MakeRobust{\one}

% Big-O notation 
\newcommand{\Oh}[2][2]{\mathcal{O}\left(#2^{#1}\right)}

% (anti) commutator
\newcommand{\com}[2]{\left[#1, #2 \right]}
\newcommand{\acom}[2]{\left\{#1, #2 \right\}}

% Lie algebra
\newcommand{\liea}[2]{\mathfrak{#1}\left(#2\right)}
\newcommand{\lieg}[2]{\mathrm{#1}\left(#2\right)}

% Curly brackets
\newcommand{\C}[1]{\left\{ #1 \right\}}

% Fourier Transform
\newcommand{\F}[1]{\mathcal{F}\C{#1}}
\newcommand{\FInv}[1]{\mathcal{F}^{-1}\C{#1}}

% operator in braket
\newcommand{\inner}[3]{\left\langle #1 {\left| #2 \right|} #3 \right\rangle}

\newcommand{\T}[1]{\textrm{T}_\tau \left\{ #1 \right\}}

\usepackage{glossaries}

\makenoidxglossaries

\newglossaryentry{chi}
{
    name={$\chi$},
    description={Free parameter in the first order chiral Lagrangian. Related to the mass of the pion}
}


\title{The effective pion Lagrangian.}
\author{Martin Johnsrud}

%%%%%%%%%%%%%%%%%%%%%%%%%%%%%%%%%%%%%%%%%%%%%%%%%%%%%%%
%%%%%%% TODO %%%%%%%
% Forklare hva \alpha er (forstå hva \alpha er)
% være mer konsekevent på parantes {[()]}
% Forklare power counting og effective Lagrangian
% Sjekke partial integraion pi1 d0 pi2
% Fjerne paranteser sin(a), og ha eksponenten riktig
% Skifte notasjon på M^2 og K (progpagator)
% Bytte notasjon på q (propagaotr), samkjøre med Martin
%%%%%%%%%%%%%%%%%%%%%%%%%%%%%%%%%%%%%%%%%%%%%%%%%%%%%%%


%%%%%%%%%%%%%%%%%%%%%%%%%%%%%%%%%%%%
%%%%%%%%%%%%% SPØRSMÅL %%%%%%%%%%%%%
%%%%%%%%%%%%%%%%%%%%%%%%%%%%%%%%%%%%
% Er det ikke noe symmetribrudd hvis \mu_I = 0 ?
% Hvis ikk, hvordan er F0 og mu_I begge frie parametere?
% Setter vi faktisk m = 0? Hvordan har vi parameteren \chi hvis m = 0
% Hvordan er SU(2)(kompakt) isomorf til R^3???
% Hvor kommer EOM inn i bildet? Det ser ut som det bare er et navn...
% Kan jeg droppe l_4 termen?


\begin{document}
\maketitle 

\section{Effective Lagrangians}
%%%%%%%%%%%%%%%%%
%%%% SECTION %%%%
%%%%%%%%%%%%%%%%%

The technique used in \chpt to obtain the effective Lagrangian of the pion relies on a ``theorem'', formulated by \citeauthor{WeinbergPhenom}:
\begin{quote}
    [I]f one writes down the most general possible Lagrangian, including all terms consistent with assumed symmetry principles, and then calculates matrix elements with this Lagrangian to any given order of perturbation theory, the result will simply be the most general possible S-matrix consistent with analyticity, perturbative unitarity, cluster decomposition and the assumed symmetry principles \cite{WeinbergPhenom}.
\end{quote}
In other words, if we write down the most general Lagrange density consistent with symmetries of the underlying theory, it will result in the most general S-matrix consistent with important physical assumptions and our underlying theory.
All that is left to do is to tune the free parameters that this leaves.
As this Lagrangian contains infinitely many terms, and thus infinitely many free parameters, a method for ordering them in terms of importance is needed.
As described in \cite{Scherer2002IntroductionTC}, by rescaling the external momenta $p_\mu \rightarrow t p_\mu$ and quark masses $m_i \rightarrow t^2 m_i$, each term in the Lagrangian obtains a factor $t^D$.
The Lagrangian is then expanded as $\Ell = \sum_D \Ell_D$, where $\Ell_D$ contains all terms with a factor $t^D$.

In our case, the underlying theory is QCD with two quarks, up and down, with masses 
\begin{equation*}
    M =
    \begin{pmatrix}
        m_u & 0 \\
        0 & m_d
    \end{pmatrix}.
\end{equation*}
In the isospin limit, $m_u = m_d = m(=0?)$, the theory is invariant under global transformations in the group $G' = \lieg{SU}{2}_L \times \lieg{SU}{2}_R \times \lieg{U}{1}_V$.
All terms involving only pions is trivially invariant under $\lieg{U}{1}_V$, (HVORFOR?) so we focus on the $G = \lieg{SU}{2}_L \times \lieg{SU}{2}_R$ subgroup.
This symmetry is spontaneously broken if the quark field has a non-zero ground state expectation value $\ex{\bar q q}$, leaving only a subgroup $H = \lieg{SU}{2}_V \subseteq G$ of symmetry transformations of the vacuum state.
The Goldstein manifold $G/H = \lieg{SU}{2}_A$ is a three-dimensional Lie group, and therefore results in three (pseudo) Goldstein bosons, the pions.
There exists an isomorphism from a subset $S \subseteq M_1$ of the set of all Goldstein-fields
\begin{equation*}
    M_1 = \curly{ \pi_a: \Em_4 \longrightarrow \R^3 | \pi_a \, \mathrm{smooth} }
\end{equation*}
close to the ground state, into fields taking values in the Goldstein manifold $G/H$. (BEVISE?)(HVA ER ISOMORFISME HER?).
The \chpt effective Lagrangian will be constructed using this map, through the parametrization
\begin{align}
\label{sigma}
    \Sigma : \Em_4 & \longrightarrow \lieg{SU}{2},\\ \notag
    x & \longrightarrow \Sigma(x) = A_\alpha (U(x) \Sigma_0 U(x)) A_\alpha,
\end{align}
where 
\begin{align*}
    \Sigma_0 = \one,\, 
    A_\alpha = \exp(\frac{i \alpha}{2} \tau_1),\, 
    U(x) = \exp(i \frac{\tau_a\pi_a(x)}{2f}).
\end{align*}
$\tau_a$ are the $\SU(2)$ generators, i.e. Pauli matrices, as described in \autoref{Conventions and notation}
$\pi_a$, where $ \, a \in \curly{1, 2, 3}$, are the pion fields. These are real fields, meaning $\pi_a^\dagger = \pi_a$.

\section{Leading order Lagrangian}
%%%%%%%%%%%%%%%%%
%%%% SECTION %%%%
%%%%%%%%%%%%%%%%%

The leading order Lagrangian in \chpt is \cite{mojahed, Scherer2002IntroductionTC}
\begin{equation}
    \label{chpt lagrangian}
    \Ell_2 = 
    \frac{f^2}{4} \Tr\left[\nabla_\mu \Sigma (\nabla^\mu \Sigma)^\dagger \right] 
    + \frac{f^2}{4} \Tr \left[\chi^\dagger \Sigma + \Sigma^\dagger \chi \right].
\end{equation}
$\chi$ and $f$ are the free parameters of the theory. $f$ is the pion decay constant, while \gls{chi} $= 2B_0 M$. Here, $M$ is the mass matrix of the quarks, which in the isospin limit are taken to be equal. That is, $m = m_u = m_d$. $B_0$ is related to the quark condensate through $3 f^2 B_0 = - \ex{\bar q q}$. (ER DET SANT NÅR $\mu_I \neq 0?$) The covariant derivative is defined by
\begin{equation*}
    \nabla_\mu \Sigma = \partial_\mu \Sigma - i[v_\mu, \Sigma], \quad 
    (\nabla_\mu \Sigma)^\dagger 
    = \partial_\mu \Sigma^\dagger - i[v_\mu, \Sigma^\dagger], \quad
    v_\mu = \frac{1}{2} \mu_I \delta_\mu^0 \tau_3 ,
\end{equation*}
where $\mu_I$ is the isospin chemical potential.
This Lagrangian results in a pion mass of $M_\pi^2 = 2B_0m$. 
To get the series expansion of $\Sigma$ in powers of $\pi/f$, we start by using the fact that $\tau_a^2 = \one$ to write
\begin{equation*}
A_\alpha 
= \sum_n^\infty \frac{1}{n!} \left(\frac{i \alpha}{2} \tau_1 \right)^n 
= \sum_n^\infty 
\left[
    \frac{\one}{(2n)!} \left(\frac{i \alpha}{2}\right)^{(2n)} 
    + \frac{\tau_1}{(2n + 1)!} \left(\frac{i\alpha}{2}\right)^{(2n + 1)}
\right] 
= \one \cos(\frac{\alpha}{2}) + i \tau_1 \sin(\frac{\alpha}{2}).
\end{equation*}
Using the expansion of the exponential, 
\begin{align*}
    U = \exp(\frac{i \pi_a \tau_a}{2f}) = 
    1
    + \frac{i \pi_a \tau_a}{2f} 
    + \frac{1}{2}\left(\frac{i \pi_a \tau_a}{2f}\right)^2 
    + \frac{1}{6}\left(\frac{i \pi_a \tau_a}{2f}\right)^3 
    + \frac{1}{24}\left(\frac{i \pi_a \tau_a}{2f}\right)^4 
    + \Oh[5]{(\pi/f)}
\end{align*}
the inner part of the definition of $\Sigma$, as given in Eq. \eqref{sigma}, has the expansion
\begin{align*}
    U\Sigma_0U & = 
    \left(
        1
        + \frac{i \pi_a \tau_a}{2f} 
        + \frac{1}{2}\left(\frac{i \pi_a \tau_a}{2f}\right)^2 
        + \frac{1}{6}\left(\frac{i \pi_a \tau_a}{2f}\right)^3 
        + \frac{1}{24}\left(\frac{i \pi_a \tau_a}{2f}\right)^4 
    \right)\\
    & \times
    \left(
        1
        + \frac{i \pi_a \tau_a}{2f} 
        + \frac{1}{2}\left(\frac{i \pi_a \tau_a}{2f}\right)^2 
        + \frac{1}{6}\left(\frac{i \pi_a \tau_a}{2f}\right)^3 
        + \frac{1}{24}\left(\frac{i \pi_a \tau_a}{2f}\right)^4 
    \right)
    + \Oh[5]{(\pi/f)}\\
    &=
    1 + \frac{i \pi_a \tau_a}{f}
    + 2 \left( \frac{i \pi_a \tau_a}{2f} \right)^2
    + \frac{4}{3} \left( \frac{i \pi_a \tau_a}{2f} \right)^3
    + \frac{2}{3} \left( \frac{i \pi_a \tau_a}{2f} \right)^4
    + \Oh[5]{(\pi/f)}
\end{align*}
The symmetry of $\pi_a\pi_b$ means that
\begin{align*}
% Identitites
    & (\pi_a \tau_a)^2
    = 
    \pi_a \pi_b \frac{1}{2} \acom{\tau_a}{\tau_b} 
    =
    \pi_a \pi_a, \quad
    (\pi_a \tau_a)^3
    =
    \pi_a \pi_a \pi_b \tau_b,\quad
    (\pi_a \tau_a)^4
    =
    \pi_a \pi_a \pi_b \pi_b.
\end{align*}
This gives us the expression
\begin{align*}
% Final expression
    & U\Sigma_0U 
    =
    1
    + i \frac{\pi_a \tau_a}{f} 
    - \frac{\pi_a^2}{2f^2}
    - i \frac{\pi_a^2\pi_b \tau_b}{6f^3}
    + \frac{\pi_a^2\pi_b^2}{24f^4}
    + \Oh[5]{(\pi/f)},
\end{align*}
which we used to get the series expansion of $\Sigma$ up to $\Oh[5]{(\pi/f)}$
\begin{align*}
    \Sigma & =  \Big( \cos(\frac{\alpha}{2}) + i \tau_1 \sin(\frac{\alpha}{2}) \Big) 
    \left(
        1
        + i \frac{\pi_a \tau_a}{f} 
        - \frac{\pi_a^2}{2f^2}
        - i \frac{\pi_a^2\pi_b \tau_b}{6f^3}
        + \frac{\pi_a^2\pi_b^2}{24f^4}    
    \right)
    \Big( \cos(\frac{\alpha}{2}) + i \tau_1 \sin(\frac{\alpha}{2}) \Big) \\
    & =
    \left(
        1
        + i \frac{\pi_a \tau_a}{f} 
        - \frac{\pi_a^2}{2f^2}
        - i \frac{\pi_a^2\pi_b \tau_b}{6f^3}
        + \frac{\pi_a^2\pi_b^2}{24f^4}    
    \right)
    \cos(\frac{\alpha}{2})^2 \\
    & -
    \left(
        1
        + i \frac{\pi_a}{f} \tau_1\tau_a\tau_1
        - \frac{\pi_a^2}{2f^2}
        - i \frac{\pi_a^2\pi_b}{6f^3} \tau_1\tau_b\tau_1
        + \frac{\pi_a^2\pi_b^2}{24f^4}
    \right)
    \sin(\frac{\alpha}{2})^2\\
    & + i
    \left(
        2 \tau_1
        + i \frac{\pi_a}{f} \acom{\tau_1}{\tau_a}
        - 2\tau_1 \frac{\pi_a^2}{2f^2}
        - i \frac{\pi_a^2\pi_b}{6f^3} \acom{\tau_1}{\tau_b}
        + 2\tau_1 \frac{\pi_a^2\pi_b^2}{24f^4}
    \right)
    \sin(\frac{\alpha}{2})\cos(\frac{\alpha}{2}).
\end{align*}
Using trigonometric identities and the commutator,
\begin{align*}
    \cos(\frac{\alpha}{2})^2 - \sin(\frac{\alpha}{2})^2 = \cos(\alpha), \quad 2 \cos(\frac{\alpha}{2})\sin(\frac{\alpha}{2}) = \sin(\frac{\alpha}{2}), \quad
    \tau_1 \tau_a \tau_1
    = -\tau_a + 2 \delta_{1a}\tau_1,
\end{align*}
the final expression of $\Sigma$ to $\Oh[5]{(\pi/f)}$ is
\begin{align}
    \Sigma =
     \left(
        1 
        - \frac{\pi_a^2}{2f^2}
        + \frac{\pi_a^2\pi_b^2}{24f^4}
    \right)
    (\cos(\alpha) + i \tau_1 \sin(\alpha))
    + 
    \left(
        \frac{\pi_a}{f} 
        - \frac{\pi_b^2\pi_a}{6f^3} 
    \right)
    \left(
        i\tau_a - 2i \delta_{a1}\tau_1\sin(\frac{\alpha}{2})^2 - \delta_{1a} \sin(\alpha)
    \right).
    \label{expansion of sigma}
\end{align}

The kinetic term in the \chpt Lagrangian is
\begin{equation}
    \nabla_\mu \Sigma (\nabla^\mu \Sigma)^\dagger 
    = \partial_\mu \Sigma \partial^\mu \Sigma^\dagger 
    - i([v_\mu, \Sigma] \partial^\mu \Sigma^\dagger - \hc) 
    + [v_\mu, \Sigma]\left([v^\mu, \Sigma]\right)^\dagger.
    \label{kinetic term}
\end{equation}
Using \autoref{expansion of sigma} we find the expansion of the constitutive parts of the kinetic term to be
\begin{align}
    \notag
    \partial_\mu \Sigma = &
    \left(
        \frac{-1}{f^2}
        + \frac{\pi_b^2}{6f^4}
    \right)
    (\cos(\alpha) + i \tau_1 \sin(\alpha)) (\pi_a \partial_\mu \pi_a)\\
    +&
    \left(
        \frac{\partial_\mu \pi_a}{f} 
        - \frac{\pi_b^2 \partial_\mu\pi_a 
        + 2 \pi_a \pi_b \partial_\mu\pi_b}{6f^3} 
    \right)
    \left(
        i\tau_a - 2i \delta_{a1}\tau_1\sin(\frac{\alpha}{2})^2 - \delta_{1a} \sin(\alpha)
    \right)
    \label{Sigma derivative}\\ \notag
    =& 
    \left[
        \left(
            \frac{-1}{f^2}
            + \frac{\pi_b^2}{6f^4}
        \right)
        (\pi_a \partial_\mu \pi_a)
        \cos(\alpha)
        - 
        \left(
            \frac{\partial_\mu \pi_1}{f} 
            - \frac{\pi_b^2 \partial_\mu\pi_1
            + 2 \pi_1 \pi_b \partial_\mu\pi_b}{6f^3} 
        \right)
        \sin(\alpha)
    \right]
    \\ \notag 
    - &
    \left[
        \left(
            \frac{-1}{f^2}
            + \frac{\pi_b^2}{6f^4}
        \right)
        (\pi_a \partial_\mu \pi_a)
        \sin(\alpha)
        - \left(
        \frac{\partial_\mu \pi_1}{f} 
        - \frac{\pi_b^2 \partial_\mu\pi_1
        + 2 \pi_1 \pi_b \partial_\mu\pi_b}{6f^3}
        \right)
        2 \sin(\frac{\alpha}{2})^2
    \right]
    i \tau_1 \\ \notag
    +& 
    \left(
        \frac{\partial_\mu \pi_a}{f} 
        - \frac{\pi_b^2 \partial_\mu\pi_a 
        + 2 \pi_a \pi_b \partial_\mu\pi_b}{6f^3} 
    \right)
    i \tau_a,
\end{align}
and
\begin{align}
    \notag
    [v_\mu, \Sigma^\dagger] & = - 
    \frac{1}{2} \mu_I \delta^0_\mu
    \left[
        \left(
            1 
            - \frac{\pi_a^2}{2f^2}
            + \frac{\pi_a^2\pi_b^2}{24f^4}
        \right)
        i \sin(\alpha) \com{\tau_3}{\tau_1}
        + 
        \left(
            \frac{\pi_a}{f} 
            - \frac{\pi_b^2\pi_a}{6f^3} 
        \right)
        \left(
            i\com{\tau_a}{\tau_3} 
            - 2i\delta_{a1}\sin(\frac{\alpha}{2})^2\com{\tau_3}{\tau_1}
        \right)
    \right] \\
    \notag
    & =
    \mu_I \delta^0_\mu
    \left[
        \left(
            1 
            - \frac{\pi_a^2}{2f^2}
            + \frac{\pi_a^2\pi_b^2}{24f^4}
        \right)
        \sin(\alpha) \tau_2
        + 
        \left(
            \frac{\pi_a}{f} 
            - \frac{\pi_b^2\pi_a}{6f^3} 
        \right)
        \left(
            (\delta_{a1}\tau_2-\delta_{a2}\tau_1)
            - 2\delta_{a1}\sin(\frac{\alpha}{2})^2 \tau_2
        \right)
    \right] \\
    & =
    \mu_I \delta^0_\mu
    \left\{
        \left[
        \left(
            1 
            - \frac{\pi_a^2}{2f^2}
            + \frac{\pi_a^2\pi_b^2}{24f^4}
        \right)
        \sin(\alpha)
        + 
        \left(
            \frac{\pi_1}{f} 
            - \frac{\pi_b^2\pi_1}{6f^3} 
        \right) \cos(\alpha)
        \right]
         \tau_2
        -
        \left(
            \frac{\pi_2}{f} 
            - \frac{\pi_b^2\pi_2}{6f^3} 
        \right)
        \tau_1
    \right\}.
    \label{sigma commutator}
\end{align}
Combining \autoref{Sigma derivative} and \autoref{sigma commutator} gives the following terms \footnote{The Mathematica script used to aid the calculation of the Lagrangian is available here: \url{https://github.com/martkjoh/prosjektopggave}}
\begin{align*}
    % Term 1
    & \Tr{\partial_\mu \Sigma \partial^\mu \Sigma^\dagger}
    = \frac{2}{f^2} \partial_\mu \pi_a \partial^\mu \pi_a
    + \frac{2}{3f^4}
    \left[
        (\pi_a\partial_\mu \pi_a)(\pi_b\partial^\mu \pi_b)
        -        
        (\pi_a\partial_\mu \pi_b)(\pi_b\partial^\mu \pi_a)
    \right], \\
    % Term 2
    -i  &\Tr{\com{v_\mu}{\Sigma}\partial^\mu\Sigma^\dagger - \hc}
    % & =
    % \frac{4 \mu_I}{f} \partial_0\pi_2 \sin(\alpha)
    % + \frac{4\mu_I}{f^2}(\pi_1\partial_0 \pi_2 - \pi_2 \partial_0 \pi_1) \cos(\alpha)
    % + \frac{8 \mu_I}{3f^3} \\
    % &
    % \left[
    %     \pi_1(\pi_2 \partial_0 \pi_1 - \pi_1 \partial_0 \pi_2) + \pi_3(\pi_2 \partial_0 \pi_3 - \pi_3 \partial_0 \pi_2)
    % \right] \sin(\alpha)
    % \\ &
    % - \frac{4 \mu_I}{3f^4}
    % \pi_a \pi_a (\pi_1\partial_0 \pi_2 - \pi_2 \partial_0 \pi_1) \cos(\alpha)\\
    % % Rewrite
     =
    \frac{4 \mu_I}{f} \partial_0\pi_2 \sin(\alpha)
    + \frac{8 \mu_I}{3f^3}\sin(\alpha) \pi_3(
        \pi_2 \partial_0 \pi_3 - \pi_3 \partial_0 \pi_2
        )
    \\ & \quad \quad \quad \quad \quad \quad \quad \quad \quad \quad \quad
    +
    \left(
        \frac{4\mu_I}{f^2} \cos(\alpha)
        - \frac{8 \mu_I\pi_1}{3f^3} \sin(\alpha)
        - \frac{4 \mu_I \pi_a \pi_a} {3f^4}\cos(\alpha) 
    \right) 
    (\pi_1\partial_0 \pi_2 - \pi_2 \partial_0 \pi_1), \\
    % Term 3
    & \Tr{\com{v_\mu}{\Sigma}\com{v^\mu}{\Sigma}^\dagger}
    % \\ & \quad \,
    = \mu^2
    \bigg[
        2 \sin(\alpha)^2
        +
        \left(
            \frac{2}{f} 
            - \frac{4\pi_a \pi_a}{3 f^3} 
        \right)
        \pi_1  \sin(2\alpha)
        + \left(
            \frac{2}{f^2}
            - \frac{2 \pi_a \pi_a}{3 f^4} 
        \right)
        \pi_a \pi_b k_{ab}
        % \left(\cos(2 \alpha)\pi_1 + \cos(\alpha)^2 \pi_2 - \sin(\alpha)^2 \pi_3\right)
    \bigg], 
    \\
    % Mass Term
    & \Tr{\Sigma + \Sigma^\dagger}
    = 4 \cos(\alpha) 
    - \frac{4 \pi_1}{f} \sin(\alpha) 
    - \frac{2 \pi_a \pi_a}{f^2} \cos(\alpha)
    + \frac{2 \pi_1 \pi_a \pi_a}{3 f^3} 
    + \frac{(\pi_a \pi_a)^2}{6 f^4}\cos(\alpha), 
    \end{align*}
where $k_{ab} =\delta_{1a} \delta_{1b} \cos(\alpha)^2  + \delta_{2a}\delta_{2b} - \delta_{ab} \sin(\alpha)^2$.
If we write the Lagrangian as show in \autoref{chpt lagrangian} as $\Ell_2 = \Ell_2^{(0)} + \Ell_2^{(1)} + \Ell_2^{(2)} +...$, where $\Ell_2^{(n)}$ contains all terms of order $\Oh[n]{(\pi/f)}$, then the result of the series expansion is
\begin{align}
%%%%%%%%%%%%%%%%%%
%% zeroth order %%
%%%%%%%%%%%%%%%%%%
\Ell_2^{(0)}
&  =
    f^2   
    \left(
        2B_0m \cos(\alpha )
        + \frac{1}{2} \mu^2 \sin^2(\alpha )
    \right),
    \label{L0}
\\
%%%%%%%%%%%%%%%%%%
%% first order %%
%%%%%%%%%%%%%%%%%%
\label{L1}
\Ell_2^{(1)}
& =
    f 
    (
        \mu_I^2\cos (\alpha)
        - 2B_0m
    )\sin (\alpha) \pi_1 
    + f \mu_I \sin(\alpha) \partial_0\pi_2,
\\
%%%%%%%%%%%%%%%%%%
%% second order %%
%%%%%%%%%%%%%%%%%%
\Ell_2^{(2)}
& =
    \frac{1}{2} \partial_\mu\pi_a\partial^\mu\pi_a
    + \mu_I \cos(\alpha) \left( \pi_1 \partial_0\pi_2 - \pi_2\partial_0\pi_1 \right)
    - B_0m \cos(\alpha) \pi_a \pi_a
    + \frac{1}{2} \mu_I ^2 \pi_a \pi_b k_{ab},
\label{L2}
    % \\ \notag &
    % - \frac{1}{2}
    % \left[
    %     \pi_1^2 \left(\chi\cos(\alpha) - \mu_I{}^2\cos(2\alpha) \right)
    %     + \pi_2^2 \left(\chi\cos(\alpha) - \mu_I{}^2\cos(\alpha)^2 \right)
    %     + \pi_2^2 \left(\chi\cos(\alpha) + \mu_I{}^2\sin(\alpha)^2 \right)
    % \right]
\\
%%%%%%%%%%%%%%%%%%
%% third order %%
%%%%%%%%%%%%%%%%%%
\label{L3}
\Ell_2^{(3)}
& =
    \frac{\pi_a\pi_a \pi_1}{6f}
    (2B_0m \sin (\alpha )-2\mu_I{}^2 \sin (2\alpha ))
    + 
    \frac{2 \mu_I}{3 f}
    \left(
        \pi_2 \pi_1\partial_0\pi_1
        - \pi_1^2 \partial_0\pi_2
        - \pi_3^2\partial_0\pi_2
        + \pi_2 \pi_3 \partial_0\pi_3
    \right)\sin (\alpha),
\\
%%%%%%%%%%%%%%%%%%
%% fourth order %%
%%%%%%%%%%%%%%%%%%
\notag
\Ell_2^{(4)}
& =
\frac{1}{6f^2}
\curly{    
    \frac{1}{2} B_0m (\pi_a\pi_a)^2 \cos(\alpha)
    -
    \left[
        (\pi_a \pi_a) (\partial_\mu \pi_b \partial^\mu \pi_b )
        - (\pi_a \partial_\mu \pi_a)(\pi_b \partial^\mu \pi_b )
    \right]
}
\\
&
- \frac{\mu_I \pi_a\pi_a}{3f^2}
\left[
    \bigg(\pi_1\partial_0 \pi_2 - \pi_2 \partial_0 \pi_1\bigg) 
    \cos(\alpha)
    + \frac{1}{2} \mu_I ^2 \pi_a \pi_b k_{ab}
\right].
\label{L4}
\end{align}

The equation of motion for the leading order Lagrangian is found by using the principle of least action.
Choosing the parametrization $\Sigma = \exp(i \pi_a \tau_a)$, a variation $\pi_a \rightarrow \pi_a + \varepsilon \eta_a = \pi_a + \delta \pi_a$ gives a variation in $\Sigma$, $\delta \Sigma = i \tau_a \delta \pi_a \Sigma $. 
The variation of the action is then 
\begin{align*}
    \delta S = \int_\Omega \dd x \, \frac{f^2}{4}
    \Tr{
        (\nabla_\mu \delta \Sigma) (\nabla^\mu \Sigma)^\dagger
        + (\nabla_\mu \Sigma) (\nabla^\mu \delta \Sigma)^\dagger
        + \chi \delta \Sigma^\dagger + \delta \Sigma \chi^\dagger
    }.
\end{align*}
Using the properties of the covariant derivative to do partial integration, as show in \autoref{Covarinat derivative}, as well as the $\delta(\Sigma \Sigma^\dagger) = (\delta\Sigma)\Sigma^\dagger + \Sigma (\delta \Sigma)^\dagger = 0$, the variation of the action can be written
\begin{align*}
    \delta S 
    & = \frac{f^2}{4} \int_\Omega \dd x\, 
    \Tr{
        - \delta \Sigma \nabla^2 \Sigma^\dagger
        + (\nabla^2 \Sigma) (\Sigma^\dagger \delta \Sigma \Sigma^\dagger)
        - \chi (\Sigma^\dagger \delta \Sigma \Sigma^\dagger)
        + \delta \Sigma \chi^\dagger
    } \\
    % & = 
    % \frac{f^2}{4} \int_\Omega \dd x\, 
    % \Tr{\delta \Sigma \left[
    %     \Sigma^\dagger (\nabla^2 \Sigma)\Sigma^\dagger
    %     - \nabla^2 \Sigma^\dagger
    %     - \Sigma^\dagger \chi \Sigma^\dagger
    %     + \chi^\dagger
    % }\right] \\
    & = 
    \frac{f^2}{4} \int_\Omega \dd x\, 
    \Tr{
        \delta \Sigma \Sigma^\dagger 
        \left[
            (\nabla^2 \Sigma)\Sigma^\dagger
            - \Sigma \nabla^2 \Sigma^\dagger
            - \chi \Sigma^\dagger
            + \Sigma \chi^\dagger
        \right]
        } \\
    & = 
    i \frac{f^2}{4} \int_\Omega \dd x\, 
    \Tr{\tau_a 
    \left[
         (\nabla^2 \Sigma)\Sigma^\dagger
        - \Sigma \nabla^2 \Sigma^\dagger
        - \chi \Sigma^\dagger
        + \Sigma \chi^\dagger
    \right]
    } 
    \delta \pi_a = 0
\end{align*}  
As the variation is arbitrary, the equation of motion to leading order is
\begin{equation}
    \Tr{
        \tau_a 
        \left[
            (\nabla^2 \Sigma)\Sigma^\dagger
            - \Sigma \nabla^2 \Sigma^\dagger
            - \chi \Sigma^\dagger
            + \Sigma \chi^\dagger
        \right]
    } = 0.
\end{equation}
This may be rewritten as a matrix equation. 
Using that 
\begin{align*}
    \Tr{(\nabla_\mu \Sigma)\Sigma^\dagger}
    = 
    \Tr{i \tau_a (\partial_\mu \pi_a )\Sigma \Sigma^\dagger}
    - i\Tr{[v_\mu, \Sigma]\Sigma^\dagger}
    = 0,
\end{align*}
we can se that $\Tr{(\nabla^2 \Sigma)\Sigma^\dagger - \Sigma \nabla^2 \Sigma^\dagger} = 0$, and the equation of motion may be written as
\begin{equation}
    \label{EOM matrix form}
    \mathcal{O}_\mathrm{EOM}^{(2)}(\Sigma) 
    = 
    (\nabla^2 \Sigma)\Sigma^\dagger
    - \Sigma \nabla^2 \Sigma^\dagger
    - \chi \Sigma^\dagger
    + \Sigma \chi^\dagger
    - \frac{1}{2}
    \Tr{ \chi \Sigma^\dagger - \Sigma \chi^\dagger} = 0.
\end{equation}


\section{Next to leading order Lagrangian}
%%%%%%%%%%%%%%%%%
%%%% SECTION %%%%
%%%%%%%%%%%%%%%%%

The next to leading order Lagrangian density is, assuming no external fields (ER DET RIKTIG FORMULERING? GAUGE FIELDS?)
\begin{align}
    \notag
    \Ell_4 
    % & = 
    % \frac{l_1}{4} \Tr{\nabla_\mu \Sigma (\nabla^\mu \Sigma)^\dagger}^2
    % + \frac{l_2}{4} \Tr{\nabla_\mu \Sigma (\nabla_\nu \Sigma)^\dagger} 
    % \Tr{\nabla^\mu \Sigma (\nabla^\nu \Sigma)^\dagger} 
    % +
    % \frac{l_3 + l_4}{16} \Tr{\chi \Sigma^\dagger + \Sigma \chi^\dagger}^2
    % \\ \notag
    % &
    % + \frac{l_4}{4}\Tr{\nabla_\mu \Sigma (\nabla^\mu \Sigma)^\dagger} \Tr{\chi \Sigma^\dagger + \Sigma \chi^\dagger}
    % - \frac{l_7}{16} \Tr{\chi \Sigma^\dagger - \Sigma \chi^\dagger}^2
    % + \frac{h_1 + h_2 - l_4}{4} \Tr{\chi \chi^\dagger} \\
    % & + 
    % \frac{h_1 - h_3 - l_4}{16} 
    % \left[
    %     \Tr{\chi \Sigma^\dagger + \Sigma \chi^\dagger}^2
    %     +\Tr{\chi \Sigma^\dagger - \Sigma \chi^\dagger}^2
    %     -2\Tr{\left(\chi \Sigma^\dagger\right)^2 + \left( \Sigma \chi^\dagger\right)^2}
    % \right] \\
    % \notag
    & = 
    \frac{l_1}{4} \Tr{\nabla_\mu \Sigma (\nabla^\mu \Sigma)^\dagger}^2
    + \frac{l_2}{4} \Tr{\nabla_\mu \Sigma (\nabla_\nu \Sigma)^\dagger} 
    \Tr{\nabla^\mu \Sigma (\nabla^\nu \Sigma)^\dagger} 
    +
    \frac{l_3 + h_1 - h_3 }{16} \Tr{\chi \Sigma^\dagger + \Sigma \chi^\dagger}^2
    \\ \notag
    &
    + \frac{l_4}{4}\Tr{\nabla_\mu \Sigma (\nabla^\mu \Sigma)^\dagger} \Tr{\chi \Sigma^\dagger + \Sigma \chi^\dagger}
    + \frac{h_1 - h_3 - l_4-l_7}{16} \Tr{\chi \Sigma^\dagger - \Sigma \chi^\dagger}^2
    + \frac{h_1 + h_2 - l_4}{4} \Tr{\chi \chi^\dagger} \\
    & -
    \frac{h_1 - h_3 - l_4}{8} 
        \Tr{\left(\chi \Sigma^\dagger\right)^2 + \left( \Sigma \chi^\dagger\right)^2}
    \label{NLO Lagrangian}
\end{align}
The form of the Lagrangian is not unique.
Changing the parametrization of $\Sigma$ by $\Sigma(x) \rightarrow \Sigma'(x), \quad \Sigma(x) = e^{iS(x)} \Sigma'(x), \, S(x) \in \liea{su}{2}$ leads to a new Lagrangian density, $\Ell[\Sigma] = \Ell[\Sigma'] + \Delta \Ell[\Sigma']$.
By tuning $S(x)$ it is possible to eliminate terms that naively appeared to be independent.
When demanding that $\Sigma'$ obey the same symmetries as $\Sigma$, the most general transformation to $\Oh[2]{t}$ is
\begin{equation*}
    S_{2} = 
    i \alpha_2 
    \left[
        (\nabla^2 \Sigma') \Sigma^\dagger - \Sigma' (\nabla^2 {\Sigma'})^\dagger
    \right]
    + i \alpha_2
    \left[
        \chi \Sigma'^\dagger - \Sigma' \chi^\dagger 
        - \frac{1}{2} \Tr{\chi \Sigma'^\dagger - \Sigma' \chi^\dagger}
    \right].
\end{equation*}
$\alpha_1$ and $\alpha_2$ are arbitrary real numbers. To $\Oh[4]{t}$, an arbitrary new parametrization gives the new Lagrangian
\begin{align*}
    \Ell_2\left[e^{i S_2}\Sigma '\right]
    & =
    \frac{f^2}{f}\Tr{[\nabla_\mu (1 +i S_2)\Sigma'][\nabla^\mu \Sigma'^\dagger  (1 - i S_2)]}
    + \frac{f^2}{4} \Tr{\chi\Sigma'^\dagger (1 - i S_2) + (1 +i S_2)\Sigma' \chi^\dagger} \\
    & = \Ell[\Sigma'] + 
    i \frac{f^2}{4}
    \Tr{[\nabla_\mu (S_2\Sigma')][\nabla^\mu\Sigma']^\dagger 
    -  [\nabla_\mu\Sigma'][\nabla^\mu (\Sigma'^\dagger  S_2) ]}
    - i \frac{f^2}{4} \Tr{\chi \Sigma'^\dagger S_2 - S_2 \Sigma' \chi^\dagger}
\end{align*}
Using the properties of the covariant derivative, as described in \autoref{Covarinat derivative}, we may use the product rule and partial integration to write the difference between the two Lagrangians as
\begin{align*}
    \Delta \Ell[\Sigma'] 
    & = 
    i \frac{f^2}{4}
    \Tr{
        (\nabla_\mu S_2)
        (\Sigma' \nabla^\mu \Sigma'^\dagger - (\nabla^\mu \Sigma') \Sigma'^\dagger) 
    }
    - i \frac{f^2}{4} \Tr{\chi \Sigma'^\dagger  S_2 - S_2 \Sigma' \chi^\dagger} \\
    & = 
    i \frac{f^2}{4}
    \Tr{
        S_2
        \left[
            \Sigma'^\dagger\nabla^2 \Sigma' - ( \nabla^2 \Sigma') \Sigma'^\dagger 
            - \chi\Sigma'^\dagger + \Sigma' \chi^\dagger
        \right]
    }.
\end{align*}
Using the equation of motion as given \autoref{EOM matrix form}, and the fact that $\Tr{S_2} = 0$, we may write this difference as
\begin{align*}
    \Delta \Ell[\Sigma'] = \frac{f^2}{4} \Tr{i S_2 \mathcal{O}_\mathrm{EOM}^{(2)}(\Sigma')}.
\end{align*}
As the physics are unchanged under a reparametrization, any term that can be rewritten in the form $\Delta \Ell$, for any $\alpha_1, \alpha_2 \in \R$ can be removed from the Lagrangian.
$\Delta \Ell_2$ is of order $\Oh[4]{t}$, and it can thus be used to remove terms from $\Ell_4$.
 
To expand this $\Ell_4$ to $\Oh[2]{(\pi/f)}$, we us the result from \autoref{Sigma derivative} and \autoref{sigma commutator},
\begin{align*}
    \Sigma & =
    \left(
       1 
       - \frac{\pi_a^2}{2f^2}
   \right)
   (\cos(\alpha) + i \tau_1 \sin(\alpha))
   +  \frac{\pi_a}{f}    \left(
       i\tau_a 
       - \delta_{1a} 2i \sin(\frac{\alpha}{2})^2 \tau_1 
       - \delta_{1a} \sin(\alpha)
   \right), \\
    \partial_\mu \Sigma 
    & = 
    - \frac{\pi_a \partial_\mu \pi_a}{f^2}
    \left(\cos(\alpha) + i \tau_1 \sin(\alpha)\right)
    + \frac{\partial_\mu \pi_a}{f}
    \left(
        i\tau_a 
        -2 i \delta_{a1} \sin(\frac{\alpha}{2})^2 \tau_1 
        - \delta_{a1}\sin(\alpha)\right), \\
    [v_\mu, \Sigma^\dagger] 
    & =
    \mu_I \delta^0_\mu
    \left[
        \left( 1 - \frac{\pi_a^2}{2f^2} \right) \sin(\alpha) \tau_2
        + \frac{\pi_a}{f}
        \left(
            \delta_{a1} \cos(\alpha) \tau_2 - \delta_{a2} \tau_1
        \right)
    \right].
    \end{align*}
Up to and including $\Oh{(\pi/f)}$, this gives
\begin{align*}
    %%%%%%%%%%%%%%%%%%
    % parital square %
    %%%%%%%%%%%%%%%%%%
    % & \partial_\mu \Sigma \partial_\nu \Sigma^\dagger
    %  =
    % \frac{\partial_\mu \pi_a \partial_\nu \pi_b}{f^2}
    % \Bigg[
    %     \tau_a \tau_b
    %     + 2\sin(\frac{\alpha}{2})^2
    %     \left(
    %         2 \delta_{1a}\delta_{1b} 
    %         - \delta_{1a} \tau_1 \tau_b - \delta_{1b} \tau_a \tau_1
    %     \right)
    %     + i 
    %         \left(
    %             \delta_{1a}\tau_b - \delta_{1b}\tau_a
    %         \right)\sin(\alpha)
    % \Bigg], \\
    \Tr{\partial_\mu \Sigma \partial_\nu \Sigma^\dagger}
    & =
   2 \frac{\partial_\mu \pi_a \partial_\nu \pi_a}{f^2} \\
    %%%%%%%%%%%%%%
    % Cross term %
    %%%%%%%%%%%%%
    % & \partial_\mu \Sigma \com{v_\nu}{\Sigma^\dagger} 
    % - \hc
    %  = 
    % 2 i \mu_I \delta_\nu^0
    % \Bigg\{
    %     \frac{\partial_\mu \pi_a}{f}
    %     \left(
    %         \tau_a - 2 \delta_{1 a} \sin(\frac{\alpha}{2})^2 \tau_1
    %     \right)
    %     \left[
    %         \sin(\alpha) \tau_2 
    %         +\frac{\pi_b}{f}
    %         \left(
    %             \delta_{b1} \cos(\alpha) \tau_2 - \delta_{2b} \tau_1
    %         \right)
    %     \right]
    %     % \\
    %     % & \quad \quad \quad \quad \quad
    %     - \frac{\pi_a \partial_\mu \pi_a}{f^2}
    %     \sin(\alpha)^2 \tau_1 \tau_2 
    % \Bigg\}, \\
    \Tr{\partial_\mu \Sigma \com{v_\nu}{\Sigma^\dagger} - \hc}
    & = 2 i \mu_I 
    \left[
        (\delta_\mu^0\partial_\nu + \delta_\nu^0\partial_\mu)\pi_2 \sin(\alpha) + 
        \cos(\alpha)
        \frac{\pi_1 (\delta_\mu^0\partial_\nu + \delta_\nu^0\partial_\mu)\pi_2 
        - \pi_2 (\delta_\mu^0\partial_\nu + \delta_\nu^0\partial_\mu)\pi_1}{f^2}
    \right]
    \\
    %%%%%%%%%%%%%%%%%
    % Comm. squared %
    %%%%%%%%%%%%%%%%%
    \Tr{\com{v_\nu}{\Sigma} \com{v_\nu}{\Sigma^\dagger}}
    & =   -2 \mu_I{}^2 \delta_\mu^0 \delta_\nu^0 
    \left[
        \sin(\alpha)^2 + \frac{\pi_1}{f} \sin(2\alpha) 
        + \frac{\pi_a \pi_b}{f^2} 
        k_{ab}
    \right].
\end{align*}
Using the form of the Pauli matrices, we may write $\chi$ as 
\begin{align*}
    \chi = 2 B_0 M = 2 B_0 (\bar m \one + \Delta m \tau_3),
\end{align*}
where $\bar m  = (m_u + m_d)/2, \, \Delta m = (m_u - m_d)/2$, which gives 
\begin{align*}
    \chi \Sigma^\dagger + \Sigma \chi^\dagger
    %  = 2 B_0 \left[
    %     \bar m (\Sigma^\dagger + \Sigma) 
    %     + \Delta m ( \tau_3 \Sigma^\dagger + \Sigma \tau_3)
    % \right] \\
    % & = 2 B_0 \Bigg\{
    %     2 \bar m 
    %     \left[
    %         \left(
    %             1 
    %             - \frac{\pi_a^2}{2f^2}
    %         \right)
    %         \cos(\alpha)
    %         - \frac{\pi_a}{f}    
    %         \delta_{1a} \sin(\alpha)
    %     \right] \\
    % &
    % + \Delta m 
    % \left[
    %     \left(
    %         1 
    %         - \frac{\pi_a^2}{2f^2}
    %     \right)
    %     (2\cos(\alpha) 
    %     + i \com{\tau_1}{\tau_3} \sin(\alpha))
    %     +  \frac{\pi_a}{f}    
    %     \left(
    %         i\com{\tau_a}{\tau_3} 
    %         - \delta_{1a} 2i \sin(\frac{\alpha}{2})^2 \com{\tau_1}{\tau_3}
    %         - 2 \delta_{1a} \sin(\alpha)
    %     \right)
    % \right]
    % \Bigg\} \\
    % & = 2 B_0 \Bigg\{
    %     2 (\bar m + \Delta m)
    %     \left[
    %         \left(
    %             1 
    %             - \frac{\pi_a^2}{2f^2}
    %         \right)
    %         \cos(\alpha)
    %         - \frac{\pi_a}{f}    
    %         \delta_{1a} \sin(\alpha)
    %     \right] \\
    % &
    % - 2 i \Delta m 
    % \left[
    %     \left(
    %         1 
    %         - \frac{\pi_a^2}{2f^2}
    %     \right)
    %     \sin(\alpha) i \tau_2
    %     +  \frac{\pi_a}{f}
    %     \left(
    %         i(\delta_{a1}\tau_2 - \delta_{a2}\tau_1)
    %         - \delta_{1a} 2 \sin(\frac{\alpha}{2})^2 i\tau_2
    %     \right)
    % \right]
    % \Bigg\} \\
    & = 4 B_0 \Bigg\{
         (\bar m + \Delta m \tau_3)
        \left[
            \left(
                1 
                - \frac{\pi_a^2}{2f^2}
            \right)
            \cos(\alpha)
            - \frac{\pi_1}{f}    
            \sin(\alpha)
        \right]\\
    & \quad \quad \quad
    + \Delta m 
    \left[
        \left(
            1 
            - \frac{\pi_a^2}{2f^2}
        \right)
        \sin(\alpha) \tau_2
        +  \frac{\pi_a}{f}
        \left(
            \delta_{a1} \cos(\alpha) \tau_2 - \delta_{a2}\tau_1
        \right)
    \right]
    \Bigg\}, \\
    %%%%%%%%%%%%%%
    % Difference %
    %%%%%%%%%%%%%%
    \chi \Sigma^\dagger  - \Sigma \chi^\dagger
    % = 2 B_0 
    % \left[
    %     \bar m (\Sigma^\dagger - \Sigma) 
    %     + \Delta m ( \tau_3 \Sigma^\dagger - \Sigma \tau_3)
    % \right] \\
    % & = - 2 B_0 \Bigg\{
    %     2 \bar m 
    %     \left[
    %         \left(
    %             1 - \frac{\pi_a^2}{2f^2}
    %         \right)
    %         (i \tau_1 \sin(\alpha))
    %         +  \frac{\pi_a}{f}    \left(
    %             i\tau_a 
    %             - \delta_{1a} 2i \sin(\frac{\alpha}{2})^2 \tau_1 
    %         \right)        
    %     \right] \\
    % &
    % i \Delta m 
    % \left[
    %     \left(
    %         1 
    %         - \frac{\pi_a^2}{2f^2}
    %     \right)
    %     (\sin(\alpha) \acom{\tau_3}{\tau_1})
    %     +  \frac{\pi_a}{f}    
    %     \left(
    %         \acom{\tau_3}{\tau_a}
    %         - \delta_{1a} 
    %         2 \sin(\frac{\alpha}{2})^2 \acom{\tau_3}{\tau_1}
    %     \right)        
    % \right]
    % \Bigg\} \\
    & = - 4 i B_0 \Bigg\{
        \bar m 
        \left[
            \left(
                1 - \frac{\pi_a^2}{2f^2}
            \right)
            \sin(\alpha) \tau_1
            +  \frac{\pi_a}{f}    \left(
                \tau_a 
                - \delta_{1a} 2 \sin(\frac{\alpha}{2})^2 \tau_1 
            \right)        
        \right]
        + \Delta m 
            \frac{\pi_3}{f}
    \Bigg\}.
\end{align*}
This results in the terms, to $\Oh[2]{(\pi/f)}$
\begingroup
\allowdisplaybreaks
\begin{align*}
    %%%%%%%
    % l_1 %
    %%%%%%%
    & \Tr{\nabla_\mu \Sigma (\nabla^\mu \Sigma)^\dagger}^2 
    =
    \Tr{\partial_\mu \Sigma \partial^\mu \Sigma ^\dagger
    - i (\partial_\mu \Sigma \com{v^\mu}{\Sigma^\dagger} - \hc) 
    - \com{v_\mu}{\Sigma}\com{v^\mu}{\Sigma^\dagger} 
    }^2 \\
    &\quad  =
    \frac{8 \mu_I^2}{f^2} 
    (\partial_\mu \pi_a \partial^\mu \pi_a + 2 \partial_\mu \pi_2 \partial^\mu \pi_2)
    \sin(\alpha)^2 \\
    &\quad  + 16 \mu_I^3 \left[
        \frac{\partial_0 \pi_2}{f}\
            \sin(\alpha)^3
        + \frac{3 \pi_1 \partial_0 \pi_2 - \pi_2 \partial_0 \pi_1}{f^2} 
        \cos(\alpha) \sin(\alpha)^2
    \right] \\
    & \quad + 4 \mu_I^4 
    \left\{
        \sin(\alpha)^4
        + 2 \sin(\alpha) ^2 
        \left[
            \frac{\pi_1}{f}\sin(2 \alpha) 
            + \frac{\pi_a \pi_b}{f^2}        
            \left(k_{ab} + 2\cos(\alpha)^2 \delta_{a1}\delta_{a2}\right)
        \right]
    \right\}, \\
    %%%%%%%
    % l_2 %
    %%%%%%%
    & \Tr{\nabla_\mu \Sigma (\nabla_\nu \Sigma)^\dagger} \Tr{\nabla^\mu \Sigma (\nabla^\nu \Sigma)^\dagger}\\
    & \quad = \frac{4 \mu_I{}^2}{f^2}
    \left(
        \partial_0 \pi_a\partial_0 \pi_a + \partial_0 \pi_2\partial_0 \pi_2 + \partial_\mu \pi_2\partial^2\mu \pi_2
    \right) \sin(\alpha)^2 \\
    &\quad  + 16 \mu_I^3 \left[
        \frac{\partial_0 \pi_2}{f}\
            \sin(\alpha)^3
        + \frac{3 \pi_1 \partial_0 \pi_2 - \pi_2 \partial_0 \pi_1}{f^2} 
        \cos(\alpha) \sin(\alpha)^2
    \right] \\
    & \quad + 4 \mu_I^4 
    \left\{
        \sin(\alpha)^4
        + 2 \sin(\alpha) ^2 
        \left[
            \frac{\pi_1}{f}\sin(2 \alpha) 
            + \frac{\pi_a \pi_b}{f^2}        
            \left(k_{ab} + 2\cos(\alpha)^2 \delta_{a1}\delta_{a2}\right)
        \right]
    \right\}, \\
    %%%%%%%
    % l_4 %
    %%%%%%%
    & \Tr{\nabla_\mu \Sigma (\nabla^\mu \Sigma)^\dagger} 
    \Tr{\chi \Sigma^\dagger + \Sigma \chi^\dagger } \\
    & \quad =
    8 B_0 \bar m 
    \Bigg\{
        2 \cos(\alpha) \frac{\partial_\mu \pi_a \partial^\mu \pi_a}{f^2}
        + 4 \mu_I 
        \left[
            \frac{\partial_0 \pi_2}{2 f} \sin(2 \alpha)
            + \frac{
                \pi_1 \partial_0 \pi_2 \cos(2\alpha) 
                - \pi_2 \partial_0 \pi_1\cos(\alpha)^2
                }
                {f^2}
        \right]\\
        & \quad + \mu_I{}^2
        \left[
            2\cos(\alpha)\sin(\alpha)^2 
            - 2\sin(\alpha) 
            \frac{\pi_1}{f}
            \left(3\sin(\alpha)^2 - 1\right)
            + \frac{                
                \pi_1^2(2 - 9 \sin(\alpha)^2)
                + \pi_2^2 (2 - 3 \sin(\alpha)^2)
                - 3\pi_3^2\sin(\alpha)^2
            }{f^2}
            \cos(\alpha)
        \right]
    \Bigg\}, \\
    %%%%%%%
    % l_3 %
    %%%%%%%
    & \Tr{\chi \Sigma^\dagger + \Sigma \chi^\dagger }^2
    = (8 B_0 \bar m)^2 
    \left[
        \cos(\alpha)^2 
        - \frac{\pi_1}{2f} \sin(2 \alpha) 
        + \frac{1}{f^2}\left(\pi_1^2 \sin(\alpha)^2 - \pi_a \pi_a \cos(\alpha)^2)\right)
    \right], \\
    %%%%%%%
    % l_7 %
    %%%%%%%
    & \Tr{\chi \Sigma^\dagger - \Sigma \chi^\dagger }^2
     = - 16 \left( \frac{2 \Delta m B_0 \pi_3}{f} \right)^2, \\
    %%%%%%%
    % h_3 %
    %%%%%%%
    & \Tr{\left(\chi \Sigma^\dagger\right)^2 + \left(\Sigma \chi^\dagger \right)^2}
    = 16 B_0 \bar m
    \left(
        \cos(2 \alpha) 
        - \pi_1 \sin(2\alpha) 
        - \cos(\alpha) \pi_a \pi_a
        -\cos(2 \alpha) \pi_1^2
    \right), \\
    %%%%%%%
    % h_2 %
    %%%%%%%
    & \Tr{\chi \chi^\dagger} = {\bar m}^2 + {\Delta m}^2.
\end{align*}
\endgroup

\section{Minimizing energy}
%%%%%%%%%%%%%%%%%
%%%% SECTION %%%%
%%%%%%%%%%%%%%%%%
Minimizing the effective action $\Gamma[\pi]$ with respect to the expectation value $\pi_a(x) = \ex{\hat \pi_a}$ gives the ground state expectation value off the field. 
The first order approximation to this is give by the classical potential. (HVORFOR?)
The static Hamiltonian density $\He^{(0)}$, which we get from \autoref{L0} through
\begin{equation*}
    \He_2^{(0)} = - \Ell_2^{(0)} = 
    -f^2   
    \left(
        2B_0m \cos(\alpha )
        + \frac{1}{2} \mu^2 \sin^2(\alpha )
    \right),
\end{equation*}
is minimized with respect to $\alpha$. This is achieved at 
\begin{align}
    &\dv{\alpha} \He_2^{(0)} 
    = f^2\left(2B_0m - \mu_I^2\cos(\alpha)\right)\sin(\alpha) 
    = 0.
\end{align}
This gives the solution set and minimization criterion
\begin{align}
    \alpha = \pi n, \, n \in \mathbb{Z} \quad
    \mathrm{or} \quad
    \cos(\alpha) = \frac{2B_0m}{\mu_I{}^2}.
\end{align}
We see that the linear part of the potential from \autoref{L1}, $\Ve^{(1)} = f(\mu_I{}^2\cos(\alpha) - 2B_0m)\sin(\alpha)\pi_1 = 0$ if and only if the criterion for minimization is fulfilled, as we expect (HVORFOR??).

\section{Propagator}
%%%%%%%%%%%%%%%%%
%%%% SECTION %%%%
%%%%%%%%%%%%%%%%%

We may write the quadratic part of the Lagrangian \autoref{L2} as 
\begin{align}
    \Ell^{(2)}
    =
    \frac{1}{2} \delta_{ab} \partial_\mu \pi_a \partial^\mu \pi_b
    + K_{ab} \pi_a \partial_0 \pi_b
    - \frac{1}{2} M^2_{ab} \pi_a \pi_b,
\end{align}
where  (Er det mer naturlig å ha $m_\pi^2 = 2 B_0 m$?)
\begin{align*}
    M^2 & = \mathrm{diag} 
    \left(
        2 B_0 m \cos(\alpha) - \mu_I^2 \cos(2\alpha),\quad 
        2 B_0 m \cos(\alpha) - \mu_I^2 \cos(\alpha)^2,\quad
        2 B_0 m \cos(\alpha) + \mu_I^2 \sin(\alpha)^2
    \right),
    \\
    K & =
    \begin{pmatrix}
        0 & \mu_I\cos(\alpha) & 0 \\
        -\mu_I\cos(\alpha) & 0 & 0 \\
        0 & 0 & 0
    \end{pmatrix}.
\end{align*}
The components of the Euler-Lagrange equations of this field are
\begin{equation*}
    \pdv{\Ell}{\pi_c} = K_{cb} \partial_0 \pi_b - M^2_{cb} \pi_b, \quad
    \pdv{\Ell}{(\partial_\mu \pi_c)} = \partial^\mu \pi_c - K_{ca}\delta^\mu_0 \pi_a.
\end{equation*}
This gives the equation of motion for the field
\begin{equation}
    \partial^\mu \partial_\mu \pi_a + M^2_{a b} \pi_b 
    = 2 K_{ab}\partial_0 \pi_b 
    = 2 \mu_I \cos(\alpha) (\delta_{a1}\pi_2 - \delta_{a2}\pi_1).
\end{equation}
The propagator of the pion field is defined by
\begin{equation}
    \left(
        \delta_{ab}\partial_\mu\partial^\mu - 2 K_{ab} \partial_0 + M^2_{ab} 
    \right) D_{bc}(x, x') 
    = -i \delta(x - x') \delta_{ac}.
\end{equation}
The momentum space propagator, as defined in the \autoref{Conventions and notation}, fulfills
\begin{equation*}
    -(\delta_{ab}p^2 - 2 K_{ab} i p_0 - M_{ab}^2) \tilde D_{bc}(p) 
    := A_{ab} \tilde D_{bc}(p) = -i \delta_{ac},
\end{equation*}
where
\begin{equation*}
    A = -
    \begin{pmatrix}
        p^2 - M^2_{11}              & -2 i p_0 K_{12}   & 0             \\
        2 i p_0 K_{12}  & p^2 - M^2_{22}                & 0             \\
        0                           & 0                 & p^2 - M^2_{33}
    \end{pmatrix}.
\end{equation*}
The spectrum of the particles is given by solving $\det(A) = 0$ for $p^0$. With $p = (p_0, p_i) = (p_0, q)$ as the four momentum, this gives
\begin{align*}
    \det(A) & = A_{33} \left(A_{11} A_{22} + A_{12}^2\right)
    = \left(p^2 - M^2_{33}\right)
    \left[
        \left(p^2 - M^2_{11}\right)
        \left(p^2 - M^2_{22}\right)
        - \left(2 p_0 K_{12}\right)^2
    \right] = 0,
\end{align*}
This equation has the solutions
\begin{align}
    E_0^2 &= q^2 + M_{33}^2, \\
    E_\pm^2
    % & = q^2 + \frac{1}{2} 
    % \left[
    %     M_{11}^2 + M_{22}^2 + (2K_{12})^2 
    % \right]
    % \pm \frac{1}{2}
    % \sqrt{
    %     \left[
    %     4q^2 + 2\left(M_{11}^2 + M_{22}^2\right)
    %     + \left(2K_{12}\right)^2
    %     \right]
    %     \left(2K_{12}\right)^2
    %     + \left(M_{11}^2 - M_{22}^2\right)^2
    % }. \\
    & = q^2 + \frac{1}{2} 
    \left[
        M_{11}^2 + M_{22}^2 + (2K_{12})^2 
    \right]
    \pm \frac{1}{2}
    \sqrt{
        4q^2\left(2K_{12}\right)^2 +
        \left(
            M_{11}^2 + M_{22}^2 + 2K_{12}^2
        \right)^2
        - 4 M_{11}^2 M_{22}^2
    }.
\end{align}
where $K_{12} = \mu_I \cos(\alpha)$. 
This gives the effective masses
\begin{align}
    m_0^2 &= M_{33}^2, \\
    m_\pm^2
    & = q\frac{1}{2} 
    \left[
        M_{11}^2 + M_{22}^2 + (2K_{12})^2 
    \right]
    \pm \frac{1}{2}
    \sqrt{
        \left(
            M_{11}^2 + M_{22}^2 + 2K_{12}^2
        \right)^2
        - 4 M_{11}^2 M_{22}^2
    }.
\end{align}
The propagator may then be obtained as described in \autoref{Conventions and notation}, (Sjekk fortegn off-diag.)
\begin{align}
    \notag
    D & = - i A^{-1} = \frac{i}{\det(A)}
    \begin{pmatrix}
        A_{22} A_{33} & -A_{12}A_{33} & 0 \\
        A_{12}A_{33} & A_{11}A_{33} & 0 \\
        0 & 0 & A_{11}A_{22} + A_{12}^2
    \end{pmatrix} \\
    & = i
    \begin{pmatrix}
        \frac{
            p^2 - M_{22}^2
        }
        {
            (p_0^2 - E_+^2)(p_-^2 - E_-^2)
        } 
        & \frac{
            2ip_0K_{12}
        }
        {
            (p_0^2 - E_+^2)(p_-^2 - E_-^2)
        } & 0 \\
        \frac{
            - 2ip_0K_{12}
        }
        {
            (p_0^2 - E_+^2)(p_-^2 - E_-^2)
        }
        & \frac{
            p^2 - M_{11}^2
        }
        {
            (p_0^2 - E_+^2)(p_-^2 - E_-^2)
        } & 0 \\
        0 & 0 & 
        \frac{1}{p_0^2 - E_0^2}
    \end{pmatrix}.
\end{align}



\printbibliography

\clearpage
\begin{appendices}
    Throughout this text, \emph{natural units} are used.
These units are defined so that
\begin{equation}
    \hbar = c = k_B = 1,
\end{equation}
where $\hbar$ is the Planck reduced constant, $k_B$ is the Boltzmann constant, and $c$ is the speed of light.
These constants will therefore be dropped from all expressions.
They can be reintroduced using dimensional analysis.
In natural units, \emph{mass dimension} is the only engineering dimension.
Dimensionfull results are given in units of electronvolt (eV), or pion-masses, 
\begin{equation}
    m_\pi = 131 \, \text{MeV}.
\end{equation}

The Minkowski metric convention used is the ``mostly minus'',
\begin{equation}
    g_{\mu \nu} = \mathrm{diag}(1, -1, -1, -1).
\end{equation}
The Fourier transform used in this text is defined by
\begin{align*}
    \F{f(x)}(p) = \tilde f(p) = \int \dd x\, e^{i p x}f(x), \quad 
    \FInv{\tilde f(p)}(x) = f(x) = \int \frac{\dd p}{2 \pi}\, e^{- i p x} \tilde f(p).
\end{align*}
We employ the \emph{Einstiein summation convention}, in which pairwise matching indices are summed.
That is,
\begin{equation}
    a_i b_i = \sum_i a_i b_i = a_1 b_1 + \dots.
\end{equation}
For Minkowski-space indices, $\mu$, $\nu$, $\rho$ and $\sigma$, the metric raises and lower indices, and summation should always be over one raised and one lowered index,
\begin{equation}
    a_\mu b^\mu = g_{\mu\nu} a^\mu b^\nu 
    = a^0 b^0 - a^1 b^1 - \dots.
\end{equation}

    \section{Covariant derivative}

In \chpt at finite isospin chemical potential $\mu_I$, the covariant derivative acts on functions $A(x): \Em_4 \rightarrow \lieg{SU}{2}$, where $\Em_4$ is the space-time manifold. It is defined as 
\begin{equation}
    \nabla_\mu A(x) = \partial_\mu A(x) - i \com{v_\mu}{A(x)}, 
    \quad v_\mu = \frac{1}{2} \mu_I \delta_\mu^0 \tau_3.
\end{equation}
The covariant derivative obeys the product rule, as
\begin{equation*}
    \nabla_\mu (A B) 
    = (\partial_\mu A) B + A (\partial_\mu B) - i \com{v_\mu}{AB}
    = (\partial_\mu A - i \com{v_\mu}{A})B + A(\partial_\mu B- i \com{v_\mu}{B}) 
    = (\nabla_\mu A)B + A (\nabla_\mu B).
\end{equation*}
Decomposing a 2-by-2 matrix $M$, as described in \autoref{section:algebra bases}, shows that the trace of the commutator of $\tau_b$ and $M$ is zero:
\begin{equation*}
    \Tr{\com{\tau_a}{M}]} = M_b\Tr{ \com{\tau_a}{\tau_b}} = 0.
\end{equation*}
Together with the fact that $\Tr{\partial_\mu A} = \partial_\mu \Tr{A}$, this gives the product rule for invariant traces:
\begin{equation*}
    \Tr{A \nabla_\mu B} = \partial_\mu \Tr{AB} - \Tr{(\nabla_\mu A) B}.
\end{equation*}
This allows for the use of the divergence theorem when doing partial integration.
Let $\Tr{K^\mu}$ be a space-time vector, and $\Tr{A}$ scalar. 
Let $\Omega$ be the domain of integration, with coordinates $x$ and $\partial \Omega$ its boundary, with coordinates $y$. Then, 
\begin{align*}
    \int_\Omega \dd x \, \Tr{A \nabla_\mu K^\mu} = \int_{\partial\Omega} \dd y\, n_\mu \Tr{A K^\mu} - \int_\Omega \dd x \, \Tr{(\nabla_\mu A) K^\mu},
\end{align*}
where $n_\mu$ is the normal vector of $\partial \Omega$~\cite{Carroll:space-time}.
This makes it possible to do partial integration and discard surface terms in the \chpt Lagrangian, given the assumption of no variation on the boundary.

    \section{Integrals}
\subsection{Gaussian integrals}
A useful integral is the Gaussian integral,
\begin{equation}
    \int_\R \dd x \, \exp(- \frac{1}{2} a x^2) = \sqrt{\frac{2 \pi}{a}},
\end{equation}
for $a \in \R$. The imaginary version,
\begin{equation}
    \int_R \dd x \, \exp(i \frac{1}{2} a x^2 ) 
\end{equation}
does not converge. However, if we change the contour of integration slightly, by rotating it clockwise to $C = \R(1 + i\epsilon)$,
\begin{figure}
    \begin{subfigure}{0.4\textwidth}
        \begin{tikzpicture}
            \draw (-2, 0) -- (2, 0) node[right] {$\mathrm{Re}(x)$};
            \draw (0, -2) -- (0, 2) node[above] {$\mathrm{Im}(x)$};
            \draw[->, thick] (-1.8, 0) -- (1.8, 0);
        \end{tikzpicture}    
    \end{subfigure}
    \begin{subfigure}{0.18\textwidth}
        \begin{tikzpicture}
            \draw[->] (-1, 0) -- (1, 0);
        \end{tikzpicture}
    \end{subfigure}
    \begin{subfigure}{0.4\textwidth}
        \begin{tikzpicture}
            \draw (-2, 0) -- (2, 0) node[right] {$\mathrm{Re}(x)$};
            \draw (0, -2) -- (0, 2) node[above] {$\mathrm{Im}(x)$};
            \draw[->, thick] (-1.8, -0.1) -- (1.8, 0.1);
        \end{tikzpicture}    
    \end{subfigure}
\end{figure}

    \chapter{Functional derivatives}
\label{section:Functional derivative}

Functional derivatives generalize the notion of gradients and directional derivatives.
A function $f(x)$ has a gradient
\begin{equation}
    \dd f_{x_0} = \pdv{f}{x_i} \Big|_{x_0} \dd x_i.
\end{equation}
The derivative in a particular direction $v = v^i \partial_i$ is 
\begin{equation}
    \dv{\epsilon} f(x_i + \epsilon v_i) \Big|_{\epsilon = 0} 
    = f(x) + \dd f_x (v) = f(x) + \pdv{f}{x^i}v_i.
\end{equation}
This is generalized to functionals through the definition of the functional derivative and the variation of a functional.
Let $F[f]$ be a functional, i.e., a machine that takes in a function and returns a number or a function.
The obvious example in our case is the action, which takes in one or more field configurations, and returns a single real number.
We will assume here that the functions have the domain $\Omega$, with coordinates $x$.
The functional derivative is defined as
\begin{equation}
    \delta F[f]
    =
    \dv{\epsilon} F[f + \epsilon \eta] \Big|_{\epsilon = 0}
    = \int_\Omega \dd x \, \frac{\delta F[f]}{\delta f(x)} \eta(x).
\end{equation}
$\eta(x)$ is here an arbitrary function, but we will make the assumption that it as well as all its derivatives are zero at the boundary of its domain $\Omega$, i.e., $\eta(\partial \Omega) = 0$.
This allows us to discard surface terms stemming from partial integration, which we will use frequently.
We may use the definition to derive one of the fundamental relations of functional derivation.
Take the functional $F[f] = f(x)$. 
Then,
\begin{equation}
    \label{Functional derivative delta identity}
    \delta F[f] = \dv{\epsilon} [f(x) + \epsilon \eta(x)] \Big|_{\epsilon = 0}
    = \eta(x) = \int \dd y \, \delta(x - y) \eta(y)
\end{equation}
This leads to the identity
\begin{equation}
    \frac{\delta f(x)}{\delta f(y)} = \delta(x - y),
\end{equation}
for any function $f$.
Higher functional derivatives are defined similarly, by applying functional variation repeatedly
\begin{equation}
    \delta^n F[f] = \dv{\epsilon} \delta^{n-1}F[f + \epsilon \eta_n] \big|_{\epsilon=0}
    = \int \left(\prod_{i=1}^n \dd x_i\right)
    \frac{\delta^n F[f]}{ \delta f(x_n)\dots\delta f(x_1)} \left(\prod_{i=1}^n \eta_i(x_i)\right).
\end{equation}
If we can write the functional $F[f]$ in terms of a new variable, $g(y)$, then the chain rule for functional derivatives is
\begin{equation}
    \fdiff{F[f]}{f(x)} = \int \dd y \, \fdiff{F[f]}{g(y)} \fdiff{g(y)}{f(x)}.
\end{equation}

A functional may be expanded in a generalization of the Taylor series, 
\begin{equation}
    F[f_0 + f] = F[f_0] + \int_\Omega \dd x \, f(x) \frac{\delta F[f_0]}{\delta f(x)}
    + \frac{1}{2!}\int_\Omega \dd x \dd y \, f(x) f(y) \frac{\delta^2 F [f_0]}{\delta f(x) \delta f(y)}
    + \dots
\end{equation}
Her, the notation
\begin{equation}
    \fdiff{F[f_0]}{f(x)}
\end{equation}
indicate that the functional $F[f]$ is first differentiated with respect to $f$, the evaluated at $f = f_0$.
As an example, the Klein-Gorodn action
\begin{equation}
    S[\varphi] = - \frac{1}{2}\int_\Omega \dd x \, \varphi(x) (\partial^2 + m^2) \varphi(x).
\end{equation}
Using \autoref{Functional derivative delta identity} and partial integration,
\begin{align}
    \nonumber
    \funcdv{\varphi(x)} S[\varphi] 
    & = 
    - \frac{1}{2} \int_\Omega \dd y \, 
    [\delta(x - y)(\partial_y^2 + m^2)\varphi(y) + \varphi(y) (\partial_y^2 + m^2)\delta(x - y)] \\
    & = 
    - \int_\Omega \dd y \, 
    \delta(x - y)(\partial_y^2 + m^2)\varphi(y) 
    = (\partial_x^2 + m^2)\varphi(x)
\end{align}
The second derivative is
\begin{equation}
    \frac{\delta^2S[\varphi]}{\delta \varphi(x)\delta \varphi(y)}
    =
    \funcdv{\varphi(x)} (\partial_y^2 + m^2)\varphi(y)
    = 
    (\partial_y^2 + m^2) \delta(x - y).
\end{equation}


\subsection*{Gaussian integrals}
\label{section:gaussian integrals}

\begin{figure}[ht]
    \centering
    \begin{tikzpicture}
        \draw (-2, 0) -- (2, 0) node[right] {$\mathrm{Re}(x)$};
        \draw (0, -2) -- (0, 2) node[above] {$\mathrm{Im}(x)$};
        \draw[->, thick] (-1.75, 0.1) -- (1.8, 0.1);
        \draw[->, thick] (1.8, 0.15) arc (10:45:1.8);
        \draw[->, thick] ({1.8/sqrt(2)}, {1.8/sqrt(2)}) -- ({-1.8/sqrt(2)}, {-1.8/sqrt(2)});
        \draw[->, thick] ({-1.8/sqrt(2)}, {-1.8/sqrt(2)}) arc (225:180:1.8);
    \end{tikzpicture}
    \caption{Wick rotation}
    \label{Wick rotation}
\end{figure}


A useful integral is the Gaussian integral,
\begin{equation}
    \int_\R \dd x \, \exp(- \frac{1}{2} a x^2) = \sqrt{\frac{2 \pi}{a}},
\end{equation}
for $a \in \R$. The imaginary version,
\begin{equation}
    \int_R \dd x \, \exp(i \frac{1}{2} a x^2 )
\end{equation}
does not converge. However, if we change $a \rightarrow a + i\epsilon$, the integrand is exponentially suppressed.
\begin{equation}
    f(x) = \exp(i \frac{1}{2}a x^2) \rightarrow
    \exp(i\frac{1}{2}a x^2 - \frac{1}{2} \epsilon  x^2).
\end{equation}
As the integrand falls exponentially for $x\rightarrow \infty$ and contains no poles in the upper right nor lower left quarter of the complex plane, we may perform a wick rotation by closing the contour as shown in \autoref{Wick rotation}.
This gives the result
\begin{equation}
    \label{complex gauss 1D}
    \int_\R \dd x \, \exp(i \frac{1}{2}(a + i\epsilon) x^2) 
    = \int_{\sqrt{i}\R} \dd x \, \exp(i\frac{1}{2} ax^2)
    = \sqrt{i} \int_\R \dd y\, \exp(-\frac{1}{2} (-a) y^2) = \sqrt{\frac{2 \pi i}{(-a)}}
\end{equation}
where we have made the change of variable $y = (1+i)/\sqrt{2} x = \sqrt{i} x$.
In $n$ dimensions, the Gaussian integral formula generalizes to
\begin{equation}
    \int_{\R^n} \dd^n x \, \exp{-\frac{1}{2} x_n A_{nm} x_m } =\sqrt{\frac{(2 \pi)^n}{\det(A)}},
\end{equation}
where $A$ is a matrix with $n$ real, positive eigenvalues.
We may also generalize \autoref{complex gauss 1D},
\begin{align}
    \int_{\R^n} \dd^n x \, \exp{i\frac{1}{2} x_n( A_{nm} + i \epsilon \delta_{nm}) x_m } =\sqrt{\frac{(2 \pi i )^n}{\det(-A)}}.
\end{align}
The final generalization is to functional integrals,
\begin{align}
    \int \D \varphi \, \exp(- \frac{1}{2} \int \dd x \, \varphi(x) A \varphi(x) )
    = C (\det(A))^{-1/2},
    \int \D \varphi \, \exp(i\frac{1}{2} \int \dd x \, \varphi(x) A \varphi(x) )
    = C (\det(-A))^{-1/2}.
\end{align}
$C$ is here a divergent constant, but will either fall away as we are only looking at the logarithm of $I_\infty$ and are able to throw away additive constants, or ratios between quantities which are both multiplied by $C$.

The Gaussian integral can be used for the stationary phase approximation.
In one dimension, it is
\begin{equation}
    \int \dd x \, \exp{i \alpha f(x)} 
    \approx \sqrt{\frac{2 \pi }{f''(x_0)}}\exp{ f(x_0)}, 
    \, \alpha\rightarrow \infty,
\end{equation}
where the point $x_0$ is defined by $ f'(x_0) = 0$. 
The functional generalization of this is
\begin{equation}
    \int \D \varphi \exp{i S[\varphi]}
    \approx 
    C \det(- \frac{\delta^2 S[\varphi_0]}{\delta \varphi^2})
    \exp{i \alpha S[\varphi_0]  },
\end{equation}
Here, $S[\varphi]$ is a general functional of $\varphi$, we have used the Taylor expansion, and $\varphi_0$ obeys
\begin{equation}
    \funcdv{\varphi(x)}{S[\varphi_0]} = 0.
\end{equation}


\end{appendices}
    

\clearpage
\printnoidxglossaries

\end{document}