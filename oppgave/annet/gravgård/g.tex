\documentclass{article}

\usepackage[left=2.5cm, right=2.5cm, top=2cm, bottom=2cm]{geometry}

\usepackage{amsmath}
\usepackage{amsfonts}
\usepackage{amssymb}
\usepackage{physics}
\usepackage{slashed}
\usepackage{hyperref}
\usepackage[title]{appendix}
\usepackage[intlimits]{mathtools}
\usepackage[export]{adjustbox}
\usepackage{esvect}
\usepackage[capitalize]{cleveref}


\usepackage{pgfplots}
\pgfplotsset{compat=1.16}

\usepackage{tikz}
\usepackage[compat=1.1.0]{tikz-feynman}

\usepackage{caption}
\usepackage{subcaption}

\usepackage[ISO]{diffcoeff} % Get straight d's when differantiating

\usepackage{tocloft}    % Nice table of contents

\usepackage[section]{placeins} % Get floats right
\usepackage[super]{nth}


\setlength\cftparskip{1pt}
\setlength\cftbeforechapskip{0pt}

\setlength{\parindent}{0em}
\setlength{\parskip}{0.8em}

\def\equationautorefname~#1\null{Eq.~(#1)\null}

% Simple shortcuts
\newcommand{\Ell}{\mathcal{L}}      % Lagrangian L
\newcommand{\He}{\mathcal{H}}       % Hamiltonian H
\newcommand{\Ve}{\mathcal{V}}       % Potential V
\newcommand{\Em}{\mathcal{M}}       % Manifold M
\newcommand{\Ef}{\mathcal{F}}       % Fancy f
\newcommand{\R}{\mathbb{R}}         % Real numbers
\newcommand{\chpt}{$\chi$PT }       % Chiral pertubation theory
\newcommand{\SU}{\mathrm{SU}}       % SU(n)
\newcommand{\eps}{\varepsilon}      % nice epsilon
% \newcommand{\one}{\mathbb{1}}       % Identity
\newcommand{\hc}{\mathrm{h.c.}}     % Hermitian conjugate
\newcommand{\ex}[1]{\expectationvalue{#1}}
\newcommand{\D}{\mathcal{D}}

\newcommand{\one}{\text{\usefont{U}{bbold}{m}{n}1}}
\MakeRobust{\one}

% Big-O notation 
\newcommand{\Oh}[2][2]{\mathcal{O}\left(#2^{#1}\right)}

% (anti) commutator
\newcommand{\com}[2]{\left[#1, #2 \right]}
\newcommand{\acom}[2]{\left\{#1, #2 \right\}}

% Lie algebra
\newcommand{\liea}[2]{\mathfrak{#1}\left(#2\right)}
\newcommand{\lieg}[2]{\mathrm{#1}\left(#2\right)}

% Curly brackets
\newcommand{\curly}[1]{\left\{ #1 \right\}}

% Fourier Transform
\newcommand{\F}[1]{\mathcal{F}\curly{#1}}
\newcommand{\FInv}[1]{\mathcal{F}^{-1}\curly{#1}}

% operator in braket
\newcommand{\inner}[3]{\left\langle #1 {\left| #2 \right|} #3 \right\rangle}

\newcommand{\T}[1]{\textrm{T} \left\{ #1 \right\}}

\title{Gravgård}
% \author{Martin Johnsrud}
\vspace{-8ex}
\date{}

\begin{document}
    \maketitle
    Gravgården er hvor tekster går for å dø. Ting som er fjernet fra main, men som jeg ikke helt klarer å slette enda
    \section{path integral}
    By Wick's theorem, an $n$-point correlated is given by the sum of all Feynman diagrams with $n$ external vertices.
    The factor $Z[0]^{-1}$ divides out all \emph{vacuum bubbles}, that is diagrams without external vertices.
    We can show this by considering 

    Here we defined the \emph{generating functional for connected diagrams}, $W[J]$.
    The reason for the name will become apparent later. (HUSK Å REFFERE TILBAKE)
    The expectation value of some function of the field-configuration, $A = A[\varphi]$, in the precesense of the source $J$ is
    \begin{equation}
        \ex{A}_J = \frac{1}{Z[J]} A\left( -i  \fdv{J}\right) Z[J].
    \end{equation}
    (DEFINE FUNCTIONAL DERIVATIVE)
    The expectation value of the field defines a functional,
    \begin{equation}
        \label{calssical field functional}
        \varphi[J](x) = \ex{\varphi(x)}_J = 
        \frac{1}{Z[J]} \left( -i  \fdv{J}\right) Z[J]
        = \fdv{J(x)} W[J],
    \end{equation}
    and is sometimes called the \emph{classical field}.
    The notation $\Ef[f](x)$ means that $\Ef$ is a functional which takes in a function $f$, and returns the new function $(\Ef[f])(x)$.
    One example is the Lagrangian density, which takes in a field, and returns a function which has a value for each point in space-time.
    We can reverse this relationship, by defining the functional $J[\varphi](x)$ as \emph{the current which causes the classical field $\varphi$}.
    That is, if $\varphi[J_0](x) = \varphi_0(x)$ for some source $J_0$, then $J[\varphi_0] = J_0$

\section{free scalar}
    Comparing with the definitions of the thermal propagator in \autoref{Conventions and notation}, we can write the free energy compactly as
    \begin{equation}
        \label{free energy from propagator}
        \beta F = \frac{1}{2} \Tr{\ln[D_0^{-1}(K, K'))]} 
        = \frac{1}{2} \Tr{\ln[\beta^2 D_0^{-1}(K)]}.
    \end{equation}

\section{interacting scalar}
    Notice that the constant factor from the Jacobian due to the change of variable $\varphi \rightarrow \tilde \varphi$ does not affect the expectation value, as the same factor is in both the numerator and denominator.
    If the quantity $A$ is a function of the momentum-space fields, $A = A[\tilde \varphi(K)]$, then this expectation value takes the form
    \begin{equation}
        \ex{A}_0 = 
        \frac{
            \int_{\tilde S} \D \tilde \varphi(K) \, f[\tilde\varphi(K)] \, 
            \exp{- \frac{1}{2} \langle \tilde \varphi^*, D \tilde \varphi \rangle}
            }
        {
            \int_{\tilde S} \D \tilde \varphi(K) \,
            \exp{
                - \frac{1}{2} \langle \tilde \varphi^*, D \tilde \varphi \rangle
                }
        }.
    \end{equation}
    where, as before, 
    \begin{equation}
        \langle \tilde \varphi^*, D \tilde \varphi \rangle
        = 
        \int_\Omega 
                \tilde \dd K\, [\beta^2(\omega_n^2 + \omega_n^2)] |\tilde \varphi(K)|^2
    \end{equation}
    The exponential form of $Z[J]$ leads straight forwardly to Wick's theorem, which states that an expectation value of $2n$ fields is a sum of \emph{all possible, distinct} combination of $n$ propagators.
    To write this in a formal way, we define the functions $a$ and $b$, which define a way to pair up $2m$ elements.
    The domain of the functions are the integers between $1$ and $m$, the image a subset of the integers between $1$ and $2m$ of size $m$.
    A valid pairing is a set $\{(a(1), b(1)), \dots (a(m), b(m))\}$, where all elements $a(i)$ and $b(j)$ are different, such all integers up to and including $2m$ are featured.
    A pair is not directed, so $(a(i), b(i))$ is the same pair as $(b(i), a(i))$.
    Wick theorem states that,
    \begin{equation}
        \ex{\prod_{i=1}^{2m} \varphi(X_i)  }_0
        = \sum_{\{(a, b)\}} \ex{\varphi(X_{a(i)}) \varphi(X_{b(i)})},
    \end{equation}
    where the sum is over all tuples $(a, b)$ that define a valid and unique pairing.

    \begin{align}
        \feynmandiagram [inline=(a.base), small, horizontal=i1 to f2]
        {
        {i1} -- [fermion, edge label'=$K_1$] a[dot] 
        -- [anti fermion, edge label'=$K_3$] {f1},
        {i2} -- [fermion, edge label'=$K_2$] a -- [anti fermion, edge label'=$K_4$] {f2},
        };
        \feynmandiagram[horizontal= i to f]{
            i[particle=$K$] -- [fermion] f,
        };
    \end{align}
    The expression is the integrated over all \emph{internal} momenta.
    The factor $1/4!$ is removed as a general Feynman diagram represent all diagrams with the same form, but different pairing of the momenta.
    Some diagrams are more symmetric, such that an exchange of momenta still gives \emph{the same pairing}. 
    
\section{effective action}

In free theory, we may write
\begin{equation}
    W[J] = \frac{1}{2} \int \dd^4 x \dd^4y J(x) D_0(x - y) J(y),
\end{equation}
where $D_0$ is the free propagator.
We may reverse the relation \autoref{calssical field functional} to write the source in terms of the field,
\begin{equation}
    J = D_0^{-1} \varphi(x)
\end{equation}
This is the field equation for the free field with a source.
For the scalar Klein-Gordon field, $D_0^{-1} = \partial^2 + m^2$
Inserting these two relation into the definition of the effective action, and assuming we can do partial integration with $D_0^{-1}$, we get
\begin{equation}
    \Gamma[\varphi] = W[J] - \int \dd^4x J(x)\varphi(x)
    = 
    \int \dd x( 
        \frac{1}{2}\int \dd y (D_0^{-1} \varphi ) D_0 (D_0^{-1} \varphi ) 
        - (D_0^{-1} \varphi ) \varphi
        )
    = - \frac{1}{2} \int \dd^4 x \varphi(x) D_0^{-1} \varphi(x)
\end{equation}
This is the classical action.
Thus, the effective action $\Gamma$ and the classical action $S$ are the same to first order in perturbation theory.


Let $\varphi^*$ solve the quantum mechanical version of the equation of motion, i.e.
\begin{equation}
    \fdv{\Gamma[\varphi^*]}{\varphi} = 0.
\end{equation}
We can Taylor-expand the classical action around this point, by setting $\varphi(x) = \varphi^*(x) + \eta(x)$ for some function $\eta$.
The generating functional becomes
\begin{align}
    Z[J] 
    = \int \D (\varphi^* + \eta) \, 
    \exp{i S[\varphi^* + \eta] + i \int \dd^4 x J (\varphi^* + \eta) }
\end{align}
The functional version of a Taylor expansion is
\begin{equation}
    S[\varphi^* + \eta] = 
    S[\varphi^*]
    + \int \dd x \fdv{S[\varphi^*]}{\varphi(x)} \eta(x)
    + \frac{1}{2} \int \dd x \dd y\,  \frac{\delta^2 S[\varphi^*]}{\delta\varphi(x)\delta\varphi(y)} \eta(x) \eta(y)
    + \dots
\end{equation}
Inserting this into $Z[J]$, with $S_I$ to denote the derivatives of higher order than $2$, we get
\begin{align*}
    &Z[J] = \\ 
    &\int \D \eta  
    \exp{
        i \int \dd^4 x \left(  \Ell[\varphi^*] + J \varphi^*  \right)
        + i \int \dd x \left(  \fdv{S[\varphi^*]}{\varphi(x)} + J(x) \right) \eta(x)
        + i \frac{1}{2} \int \dd x \dd y\,  
        \frac{\delta^2 S[\varphi^*]}{\delta\varphi(x)\delta\varphi(y)} \eta(x) \eta(y) 
        + i S_I[\eta]
        }
\end{align*}
In the first term we used the definition of the classical action. This term is constant with respect to $\eta$, and may therefore be taken outside the path integral.
The next term is the classical equation of motion with a source, 
\begin{equation}
    \fdv{S[\varphi]}{\varphi(x)} = - J(x),
\end{equation} 
evaluated at $\varphi^*$.
\begin{equation}
    \fdv{S[\varphi^*]}{\varphi(x)} + J(x)
    = \fdv{\Gamma[\varphi^*]}{\varphi(x)} + J(x)
    + \left(\fdv{S[\varphi^*]}{\varphi(x)} - \fdv{\Gamma[\varphi^*]}{\varphi(x)} \right)
    = \left(\fdv{S[\varphi^*]}{\varphi(x)} - \fdv{\Gamma[\varphi^*]}{\varphi(x)} \right)
    := \delta J
\end{equation}
The second to last term is a Gaussian integral, and may be evaluated as described in \autoref{section:gaussian integrals},
\begin{equation}
    \int \D \eta \, \exp(
        i \frac{1}{2} \int \dd x \dd y\,  
        \frac{\delta^2 S[\varphi^*]}{\varphi(x)\varphi(y)} \eta(x) \eta(y)
        )
        = C \det\left( \frac{\delta^2 S[\varphi^*]}{\delta \varphi^2} \right)^{-1/2}
\end{equation}
This leaves us with 
\begin{align}
    \label{generating functional}
    W[J] 
    & = -i \ln(Z) \\
    & = 
    \int\dd^4 x \, \left(\Ell[\varphi^*] + J \varphi^*\right)
    - \frac{1}{2} \Tr{\ln\left( - \frac{\delta^2 S[\varphi^*]}{\delta \varphi^2} \right)}
    + \int \D \eta \, \exp{i \int \dd^4 x \delta J(x) \eta  }
    + \int \D \eta \, e^{iS_I}
\end{align}
$\delta J$ is ultimately dependent on our choice of $J$ to define $\varphi$.
It contributes to the expectation value of $\eta$, through tadpole diagrams
\begin{align}
    \ex{\eta}_{j=0} = 
    \feynmandiagram [horizontal=a to b]{
    a --[inline=(b.base), fermion] b[blob]
    }; 
\end{align}
This can be removed by using the renormalization condition
\begin{equation}
    \feynmandiagram [horizontal=a to b]{
    a --[inline=(b.base), fermion] b[blob]
    };
    = 0.
\end{equation}


FULL LAGRANGIAN !!!!!
\begin{align*}
    & \Ell = \\
& 
    f^2   
    \left(
        2B_0m \cos{\alpha}
        + \frac{1}{2} \mu^2 \sin^2{\alpha}
    \right),
\\
& +
    f 
    (
        \mu_I^2\cos{\alpha}
        - 2B_0m
    ) \pi_1 \sin{\alpha}
    + f \mu_I \partial_0\pi_2 \sin{\alpha},
\\
& +
    \frac{1}{2} \partial_\mu\pi_a\partial^\mu\pi_a
    + \mu_I \cos{\alpha} \left( \pi_1 \partial_0\pi_2 - \pi_2\partial_0\pi_1 \right)
    - B_0m  \pi_a \pi_a \cos{\alpha}
    + \frac{1}{2} \mu_I ^2 \pi_a \pi_b k_{ab},
\\
& +
    \frac{\pi_a\pi_a \pi_1}{6f}
    (2B_0m \sin{\alpha}-2\mu_I{}^2 \sin{2\alpha})\\
    &
    -
    \frac{2 \mu_I}{3 f}
    \left[
        \pi_1(\pi_1 \partial_0\pi_2 - \pi_2\partial_0\pi_1)
        +
        \pi_3(\pi_3\partial_0\pi_2-\pi_2 \partial_0\pi_3)
    \right]
    \sin{\alpha},
\\
& +
\frac{1}{6f^2}
\curly{    
    \frac{1}{2} B_0m (\pi_a\pi_a)^2 \cos{\alpha}
    -
    \left[
        (\pi_a \pi_a) (\partial_\mu \pi_b \partial^\mu \pi_b )
        - (\pi_a \partial_\mu \pi_a)(\pi_b \partial^\mu \pi_b )
    \right]
}
\\
&
- \frac{\mu_I \pi_a\pi_a}{3f^2}
\left[
    \left(\pi_1\partial_0 \pi_2 - \pi_2 \partial_0 \pi_1\right)
    \cos{\alpha}
    + \frac{1}{2} \mu_I \pi_a \pi_b k_{ab}
\right].
 \\
 & + 
  \frac{l_1}{4} 
 \bigg(
     \frac{8 \mu_I^2}{f^2} 
     % term 1
     (\partial_\mu \pi_a \partial^\mu \pi_a + 2 \partial_\mu \pi_2 \partial^\mu \pi_2)
     \sin^2{\alpha}
     + 16 \mu_I^3 
     \left[
         % term 2
         \frac{\partial_0 \pi_2}{f}\sin^3{\alpha}
         + \frac{1}{f^2} 
         % term 3
         \left(
             3 \pi_1 \partial_0 \pi_2 - \pi_2 \partial_0 \pi_1
         \right)
             \cos{\alpha} \sin^2{\alpha}
     \right] \\
     & + 4 \mu_I^4 
     \left\{
         % term 4
         \sin^4{\alpha}
         + 2 \sin^2{\alpha}
         \left[
             % term 5
             \frac{\pi_1}{f}\sin{2\alpha}
             % term 6
             + \frac{\pi_a \pi_b}{f^2}        
             \left(k_{ab} + 2\delta_{a1}\delta_{a2}\cos^2{\alpha} \right)
         \right]
     \right\}
 \bigg)
 % 
 \\
 % \Tr{\nabla_\mu \Sigma (\nabla^\mu \Sigma)^\dagger}^2
 & + \frac{l_2}{4} 
 \bigg(
     \frac{4 \mu_I{}^2}{f^2}
     % term 7
     \left(
         \partial_0 \pi_a\partial_0 \pi_a 
         + \partial_0 \pi_2\partial_0 \pi_2 
         + \partial_\mu \pi_2\partial^\mu \pi_2
     \right) 
     \sin^2{\alpha}
     + 16 \mu_I^3 
     \left[
         % term 8
         \frac{\partial_0 \pi_2}{f} \sin^3{\alpha}
         % term 9
         + \frac{1}{f^2} \left(3 \pi_1 \partial_0 \pi_2 - \pi_2 \partial_0 \pi_1\right)
         \cos{\alpha} \sin^2{\alpha}
     \right] \\
     & + 4 \mu_I^4 
     \left\{
         % term 9
         \sin^4{\alpha}
         + 2 \sin^2{\alpha}
         \left[
             % term 10
             \frac{\pi_1}{f}\sin{2\alpha}
             % term 11
             + \frac{\pi_a \pi_b}{f^2}
             \left(k_{ab} + 2 \delta_{a1}\delta_{a2} \cos^2{\alpha} \right)
         \right]
     \right\}
 \bigg) \\
 %\Tr{\nabla_\mu \Sigma (\nabla_\nu \Sigma)^\dagger} 
 %\Tr{\nabla^\mu \Sigma (\nabla^\nu \Sigma)^\dagger} 
 & +
 \frac{l_3 + h_1 - h_3 }{16}
 \bigg(
     (8 B_0 \bar m)^2
     \left[
         % term 12
         \cos^2{\alpha}
         % term 13
         - \frac{\pi_1}{f} \sin{2\alpha}
         % term 14
         + \frac{1}{f^2}\left(\pi_1^2 \sin^2{\alpha} - \pi_a \pi_a \cos^2{\alpha}\right)
     \right]    
 \bigg)
 %\Tr{\chi \Sigma^\dagger + \Sigma \chi^\dagger}^2
 \\
 &
 + \frac{l_4}{4}
 \bigg(
     8 B_0 \bar m
     \Bigg\{
         % term 
         2 \frac{\partial_\mu \pi_a \partial^\mu \pi_a}{f^2} \cos{\alpha}
         + 4 \mu_I 
         \left[
             \frac{\partial_0 \pi_2}{2 f} \sin{2\alpha}
             + \frac{1}{f^2}
             \left(
                 \pi_1 \partial_0 \pi_2 \cos{2\alpha}
                 - \pi_2 \partial_0 \pi_1\cos^2{\alpha}
             \right)
         \right]\\\notag
         & \quad + \mu_I{}^2
         \left[
             2\cos{\alpha}\sin^2{\alpha} 
             - 2 \frac{\pi_1}{f} \sin{\alpha}
             \left(3\sin^2{\alpha} - 1\right)
             + \frac{1}{f^2}
             \left(                
                 \pi_1^2[2 - 9 \sin^2{\alpha}]
                 + \pi_2^2 [2 - 3 \sin^2{\alpha}]
                 - 3\pi_3^2\sin^2{\alpha}
             \right)
             \cos{\alpha}
         \right]
     \Bigg\}    
 \bigg) \\
 %\Tr{\nabla_\mu \Sigma (\nabla^\mu \Sigma)^\dagger} \Tr{\chi \Sigma^\dagger + \Sigma \chi^\dagger}
 & + \frac{h_1 - h_3 - l_4-l_7}{16} 
 \bigg(
     - 16 \left( \frac{2 \Delta m B_0 \pi_3}{f} \right)^2
 \bigg)
 %\Tr{\chi \Sigma^\dagger - \Sigma \chi^\dagger}^2
 + \frac{h_1 + h_2 - l_4}{4} \left( 8B_0^2\left( {\bar m}^2 + {\Delta m}^2\right)\right) \\
 & -
 \frac{h_1 - h_3 - l_4}{8}
 \bigg(
     \left(
         \cos{2\alpha} 
         - 2\frac{\pi_1}{f} \sin{2\alpha}
         - 2\frac{\pi_a \pi_a}{f^2} \cos^2{\alpha}
         + 2\frac{\pi_1^2}{f^2} \sin^2{\alpha}
     \right)
     + 16 B_0^2 \Delta m^2
     \left(
         1- 2\frac{ \pi_3^2}{f^2}
     \right)    
 \bigg) \\
&  + \mathcal{O}\left[ t^6 \left(\frac{\pi}{f}\right)^5 \right]
\end{align*}


% The next to leading order Lagrangian is therefore


% \begin{align*}
%     & \Ell_4 = \\
%     & \frac{l_1}{4} 
%     \bigg(
%         % term 1 pi^2
%         \frac{8 \mu_I^2}{f^2} 
%         (\partial_\mu \pi_a \partial^\mu \pi_a + 2 \partial_\mu \pi_2 \partial^\mu \pi_2)
%         \sin^2{\alpha}
%         + 16 \mu_I^3 
%         \left[
%             % term 2 pi
%             \frac{\partial_0 \pi_2}{f}\sin^3{\alpha}
%             + \frac{1}{f^2} 
%             % term 3 pi^2
%             \left(
%                 3 \pi_1 \partial_0 \pi_2 - \pi_2 \partial_0 \pi_1
%             \right)
%                 \cos{\alpha} \sin^2{\alpha}
%         \right] \\
%         & + 4 \mu_I^4 
%         \left\{
%             % term 4 1
%             \sin^4{\alpha}
%             + 2 \sin^2{\alpha}
%             \left[
%                 % term 5 pi
%                 \frac{\pi_1}{f}\sin{2\alpha}
%                 % term 6 pi^2
%                 + \frac{\pi_a \pi_b}{f^2}        
%                 \left(k_{ab} + 2\delta_{a1}\delta_{a2}\cos^2{\alpha} \right)
%             \right]
%         \right\}
%     \bigg)
%     % 
%     \\
%     % \Tr{\nabla_\mu \Sigma (\nabla^\mu \Sigma)^\dagger}^2
%     & + \frac{l_2}{4} 
%     \bigg(
%         \frac{4 \mu_I{}^2}{f^2}
%         % term 7 pi^2
%         \left(
%             \partial_0 \pi_a\partial_0 \pi_a 
%             + \partial_0 \pi_2\partial_0 \pi_2 
%             + \partial_\mu \pi_2\partial^\mu \pi_2
%         \right) 
%         \sin^2{\alpha}
%         + 16 \mu_I^3 
%         \left[
%             % term 8 pi
%             \frac{\partial_0 \pi_2}{f} \sin^3{\alpha}
%             % term 9 pi^2
%             + \frac{1}{f^2} \left(3 \pi_1 \partial_0 \pi_2 - \pi_2 \partial_0 \pi_1\right)
%             \cos{\alpha} \sin^2{\alpha}
%         \right] \\
%         & + 4 \mu_I^4 
%         \left\{
%             % term 9 1
%             \sin^4{\alpha}
%             + 2 \sin^2{\alpha}
%             \left[
%                 % term 10 pi
%                 \frac{\pi_1}{f}\sin{2\alpha}
%                 % term 11 pi^2
%                 + \frac{\pi_a \pi_b}{f^2}
%                 \left(k_{ab} + 2 \delta_{a1}\delta_{a2} \cos^2{\alpha} \right)
%             \right]
%         \right\}
%     \bigg) \\
%     %\Tr{\nabla_\mu \Sigma (\nabla_\nu \Sigma)^\dagger} 
%     %\Tr{\nabla^\mu \Sigma (\nabla^\nu \Sigma)^\dagger} 
%     & +
%     \frac{l_3 + h_1 - h_3 }{16}
%     \bigg(
%         (8 B_0 \bar m)^2
%         \left[
%             % term 12 1
%             \cos^2{\alpha}
%             % term 13 pi
%             - \frac{\pi_1}{f} \sin{2\alpha}
%             % term 14 pi^2
%             + \frac{1}{f^2}\left(\pi_1^2 \sin^2{\alpha} - \pi_a \pi_a \cos^2{\alpha}\right)
%         \right]    
%     \bigg)
%     %\Tr{\chi \Sigma^\dagger + \Sigma \chi^\dagger}^2
%     \\
%     &
%     + \frac{l_4}{8}
%     \bigg(
%         8 B_0 \bar m
%         \Bigg\{
%             % term 15 pi^2
%             2 \frac{\partial_\mu \pi_a \partial^\mu \pi_a}{f^2} \cos{\alpha}
%             + 4 \mu_I 
%             \left[
%                 % term 16 pi
%                 \frac{\partial_0 \pi_2}{2 f} \sin{2\alpha}
%                 % term 17 pi^2
%                 + \frac{1}{f^2}
%                 \left(
%                     \pi_1 \partial_0 \pi_2 \cos{2\alpha}
%                     - \pi_2 \partial_0 \pi_1\cos^2{\alpha}
%                 \right)
%             \right]\\
%             & + \mu_I{}^2
%             \left[
%                 %term 18 1
%                 2\cos{\alpha}\sin^2{\alpha} 
%                 % term 19 pi
%                 - 2 \frac{\pi_1}{f} \sin{\alpha}
%                 \left(3\sin^2{\alpha} - 1\right)
%                 % term 20 pi^2
%                 + \frac{1}{f^2}
%                 \left(                
%                     \pi_1^2[2 - 9 \sin^2{\alpha}]
%                     + \pi_2^2 [2 - 3 \sin^2{\alpha}]
%                     - 3\pi_3^2\sin^2{\alpha}
%                 \right)
%                 \cos{\alpha}
%             \right]
%         \Bigg\}    
%     \bigg) \\
%     %\Tr{\nabla_\mu \Sigma (\nabla^\mu \Sigma)^\dagger} \Tr{\chi \Sigma^\dagger + \Sigma \chi^\dagger}
%     & 
%     % term 21 pi^2
%     + \frac{h_1 - h_3 - l_4-l_7}{16} 
%     \bigg(
%         - 16 \left( \frac{2 \Delta m B_0 \pi_3}{f} \right)^2
%     \bigg)
%     %\Tr{\chi \Sigma^\dagger - \Sigma \chi^\dagger}^2
%     % term 22 1
%     + \frac{h_1 + h_3 - l_4}{4} \left( 8B_0^2\left( {\bar m}^2 + {\Delta m}^2\right)\right) \\
%     & -
%     \frac{h_1 - h_3 - l_4}{8}
%     \bigg(
%         16 B_0^2 \bar m^2
%         \left(
%             % term 23 1
%             \cos{2\alpha} 
%             % term 24 pi
%             - 2\frac{\pi_1}{f} \sin{2\alpha}
%             % term 25 pi^2
%             - 2\frac{\pi_a \pi_a}{f^2} \cos^2{\alpha}
%             % term 26 pi^2
%             + 2\frac{\pi_1^2}{f^2} \sin^2{\alpha}
%         \right)
%         + 16 B_0^2 \Delta m^2
%         \left(
%             % term 27 1
%             1
%             % term 28 pi^2
%             - 2\frac{ \pi_3^2}{f^2}
%         \right)    
%     \bigg)
%     %\Tr{\left(\chi \Sigma^\dagger\right)^2 + \left( \Sigma \chi^\dagger\right)^2}
% \end{align*}



The different terms of the NLO Lagrangian is
\begin{align*}
    \Ell_4^{(0)} & =
    %
    %
    % zeroth order
    %
    %
    % term 4 1
    \frac{l_1}{4} 4 \mu_I^4 \sin^4{\alpha}
    % term 9 1
    + \frac{l_2}{4}  4 \mu_I^4 \sin^4{\alpha}
    % term 12 1
    + \frac{l_3 + h_1 - h_3}{16} (8 B_0 \bar m)^2 \cos^2{\alpha}
    %term 18 1
    + \frac{l_4}{8} 8 B_0 \bar m \mu_I{}^2 2\cos{\alpha}\sin^2{\alpha} \\
    & 
    % term 22 1
    + \frac{h_1 + h_3 - l_4}{4} \left( 8B_0^2\left( {\bar m}^2 + {\Delta m}^2\right)\right)
    % term 23 1 +  term 27 1
    - \frac{h_1 - h_3 - l_4}{8}
    \bigg(
        16 B_0^2 \bar m^2 \cos{2\alpha} 
        + 16 B_0^2 \Delta m^2
    \bigg) \\
    & = 
    (l_1 + l_2)\mu_I^4 \sin^4{\alpha}
    + l_3 (2 B_0 \bar m)^2 \cos^2{\alpha}
    + l_4
    [
        2 B_0 \bar m \mu_I{}^2 \cos{\alpha}\sin^2{\alpha}
        - 2 B_0^2
        (\bar m^2 (1 - \cos 2 \alpha )) 
    ]
    \\
    & 
    + h_1
        [
            2 B_0^2 \bar m^2(1 - \cos 2 \alpha)
            + (2 B_0 \bar m)^2 \cos^2\alpha
        ]
    + h_3
        [
            2 B_0^2 \bar m^2(1 + \cos 2 \alpha) 
            - (2 B_0 \bar m)^2 \cos^2(\alpha)
            + 4 B_0 \Delta m^2
        ]
    \\
    & =
    (l_1 + l_2)\mu_I^4 \sin^4{\alpha}
    + l_3 (2 B_0 \bar m)^2 \cos^2{\alpha}
    + l_4 
    [
        2 B_0 \bar m \mu_I{}^2 \cos{\alpha}
        -
        (2 B_0\bar m) ^2
    ] \sin^2{\alpha}
    + h_1 (2B_0 \bar m)^2
    + h_3 (2B_0 \Delta m)^2
    \\
    & =
    (l_1 + l_2)\mu_I^4 \sin^4{\alpha}
    + (l_3 + l_4)(2 B_0 \bar m)^2 \cos^2{\alpha}
    + l_4 (2 B_0 \bar m ) \mu_I{}^2 \cos{\alpha} \sin^2{\alpha}
    - l_4 (2B_0 \bar m )^2
    + h_1 (2B_0 \bar m)^2
    + h_3 (2B_0 \Delta m)^2
    \\
    %
    %
    % first order
    %
    %
    \Ell_4^{(1)} & =
        % term 2 pi
    \frac{l_1}{4} 
    \bigg(
        16 \mu_I^3 
        \left[
            \frac{\partial_0 \pi_2}{f}\sin^3{\alpha}
        \right]
    % term 5 pi
    + 4 \mu_I^4 
        \left\{
            2 \sin^2{\alpha}
            \left[
                \frac{\pi_1}{f}\sin{2\alpha}
            \right]
        \right\}
    \bigg)
    % term 8 pi
    + \frac{l_2}{4} 
    \bigg(
        16 \mu_I^3 
        \left[
            \frac{\partial_0 \pi_2}{f} \sin^3{\alpha}
        \right] 
    % term 10 pi
    + 4 \mu_I^4 
        \left\{
            2 \sin^2{\alpha}
            \left[
                \frac{\pi_1}{f}\sin{2\alpha}
            \right]
        \right\}
    \bigg) \\
    % term 13 pi
    & +
    \frac{l_3 + h_1 - h_3 }{16}
    \bigg(
        (8 B_0 \bar m)^2
        \left[
            - \frac{\pi_1}{f} \sin{2\alpha}
        \right]    
    \bigg)
    %%%%%%%%
    + \frac{l_4}{4}
    \bigg(
        8 B_0 \bar m
        \Bigg\{
            % term 16 pi
            4 \mu_I 
            \left[
                \frac{\partial_0 \pi_2}{2 f} \sin{2\alpha}
            \right]
            + \mu_I{}^2
            \left[
                % term 19 pi
                - 2 \frac{\pi_1}{f} \sin{\alpha}
                \left(3\sin^2{\alpha} - 1\right)
            \right]
        \Bigg\}
    \bigg) 
    \\
    & -
    \frac{h_1 - h_3 - l_4}{8}
    \bigg(
        16 B_0^2 \bar m^2
        \left(
            % term 24 pi
            - 2\frac{\pi_1}{f} \sin{2\alpha}
        \right)
    \bigg) \\
    & = 
    % term 2 pi
    (l_1 + l_2)
    \bigg(
        4 \mu_I^3 
        \frac{\partial_0 \pi_2}{f}\sin^3{\alpha}
    % term 5 pi
    + \mu_I^4 
    2 \sin^2{\alpha} \frac{\pi_1}{f}\sin{2\alpha}
    \bigg)
    \\
    % term 13 pi
    & -
    (l_3 + h_1 - h_3 )
    (2 B_0 \bar m)^2
    \frac{\pi_1}{f} \sin{2\alpha}
    %%%%%%%%
    + l_4
    2 B_0 \bar m
    \Bigg\{
        % term 16 pi
        4 \mu_I 
        \left[
            \frac{\partial_0 \pi_2}{2 f} \sin{2\alpha}
        \right]
        - 2 \mu_I{}^2
        % term 19 pi
            \frac{\pi_1}{f} \sin{\alpha}
        \left(3\sin^2{\alpha} - 1\right)
    \Bigg\}    
    \\
    & +
    (h_1 - h_3 - l_4)(2B_0 \bar m)^2 
    % term 24 pi
    \frac{\pi_1}{f} \sin{2\alpha} 
    \\
    & = 
    % term 2 pi
    (l_1 + l_2)
    \frac{1}{f}
    \left(
        4 \mu_I^3 
        \partial_0 \pi_2 \sin^3{\alpha}
    % term 5 pi
    + \mu_I^4 
    2 \sin^2{\alpha} \pi_1 \sin{2\alpha}
    \right)
    % term 13 pi
    -
    (l_3 + l_4)
    \frac{1}{f}
    (2 B_0 \bar m)^2
    \pi_1 \sin{2\alpha}
    %%%%%%%% 
    \\
    & + l_4
    2 B_0 \bar m
    \frac{1}{f}
    \left[
        % term 16 pi
        2 \mu_I 
        \partial_0 \pi_2 \sin{2\alpha}
        - 2 \mu_I{}^2
        % term 19 pi
        \pi_1 \sin{\alpha}
        \left(3\sin^2{\alpha} - 1\right)
    \right]
    \\
    %
    %
    % second order
    %
    %
    \Ell_4^{(2)} & = 
    \frac{l_1}{4} 
    \bigg(
        % term 1 pi^2
        \frac{8 \mu_I^2}{f^2} 
        (\partial_\mu \pi_a \partial^\mu \pi_a + 2 \partial_\mu \pi_2 \partial^\mu \pi_2)
        \sin^2{\alpha}
        % term 3 pi^2
        + 16 \mu_I^3 
        \left[
            \left(
                3 \pi_1 \partial_0 \pi_2 - \pi_2 \partial_0 \pi_1
            \right)
                \cos{\alpha} \sin^2{\alpha}
        \right] \\
        % term 6 pi^2
        & + 4 \mu_I^4 
        \left\{
            2 \sin^2{\alpha}
            \left[
                \frac{\pi_a \pi_b}{f^2}        
                \left(k_{ab} + 2\delta_{a1}\delta_{a2}\cos^2{\alpha} \right)
            \right]
        \right\}
    \bigg)
    % 
    \\
    & + \frac{l_2}{4} 
    \bigg(
        \frac{4 \mu_I{}^2}{f^2}
        % term 7 pi^2
        \left(
            \partial_0 \pi_a\partial_0 \pi_a 
            + \partial_0 \pi_2\partial_0 \pi_2 
            + \partial_\mu \pi_2\partial^\mu \pi_2
        \right) 
        \sin^2{\alpha}
        + 
        % term 9 pi^2
        16 \mu_I^3 
        \left[
            \frac{1}{f^2} \left(3 \pi_1 \partial_0 \pi_2 - \pi_2 \partial_0 \pi_1\right)
            \cos{\alpha} \sin^2{\alpha}
        \right] \\
        % term 11 pi^2
        & + 4 \mu_I^4 
        \left\{
            2 \sin^2{\alpha}
            \left[
                + \frac{\pi_a \pi_b}{f^2}
                \left(k_{ab} + 2 \delta_{a1}\delta_{a2} \cos^2{\alpha} \right)
            \right]
        \right\}
    \bigg) \\
    & +
    \frac{l_3 + h_1 - h_3 }{16}
    \bigg(
        (8 B_0 \bar m)^2
        \left[
            % term 14 pi^2
            + \frac{1}{f^2}\left(\pi_1^2 \sin^2{\alpha} - \pi_a \pi_a \cos^2{\alpha}\right)
        \right]    
    \bigg)
    \\
    &
    + \frac{l_4}{4}
    \bigg(
        8 B_0 \bar m
        \Bigg\{
            % term 15 pi^2
            2 \frac{\partial_\mu \pi_a \partial^\mu \pi_a}{f^2} \cos{\alpha}
            + 4 \mu_I 
            \left[
                % term 17 pi^2
                + \frac{1}{f^2}
                \left(
                    \pi_1 \partial_0 \pi_2 \cos{2\alpha}
                    - \pi_2 \partial_0 \pi_1\cos^2{\alpha}
                \right)
            \right]\\
            & + \mu_I{}^2
            \left[
                % term 20 pi^2
                + \frac{1}{f^2}
                \left(                
                    \pi_1^2[2 - 9 \sin^2{\alpha}]
                    + \pi_2^2 [2 - 3 \sin^2{\alpha}]
                    - 3\pi_3^2\sin^2{\alpha}
                \right)
                \cos{\alpha}
            \right]
        \Bigg\}    
    \bigg) \\
    & 
    % term 21 pi^2
    + \frac{h_1 - h_3 - l_4-l_7}{16} 
    \bigg(
        - 16 \left( \frac{2 \Delta m B_0 \pi_3}{f} \right)^2
    \bigg) \\
    & -
    \frac{h_1 - h_3 - l_4}{8}
    \bigg(
        16 B_0^2 \bar m^2
        \left(
            % term 25 pi^2
            - 2\frac{\pi_a \pi_a}{f^2} \cos^2{\alpha}
            % term 26 pi^2
            + 2\frac{\pi_1^2}{f^2} \sin^2{\alpha}
        \right)
        + 16 B_0^2 \Delta m^2
        \left(
            % term 28 pi^2
            - 2\frac{ \pi_3^2}{f^2}
        \right)    
    \bigg)
    \\
    & = 
    %%%%%%%%%%%%
    %%%%%%%%%%%%
    %%%%%%%%%%%%
    l_1
    \frac{2 \mu_I^2}{f^2}
        (\partial_\mu \pi_a \partial^\mu \pi_a + 2 \partial_\mu \pi_2 \partial^\mu \pi_2)
        \sin^2{\alpha}
    + l_2 
    \frac{4 \mu_I{}^2}{f^2}
    % term 7 pi^2
    \left(
        \partial_0 \pi_a\partial_0 \pi_a 
        + \partial_0 \pi_2\partial_0 \pi_2
        + \partial_\mu \pi_2 \partial^\mu \pi_2 
    \right) 
    \sin^2{\alpha}
    \\
    & + 
    \frac{l_1 + l_2}{f^2}
    \left[>
        % term 3 pi^2
        4 \mu_I^3 
        \left( 3 \pi_1 \partial_0 \pi_2 - \pi_2 \partial_0 \pi_1 \right)
        \cos{\alpha} \sin^2{\alpha}
        % term 6 pi^2
        + 2 \mu_I^4  \sin^2{\alpha} \, \pi_a \pi_b 
            \left(k_{ab} + 2\delta_{a1}\delta_{a2}\cos^2{\alpha} \right)
    \right]
    % 
    \\
    & +
    \frac{l_3 + l_4}{f^2}
    (2 B_0 \bar m)^2
    % term 14 pi^2
    \left(\pi_1^2 \sin^2{\alpha} - \pi_a \pi_a \cos^2{\alpha}\right)
    % \\
    % &
    +  \frac{l_4}{f^2}
    2 B_0 \bar m
    \bigg[
    % term 15 pi^2
    2 \partial_\mu \pi_a \partial^\mu \pi_a \cos{\alpha}
    + 4 \mu_I 
    % term 17 pi^2
    \left(
        \pi_1 \partial_0 \pi_2 \cos{2\alpha}
        - \pi_2 \partial_0 \pi_1\cos^2{\alpha}
    \right)
    \\
    & 
    + \mu_I{}^2
    % term 20 pi^2
    \left(                
        \pi_1^2[2 - 9 \sin^2{\alpha}]
        + \pi_2^2 [2 - 3 \sin^2{\alpha}]
        - 3\pi_3^2\sin^2{\alpha}
    \right)
    \cos{\alpha}
    \bigg]
    % \\
    % & 
    % term 21 pi^2
    + \frac{l_7}{f^2}
    \left( 2 \Delta m B_0 \right)^2 \pi_3^2
\end{align*}

Calculating the free energy density:

\begin{align}
    \Ef &=
    - f^2 \left(m_\pi^2 \cos \alpha + \frac{1}{2}\mu_I^2 \sin^2 \alpha\right)
    + \Ef^{(1)}_{\mathrm{fin}, \pi_\pm}
    \\ 
    &
    - \frac{1}{2}\frac{1}{(4 \pi)^2}
    \bigg[
        \frac{3}{2}
        \left(
            \frac{3}{2}m_\pi^4 \cos^4 \alpha + m_\pi^2 \mu_I^2 \cos \alpha \sin^2\alpha + \mu_I^4 \sin^4 \alpha
        \right)
        + 
        \frac{1}{3}
        \left( 
            \bar l_1 + 2 \bar l_2 - 3
        \right) \mu_I^4 \sin^4 \alpha
        +
        \\ 
        &
        \frac{1}{2}
        \left(
            - \bar l_3 + 4 \bar l_4 - 3
        \right) m_\pi^4 \cos^2\alpha 
        + 2 \left(\bar l_4 - 1\right)
        m_\pi^2 \mu_I^2 \cos\alpha \sin^2 \alpha
    \bigg]
    \\
    &
    - \frac{1}{2} \frac{1}{(4 \pi)^2}
    \bigg[
        \frac{1}{\epsilon}
        \left(
            \frac{3}{2}m_\pi^4 \cos^4 \alpha + m_\pi^2 \mu_I^2 \cos \alpha \sin^2\alpha + \mu_I^4 \sin^4 \alpha
        \right)
        + 
        \left(\ln\frac{\mu^2}{m_3^2} + \frac{1}{2} \ln \frac{\mu^2}{\tilde m^2_2}\right)
        m_\pi^3 \cos^2 \alpha
        \\
        &
        + \ln \frac{\mu}{m_3^2}(\mu_I^4 \sin^4 \alpha + 2m_\pi^2 \mu_I^2 \cos \alpha \sin^2 \alpha)
        - 
        \left(\frac{1}{\epsilon} + \ln \frac{\mu^2}{M^2}\right) 
        \left(
            \mu_I^4\sin^4\alpha + \frac{3}{2} m_\pi^4 \cos^2 \alpha
            + 2 m_\pi^2 \mu_I^2 \cos\alpha \sin^2 \alpha
        \right)
    \bigg]
    \\
    &
    =
    - f^2 \left(m_\pi^2 \cos \alpha + \frac{1}{2}\mu_I^2 \sin^2 \alpha\right)
    + \Ef^{(1)}_{\mathrm{fin}, \pi_\pm}
    \\ 
    &
    - \frac{1}{2}\frac{1}{(4 \pi)^2}
    \bigg[
        \frac{1}{3}
        \left( 
            \bar l_1 + 2 \bar l_2 - 3 + \frac{3^2}{2}
        \right) \mu_I^4 \sin^4 \alpha
        +
        \frac{1}{2}
        \left(
            - \bar l_3 + 4 \bar l_4 - 3 + \frac{3^2}{2}
        \right) m_\pi^4 \cos^2\alpha 
        + 2 \left(\bar l_4 - 1 + \frac{3}{4}\right)
        m_\pi^2 \mu_I^2 \cos\alpha \sin^2 \alpha
    \\
    &
    \left(
        \mu_I^4\sin^4\alpha + m_\pi^4 \cos^2 \alpha
        + 2 m_\pi^2 \mu_I^2 \cos\alpha \sin^2 \alpha
    \right)
    \ln \frac{M}{m_3^2}
    + \frac{1}{2} m_\pi^2 \cos^2 \alpha \, \ln \frac{M^2}{\tilde m^2_2}
    \bigg]
    \\
    &
    =
    - f^2 \left(m_\pi^2 \cos \alpha + \frac{1}{2}\mu_I^2 \sin^2 \alpha\right)
    + \Ef^{(1)}_{\mathrm{fin}, \pi_\pm}
    - \frac{1}{2}\frac{1}{(4 \pi)^2}
    \bigg[
        \frac{1}{3}
        \left( 
            \bar l_1 + 2 \bar l_2 + \frac{3}{2} + 3 \ln \frac{M^2}{m_3^2}
        \right) \mu_I^4 \sin^4 \alpha
        \\ 
        &
        +
        \frac{1}{2}
        \left(
            - \bar l_3 + 4 \bar l_4 + \frac{3}{2} + 2\ln \frac{M^2}{m_3^2}
            + \ln \frac{M^2}{\tilde m^2_2}
        \right) m_\pi^4 \cos^2\alpha 
        + 2 \left(\bar l_4 - \frac{1}{4} + \frac{1}{2}\ln \frac{M^2}{m_3^2}\right)
        m_\pi^2 \mu_I^2 \cos\alpha \sin^2 \alpha
    \bigg].
\end{align}

\section{Forsøk på CCWZ}


Goldstone's theorem tells us that if a theory is invariant under the actions of a group $G$, while the ground state of that theory i.e. the symmetry is broken, then there will appear massless modes.
The low energy dynamics of the theory will be dominated by these modes, as they can be exited by arbitrarily small perturbation to the ground state.
If we want to treat the theory perturbatively, then, the original degrees of freedom might not be the best way to treat the theory.
For example, in QCD, the approximate chiral symmetries are apparent when describing the theory using the quark spinors, $\psi_{i, \alpha}$.
However, low energy QCD is notoriously non-perturbative, due to the strong coupling of the strong force.
Thus, we seek a way to be able to expand the theory in powers of the momenta of the Goldstone modes.

For concreteness, consider a theory consisting of $N$ real fields $\varphi_i(x)$.
The underlying fields of the theory might be complex, or Grassmann-number valued.
However, in the case of $N$ complex fields, they can be described as $2N$ real fields, and the transformation of real Grassmann-numbers, for our purposes, follows the same rules as real numbers.
We can then assume that $G$ is a real, compact Lie Group. 
The action of $g\in G$ can then be represented as a matrix $M_{ij}(g)$ acting on $\varphi_j$, and infinitesimal transformations has the form $M_{ij} = \delta_{ij} + i \epsilon n_{\alpha} t^\alpha_{ij}$.
Here, $t_{ij}^\alpha$ is the generators of $g$, and a basis for the Lie Group corresponding to $G$, and $n^{\alpha}$ is a normal vector.

Our goal is to remove the Goldstone-modes from the original degrees of freedom, $\varphi_i(x)$, by transforming them into a restricted set $\tilde \varphi_i(x)$.
The condition that $\tilde \varphi_i$ is 
\begin{equation}
    \tilde \varphi_i(x) t^\alpha_{ij} \varphi^*_{j} = 0,
\end{equation}
where $\varphi^*$ is the vacuum.
If $t^\alpha$ is not a broken generator, then this is trivially fulfilled, as $t^\alpha _{ij}\varphi_{j} = 0$.
Thus, this is one constraint per broken generator. 

Consider now the quantity
\begin{equation}
    V_{\varphi}(g) = \varphi_i g_{ij} \varphi^*_j.
\end{equation}
This is a continuous bounded function of $g$, as $G$ is compact.
This means that it has a maximum.
Given an arbitrary function $\varphi(x)$, there is a function $g(x)$ that maximizes $V$ for each $x$.
At this maximum, $V$ is stationary, and thus invariant under a small change in $g(x)$, 
$\delta g(x) = i \epsilon n_\alpha g(x) t^\alpha$.
Thus,
\begin{equation}
    \delta V_{\varphi(x)}(g(x)) = i n_\alpha \epsilon \varphi_i(x) g_{ij}(x) t^\alpha_{jk} \varphi_k = 0.
\end{equation}
As $n_\alpha$ is arbitrary, this gives us our transformed field,
\begin{equation}
    \tilde \varphi_i(x) = \varphi_j(x) g_{ij}(x)
\end{equation}

Let $H \subset G$ be the non-broken group left after the broken symmetry.
If this set is non-empty, then the choice of $g_{ij}(x)$ is highly non-unique.
This is becuse $h \in H$, $h_{ij} \varphi^*_j = \varphi^*_i$, by definition. 
Thus, $V_\varphi(gh) = V_\varphi(g)$, and if $g(x)$ maximizes $V_{\varphi(x)}$, so does $g(x)h$.
We therefore consider $g$ and $gh$ equivalent.
This is an equivalence relation, in the sens that it is reflective, symmetric and transiative.
This partitions $G$ into equivalence classes, where all the elements of the right coset,
\begin{equation}
    gH = \{g h| \forall h \in H  \}
\end{equation}
are equivalent to $g$.
The set of cosets, called the quotient group $G / H$, is a new group.
We only need one representative element for each coset, i.e. we have a bijective function from the equivalence classes of $g$ and $G / H$.

% The symmetry transformations $g \in G$ is then represented by a matrix $M(x)$ acting on $\varphi_i$, so a transformed field $\varphi'_i$ is related to the original field $\varphi_i$ through $\varphi_i(x) = M_{ij}(g)\varphi_{j}(x)$, where the summation is implied.
% An infinitesimal transformation matrix is represented by $M_{ij}(g) = \delta_{ij} + i \epsilon t_{ij}(g) $.
% $t_{ij}(g)$ is the generators of the transformation $M$.
% The generators of all elements $g \in G$ form a vector space, called a Lie Algebra.

We now insert $\varphi_i(x)$ into the Lagrangian.
The original theory was invariant under global transformations $g \in G$.
This means that any terms in the Lagrangian $f(\varphi)$ that does not depend on derivatives of $\varphi(x)$ only depend on $\tilde \varphi(x)$, as $f(\varphi(x)) = f(\tilde \varphi(x))$.
The derivative of $\varphi(x)$ is
\begin{equation}
    \partial_\mu \varphi(x) = [\partial_\mu \tilde \varphi_i(x) + \tilde \varphi(x) (\partial_\mu g^{-1}(x)) g(x) ]g^{-1}(x),
\end{equation}
As terms that depend on the derivatives of $\varphi(x)$ also are invariant under a global transformation, all terms that depend on $g$ will be proportional to at least one derivative of $g(x)$.
Thus, there will be no mass terms, and all terms can be ordered in terms of powers of the momenta of the Goldstone bosons.
The field $g(x)$ takes on values in $G / H$, and can therefore be parametrized as 
\begin{equation}
    g(x) = \exp{i \xi_a(x) t^a},
\end{equation}
where $t^a$ are the generators of $G / H$. 
These new fields $\xi_a$ are identified with the Goldstone bosons.
We see that there are one field per broken generator, as $|G/H| = |G| - |H|$.
These fields might in general transform non-linearly under G.
We may deduce heir new transformation rule by the fact that we might write any transformation $g' g(\xi(x))$ as an element in $G/H$, which we can write as $g(\xi'(x))$ for some $\xi$, and an element in $H$.
Thus, a transformation $\varphi(x) \rightarrow g' \varphi(x)$ induces a transfomration $\xi \rightarrow \xi'$ defined by 
\begin{equation}
    g' g(\xi) = g(\xi') h(\xi, g)
\end{equation}

\subsection*{More from CCWZ}
where $\Sigma$ is a function is a function from space-time, $\mathcal{M}_4$ to the symmetry group $G$,
\begin{equation}
    \Sigma : \mathcal{M}_4 \longrightarrow G.
\end{equation}
$G$ is a connected Lie group, which means we can connect it to the identity by a continuous map $\Sigma_t$, such that $\Sigma_0 = \mathrm{id}$, $\Sigma_1 = \Sigma$.
$G$ is a $n$ dimensional manifold, which can be parametrized by $n$ real coordinates, $\xi_a$.
Furthermore, close to the identity we have
\begin{equation}
    \Sigma_\epsilon \sim \id + i \epsilon \eta_\alpha T_\alpha, \quad \epsilon \rightarrow 0.
\end{equation}
$T_\alpha = \diff{\Sigma}{\xi_\alpha}|_0$ are the generators of the Lie group, and form a Lie algebra $\mathfrak{g}$, with the Lie bracket
\begin{equation}
    [T_\alpha, T_\beta] = iC_{\alpha\beta}^\gamma T_\gamma.
\end{equation}
$C_{ab}^c$ are called the structure constants of the Lie algebra.
The generators get their name as any part of a connected Lie group can be written as 
\begin{equation}
    g(\eta) = \exp{i\eta_\alpha T_\alpha}.
\end{equation}
For matrix groups, the lie bracket is the commutator, and the exponential is defined through the series expansion.
As $H$ form a subgroup of $G$, it has its own set of $m = \dim H$ generators, $ x_{i}$.
The remaining set of commutators, $t_{a}$ are the broken generators.
We can write the commutator as
\begin{align}
    [T_\alpha, T_\beta] = i C_{\alpha \beta }^{k} \hat x_k + i C_{\alpha \beta }^c t_c
\end{align}
As $H$ is a subgroup, its commutator must form a closed algebra, thus
\begin{align}
    [x_i, x_j] &= i C_{i j}^{k} x_k,\\
    [x_i, t_a] &= i C_{i a}^b t_b, \\
    [t_a, t_b] &= i C_{ab}^k x_k + i C_{ab}^c t_c.
\end{align}
The second line comes from the Jacobi-identity,(DERIVE?) which means that the structure constants $C_{\alpha \beta \gamma}$ are total antisymmetric.
As $C_{ija} = 0$, we also have that $C_{iaj} = 0$.


\section{Old dim-reg}

Returning to the temperature-independent part, we use dimensional regularization to see its singular behavior.
To that end, we define
\begin{equation}
    \label{def dimreg integral}
    \Phi_n(m, d, \alpha) 
    = \mu^{n - d}\int_{\tilde \Omega} \frac{\dd^d k}{(2 \pi)^d} (k^2 + m^2)^{-\alpha},
\end{equation}
so that $\Ef_0 = \Phi_3(m, 3, 1/2) / 2$.
The parameter $\mu$ has the dimensions of $k$, and is inserted to ensure that $\Phi_n$ does not change physical dimension for $d \neq n$.
Furthermore, as non-rational exponents are defined through the exponential functions, this parameter is needed to make the expression well-defined.
Dimensional regularization takes uses the formula for integration of spherically symmetric function in $d$-dimensions,
\begin{equation}
    \int_{\R^d} \dd^d x \, f(r) 
    = \frac{2 \pi^{d/2}}{\Gamma(d/2)} \int_\R \dd r \, r^{d-1}f(r),
\end{equation}
where $r = \sqrt{x_i x_i}$ is the radial distance, and $\Gamma$ is the Gamma function.
The factor in the front of the integral comes from the solid angle.
By extending this formula from integer-valued $d$ to real numbers, the function we defined becomes
\begin{equation}
    \Phi_n
    = \frac{2 \pi^{d/2} \mu^{n - d} }{\Gamma(d/2)} \int_\R \dd k \, 
    \frac{k^{d-1}}{(k^2 + m^2)^\alpha}
    = \frac{m^{n-2\alpha}}{(4 \pi)^{d / 2}\Gamma(d/2)} 
    \left(\frac{m}{\mu}\right)^{d-n} 
    2 \int_\R \dd z \, \frac{z^{d - 1}}{(1 + z)^\alpha}, 
\end{equation}
where we have made the change of variables $m z = k$.
We make one more change of variable to the integral,
\begin{equation}
    I = 2 \int_\R \dd z \, \frac{z^{d - 1}}{(1 + z)^\alpha}
\end{equation}
Let
\begin{equation}
    z^2 = \frac{1}{s} - 1 \implies 2 z \dd z = - \frac{\dd s}{s^2}
\end{equation}
Thus,
\begin{equation}
    I = \int_0^a \dd s \, s^{\alpha - d/2 - 1} (1 - z)^{d/2 - 1}.
\end{equation}
This is the beta function, which can be written in terms of Gamma functions~\cite{Peskin:IntroQFT}
\begin{equation}
    I = B\left(\alpha - \frac{d}{2}, \frac{d}{2}\right) 
    = \frac{\Gamma\left(\alpha - \frac{d}{2}\right) \Gamma\left(\frac{d}{2}\right)}{\Gamma(\alpha)}.
\end{equation}
Combining this gives
\begin{equation}
    \label{result dimreg}
    \Phi_n(m, d, \alpha) = \frac{m^{n - 2\alpha}}{(4 \pi)^{d / 2}}
    \frac{
        \Gamma \left(\alpha - \frac{d}{2} \right) 
    }
    {\Gamma(\alpha)}
    \left(\frac{m^2}{\mu^2}\right)^{\flatfrac{(d-n)}{2}} 
    .
\end{equation}
Inserting $n=3$, $d = 3 - 2\epsilon$ and $\alpha = -1/2$, we get
\begin{equation}
    \Phi_3(m, 3 - 2\epsilon, -1/2)
    =
    \frac{m^4}{(4 \pi)^{d/2}\Gamma(-1/2)} \Gamma(-2 + \epsilon) \left(\frac{m^2}{\mu^2}\right)^{-\epsilon}
    =
    - \frac{m^4}{(4 \pi)^{2}}
    \left(\frac{m^2}{4 \pi \mu^2}\right)^{- \epsilon}
    \frac{\Gamma(\epsilon)}{(\epsilon - 2)(\epsilon - 1)},
\end{equation}
where we have used the defining property $\Gamma(z + 1) = z\Gamma(z)$ and $\Gamma(1/2) = \sqrt \pi$.
Expanding around $\epsilon = 0$ gives
\begin{align}
    \left(\frac{m^2}{4 \pi \mu^2}\right)^{- \epsilon}
    &\sim 1 + \epsilon \ln\left(4 \pi \frac{\mu^2}{m^2}\right),\\
    \Gamma(\epsilon) 
    & \sim \frac{1}{\epsilon} - \gamma, \\
    \frac{1}{(\epsilon - 2)(\epsilon - 1)}
    &\sim \frac{1}{2}\left(1 + \frac{3}{2} \epsilon\right).
\end{align}
The singular behavior of the time-independent term is therefore
\begin{align}
    J_0 \sim
    - \frac{1}{4}\frac{m^4}{(4 \pi)^2}
    \left[
        \frac{1}{\epsilon} 
        - \gamma + \frac{3}{2}
        + \ln\left(4 \pi \frac{\mu^2}{m^2}\right)
    \right].
\end{align}

With this regulator, one can then add counter-terms to cancel the $\frac{1}{\epsilon}$-divergence.
The exact form of the counter-term is convention.
One may also cancel the finite contribution due to the regulator.
The minimal subtraction, or $\mathrm{MS}$, scheme, is to only subtract the divergent term, as the name suggest.
We will use the modified minimal subtraction, or $\overline{ \mathrm{MS}}$, scheme.
In this scheme, one also removes the $-\gamma$ and $\ln(4 \pi)$ term,
which can be interpreted as changing the parameter $\mu$
\begin{equation}
    -\gamma + \ln(4\pi \frac{\mu^2}{m^2}) \rightarrow \ln(\frac{\mu^2}{m^2}),
\end{equation}
which leads to the expression
\begin{equation}
    \label{free field regularized energy}
    J_0 \sim
    - \frac{1}{4}\frac{m^4}{(4 \pi)^2}
    \left[
        \frac{1}{\epsilon} 
        + \frac{3}{2}
        + \ln\left(\frac{\mu^2}{m^2}\right)
    \right].
\end{equation}

\begin{align}
    \tilde \Ef_1
    & = 
    \mu^{-2\epsilon} \frac{1}{2} \frac{1}{(4\pi)^2}
    \bigg[
        \frac{1}{2}
        \left(
            \frac{1}{\epsilon} + \frac{3}{2} +\ln \frac{\mu^2}{\bar m^2} - \frac{1}{2} \frac{(2\mu_I^2 - \bar m^2)}{\bar m^2}\alpha^2
        \right)[\bar m^4 + \bar m^2 (2\mu_I^2 - \bar m^2)\alpha^2]
        \\
        & +
        \left(
            \frac{1}{\epsilon} + \frac{3}{2} +\ln \frac{\mu_I^2}{\bar m^2} + \frac{1}{2}\frac{\bar m^2 + \mu_I^2}{\bar m^2}\alpha^2
        \right)[\bar m^4 - \bar m^2(m^2 + \mu_I^2) \alpha^2]
        \\
        &
        +
        \frac{3}{4}
        \left(
            \frac{4}{3}\bar l_4 - \frac{1}{3}\bar l_3 - 1 - \frac{1}{\epsilon} 
            - \ln\frac{\tilde \mu^2}{M^2}
        \right)\bar m^4
        -
        \left(
            \bar l_4 - 1 - \frac{1}{\epsilon} - \ln\frac{\tilde \mu^2}{M^2}
        \right) 
        \bar m^2\mu_I^2
    \bigg] \alpha^2
    \\
    &=
    \mathrm{const.}
    +
    \mu^{-2\epsilon}\frac{1}{2} \frac{1}{(4\pi)^2}
    \bigg[
        \left(
            -\frac{3}{2}\frac{1}{\epsilon} - \frac{3}{4}\frac{1}{\epsilon} + \frac{1}{\epsilon}
            + \frac{3}{2} + \ln \frac{\mu_I^2}{\bar m^2} 
        \right)
        \bar m^4
    \bigg]
\end{align}


\subsection*{Landau ting}

Due to symmetry of the system under $\alpha \rightarrow -\alpha$, the expansion of $\Ef$ should only contain even powers.
This can be certified to leading order by explicit calculation.
We therefore write,
\begin{equation}
    \Ef = \Ef(\alpha = 0) + a \alpha^2 - \frac{1}{2} b \alpha^4 + \frac{1}{3} c \alpha^6
    + \Oh[8]{\alpha}.
\end{equation}
We assume that near $\bar m = \mu_I$, we can write $a = -a_0(\mu_I - \bar m), \, b = -b0 ,½$ and $c = c_0$, where all the constants are positive.
The equation for $\alpha'$ is now
\begin{equation}
    2\alpha [a - \alpha^3 (b - c \alpha^3 )] = 0
\end{equation}
We still have the $\alpha' = 0$ for $\mu_I < \bar m$.
For $\mu_I> \bar m$, we get the solutions
\begin{equation}
    \alpha^3 = \frac{1}{2} \left( \frac{b}{c} \pm \sqrt{\left(\frac{b}{c} \right)^2- 4 \frac{a}{c}} \right)
    = \frac{b_0}{2c_0} \left(1 \pm \sqrt{1- 4 (\mu_I - \bar m) \frac{a_0c_0}{b_0^2}} \right).
\end{equation}
Taking the second derivative, 
\begin{equation}
    \Ef' = a - (3 b - 5c \alpha^2)\alpha^2,
\end{equation}

\subsection*{Gammel intro}
The $\lieg{SU}{2}_L \times \lieg{SU}{2}_R$ symmetry of QCD is spontaneously broken if the quark field has a non-zero ground state expectation value $\ex{\bar q q}$, leaving only a subgroup $H = \lieg{SU}{2}_V \subseteq G$ of symmetry transformations of the vacuum state.
The Goldstone manifold $G/H = \lieg{SU}{2}_A$ is a three-dimensional Lie group, and therefore results in three (pseudo) Goldstone bosons, the pions.
There exists an isomorphism from a subset $S \subseteq M_1$ of the set of all Goldstone-fields
\begin{equation*}
    M_1 = \curly{ \pi_a: \Em_4 \longrightarrow \R^3 | \pi_a \, \mathrm{smooth} }
\end{equation*}
close to the ground state, into fields taking values in the Goldstone manifold $G/H$. (BEVISE?)(HVA ER ISOMORFISME HER?).


\end{document}