\section{Leading order Lagrangian}

The leading order \chpt Lagrangian is made up of the terms \cref{leading order chi sigma term} and \cref{leading order term sigma}, and reads
\begin{equation}
    \label{chpt lagrangian}
    \Ell_2 = 
    \frac{1}{4} f^2 \Tr{\nabla_\mu \Sigma (\nabla^\mu \Sigma)^\dagger}
    + \frac{1}{4} f^2 \Tr{\chi^\dagger \Sigma + \Sigma^\dagger \chi}.
\end{equation}
In the ground state, we set the external scalar current $s = m$, where $m$ is the mass matrix \cref{Mass matrix}, so
\begin{equation}
    \chi = 2 B_0 m = \bar m^2 \one + \Delta m^2 \tau_3,
\end{equation}
where we have defined
\begin{equation}
    \bar m^2 = B_0(m_u + m_d), \quad \Delta m^2 = B_0 (m_u - m_d).
\end{equation}
In this section, we will expand this Lagrangian in $\pi/f$, which we will use to calculate the free energy density.
To get the series expansion of $\Sigma$ in powers of $\pi/f$, we start by using the fact that $\tau_a^2 = \one$ to write
\begin{equation}
    \label{A}
    A_\alpha 
    = \sum_n^\infty \frac{1}{n!} \left(\frac{i \alpha}{2} \tau_1 \right)^n 
    = \sum_n^\infty 
    \left[
        \frac{\one}{(2n)!} \left(\frac{i \alpha}{2}\right)^{(2n)} 
        + \frac{\tau_1}{(2n + 1)!} \left(\frac{i\alpha}{2}\right)^{(2n + 1)}
    \right] 
    = \one \cos{\frac{\alpha}{2}} + i \tau_1 \sin{\frac{\alpha}{2}}.
\end{equation}
The series expansion of $U$ is
\begin{align*}
    U = \exp(\frac{i \pi_a \tau_a}{2f}) = 
    1
    + \frac{i \pi_a \tau_a}{2f} 
    + \frac{1}{2}\left(\frac{i \pi_a \tau_a}{2f}\right)^2 
    + \frac{1}{6}\left(\frac{i \pi_a \tau_a}{2f}\right)^3 
    + \frac{1}{24}\left(\frac{i \pi_a \tau_a}{2f}\right)^4 
    + \Oh[5]{(\pi/f)},
\end{align*}
which we use to calculate the expansion of the inner part of $\Sigma$, as given in \cref{sigma},
\begin{align*}
    U\Sigma_0U & = 
    \left(
        1
        + \frac{i \pi_a \tau_a}{2f} 
        + \frac{1}{2}\left(\frac{i \pi_a \tau_a}{2f}\right)^2 
        + \frac{1}{6}\left(\frac{i \pi_a \tau_a}{2f}\right)^3 
        + \frac{1}{24}\left(\frac{i \pi_a \tau_a}{2f}\right)^4 
    \right)\\
    & \times
    \left(
        1
        + \frac{i \pi_a \tau_a}{2f} 
        + \frac{1}{2}\left(\frac{i \pi_a \tau_a}{2f}\right)^2 
        + \frac{1}{6}\left(\frac{i \pi_a \tau_a}{2f}\right)^3 
        + \frac{1}{24}\left(\frac{i \pi_a \tau_a}{2f}\right)^4 
    \right)
    + \Oh[5]{(\pi/f)}\\
    &=
    1 + \frac{i \pi_a \tau_a}{f}
    + 2 \left( \frac{i \pi_a \tau_a}{2f} \right)^2
    + \frac{4}{3} \left( \frac{i \pi_a \tau_a}{2f} \right)^3
    + \frac{2}{3} \left( \frac{i \pi_a \tau_a}{2f} \right)^4
    + \Oh[5]{(\pi/f)}.
\end{align*}
The symmetry of $\pi_a\pi_b$ means that
\begin{align*}
% Identitites
    & (\pi_a \tau_a)^2
    = 
    \pi_a \pi_b \frac{1}{2} \acom{\tau_a}{\tau_b} 
    =
    \pi_a \pi_a, \quad
    (\pi_a \tau_a)^3
    =
    \pi_a \pi_a \pi_b \tau_b,\quad
    (\pi_a \tau_a)^4
    =
    \pi_a \pi_a \pi_b \pi_b.
\end{align*}
This gives us the expression
\begin{align*}
% Final expression
    & U\Sigma_0U 
    =
    1
    + i \frac{\pi_a \tau_a}{f} 
    - \frac{\pi_a^2}{2f^2}
    - i \frac{\pi_a^2\pi_b \tau_b}{6f^3}
    + \frac{\pi_a^2\pi_b^2}{24f^4}
    + \Oh[5]{(\pi/f)}.
\end{align*}
We combine this result with \cref{A} to get an expression for $\Sigma$ up to $\Oh[5]{(\pi/f)}$
\begin{align*}
    \Sigma 
    & =
    \left(
        1
        + i \frac{\pi_a \tau_a}{f} 
        - \frac{\pi_a^2}{2f^2}
        - i \frac{\pi_a^2\pi_b \tau_b}{6f^3}
        + \frac{\pi_a^2\pi_b^2}{24f^4}    
    \right)
    \cos^2{\frac{\alpha}{2}} \\
    & -
    \left(
        1
        + i \frac{\pi_a}{f} \tau_1\tau_a\tau_1
        - \frac{\pi_a^2}{2f^2}
        - i \frac{\pi_a^2\pi_b}{6f^3} \tau_1\tau_b\tau_1
        + \frac{\pi_a^2\pi_b^2}{24f^4}
    \right)
    \sin^2{\frac{\alpha}{2}}\\
    & + i
    \left(
        2 \tau_1
        + i \frac{\pi_a}{f} \acom{\tau_1}{\tau_a}
        - 2\tau_1 \frac{\pi_a^2}{2f^2}
        - i \frac{\pi_a^2\pi_b}{6f^3} \acom{\tau_1}{\tau_b}
        + 2\tau_1 \frac{\pi_a^2\pi_b^2}{24f^4}
    \right)
    \sin{\frac{\alpha}{2}}\cos{\frac{\alpha}{2}}.
\end{align*}
Using trigonometric identities and the commutator,
\begin{align*}
    \cos^2{\frac{\alpha}{2}} - \sin^2{\frac{\alpha}{2}} = \cos{\alpha}, \quad 
    2 \cos{\frac{\alpha}{2}} \sin{\frac{\alpha}{2}} = \sin{\frac{\alpha}{2}}, \quad
    \tau_1 \tau_a \tau_1
    = -\tau_a + 2 \delta_{1a}\tau_1,
\end{align*}
the final expression of $\Sigma$ to $\Oh[5]{(\pi/f)}$ is
\begin{align}
    \Sigma =
     \left(
        1 
        - \frac{\pi_a^2}{2f^2}
        + \frac{\pi_a^2\pi_b^2}{24f^4}
    \right)
    (\cos{\alpha} + i \tau_1 \sin{\alpha})
    +
    \left(
        \frac{\pi_a}{f} 
        - \frac{\pi_b^2\pi_a}{6f^3} 
    \right)
    \left(
        i\tau_a - 2i \delta_{a1}\tau_1\sin^2{\frac{\alpha}{2}} - \delta_{a1} \sin{\alpha}
    \right).
    \label{expansion of sigma}
\end{align}

The kinetic term in the \chpt Lagrangian is
\begin{equation}
    \nabla_\mu \Sigma (\nabla^\mu \Sigma)^\dagger 
    = \partial_\mu \Sigma \partial^\mu \Sigma^\dagger 
    - i \left(\partial_\mu \Sigma \com{v^\mu}{\Sigma^\dagger} - \hc \right)
    - \com{v_\mu}{\Sigma}\com{v_\mu}{\Sigma^\dagger}.
    \label{kinetic term}
\end{equation}
Using \cref{expansion of sigma} we find the expansion of the constitutive parts of the kinetic term to be
\begin{align}
    \notag
    \partial_\mu \Sigma 
    = &
    \left[
        \left(
            \frac{-1}{f^2}
            + \frac{\pi_b^2}{6f^4}
        \right)
        (\pi_a \partial_\mu \pi_a)
        \cos{\alpha}
        - 
        \left(
            \frac{\partial_\mu \pi_1}{f} 
            - \frac{\pi_b^2 \partial_\mu\pi_1
            + 2 \pi_1 \pi_b \partial_\mu\pi_b}{6f^3} 
        \right)
        \sin{\alpha}
    \right]
    \\ \notag 
    - &
    \left[
        \left(
            \frac{-1}{f^2}
            + \frac{\pi_b^2}{6f^4}
        \right)
        (\pi_a \partial_\mu \pi_a)
        \sin{\alpha}
        - \left(
        \frac{\partial_\mu \pi_1}{f} 
        - \frac{\pi_b^2 \partial_\mu\pi_1
        + 2 \pi_1 \pi_b \partial_\mu\pi_b}{6f^3}
        \right)
        2 \sin^2{\frac{\alpha}{2}}
    \right]
    i \tau_1 \\ \label{Sigma derivative}
    +& 
    \left(
        \frac{\partial_\mu \pi_a}{f} 
        - \frac{\pi_b^2 \partial_\mu\pi_a 
        + 2 \pi_a \pi_b \partial_\mu\pi_b}{6f^3} 
    \right)
    i \tau_a,
\end{align}
and
\begin{align}
    % \notag
    \com{v_\mu}{\Sigma} 
    & =
    -\mu_I \delta^0_\mu
    \left\{
        \left[
        \left(
            1 
            - \frac{\pi_a^2}{2f^2}
            + \frac{\pi_a^2\pi_b^2}{24f^4}
        \right)
        \sin{\alpha}
        + 
        \left(
            \frac{\pi_1}{f} 
            - \frac{\pi_b^2\pi_1}{6f^3} 
        \right) \cos{\alpha}
        \right]
         \tau_2
        -
        \left(
            \frac{\pi_2}{f} 
            - \frac{\pi_b^2\pi_2}{6f^3} 
        \right)
        \tau_1
    \right\}.
    \label{sigma commutator}
\end{align}
Combining \cref{Sigma derivative} and \cref{sigma commutator} gives the following terms \footnote{The scripts used to aid the calculation of the Lagrangian is available at \url{https://github.com/martkjoh/prosjektopggave}}
\begin{align}
    % Term 1
    & \Tr{\partial_\mu \Sigma \partial^\mu \Sigma^\dagger}
    = \frac{2}{f^2} \partial_\mu \pi_a \partial^\mu \pi_a
    + \frac{2}{3f^4}
    \left[
        (\pi_a\partial_\mu \pi_a)(\pi_b\partial^\mu \pi_b)
        -        
        (\pi_a\partial_\mu \pi_b)(\pi_b\partial^\mu \pi_a)
    \right], \\
    % Term 2
    \nonumber
    -i  &\Tr{\partial^\mu\Sigma\com{v_\mu}{\Sigma^\dagger} - \hc}
    =
    4 \mu_I \frac{\partial_0\pi_2}{f}
    + 8 \mu_I \frac{\pi_3}{3f^3}\sin{\alpha}(
        \pi_2 \partial_0 \pi_3 - \pi_3 \partial_0 \pi_2
        ) \sin{\alpha}
    \\ & \quad \quad \quad \quad \quad \quad \quad \quad \quad \quad \quad
    +
    \left(
        \frac{4\mu_I}{f^2} \cos{\alpha}
        - \frac{8 \mu_I\pi_1}{3f^3} \sin{\alpha}
        - \frac{4 \mu_I \pi_a \pi_a} {3f^4}\cos{\alpha} 
    \right) 
    (\pi_1\partial_0 \pi_2 - \pi_2 \partial_0 \pi_1), \\
    % Term 3
    - & \Tr{\com{v_\mu}{\Sigma}\com{v^\mu}{\Sigma^\dagger}}
    = \mu_I{}^2
    \bigg[
        2 \sin^2{\alpha}
        +
        \left(
            \frac{2}{f} 
            - \frac{4\pi_a \pi_a}{3 f^3} 
        \right)
        \pi_1  \sin{2\alpha}
        + \left(
            \frac{2}{f^2}
            - \frac{2 \pi_a \pi_a}{3 f^4} 
        \right)
        \pi_a \pi_b k_{ab}
    \bigg], 
    \\
    % Mass Term
    & \Tr{\chi^\dagger \Sigma + \Sigma^\dagger\chi}
    = 
    \bar m^2 
    \left(
        4 \cos{\alpha} 
        - \frac{4 \pi_1}{f} \sin{\alpha} 
        - \frac{2 \pi_a \pi_a}{f^2} \cos{\alpha}
        + \frac{2 \pi_1 \pi_a \pi_a}{3 f^3} \sin{\alpha}
        + \frac{(\pi_a \pi_a)^2}{6 f^4}\cos{\alpha}
    \right), 
    \end{align}
where $k_{ab} =\delta_{a1} \delta_{b1} \cos{2\alpha}  + \delta_{a2}\delta_{b2}\cos^2{\alpha} - \delta_{a3}\delta_{b3} \sin^2{\alpha}$.
Notice that the mass term is independent of the difference in quark masses, $\Delta m$.
If we write the Lagrangian as show in \cref{chpt lagrangian} as $\Ell_2 = \Ell_2^{(0)} + \Ell_2^{(1)} + \Ell_2^{(2)} +...$, where $\Ell_2^{(n)}$ contains all terms of order $(\pi/f)^n$, then the result of the series expansion is
\begin{align}
%%%%%%%%%%%%%%%%%%
%% zeroth order %%
%%%%%%%%%%%%%%%%%%
\Ell_2^{(0)}
&  =
    f^2   
    \left(
        \bar m^2 \cos{\alpha}
        + \frac{1}{2} \mu^2 \sin^2{\alpha}
    \right),
    \label{L0}
\\
%%%%%%%%%%%%%%%%%%
%% first order %%
%%%%%%%%%%%%%%%%%%
\label{L1}
\Ell_2^{(1)}
& =
    f 
    (
        \mu_I^2\cos{\alpha}
        - \bar m^2
    ) \pi_1 \sin{\alpha}
    + f \mu_I \partial_0\pi_2 \sin{\alpha},
\\
%%%%%%%%%%%%%%%%%%
%% second order %%
%%%%%%%%%%%%%%%%%%
\Ell_2^{(2)}
& =
    \frac{1}{2} \partial_\mu\pi_a\partial^\mu\pi_a
    + \mu_I \cos{\alpha} \left( \pi_1 \partial_0\pi_2 - \pi_2\partial_0\pi_1 \right)
    - \frac{1}{2} \bar m^2 \pi_a \pi_a \cos{\alpha}
    + \frac{1}{2} \mu_I ^2 \pi_a \pi_b k_{ab},
\label{L2}
\\
%%%%%%%%%%%%%%%%%%
%% third order %%
%%%%%%%%%%%%%%%%%%
\notag
\Ell_2^{(3)}
& =
    \frac{\pi_a\pi_a \pi_1}{6f}
    (\bar m^2 \sin{\alpha}-2\mu_I{}^2 \sin{2\alpha})\\ \label{L3}
    &
    -
    \frac{2 \mu_I}{3 f}
    \left[
        \pi_1(\pi_1 \partial_0\pi_2 - \pi_2\partial_0\pi_1)
        +
        \pi_3(\pi_3\partial_0\pi_2-\pi_2 \partial_0\pi_3)
    \right]
    \sin{\alpha},
\\
%%%%%%%%%%%%%%%%%%
%% fourth order %%
%%%%%%%%%%%%%%%%%%
\notag
\Ell_2^{(4)}
& =
\frac{1}{6f^2}
\curly{    
    \frac{1}{4} \bar m^2 (\pi_a\pi_a)^2 \cos{\alpha}
    -
    \left[
        (\pi_a \pi_a) (\partial_\mu \pi_b \partial^\mu \pi_b )
        - (\pi_a \partial_\mu \pi_a)(\pi_b \partial^\mu \pi_b )
    \right]
}
\\
&
- \frac{\mu_I \pi_a\pi_a}{3f^2}
\left[
    \left(\pi_1\partial_0 \pi_2 - \pi_2 \partial_0 \pi_1\right)
    \cos{\alpha}
    + \frac{1}{2} \mu_I \pi_a \pi_b k_{ab}
\right].
\label{L4}
\end{align}
