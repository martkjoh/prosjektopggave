\section{QCD}

Quantum chromodynamics, or QCD, is the theory of how quarks interact using the strong force.
It has its name due to the fact that the charges of QCD are called red, green and blue, which of course is only an analogy to colors.
The two main quantities are the quark spinors, $q_f$
if we have $N_f$ free quarks, with $N_c$ colors, then the Lagrangian is given by (HUSK i!)
\begin{equation}
    \Ell[q] = \sum_{f=1}^{N_f} \sum_{c = 1}^{N_c} i \bar q_{fc} (\gamma^\mu \partial_\mu - m_f )q_{fc}
    = i  \bar q (\slashed{\partial } - m)q.
\end{equation}
In the last equality, we have hidden the indices to reduce clutter.
Each element $q_{fc}$ is a spinor, and $\gamma^\mu$ are the gamma matrices, as described in \autoref{Conventions and notation}.
Furthermore, $\bar q = q^\dagger \gamma^0$.
This Lagrangian is invariant under rations of the quarks in the color indices,
i.e. the transformation
\begin{equation}
    q_c \rightarrow q'_c = U_{cc'} q_{c'},
    \quad 
    \bar q_c = \bar U_{cc'}^\dagger q_c'
\end{equation}
where $U_{cc'}$ is an $N_c \times N_c$ unitary matrix.
The set of all $N_c\times N_c$ unitary matrices form the Lie group $U(N_c)$.
All unitary matrices $U$ can be written as $e^{i\theta} U'$, where $\theta$ is a real number, and $U'$ is a matrix with determinant, which means we can decompose the Lie group into $U(1)\times SU(N_c)_c$.
We recognize $U(1)$ as the gauge group of the electromagnetic field, which we will ignore for the time being.
$SU(N_c)$ is the group of all complex $N_c\times N_c$ matrices with determinant 1, and the subscript $c$ specifies that this is the set of color transformation, and not just some abstract group.

\subsection*{The Yang-Mills Lagrangian}
$SU(N_c)$ is the gauge group for the strong force.
Given an element $U \in SU(N_c)$, we can write
\begin{equation}
    U = \exp{i \chi_\alpha \lambda_\alpha}, \quad
    \chi_\alpha \lambda_\alpha \in \liea{su}{N_c}_c,
\end{equation}
where $\liea{su}{N_c}_c$ is the Lie algebra of $SU(N_c)_c$.
We derive the full, interacting Lagrangian of QCD by demanding that it remain invariant under a \emph{local} $SU(N_c)_c$ transformation, i.e.
\begin{equation}
    q \rightarrow \exp{i \chi_\alpha(x) \lambda_\alpha} q, \quad
    \bar q \rightarrow \exp{-i \chi_\alpha(x) \lambda_\alpha} q'.
\end{equation}
This is no problem for the mass term, however, we need to modify the kinetic derivative term.
For the Lagrangian to be gauge-invariant, this has to obey
\begin{equation}
    D_\mu q \rightarrow (D_\mu q)' = U D_\mu q.
\end{equation}
As $q' = Uq$, this implies that
\begin{equation}
    D'_\mu = U D_\mu U^\dagger.
\end{equation}
If we posit $D_\mu = \one \partial_\mu + A_\mu^\alpha \lambda_\alpha$, then
\begin{equation}
    D_\mu' 
    =U D_\mu U^\dagger 
    = U(U^\dagger \partial_\mu +  i\partial_\mu \chi_\alpha(x)\lambda_\alpha U^\dagger)
    + U A_\mu U^\dagger
    = \partial_\mu + U(A_\mu + i\partial_\mu \chi_\alpha(x))U^\dagger,
\end{equation}
where $A_\mu = A_\mu^\alpha \lambda_\alpha$.
This means that if we demand that $A_\mu$ transforms as
\begin{equation}
    A_\mu \rightarrow U(A_\mu - i \partial_\mu \chi) U^\dagger,
\end{equation}
then our new derivative operator, called the \emph{covariant derivative}, follows the desired transformation rule.
The second derivative operator,
\begin{equation}
    D_\mu D_\nu = [\partial_\mu \partial_\nu - i(\partial_\mu A_\nu + A_\mu\partial_\nu + A_\nu\partial_\mu) - A_\mu A_\nu],
\end{equation}
transforms in the same way as the first derivative.
We see that the ``operator-part'' of this derivative is symmetric in the space-time indices, which means that the commutator will just a tensor, and not an operator.
We define
\begin{equation}
    F_{\mu\nu} := i[D_\mu, D_\nu] = (\partial_\mu A_\nu - \partial_\mu A_\nu) - i[A_\mu, A_\nu]
    = (\partial_\mu A_\nu^\alpha - \partial_\nu A_\mu^\alpha + C_{\beta \gamma }^\alpha A_{\mu}^\beta A_{\nu}^\gamma ) \lambda_\alpha.
\end{equation}
This transforms as
\begin{equation}
    F_{\mu\nu} \rightarrow U F_{\mu \nu} U^\dagger.
\end{equation}

We now need to include terms governing the gauge field $A_\mu$ in the Lagrangian.
The tensor $F_{\mu\nu}$ allows construct all gauge invariant terms to dimension 4, which are
\begin{equation}
    F_{\mu \nu}^a F_a^{\mu \nu}, 
    \quad 
    \varepsilon^{\mu\nu\rho\sigma} F_{\mu \nu}^a F_{\rho \sigma}^a.
\end{equation}
Here, $\varepsilon$ is the Levi-Civita symbol.
This allows us to write down the most general gauge-invariant Lagrangian for a $SU(N_c)$ gauge theory, the Yang-Mills Lagrangian
\begin{equation}
    \Ell = i  \bar q (\slashed{D} - m)q 
    + \frac{1}{4} F_{\mu \nu}^a F_a^{\mu \nu}
    + c \varepsilon^{\mu\nu\rho\sigma} F_{\mu \nu}^a F_{\rho \sigma}^a.
\end{equation}

\subsection*{Chiral symmetry}

In addition to the color and flavor indices $s$ and $f$, the quarks also have spinor indices, $i$, which are what the $\gamma$-matrices act on.
We can define the projection operators,
\begin{equation}
    P_\pm = \frac{1}{2}(1 \pm \gamma^5),
\end{equation}
which obey $P_\pm^2 = P_\pm$, $P_+P_- = P_-P_+ = 0$ and $P_+ + P_- = 1$, as good projection operators should.
Furthermore, $P^\dagger_\pm = P_\pm$.
These project spinors down to their chiral components, called left- and right-handed spinors,
\begin{equation}
    P_+ q = q_R, \quad P_- q = q_L.
\end{equation}
From \autoref{Conventions and notation}, we have 
\begin{equation}
    \acom{\gamma^\mu}{\gamma^5} = 0,
\end{equation}
which means that 
\begin{equation}
    \bar q P_\pm = (P_{\mp}q)^\dagger \gamma^0.
\end{equation}
This means that we can write the quark part of the Lagrangian as
\begin{align*}
    \bar q (i\slashed D - m) q
    & = 
    \bar q (P_+ + P_-) (P_+ + P_-) (i\slashed D - m) q
    = (q P_-)\gamma^0 P_+ (i \slashed D - m) q + (q P_+)\gamma^0 P_- (i \slashed D - m) q \\
    & = \bar q_L (i\slashed D) q_L + \bar q_R (i\slashed D) q_R
    - \bar q_L m q_R - \bar q_R m q_L.
\end{align*}
We see that the mass-term mixes the two chiral components.
In the limit $m \rightarrow 0$, called the \emph{chiral limit}, we gain two new symmetries,
\begin{equation}
    q_R \rightarrow U_R q_R, \quad q_L \rightarrow U_L q_L,
\end{equation}
where $U_L$ and $U_R$ are unitary matrices which act on the flavor indices.
The total set of such transformations form the Lie group $U(N_f)_L \times U(N_f)_R$.


\begin{equation}
    \label{Mass matrix}
    M =
    \begin{pmatrix}
        m_u & 0 \\
        0 & m_d
    \end{pmatrix}.
\end{equation}
