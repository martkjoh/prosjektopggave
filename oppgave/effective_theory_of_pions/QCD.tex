\section{QCD}
\label{section:QCD}
This section is based on~\cite{Schwartz:QFT,Peskin:IntroQFT,Scherer2002IntroductionTC}

Quantum chromodynamics, or QCD, is the theory of how quarks interact via the strong force.
There are six flavors of quarks, labeled by the index $f$, called up, down, charm, strange, top and bottom.
The up, charm and top quark, collectively called up-type quarks, have electrical charge $+\frac{2}{3}$, while the down-type quarks---down, strange and bottom---have an electrical charge of $-\frac{1}{3}$.
Quarks have one additional degree of freedom called color and indexed by $c$, which takes on the value of red, green and blue for quarks, and anti-red, -green and -blue for anti-quarks.
The analogy to colors is meant to capture the empirical fact that quarks are only ever observed in \emph{colorless} configurations, that is either an equal amount of each color and its anti-color, or equal amounts of red, green and blue.
Quarks are spin-$\frac{1}{2}$ particles, so each quark $q_{fc}$ is a spinor.
Free quarks obey the Dirac-equation,
\begin{equation}
    \Ell[q, \bar q] = \sum_{f=1}^{N_f} \sum_{c = 1}^{N_c} \bar q_{fc} (i\gamma^\mu \partial_\mu - m_f )q_{fc}
    = \bar q (i\slashed{\partial } - m)q,
\end{equation}
where $N_f$ is the number of flavors, and $N_c$ the number of colors.
In the last equality, we have hidden the indices to reduce clutter.
The gamma matrices, $\gamma^\mu$ are described in \autoref{Conventions and notation}, and we have employed the Feynman slash notation $\gamma^\mu A_\mu = \slashed A$.
Furthermore, $\bar q = q^\dagger \gamma^0$.
This Lagrangian is invariant under rations of the quarks in the color indices, i.e. transformations
\begin{equation}
    q_c \rightarrow q'_c = U_{cc'} q_{c'},
    \quad 
    \bar q_c' = \bar U_{cc'}^\dagger q_c
\end{equation}
where $U_{cc'}$ is an $N_c \times N_c$ unitary matrix.
The set of all $N_c\times N_c$ unitary matrices form the Lie group $U(N_c) = U(1)\times SU(N_c)_c$.
$SU(N_c)_c$ is the group of all complex $N_c\times N_c$ matrices with determinant 1, and is the gauge group of the strong force.
A theory with an $SU(N)$ gauge group is called a Yang-Mills theory.

\subsection*{Yang-Mills theory}

Given an element $U \in SU(N_c)$, we can write
\begin{equation}
    U = \exp{i \chi_\alpha \lambda_\alpha}, \quad
    \chi_\alpha \lambda_\alpha \in \liea{su}{N_c},
\end{equation}
where $\liea{su}{N_c}$ is the Lie algebra of $SU(N_c)$.
We derive the Yang-Mills Lagrangian by including all normalizable and Gauge-invariant terms. 
A term is gauge invariant if it remains unchanged after a \emph{local} $SU(N_c)$ transformation, i.e.
\begin{equation}
    q \rightarrow \exp{i \chi_\alpha(x) \lambda_\alpha} q, \quad
    \bar q \rightarrow \exp{-i \chi_\alpha(x) \lambda_\alpha} \bar q.
\end{equation}
The mass term $m_f \bar q_f q_f$ is gauge invariant, while the kinetic derivative term needs to be modified.
For this term to be gauge-invariant, we have to introduce a covariant derivative, which transforms as
\begin{equation}
    D_\mu q \rightarrow (D_\mu q)' = U (D_\mu q) = D_\mu' (U q).
\end{equation}
If we posit $D_\mu = \one \partial_\mu - i g A_\mu^\alpha \lambda_\alpha $ where $A_\mu^\alpha$ is a set of vector fields, and $g$ a constant, then
\begin{equation}
    (\one \partial_\mu - i g A_\mu') U
    = U (\one \partial_\mu - i g A_\mu^\alpha \lambda_\alpha)
\end{equation}
where $A_\mu = A_\mu^\alpha \lambda_\alpha$.
This means that if we demand that $A_\mu$ transforms as
\begin{equation}
    A_\mu \rightarrow U \left(A_\mu + \frac{i}{g} \partial_\mu\right) U^\dagger ,
\end{equation}
The second derivative,
\begin{equation}
    D_\mu D_\nu = \partial_\mu \partial_\nu - ig(\partial_\mu A_\nu + A_\mu\partial_\nu + A_\nu\partial_\mu) - g^2A_\mu A_\nu,
\end{equation}
transforms in the same way as the first derivative.
We see that the ``operator-part'' of this derivative is symmetric in the space-time indices, which means that the commutator is just a tensor, and not an operator.
We define
\begin{equation}
    \label{gluon field strength tensor}
    G_{\mu\nu} 
    := \frac{i}{g}[D_\mu, D_\nu] = (\partial_\mu A_\nu - \partial_\mu A_\nu) - ig[A_\mu, A_\nu]
    = (\partial_\mu A_\nu^\alpha - \partial_\nu A_\mu^\alpha + g C_{\beta \gamma }^\alpha A_{\mu}^\beta A_{\nu}^\gamma ) \lambda_\alpha,
\end{equation}
which is the gluon field strength tensor.
This transforms as
\begin{equation}
    G_{\mu\nu} \rightarrow U G_{\mu \nu} U^\dagger.
\end{equation}

We now need to include terms governing the gauge field $A_\mu$ in the Lagrangian.
The tensor $G_{\mu\nu}$ allows construct all gauge invariant renormalizable, which are
\begin{equation}
    G_{\mu \nu}^a G_a^{\mu \nu}, 
    \quad 
    \varepsilon^{\mu\nu\rho\sigma} G_{\mu \nu}^a G_{\rho \sigma}^a.
\end{equation}
Here, $\varepsilon$ is the Levi-Civita symbol.
The Yang-Mills Lagrangian is therefore
\begin{equation}
    \Ell = \bar q ( i \slashed{D} - m)q 
    + \frac{1}{4} G_{\mu \nu}^a G_a^{\mu \nu}
    + \theta \varepsilon^{\mu\nu\rho\sigma} G_{\mu \nu}^a G_{\rho \sigma}^a,
\end{equation}
where $\theta$ is a coupling constant. 
In the case of QCD, $N_c = 3$, and the generators $\lambda_\alpha, \, \alpha \in \{1, ... 8\}$ are the $3\times3$ Gell-Mann matrices.
Furthermore, experiments indicate that $\theta = 0$, or at least very close.
There is no symmetry which forbids $\theta \neq 0$, and its absence is dubbed the strong $CP$-problem~\cite{Schwartz:QFT}.

\subsection*{Chiral symmetry}

In addition to the color and flavor indices $s$ and $f$, the quarks also have spinor indices, $i$, on which the $\gamma$-matrices act.
We can define the projection operators,
\begin{equation}
    P_\pm = \frac{1}{2}(\one \pm \gamma^5),
\end{equation}
which obey $P_\pm^2 = P_\pm$, $P_+P_- = P_-P_+ = 0$ and $P_+ + P_- = 1$, as good projection operators should.
Furthermore, $P^\dagger_\pm = P_\pm$.
These project spinors down to their chiral components, called left- and right-handed spinors,
\begin{equation}
    P_+ q = q_R, \quad P_- q = q_L.
\end{equation}
From \autoref{Conventions and notation}, we have 
\begin{equation}
    \acom{\gamma^\mu}{\gamma^5} = 0,
\end{equation}
which means that 
\begin{equation}
    \bar q P_\pm = (P_{\mp}q)^\dagger \gamma^0.
\end{equation}
Using the chiral projection operators, we can write the quark term of the QCD Lagrangian as
\begin{align*}
    \bar q (i\slashed D - m) q
    & = 
    \bar q (P_+ + P_-) (P_+ + P_-) (i\slashed D - m) q
    = (q P_-)\gamma^0 P_+ (i \slashed D - m) q + (q P_+)\gamma^0 P_- (i \slashed D - m) q \\
    & = \bar q_L (i\slashed D) q_L + \bar q_R (i\slashed D) q_R
    - \bar q_L m q_R - \bar q_R m q_L.
\end{align*}
The mass-term mixes the two chiral components.
In the case that $m_f = 0$, called the \emph{chiral limit}, the QCD Lagrangian is invariant under the transformations
\begin{equation}
    q_R \rightarrow e^{i\theta_R} U_R q_R, \quad q_L \rightarrow e^{i\theta_L} U_L q_L,
\end{equation}
where $U_L$ and $U_R$ are complex $N_f \times N_f$ matrices with determinant 1 that act on the flavor indices, and $\theta_L$ and $\theta_R$ real numbers.
The total set of such transformations form the Lie group $U(N_f)_L \times U(N_f)_R = U(1)_L \times U(1)_R \times SU(N_f)_L \times SU(N_f)_R$.
We can write
\begin{equation}
    U_R = P_- + \exp{i \eta_\alpha^R T_\alpha^R}P_+, 
    \quad
    U_L = P_+ + \exp{i \eta_\alpha^L T_\alpha^L}P_-,
\end{equation}
where $T_\alpha^L$ and $T_\alpha^R$ are the generator for $SU(N_f)_L$ and $SU(N_f)_R$.
The conserved currents corresponding to these symmetries are
\begin{align}
    L^{\mu}_\alpha & = 
    \diffp{\Ell}{(\partial_\mu q)} (-iT_\alpha^Lq_L)
    = \bar q \gamma^\mu T_\alpha^L q_L = \bar q_L \gamma^\mu T_\alpha^L q_L, \\
    R^{\mu}_\alpha
    & = \bar q_R \gamma^\mu T_\alpha^R q_R, \\
    J_R^\mu &= \diffp{\Ell}{(\partial_\mu q)} (- iq_L) = \bar q_R \gamma^\mu q_R \\
    J_L^\mu &= \bar q_L \gamma^\mu q_L
\end{align}
We have defined the currents with a positive sign to conform to common convention.
When we introduce a non-zero mass, the symmetry of the Lagrangian is explicitly broken.
However, we see that when right and left-handed quarks are transformed in the same way, i.e. $\eta_R = \eta_L$, then the Lagrangian remain invariant.
The diagonal subgroup of $SU(N_f)_L\times SU(N_f)_R$, denoted $SU(N_f)_V$, remains unbroken.
Elements of this group have the form
\begin{equation}
    U_V 
    = \exp{i \eta_\alpha^V T_\alpha^V \one}.
\end{equation}
This is a normal subgroup of $SU(N_f)_L\times SU(N_f)_R$, and the quotient $SU(N_f)_L\times SU(N_f)_R/ SU(N_f)_V$ therefore forms a Lie group, $ SU(N_f)_A$, with elements of the form
\begin{equation}
    U_A
    = \exp{i \eta_\alpha^A T_\alpha^A \gamma^5},
\end{equation}
Similarly, the diagonal subgroup of $U(1)_L\times U(1)_R$ have an unbroken, normal subgroup $U(1)_V$, with elements of the form $e^{i\theta_A\one}$, and a quotient group, $U(1)_A$, with elements of the form $e^{i \theta_V \gamma^5}$.
In the chiral limit, these groups are all symmetries of the Lagrangian, and the corresponding conserved currents are
\begin{align}
    V^\mu_\alpha &= R^{\mu}_\alpha + L^{\mu}_\alpha = \bar q \gamma^\mu T^V_\alpha q, \\
    A^\mu_\alpha &= R^{\mu}_\alpha - L^{\mu}_\alpha = \bar q \gamma^\mu \gamma^5T^A_\alpha q, \\
    J_V^\mu & = J_R^{\mu} + L^{\mu} = \bar q \gamma^\mu q, \\
    J_A^\mu & = J_L^{\mu} - L^{\mu} = \bar q \gamma^\mu \gamma^5 q.
\end{align}

The currents corresponding to $U(1)_V$ and $U(1)_A$ are called the vector and axial currents, as they transform as vectors and axial-vectors under space-time transformations.
The conserved charge corresponding to the vector current,
\begin{equation}
    Q = \int \dd^3 x \, \bar q \gamma^0 q,
\end{equation}
is the quark number, which is $3$ times the \emph{baryon number}.
Up until now, we have only made classical considerations.
In the full quantum theory, the baryon number remains conserved.
The axial current $J^\mu_A$ however is the subject to an anomaly.
As the integration measure $\D\bar q \D q$ is not invariant under transformations in $U(1)_A$, as we saw earlier is required, the quantum correction to $\partial_\mu J^\mu_A$ are non-zero.
The remaining symmetries, $U(1)_V \times SU(N_f)_L \times SU(N_f)_R$ and how they are broken is the foundation for chiral perturbation theory, which is the subject of the next section.
