\section{Chiral perturbation theory}
\label{section:chiral pertubation theory}
This section is base on \cite{Schwartz:QFT,weinberg_1996_vol2,Scherer2002IntroductionTC}, in addition to the original work~\cite{Gasser-Leutwyler:chiral,WeinbergPhenom,Scherer:PhysRevD.53.315}

In this paper, we will consider the interaction of the two lightes quarks, the up and down quarks $u$ and $d$, i.e. $N_f = 2$.
In this case, the symmetry group of roations in the flavor matrices are $G_f = U(1) \times SU(2)_L \times SU(2)_R$.
The generators of $SU(2)$ are $T_\alpha = \tau_\alpha / 2$, where $\tau_\alpha$ are the Dirac matrices, as described in \autoref{Conventions and notation}.
The mass matrix of the quarks is
\begin{equation}
    \label{Mass matrix}
    m =
    \begin{pmatrix}
        m_u & 0 \\
        0 & m_d
    \end{pmatrix}
    = \frac{1}{2} (m_u + m_d) \one + \frac{1}{2}(m_u - m_d) \tau_3,
\end{equation}
where $m_u \approx 2.16 \, \text{MeV}$, and $m_d \approx 4.67 \, \text{MeV}$~\cite{PDG}.
This means that the $G_f$ symmetry is \emph{explicitly} broken.
In what is called the isospin limit, $m_u = m_d$, the subgroup $U(1) \times SU(2)_V$ remains intact.
However, as $m_u \neq m_d$, also the isospin subgroup $SU(2)_V$ is broken.
The difference between these masses is small, which is why isospin is a good quantum number.
Even though the underlying symmetry is only approximate, we can still apply the formalism from Goldstone's theorem and the CCWZ construction by including a small mass term for the Goldstone bosons, which in this case are called \emph{Pseudo Goldstone bosons}.

We now focus on the subgroup $G = SU(2)_L \times SU(2)_R$.
This approximate symmetry of the two-flavor QCD-Lagrangian is spontaneously broken by the ground state.
As quarks $q$ are spinors, a non-zero expectation value of the quark field would break Lorentz-invariance.
Instead, the spontaneous symmetry breaking is characterized by the \emph{scalar quark condensate}, $\ex{\bar q q}$.
The scalar quantity $\bar q q$ is invariant under isospin transformations $H = SU(2)_V$, but not under $SU(2)_A$.
The Goldstone manifold is therefore $G/H = SU(2)_L\times SU(2)_R/SU(2)_V = SU(2)_A$.
To model the low energy dynamics of QCD, we start with the massless QCD Lagrangian,
\begin{align*}
    \Ell^0_\mathrm{QCD}[q, \bar q, A_\mu] 
    = i \bar q \slashed{D} q - \frac{1}{4}G_{\mu \nu}^\alpha G^{\mu \nu}_\alpha
\end{align*}
Following~\cite{Scherer2002IntroductionTC,Gasser-Leutwyler:chiral}, we couple quarks to external currents.
These can be used to model external fields, or to capture the symmetry breaking mass term.
As found in \autoref{section:QCD}, the conserved currents are
\begin{equation}
    V_a^\mu = \frac{1}{2} \bar q \gamma^\mu \tau_a q, \quad
    A_a^\mu = \frac{1}{2} \bar q \gamma^\mu \gamma^5 \tau_a q, \quad
    J^\mu = \bar q \gamma^\mu q.
\end{equation}
In addition, external currents can couple to the scalar and pseudo-scalar quark bilinears
\begin{equation}
    \bar q q, \quad \bar q \gamma^5 q, 
    \quad \bar q \tau_a q, \quad \bar q \gamma^5 \tau_a q.
\end{equation}
The external Lagrangian is thus
\begin{align}
    \Ell_\text{ext} 
    & = 
    - (\bar q q )\, s_0 + (i \bar q \gamma^5 q) \, p_0
    - (\bar q \tau_a q)\, s_a + (i \bar q \gamma^5 \tau_a q) \, p_a
    + \frac{1}{3} J_\mu v_{(s)}^\mu 
    + V_\mu^a v_a^\mu + A_\mu^a a_a^\mu \\
    & = 
    - \bar q \left(s + i \gamma^5 p \right) q
    + \bar q \left(\frac{1}{3} v^\mu_{(s)} + v^\mu + a^\mu \gamma^5\right) \gamma_\mu q.
\end{align}
Here, $v, v_{(s)}, a, s$ and $p$ are the external currents, where we denote
\begin{equation}
    s = s_0 \one + s_a \tau_a, \quad
    p = p_0 \one + p_a \tau_a, \quad
    v^\mu = \frac{1}{2} v_a^\mu \tau_a, \quad
    a^\mu = \frac{1}{2} a_a^\mu \tau_a.
\end{equation}
Setting $v = v_{(s)} = a = s = p = 0$ recover the chiral limit.
To include the effect of the quark masses, we set $s = m$.
We denote all the external currents as $j = (v, v_{(s)}, a, s, p )$.
As with the conserved currents, we define
\begin{equation}
    v_\mu = \frac{1}{2}(r_\mu + l_\mu),
    \quad
    a_\mu = \frac{1}{2}(r_\mu - l_\mu).
\end{equation}
With this, as well as 
\begin{equation}
    \bar q (s - i \gamma^5 p) q
    = \bar q_R (s - i p) q_L + \bar q_L (s + i p) q_R,
\end{equation}
we can write the external Lagrangian as
\begin{equation}
    \Ell_\text{ext} 
    = - \bar q_R (s + i p) q_L - \bar q_L (s - i p) q_R
    + \frac{1}{3} (J_R^\mu + J_V^\mu)(v_{(s)})_\mu
    + R_\mu^a r^\mu_a + L_\mu^a l^\mu_a
\end{equation}
The generating functional is then
\begin{equation}
    Z[j] 
    = 
    \int \D \bar q \D q \D A_\mu \, 
    \exp{
        i \int \dd^4 x 
        \left( 
            \Ell^0_\text{QCD}[q, \bar q, A_\mu] + \Ell_\text{ext}[q, \bar q, j]
        \right)
    }
\end{equation}
We will now use the CCWZ construction and effective field theory to construct the effective Lagrangian, which obeys
\begin{equation}
    Z[j] = \int \D \pi \, \exp{i \int \dd^4 x \, \Ell_\text{eff}[\pi]},
\end{equation}
where $\pi$ are the Goldstone bosons.

\subsection*{Weinberg's power counting scheme}
\label{subsection:power counting}

We now have a relationship between the underlying Lagrangian and the effective Lagrangian, of the form \autoref{integrating out degrees of freedom}.
However, as discussed at earlier, we are not able to solve QCD at low energies, and have therefore no way to integrate out the high energy degrees of freedom.
Instead, we have to make use of Weinberg's ``theorem''.
Using the CCWZ-construction, we have to the most general Lagrangian obeying the symmetries of QCD, in which case we have not laid any constraints on the theory which were not present already.
This, however, will result in a theory with an infinite number of free parameters, as well as being rather unwieldy.
To amend this, we need a scheme to order the terms such that we can compute observable perturbativley.
We are working in the low-energy regime, so it is natural to expand in pion momenta.
As we saw in the CCWZ-construction, the terms in the Lagrangian will be made up of combinations the terms $e_\mu$ and $d_\mu$ of the Maurer-Cartan form, $i \Sigma \partial_\mu \Sigma = e_\mu + d_\mu$.
All terms in the effective Lagrangian will therefore be proportional to a certain number of derivatives, which Lorentz invariance demands to be even.

Consider the matrix element $\mathcal M$ for a given Feynman diagram with external pion lines with momenta $q_n$, where both the energies and momentum is less or equal to some energy scale $Q$.
If scale $Q\rightarrow tQ$, and consequently also the external momenta $q_n \rightarrow tq_n$, momentum conservation at each vertex ensures that each internal momentum $p$ of the diagram scales as $p \rightarrow tp$.
Assume this diagram is made up of $V_i$ copies of the vertex $i$, which contain $d_i$ derivatives of the pions.
Each of these vertices scale as $t^{d_i}$.
The propagators contribute a factor $p^{-2}$ and will therefore scale as $t^{-2}$, and the integration measure $\dd^4 p$ scales as $t^4$.
This means that a matrix element with $L$ loops and $I$ internal lines scales as
\begin{equation}
    \mathcal M(q) \rightarrow \mathcal M(t q) = t^D \mathcal M(q),
\end{equation}
where 
\begin{equation}
    D = \sum_i V_i d_i - 2 I + 4 L.
\end{equation}

$D$ is called the \emph{chiral dimension} of $\mathcal M$.
Using the formula \cref{Number of loops} for number of loops in a Feynman diagram, we get
\begin{equation}
    D = \sum_i V_i(d_i - 2) + 2 L + 2.
\end{equation}
For a low energys scales $Q$, the larges contribution will come from the matrix elements of smalles chiral dimension $D$.
A genereal process will consist of a sum of matrix elements of different chiral dimension.
We can expand this element in powers of the pion momenta by using $t$ as the expansion parameter.
The leading order term will be those where $L = 0$ and $d_i = 2$ so that $D = 2$, i.e. the tree-level contribution from terms in the Lagrangian with only one derivative.
Next is $D = 4$, which contains both tree-level contributions from terms with $d_i = 4$ and a one-loop contribution from $d_i = 2$.
We therefore expand the effective Lagrangian as 
\begin{equation}
    \Ell_\text{eff} = \Ell_2 + \Ell_4 + ...,
\end{equation}
where $\Ell_{2n}$ contains $2n$ derivatives.
This is equivalent to scaling the space-time cooridnates as $x^\mu \rightarrow tx^\mu$, and expanding the Lagrangian in powers of $t$.

We must also allow for the fact that pions have non-zero mass, and the fact that there are a finite isospin chemical potetntial.
This is all implemented by external currents, wich we also have to assume are small.
The external pions are on-shell, the pion mass $m_\pi$ must be less than the energy scale $Q$.
As we will see, this corresponds to scaling the quark masses as $m_q \rightarrow t^2 m_q$.
Simalarly, $\mu_I$ must also be less than $Q$, which means that we scale it as $\mu_I\rightarrow t \mu_I$.
Following these rules, each term in the effective Lagrangian will have a well-defined chiral dimension, ensuring a consistent series expansion~\cite{weinberg_1996_vol2,WeinbergPhenom,Scherer2002IntroductionTC}.

\subsection*{Non-linear realization}

To parametrize the low energy behavior of QCD, we must find a representative element of the coset space of $G$, $G/H = SU(2)_A$.
Let $g\in G$. 
We write $g = (L, R)$, where $R \in SU(2)_R$, $L \in SU(2)_L$.
Elements $h \in H$ are then of the form $h = (V, V)$.
A general element g can be written as
\begin{equation}
    g = (L, R) = (1, R L^\dagger) (L, L).
\end{equation}
Since $(L, L) \in H$, this means that we can write the coset $g H$ as $(1, R L^\dagger)H$, which gives a way to choose a representative element for each coset.
We identify
\begin{equation}
    \Sigma = R L^\dagger. 
\end{equation}
This is our standard form, and therefore makes it possible to obtain the transformation properties, which is given by the function $h(g, \xi)$.
For $\tilde g \in G$, we have 
\begin{equation}
    \tilde g (1, \Sigma)
    = (\tilde L, \tilde R) (1, R L^\dagger)
    = (1, \tilde R (R L^\dagger) \tilde L^\dagger) (\tilde L, \tilde L)
    = (1, \tilde R \Sigma \tilde L) \tilde h.
\end{equation}
This gives the transformation rule
\begin{equation}
    \Sigma \rightarrow \Sigma' = R \Sigma L^\dagger.
\end{equation}
Under transformations by $h = (V, V^\dagger) \in H$, we have
\begin{equation}
    \Sigma \rightarrow \Sigma' = V \Sigma V^\dagger.
\end{equation}
As $\partial_\mu  (1, \Sigma) = (0, \partial_\mu \Sigma)$, the constituents of the Mauer-Cartan form are
\begin{equation}
    d_\mu = i \Sigma(x)^\dagger \partial_\mu \Sigma(x),\quad
    e_\mu = 0.
\end{equation}
Using $\partial_\mu [\Sigma(x)^\dagger\Sigma(x)] = 0 $, we can write
\begin{equation}
    d_\mu d^\mu = 
    - \Sigma(x)^\dagger [\partial_\mu \Sigma(x)] \Sigma(x)^\dagger [\partial^\mu \Sigma(x)]
    =\Sigma(x)^\dagger [\partial_\mu \Sigma(x)] [\partial^\mu \Sigma(x)^\dagger] \Sigma(x).
\end{equation}
In \autoref{seciton:ccwz construction}, we found the lowest order terms, \cref{first order terms CCWZ}.
However, as $d_\mu \in \lieg{su}{2}$, which we represent by the traceless Pauli matrices, we have
\begin{equation}
    \Tr{d_\mu} = 0.
\end{equation}
This leaves us with the single leading order term
\begin{equation}
    \Tr{d_\mu d^\mu} = \Tr{\partial_\mu \Sigma (\partial^\mu \Sigma)^\dagger},
\end{equation}
where we have used the cyclic property of the trace.

Constructing the effective action out of Lagrangians that are invariant under $G$, however, is too restrictive to get the most general effective Lagrangian.
This only allows for an even number of $d_\mu$'s, and observed processes such as the decay of the neutral pion through $\pi^0 \rightarrow \gamma \gamma$ would not be possible~\cite{Scherer2002IntroductionTC}.
This is because we have not allowed for terms which changes the Lagrangian with a divergence term, as discussed in \autoref{section:symmetry}.
Terms of this type are called Wess-Zumino-Witten (WZW) terms~\cite{weinberg_1996_vol2}.
We will not consider these here, as they do not affect the thermodynamic quantities in question~\cite{Andersen:two-flavor-chpt}.

\subsection*{External currents and explicit symmetry breaking}
\label{Covarinat derivative}

Using $d_\mu$ we are able to construct any terms in the effective Lagrangian corresponding to $j=0$.
To construct the effective Lagrangian when $j \neq 0$, which is needed to capture the effects of nonzero quark masses and external currents, we treat $G =  U(1)_V \times SU(2)_L\times SU(2)_R$ as a gauge group, and the source currents as the corresponding gauge fields.
The effective Lagrangian is then constructed as the most general Lagrangian that is invariant under \emph{local} transformations in $G$.
The Ward-identities corresponding to the local symmetries of the Lagrangian are equivalent with the statement that $Z[j]$ is invariant under a gauge transformation of the external fields, in the absence of anomalies~\cite{Leutwyler:on_the_fundations}.

Following~\cite{Scherer2002IntroductionTC}, we write the gauge transformation as
\begin{equation}
    q_L \rightarrow e^{i\theta(x)/3} U_L(x) q_L, \quad
    q_R \rightarrow e^{i\theta(x)/3} U_R(x) q_R.
\end{equation}
First, we consider the $U(1)_V$ transformation.
The massless QCD Lagrangian then transforms as
\begin{equation}
    \Ell_\text{QCD}^0 = i \bar q \slashed{D} q
    \rightarrow
    i \bar q e^{-i\theta(x)/3} \slashed{D} e^{i\theta(x)/3} q
    = i \bar q \slashed{D} q - \frac{1}{3} \bar q \gamma^\mu q \partial_\mu \theta(x).
\end{equation}
This gives us the transformation rule
\begin{equation}
    v_{(s)}^\mu \rightarrow v_{(s)}^\mu - \partial^\mu \theta(x).
\end{equation}
Then, applying the $SU(2)_R$ transformation, we get
\begin{equation}
    \bar q_R \slashed{D} q_R + (i \bar q_R \gamma^\mu \tau_a  q_R) \, r^a_\mu
    \rightarrow
    i \bar q_R \slashed{D} q_R + 
    (i \bar q_R \gamma^\mu q_R ) \, U_R^\dagger \partial_\mu U_R  
    + [\bar q_R \gamma^\mu (U_R^\dagger \tau_a U_r)  q_R ] \, r^a_\mu
\end{equation}
This, and the similar expression for $U_L$, gives the gauge transformation rules
\begin{align}
    r_\mu^a \tau_a & \rightarrow U_R (r_\mu^a\tau_a + i\one \partial_\mu) U_R^\dagger, \\
    l_\mu^a \tau_a & \rightarrow U_L (l_\mu^a\tau_a + i\one \partial_\mu) U_L^\dagger, \\
    s + i p & \rightarrow U_R (s + i p) U_L^\dagger, \\
    s - i p & \rightarrow U_L (s - i p) U_R^\dagger.
\end{align}
 Goldstone fields now transform as
\begin{equation}
    \Sigma(x) \rightarrow U_R(x) \Sigma(x) U_L(x)^\dagger,
\end{equation}
and the derivative as
\begin{align}
    \nonumber
    \partial_\mu \Sigma \rightarrow & \, \partial_\mu (U_R \Sigma U_L^\dagger) \\
    &= 
    U_R (\partial_\mu \Sigma )U_L^\dagger
    + (\partial_\mu  U_R) \Sigma U_L^\dagger
    + U_R \Sigma (\partial_\mu U_L^\dagger)
    \nonumber
    \\
    & = 
    U_R
    \left[
        \partial_\mu \Sigma
        + U_R^\dagger (\partial_\mu U_L) \Sigma
        + \Sigma (\partial_\mu U_L^\dagger) U_L
    \right]
    U_L^\dagger.
    \label{Sigma partial derivative}
\end{align}
The covariant derivative is defined to transform in the same way as the object it is acting upon.
This means that the definition of the covariant derivative depends on what it acts on.
Assume $A$, $B$ and $\Sigma$ transforms as $A \rightarrow U_R A U_R^\dagger$, $B \rightarrow U_L A U_L^\dagger$, and $\Sigma \rightarrow U_R \Sigma U_L^\dagger$.
In each of these cases, we define the covariant derivative as
\begin{align}
    \label{covariant derivative general}
    \nabla_\mu A &= \partial_\mu A - i [r_\mu, A], \\
    \nabla_\mu B &= \partial_\mu B - i [B, l_\mu], \\
    \nabla_\mu \Sigma &= \partial_\mu \Sigma - i r_\mu \Sigma + i \Sigma l_\mu.
\end{align}
It follows from \cref{Sigma partial derivative} that $\nabla_\mu \Sigma$ transforms as $\Sigma$, and the proof for the other two is similar.
For quantities that do not transform under $SU(2)_L\times SU(2)_R$, the covariant derivative is the ordinary partial derivative.
If $A$, $B$ and $C = AB$ are all quantities with a well-defined covariant derivative, then
\begin{equation}
    \nabla_\mu (AB) = (\nabla_\mu A) B + A (\nabla_\mu B).
\end{equation}
We prove the special case of $a_\mu = 0$, which is what we use in this text, in which case
\begin{equation}
    \nabla_\mu \Sigma = \partial_\mu \Sigma - i [v_\mu, \Sigma].
\end{equation}
Assume $A, B$ transforms as $\Sigma$. 
Then,
\begin{align*}
    \nabla_\mu (A B)
    & = (\partial_\mu A) B + A (\partial_\mu B) - i \com{v_\mu}{AB}
    = (\partial_\mu A - i \com{v_\mu}{A})B + A(\partial_\mu B- i \com{v_\mu}{B})\\
    & = (\nabla_\mu A)B + A (\nabla_\mu B).
\end{align*}
The more general theorem follows by applying the definition of the various covariant derivatives~\cite{Scherer:PhysRevD.53.315}.
Decomposing a 2-by-2 matrix $M$, as described in \autoref{Conventions and notation}, shows that the trace of the commutator of $\tau_b$ and $M$ is zero:
\begin{equation*}
    \Tr{\com{\tau_a}{M}]} = M_b\Tr{ \com{\tau_a}{\tau_b}} = 0.
\end{equation*}
Together with the fact that $\Tr{\partial_\mu A} = \partial_\mu \Tr{A}$, this gives the product rule for invariant traces:
\begin{equation*}
    \Tr{A \nabla_\mu B} = \partial_\mu \Tr{AB} - \Tr{(\nabla_\mu A) B}.
\end{equation*}
This allows for the use of the divergence theorem when doing partial integration.
Assume $A$, $K^\mu$, and $A K^\mu$ have a well-defined covariant derivative, and that $\Tr{A K^\mu}$ is invariant under transformations by the gauge group.
Let $\Tr{K^\mu}$ be a space-time vector, and $\Tr{A}$ scalar. 
Let $\Omega$ be the domain of integration, with coordinates $x$ and $\partial \Omega$ its boundary, with coordinates $y$. Then, 
\begin{align*}
    \int_\Omega \dd x \, \Tr{A \nabla_\mu K^\mu} 
    = 
    - \int_\Omega \dd x \, \Tr{(\nabla_\mu A) K^\mu}
    + \int_{\partial\Omega} \dd y\, n_\mu \Tr{A K^\mu},
\end{align*}
where $n_\mu$ is the normal vector of $\partial \Omega$~\cite{Carroll:spacetime}.
By the assumption of no variation on the boundary, the last term is constant when varying the action, and may therefore be discarded.
This leads to a modification of the Maurer-Cartan form as well, to $i \Sigma^\dagger \nabla_\mu \Sigma$.
The leading order kinetic term now becomes
\begin{equation}
    \label{leading order term sigma}
    \Tr{d_\mu d^\mu} = \Tr{\nabla_\mu \Sigma (\nabla^\mu \Sigma)^\dagger}
\end{equation}

We define the scalar current
\begin{equation}
    \chi = 2 B_0 (s + ip),
\end{equation}
where $B_0$ is defined by the up quark condensate in the chiral limit by
\begin{equation}
    \ex{\bar u u}_{j = 0} = - f^2 B_0.
\end{equation}
Here, $f$ is the bare pion decay constant.
The pseudoscalar current is thus $\chi^\dagger = 2 B_0 (s - ip)$.
As we are going to set $\chi = 2 B_0 m$, both $\chi$ and $\chi^\dagger$ are of chiral dimension $2$.
These can be combined to give more gauge-invariant terms, such as
\begin{equation}
    \Tr{\chi^\dagger \Sigma}, \quad \Tr{\Sigma^\dagger \chi}.
\end{equation}
However, under parity transformation, $\chi \rightarrow \chi^\dagger$ and $\Sigma \rightarrow \Sigma^\dagger$~\cite{Scherer2002IntroductionTC}, so to ensure the Lagrangian is a scalar, the only allowed term is\footnote{This corresponds to the fact that pions are pseudo scalars, they transform as $\pi \rightarrow - \pi$ under parity, as they are exitations in $SU(2)_A$.}
\begin{equation}
    \label{leading order chi sigma term}
    \Tr{\chi^\dagger \Sigma + \Sigma^\dagger \chi}
\end{equation}

In the grand canonical ensemble, we modify the Lagrangian as
\begin{equation}
    \Ell \rightarrow \Ell + \mu Q,
\end{equation}
where $Q$ is a conserved charge, and $\mu$ is the corresponding chemical component.
We are interested systems with non-zero isospin, 
\begin{equation}
    Q_I = \int \dd^3 x \, V^0_3 = \int \dd^3 x \, \frac{1}{2}  \bar q \gamma^0 \tau_3 q,
\end{equation}
and the corresponding isospin chemical potential $\mu_I$.
We do this by considering the system with a external current
\begin{equation}
    v_\mu = \frac{1}{2} \mu_I \delta_\mu^0 \tau_3.
\end{equation}
The covariant derivative therefore is of chiral dimension 1.
As in the case of Yang-Mills theory, we may form a field strength tensor to create invariant terms, in this case they are
\begin{align}
    f_{\mu\nu}^R &= \partial_\mu r_\nu - \partial_\nu r_\mu - i [r_\mu, r_\nu], \\
    f_{\mu\nu}^L &= \partial_\mu l_\nu - \partial_\nu l_\mu - i [l_\mu, l_\nu].
\end{align}
In our case, these vanish, and we can therefore safely ignore terms made including them.

For $j = 0$, the vacuum is by assumption $\Sigma = \one$, and we can use the paramterization
\begin{equation}
    \Sigma(x) = \exp{i \frac{\pi_a\tau_a}{f}},
\end{equation}
where $f$ is the bare pion decay constant, $\pi_a$ are the three Goldstone bosons, and are real functions of space-time.
This ensures that $\pi = 0$ corresponds to the vacuum.
If we perform an infinitesimal isospin-transformation, and assume $\pi_a/f$ small, then
\begin{equation}
    \Sigma \rightarrow U_V \Sigma U_V^\dagger
    =
    \left(1 + i \eta_a \frac{1}{2} \tau_a\right)
    \left(1 + i \frac{1}{f} \pi_b  \tau_b\right)
    \left(1 - i \eta_c \frac{1}{2} \tau_c\right)
    % = 
    % 1 + i \pi_a \tau_a + i \frac{1}{2}[\eta_a (\tau_a\tau_b - \tau_b \tau_a)] \pi_b
    =
    1 + i\frac{1}{f}\pi_a (\delta_{ac} + i \eta_b \epsilon_{abc}) \tau_c,
\end{equation}
or
\begin{equation}
    \pi_a \rightarrow (\delta_{ac} + i \eta_b \epsilon_{abc}) \pi_c.
\end{equation}
The generators of $\pi_a$ under isospin-transforamtions are thus the adjoint representation of $\lieg{su}{2}$, and they form an isospin triplet.
For $\eta_1 = \eta_2 = 0$, i.e. transformations generated by $\tau_3$, $\pi_3$ is invariant, which means that it has quantum number $I_3 = 0$.
$\pi_1$ and $\pi_2$ do not have a definite value of the third component of isospin, but rather for the first and second component.
They are related to the observed, charged pions $\pi_+$ and $\pi_-$ by~\cite{Scherer2002IntroductionTC}
\begin{equation}
    \pi_a\tau_a
    = 
    \begin{pmatrix}
        \pi_3 & \pi_1 - i \pi_2 \\
        \pi_1 + i \pi_2 & - \pi_3
    \end{pmatrix}
    = 
    \begin{pmatrix}
        \pi_0 & \sqrt{2} \pi_+ \\
        \sqrt 2 \pi_- & - \pi_0
    \end{pmatrix}.
\end{equation}
For non-zero isospin chemical potential, however, we expect that the ground state may be rotated away from the vacuum.
Following~\cite{Andersen:two-flavor-chpt}, we parametrize the Goldstone manifold as
\begin{align}
\label{sigma}
    \Sigma(x) = A_\alpha [U(x) \Sigma_0 U(x)] A_\alpha,
\end{align}
where
\begin{align}
    \Sigma_0 = \one,\, 
    A_\alpha = \exp{\frac{i \alpha}{2} \tau_1},\, 
    U(x) = \exp{i \frac{\tau_a\pi_a(x)}{2f}}.
\end{align}
Here, $\alpha$ is a real number between $0$ and $2 \pi$, that parametrizes the ground state of the system.
This is all we need to start constructing the \chpt effective Lagrangian.
