\section{Equation of motion and redundant terms}

Changing the field parametrization that appear in the Lagrangian does not affect any of the physics, as it corresponds to a change of variables in the path integral~\cite{Scherer2002IntroductionTC,Chisholm:changeOfVar,Kamefuchi:changeOfVar}.
However, a change of variables can result in new terms in the Lagrangian.
As a result of this, terms that on the face of it appear independent may be redundant.
These terms can be eliminated by using the classical equation of motion.
In this section we show first the derivation of the equation of motion, then use this result to identify redundant terms which need not be included in the most general Lagrangian.

We derive the equation of motion for the leading order Lagrangian using the principle of least action.
Choosing the parametrization $\Sigma = \exp(i \pi_a \tau_a)$, a variation $\pi_a \rightarrow \pi_a + \delta \pi_a$ results in a variation in $\Sigma$, $\delta \Sigma = i \tau_a \delta \pi_a \Sigma $.
The variation of the leading order action,
\begin{equation}
    S_2 = \int_\Omega \dd^4x \, \Ell_2,
\end{equation}
when varying $\pi_a$ is 
\begin{align*}
    \delta S = \int_\Omega \dd x \, \frac{f^2}{4}
    \Tr{
        (\nabla_\mu \delta \Sigma) (\nabla^\mu \Sigma)^\dagger
        + (\nabla_\mu \Sigma) (\nabla^\mu \delta \Sigma)^\dagger
        + \chi \delta \Sigma^\dagger + \delta \Sigma \chi^\dagger
    }.
\end{align*}
Using the properties of the covariant derivative to do partial integration, as show in \autoref{Covarinat derivative}, as well as $\delta(\Sigma \Sigma^\dagger) = (\delta\Sigma)\Sigma^\dagger + \Sigma (\delta \Sigma)^\dagger = 0$, the variation of the action can be written
\begin{align*}
    \delta S 
    & = \frac{f^2}{4} \int_\Omega \dd x\, 
    \Tr{
        - \delta \Sigma \nabla^2 \Sigma^\dagger
        + (\nabla^2 \Sigma) (\Sigma^\dagger \delta \Sigma \Sigma^\dagger)
        - \chi (\Sigma^\dagger \delta \Sigma \Sigma^\dagger)
        + \delta \Sigma \chi^\dagger
    } \\
    & = 
    \frac{f^2}{4} \int_\Omega \dd x\, 
    \Tr{
        \delta \Sigma \Sigma^\dagger 
        \left[
            (\nabla^2 \Sigma)\Sigma^\dagger
            - \Sigma \nabla^2 \Sigma^\dagger
            - \chi \Sigma^\dagger
            + \Sigma \chi^\dagger
        \right]
        } \\
    & = 
    i \frac{f^2}{4} \int_\Omega \dd x\, 
    \Tr{\tau_a 
    \left[
         (\nabla^2 \Sigma)\Sigma^\dagger
        - \Sigma \nabla^2 \Sigma^\dagger
        - \chi \Sigma^\dagger
        + \Sigma \chi^\dagger
    \right]
    } 
    \delta \pi_a = 0.
\end{align*}  
As the variation is arbitrary, the equation of motion to leading order is
\begin{equation}
    \Tr{
        \tau_a 
        \left[
            (\nabla^2 \Sigma)\Sigma^\dagger
            - \Sigma \nabla^2 \Sigma^\dagger
            - \chi \Sigma^\dagger
            + \Sigma \chi^\dagger
        \right]
    } = 0.
\end{equation}
We define
\begin{equation}
    \mathcal O_\text{EOM}^{(2)}
    = 
    (\nabla^2 \Sigma)\Sigma^\dagger
    - \Sigma \nabla^2 \Sigma^\dagger
    - \chi \Sigma^\dagger
    + \Sigma \chi^\dagger.
\end{equation}
% This may be rewritten as a matrix equation. 
% Using that 
% \begin{align*}
%     \Tr{(\nabla_\mu \Sigma)\Sigma^\dagger}
%     = 
%     \Tr{i \tau_a (\partial_\mu \pi_a )\Sigma \Sigma^\dagger}
%     - i\Tr{[v_\mu, \Sigma]\Sigma^\dagger}
%     = 0,
% \end{align*}
% we can see that $\Tr{(\nabla^2 \Sigma)\Sigma^\dagger - \Sigma \nabla^2 \Sigma^\dagger} = 0$, and the equation of motion may be written as
% \begin{equation}
%     \label{EOM matrix form}
%     \mathcal{O}_\mathrm{EOM}^{(2)}(\Sigma) 
%     = 
%     (\nabla^2 \Sigma)\Sigma^\dagger
%     - \Sigma \nabla^2 \Sigma^\dagger
%     - \chi \Sigma^\dagger
%     + \Sigma \chi^\dagger
%     - \frac{1}{2}
%     \Tr{ \chi \Sigma^\dagger - \Sigma \chi^\dagger} = 0.
% \end{equation}

The next step in eliminating redundant terms is to change the parametrization of $\Sigma$ by $\Sigma(x) \rightarrow \Sigma'(x)$.
Here, $ \Sigma(x) = e^{iS(x)} \Sigma'(x), \, S(x) \in \liea{su}{2}$. This change leads to a new Lagrange density, $\Ell[\Sigma] = \Ell[\Sigma'] + \Delta \Ell[\Sigma']$.
We are free to choose $S(x)$, as long $\Sigma'$ still obeys the required transformation properties.
Any terms in the Lagrangian $\Delta \Ell$ due to a reparametrization can be neglected, as argued earlier.
When demanding that $\Sigma'$ obey the same symmetries as $\Sigma$,
the most general transformation to second order in Weinberg's power counting scheme  is~\cite{Scherer2002IntroductionTC}
\begin{equation}
    \label{S reparametrization}
    S_{2} = 
    i \alpha_2 
    \left[
        (\nabla^2 \Sigma') \Sigma^\dagger - \Sigma' (\nabla^2 {\Sigma'})^\dagger
    \right]
    + i \alpha_2
    \left[
        \chi \Sigma'^\dagger - \Sigma' \chi^\dagger 
        - \frac{1}{2} \Tr{\chi \Sigma'^\dagger - \Sigma' \chi^\dagger}
    \right].
\end{equation}
$\alpha_1$ and $\alpha_2$ are arbitrary real numbers. As \autoref{S reparametrization} is to second order, $\Delta \Ell$ is fourth order in Weinberg's power counting scheme.
To leading order is given by
\begin{align*}
    \Ell_2\left[e^{i S_2}\Sigma '\right]
    & =
    \frac{f^2}{4}\Tr{[\nabla_\mu (1 +i S_2)\Sigma'][\nabla^\mu \Sigma'^\dagger  (1 - i S_2)]}
    + \frac{f^2}{4} \Tr{\chi\Sigma'^\dagger (1 - i S_2) + (1 +i S_2)\Sigma' \chi^\dagger} \\
    & = \Ell[\Sigma'] + 
    i \frac{f^2}{4}
    \Tr{[\nabla_\mu (S_2\Sigma')][\nabla^\mu\Sigma']^\dagger 
    -  [\nabla_\mu\Sigma'][\nabla^\mu (\Sigma'^\dagger  S_2) ]}
    - i \frac{f^2}{4} \Tr{\chi \Sigma'^\dagger S_2 - S_2 \Sigma' \chi^\dagger}
\end{align*}
Using the properties of the covariant derivative, we may use the product rule and partial integration to write the difference between the two Lagrangians to fourth order as
\begin{align}
    \nonumber
    \Delta \Ell[\Sigma'] 
    & = 
    i \frac{f^2}{4}
    \Tr{
        (\nabla_\mu S_2)
        (\Sigma' \nabla^\mu \Sigma'^\dagger - (\nabla^\mu \Sigma') \Sigma'^\dagger) 
    }
    - i \frac{f^2}{4} \Tr{\chi \Sigma'^\dagger  S_2 - S_2 \Sigma' \chi^\dagger} \\
    & = 
    % i \frac{f^2}{4}
    % \Tr{
    %     S_2
    %     \left[
    %         \Sigma'^\dagger\nabla^2 \Sigma' - ( \nabla^2 \Sigma') \Sigma'^\dagger 
    %         - \chi\Sigma'^\dagger + \Sigma' \chi^\dagger
    %     \right]
    % }
    \label{Delta reparametrization}
    \frac{f^2}{4} \Tr{i S_2 \mathcal{O}_\mathrm{EOM}^{(2)}}.
\end{align}
Any term that can be written in the form of \autoref{Delta reparametrization} for arbitrary $\alpha_1, \alpha_2 \in \R$ is redundant, as we argued earlier, and may therefore be discarded.
$\Delta \Ell_2$ is of fourth order, and it can thus be used to remove terms from $\Ell_4$ or higher order.
