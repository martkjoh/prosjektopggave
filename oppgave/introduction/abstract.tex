Pions are particles that describe the dynamics of QCD at low energies, and it has recently been proposed that they can form compact stellar objects called pion stars.
We use two-flavor chiral perturbation theory to calculate the grand canonical free energy density to next-to-leading order, at $T = 0$ and with non-zero isospin chemical potential $\mu_I$.
At $\mu_I = m_\pi$, a pion condensate is formed and spontaneously breaks the isospin symmetry of the QCD Lagrangian.
We observe the resulting Goldstone mode.
The pion condensate phase is characterized by a non-zero isospin density.
We discuss the nature of the phase transition using Landua-theory of second-order phase transitions.
The free energy density is used to obtain the relationship between the pressure and the energy density of the system, the equation of state.
This, together with the Tolman-Oppenheimer-Volkoff equation, can be used to model pion stars, allowing for further investigation of these newly proposed objects.
