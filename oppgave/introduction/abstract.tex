Pions are particles that describe the dynamics of QCD at low energies, and it has recently been proposed that they form compact stellar objects called pion stars.
We use two-flavor chiral perturbation theory to calculate the grand canonical free energy density to next-to-leading order, at $T = 0$ and with non-zero isospin chemical potential $\mu_I$.
\todo{What part is approximate, what is broken? Find out!}
At $\mu_I = m_\pi$, the system breaks the approximate isospin symmetry of the Lagrangian, and we observe the resulting Goldstone mode.
This results in a phase transition to a pion condensate phase with non-zero isospin density.
We discuss the nature of the phase transition using Landua-theory of second-order phase transitions.
The free energy density is used to obtain the relationship between the pressure and the energy density of the system, the equation of state.
This, together with the Tolman-Oppenheimer-Volkoff equation, can be used to model pion stars, allowing for further investigation of these newly proposed objects.
