Throughout this text, \emph{natural units} are used.
These units are defined so that
\begin{equation}
    \hbar = c = k_B = 1,
\end{equation}
where $\hbar$ is the Planck reduced constant, $k_B$ is the Boltzmann constant, and $c$ is the speed of light.
These constants will therefore be dropped from all expressions.
They can be reintroduced using dimensional analysis.
In natural units, \emph{mass dimension} is the only engineering dimension.
Dimensionfull results are given in units of electronvolt (eV), or pion-masses, 
\begin{equation}
    m_\pi = 131 \, \text{MeV}.
\end{equation}

The Minkowski metric convention used is the ``mostly minus'',
\begin{equation}
    g_{\mu \nu} = \mathrm{diag}(1, -1, -1, -1).
\end{equation}
The Fourier transform used in this text is defined by
\begin{align*}
    \F{f(x)}(p) = \tilde f(p) = \int \dd x\, e^{i p x}f(x), \quad 
    \FInv{\tilde f(p)}(x) = f(x) = \int \frac{\dd p}{2 \pi}\, e^{- i p x} \tilde f(p).
\end{align*}
We employ the \emph{Einstiein summation convention}, in which pairwise matching indices are summed.
That is,
\begin{equation}
    a_i b_i = \sum_i a_i b_i = a_1 b_1 + \dots.
\end{equation}
For Minkowski-space indices, $\mu$, $\nu$, $\rho$ and $\sigma$, the metric raises and lower indices, and summation should always be over one raised and one lowered index,
\begin{equation}
    a_\mu b^\mu = g_{\mu\nu} a^\mu b^\nu 
    = a^0 b^0 - a^1 b^1 - \dots.
\end{equation}
