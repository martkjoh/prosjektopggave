The recent proposals that pions can form compact stellar objects called pion stars~\cite{new_clas_of_compact_stars,andersen:bose_einstein} have increased the importance of understanding the thermodynamics of pion condensates.
In addition to the perogative of basic science to shed light on the behavior of the standard model, it is speculated that lepton asymetires in the early univers could result in pion condensation~\cite{new_clas_of_compact_stars,abduki:Pion_condensation_in_a_dense_neutrino_gas,Wygas:Cosmic_QCD_Epoch_at_Nonvanishing_Lepton_Asymmetry,Schwarz_2009:Lepton_asymmetry_and_the_cosmic_QCD_transition}.
In this case, pion stars might have left observable traces in the form of nautrino and photon spectra from their evaporation, or in the form of graviational waves, and thus play a role in forming the universe we see today~\cite{new_clas_of_compact_stars}.

In this paper, we have discussed the theoretical fundations for chiral perturbation theory, and derived the building blocks of the effective Lagrangian governing pions.
Using the most general Lagrangian up to next-to-leading order in Weinberg's power counting scheme, we calculated the grand canonical free energy density in the case of a non-zero isospin chemical potential, to next to leading order.
From this, we calculated the equation of state, or EOS.
We find that the EOS to remain trivial for isospin chemical potential $\mu_I$ less than some critical value $\mu_I^c$, and showed that this value equals the pion mass $m_\pi$ to next-to-leading order, as expected.
Furthermore, at $\mu_I = \mu_I^c$, we showed that the system undergoes a phase transistion, where the isospin symmetry is broken.
This results in a massless Goldstone boson.
In this new phase, it becomes energetically for the system to move to an exitetd state, leading to a pion condensate.

\subsection*{Outlook}

The equation of state of a material is used in conjunction with the Tolman–Oppenheimer–Volkoff (TOV) equations to model the internal dynamics of stars,
With this one can calculate the relationship between the mass and radius of stars~\cite{Carroll:spacetime}.
In the case of pion stars, one has to include leptons in the model to ensure electric charge neutrality.
This has been done using two-flavor chiral perturbation at $T=0$~\cite{Andersen:two-flavor-chpt,andersen:bose_einstein}.
These results can be improved by taking into account the strange quark by using three-flavor chiral perturbation theory.
Furthermore, using thermal field thoery, one can take into account the effects of the non-zero temperature of real stellar objects.

(FOR SPEKULATIVT? INTETSIGENDE?)
It has been shown that a pion condensate can arise under conditions of high neutrino density and small baryon density, conditions which arise in nature in supernova explosions~\cite{abduki:Pion_condensation_in_a_dense_neutrino_gas}.
The lepton asymmetry in the universe, and the nature of neutrinos and their masses are not well understood~\cite{Schwartz:QFT,Schwarz_2009:Lepton_asymmetry_and_the_cosmic_QCD_transition}.
To more definetly answer the question of the existence of pione stars, a better understanding of these dynamics are needed.
