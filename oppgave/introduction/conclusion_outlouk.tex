% The recent proposals that pions can form compact stellar objects called pion stars~\cite{new_clas_of_compact_stars,andersen:bose_einstein} have increased the importance of understanding the thermodynamics of pion condensates.
% In addition to the perogative of basic science to shed light on the behavior of the standard model, it is speculated that lepton asymetires in the early univers could result in pion condensation~\cite{new_clas_of_compact_stars,abduki:Pion_condensation_in_a_dense_neutrino_gas,Wygas:Cosmic_QCD_Epoch_at_Nonvanishing_Lepton_Asymmetry,Schwarz_2009:Lepton_asymmetry_and_the_cosmic_QCD_transition}.
% In this case, pion stars might have left observable traces in the form of nautrino and photon spectra from their evaporation, or in the form of graviational waves, and thus play a role in forming the universe we see today~\cite{new_clas_of_compact_stars}.

The thermodynamic behavior of chiral perturbation theory at non-zero isospin chemical potential servs as fruitfull avenue for exploration of QCD at low temperatures, as it can be compared to calculations from first principles using lattice QCD.
The recent proposals that pions can form compact stellar objects called pion stars~\cite{new_clas_of_compact_stars,andersen:bose_einstein} have increased the importance of better understanding of this regime.
It is speculated that lepton asymetires in the early univers could result in pion condensation~\cite{new_clas_of_compact_stars,abduki:Pion_condensation_in_a_dense_neutrino_gas,Wygas:Cosmic_QCD_Epoch_at_Nonvanishing_Lepton_Asymmetry,Schwarz_2009:Lepton_asymmetry_and_the_cosmic_QCD_transition}.
In this case, pion stars might have left observable traces in the form of nautrino and photon spectra from their evaporation, or in the form of graviational waves, and can have thus playd a role in forming the universe we see today~\cite{new_clas_of_compact_stars}.

In this paper, we have discussed the theoretical fundations for chiral perturbation theory, and derived the building blocks of the effective Lagrangian governing pions.
Using the most general Lagrangian up to next-to-leading order in Weinberg's power counting scheme, we calculated the grand canonical free energy density in the case of a non-zero isospin chemical potential, to next to leading order.
From this, we calculated the equation of state, or EOS.
We find the EOS to remain trivial for isospin chemical potential $\mu_I$ less than some critical value $\mu_I^c$, and showed that this value equals the pion mass $m_\pi$ to next-to-leading order, as expected.
Furthermore, at $\mu_I = \mu_I^c$, we showed that the system undergoes a phase transistion, where the isospin symmetry is broken.
In this new phase, it becomes energetically favourable for the system to move to an exitetd state, leading to a pion condensate.
The pion condensate is characterized by a non-zero isospin density $n_I$, caused by the ground state being rotated away from the vacuum.
We observed a massles mode in the spectrum, as we expect from Goldstones theorem.


\subsection*{Outlook}

The equation of state of a material is used in conjunction with the Tolman–Oppenheimer–Volkoff (TOV) equations to model the internal dynamics of stars.
This enables calculation of the relationship between the mass and radius of stars~\cite{Carroll:spacetime}.
Stellar objects, such as stars, display electric charge neutrality on macroscopic scales
In the case of pion stars, one has to include leptons in the model to ensure electric charge neutrality.
Isospin density is related to the up- and down-quark density $n_u$ and $n_d$ by
\begin{equation}
    n_I 
    = \frac{1}{TV} Q_I 
    = \frac{1}{TV} \left(\ex{\bar u \gamma^0 u} - \ex{\bar d \gamma^0 d} \right) 
    = n_u - n_d,
\end{equation}
Furthermore, as the isospin chemical potential is related to the up- and down-quark chemical potential by $\mu_I/2 = \mu_u = -\mu_d$, we get the relationship $n_u = -n_d$~\cite{new_clas_of_compact_stars}.
As the up quark has charge $\frac{2}{3}e$, and the down quark $-\frac{1}{3}e$, the quark condensate alone is electricly charged.
A realistic stellar object thus must include a lepton-density $n_l$.
This gives the criterion for charge density,
\begin{equation}
    n_Q = \frac{2}{3}n_u - \frac{1}{3} n_d - n_l = n_I - n_l = 0.
\end{equation}
This equation, together with the equation of state and the TOV-equation, allows for investigation of charge-neutral pion stars.

Comparisons between the free energy density and other thermodynamic quantities obtained from chiral perturbation theory and lattice QCD at $T = 0$ are in good agreement~\cite{Andersen:two-flavor-chpt,mojahed}.
Using thermal field thoery, \chpt can be extended to finite tmperature.
Recent stuies of \chpt at non-zero temperature finds that the theory remains in good agreement with lattice QCD as well as other models for temperatures below $20 \, \text{MeV}$~\cite{andersen_mojahed:condensates_and_pressure}.
A good understanding of the themal properites of pion condensates is critical to account for the non-zero temperature of real stellar objects.

Our results can be improved by taking into account the strange quark by using three-flavor chiral perturbation theory.
The strange quark $s$ has a mass $m_s \approx 93 \, \text{MeV}$~\cite{PDG}, which is considerably larger than the up- and down-quark, but still within the strong-interaction regieme, which suggests that it can paly an important role in equation of state.
Chiral perturbation theory can be extended to include the strange quark by considering the larger group $SU(3)_L \times SU(3)_R$, consisting of rotations of all three u-, d-, and s-quarks into eachother.

(FOR SPEKULATIVT? INTETSIGENDE?)
It has been shown that a pion condensate can arise under conditions of high neutrino density and small baryon density, conditions which arise in nature in supernova explosions~\cite{abduki:Pion_condensation_in_a_dense_neutrino_gas}.
The lepton asymmetry in the universe, and the nature of neutrinos and their masses are not well understood~\cite{Schwartz:QFT,Schwarz_2009:Lepton_asymmetry_and_the_cosmic_QCD_transition}.
To more definetly answer the question of the existence of pione stars, a better understanding of these dynamics are needed.
