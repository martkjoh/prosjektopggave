The thermodynamic behavior of chiral perturbation theory at non-zero isospin chemical potential serves as a fruitful avenue for exploration of QCD at low temperatures, as it can be compared to calculations from first principles using lattice QCD.
The recent proposals that pions can form compact stellar objects called pion stars~\cite{new_clas_of_compact_stars,andersen:bose_einstein} increase the importance of better understanding of this regime.
It is speculated that lepton asymetires in the early universe could result in pion condensation~\cite{new_clas_of_compact_stars,abduki:Pion_condensation_in_a_dense_neutrino_gas,Wygas:Cosmic_QCD_Epoch_at_Nonvanishing_Lepton_Asymmetry,Schwarz_2009:Lepton_asymmetry_and_the_cosmic_QCD_transition}.
In this case, pion stars might have left observable traces in the form of neutrino and photon spectra from their evaporation or in the form of gravitational waves and can have thus played a role in forming the universe we see today~\cite{new_clas_of_compact_stars}.

In this specialization project, we have discussed the theoretical foundations for chiral perturbation theory and derived the building blocks of the effective Lagrangian governing pions.
Using the most general Lagrangian to next-to-leading order , we calculated the grand canonical free energy density in the case of a non-zero isospin chemical potential.
From this, we calculated the equation of state (EOS).
We find the EOS to remain trivial for isospin chemical potential $\mu_I$ less than some critical value $\mu_I^c$ and showed that this value equals the pion mass $m_\pi$ to next-to-leading order, as expected.
Furthermore, at $\mu_I = \mu_I^c$, we showed that the system undergoes a phase transition, breaking the isospin symmetry.
In this new phase, it becomes energetically favorable for the system to move to an excited state, leading to a pion condensate.
The pion condensate is characterized by a non-zero isospin density $n_I$, caused by the ground state being rotated away from the vacuum.
As we expect from Goldstone's theorem, we observe a massless mode in the spectrum.


\subsection*{Outlook}

The equation of state of a material is used in conjunction with the Tolman-Oppenheimer-Volkoff (TOV) equations to model the internal dynamics of stars.
This enables calculation of the relationship between the mass and radius of stars~\cite{Carroll:space-time}.
Stellar objects, such as stars, display no electric charge neutrality on macroscopic scales.
In the case of pion stars, one has to include leptons in the model to ensure electric charge neutrality.
Isospin density is related to the up- and down-quark density $n_u$ and $n_d$ by
\begin{equation}
    n_I 
    = \frac{1}{TV} Q_I 
    = \frac{1}{TV} \left(\ex{\bar u \gamma^0 u} - \ex{\bar d \gamma^0 d} \right) 
    = n_u - n_d,
\end{equation}
At zero baryon chemical potential, $\mu_B = 0$, the isospin chemical potential is related to the up- and down-quark chemical potential by $\mu_I/2 = \mu_u = -\mu_d$.
We thus get the relationship $n_u = -n_d$~\cite{new_clas_of_compact_stars}.
As the up quark has electric charge $\frac{2}{3}e$, and the down quark $-\frac{1}{3}e$, the pion condensate alone is electrically charged.
A realistic stellar object thus must include a lepton-density $n_l$ to remain neutrally charged.
This gives the criterion for charge density,
\begin{equation}
    n_Q = \frac{2}{3}n_u - \frac{1}{3} n_d - n_l = n_I - n_l = 0.
\end{equation}
Together with the equation of state and the TOV-equation, this equation allows for the investigation of charge-neutral pion stars.

Comparisons between the free energy density and other thermodynamic quantities obtained from chiral perturbation theory and lattice QCD at $T = 0$ are in good agreement~\cite{Andersen:two-flavor-chpt,mojahed}.
Using thermal field thoery, \chpt can be extended to finite temperature.
Recent studies of \chpt\, at non-zero temperature, find that the theory remains in good agreement with lattice QCD as well as other models for temperatures below $20 \, \text{MeV}$~\cite{andersen_mojahed:condensates_and_pressure}.
A good understanding of the thermal properties of pion condensates is critical to account for the non-zero temperature of real stellar objects.

Our results can be improved by using three-flavor chiral perturbation theory, taking into account the strange quark.
The strange quark $s$ has a mass $m_s \approx 93 \, \text{MeV}$~\cite{PDG}, which is considerably larger than the up and down quark, but still within the strong-interaction regime, which suggests that it can play an important role in the equation of state.
Chiral perturbation theory can be extended to include the strange quark by considering the larger group $\SU(3)_L \times \SU(3)_R$, consisting of rotations of all three quarks into each other.
