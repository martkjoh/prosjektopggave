\section{The 1PI effective action and the effective potential}

\label{section: effective action}
The generating functional for connected diagrams, $W[J]$, is dependent on the external source current $J$.
Analogously to what is done in thermodynamics and in Lagrangian and Hamiltonian mechanics, we can define a new quantity, with a different independent variable, using the Legendre transformation.
The new independent variable is 
\begin{equation}
    \varphi_J(x) := \frac{\delta W[J]}{\delta J(x)} = \ex{\varphi(x)}_J.
\end{equation}
The subscript $J$ on the expectation value indicate that it is evaluated in the presence of a source.
The Legendre transformation of $W$ is then
\begin{equation}
    \label{1PI effective action}
    \Gamma[\varphi_J]
    = W[J] - \int \dd^4 x \, J(x) \varphi_J(x).
\end{equation}
Using the definition of $\varphi_J$, we have that
\begin{equation}
    \label{effective equation of motion}
    \fdv{\varphi_J(x)} \Gamma[\varphi_J]
    = \int \dd^4 y \, \fdv{J(y)}{\varphi_J(x)} \fdv{J(y)} W[J]
    - \int \dd^4 y \, \fdv{J(y)}{\varphi_J(x)} \varphi_J(y)
    - J(x)
    = - J(x).
\end{equation}
If we compare this to the classical equations of motion of a field $\varphi$ with the action $S$,
\begin{equation}
    \frac{\delta S[\varphi]}{\delta \varphi(x)} = -J(x),
\end{equation}
we see that $\Gamma$ is an action that gives the equation of motion for the expectation value of the field, given a source current $J(x)$.

To interpret $\Gamma$ further we observe what happens if we treat $\Gamma[\varphi]$ as a classical action with a coupling $g$.
The generating functional in this new theory is
\begin{equation}
    \label{partition function with g}
    Z[J, g] = \int \D \varphi 
    \exp{ i g^{-1} \left( \Gamma[\varphi] + \int \dd^4x \varphi(x) J(x) \right) }
\end{equation}
The free propagator in this theory will be proportional to $g$, as it is given by the inverse of the equation of motion for the free theory.
All vertices in this theory, on the other hand, will be proportional to $g^{-1}$, as they are given by the higher order terms in the action $g^{-1}\Gamma$.
This means that a diagram with $V$ vertices and $I$ internal lines is proportional to $g^{I-V}$.
Regardless of what the Feynman-diagrams in this theory are, the number of loops of a connected diagram is $L = I - V + 1$.
\footnote{This is a consequence of the Euler characteristic $\chi = V - E + F$.}
To see this, we first observe that one single loop must have equally many internal lines as vertices, so the formula holds for $L = 1$.
If we add a new loop to a diagram with $n$ loops by joining two vertices, the formula still holds.
If we attach a new vertex with one line, the formula still holds, and as the diagram is connected, any more lines connecting the new vertex to the diagram will create additional loops.
This ensures that the formula holds, by induction.
As a consequence of this, any diagram is proportional to $g^{L-1}$.
This means that in the limit $g \rightarrow 0$, the theory is fully described at the tree-level, i.e. by only considering diagrams without loops.
In this limit, we may use the stationary phase approximation, as described in \autoref{section:gaussian integrals}, which gives
\begin{equation}
    Z[J, g\rightarrow 0] \approx 
    C \det(- \frac{\delta^2 \Gamma[\varphi_J]}{\delta \varphi^2})
    \exp{i g^{-1} \left(\Gamma[\varphi_J] + \int \dd^4x J \varphi_J \right)  }.
\end{equation}
This means that
\begin{equation}
    -i g \ln(Z[J, g]) 
    = g W[J, g] 
    = \Gamma[\varphi_J] + \int \dd^4x\,  J(x) \varphi_J(x) + \mathcal{O}(g),
\end{equation}
which is exactly the Legendre transformation we started out with, modulo the factor $g$.
$\Gamma$ is therefore the action which describes the full theory at the tree level.
For a free theory, the classical action $S$ equals the effective action, as there are no loop diagrams.

The propagator $D(x, y)$, which is the connected 2 point function $\ex{\varphi(x)\varphi(y)}_J$, is given by the second functional derivative of $W[J]$, times $-i$.
Using the chain rule, together with \autoref{effective equation of motion}, we get
\begin{align}
    \label{Effective action inverse propagator}
    (-i)\int \dd^4 z \frac{\delta^2 W[J]}{\delta J(x) \delta J(z)} 
    \frac{\delta^2 \Gamma[\varphi_J]}{\delta \varphi_J(z) \varphi_J(y)}
    =
    (-i)\int \dd^4 z \frac{\delta \varphi_J[z]}{\delta J(x)}
    \frac{\delta^2 \Gamma[\varphi_J]}{\delta \varphi_J(z) \varphi_J(y)}
    =
    \fdiff{}{J(x)}  \fdiff{\Gamma[\varphi_J]}{\varphi_J(y)}
    = \delta(x - y).
\end{align}
This shows that the second functional derivative of the effective action is $iD^{-1}$, where $D^{-1}$ is the inverse propagator.
The inverse propagator is the sum of all one-particle-irreducible (1PI) diagrams, with two external vertices.
More generally, $\Gamma$ is the generating functional for 1PI diagrams, which is why it is called the 1PI effective action.

\subsection*{The effective action and symmetries}
The symmetries of a theory are transformations of the physical state that leaves the governing equations unchanged.
A lot of physics is contained in the symmetries of a theory, such as the presence of conserved quantities and the systems low energy behavior.
% When we talk about symmetries of a theory, we mean that there is some sort of transformation of the physical state of the system that leaves it unchanged.
% If we take the system of the earth rotating around the sun as an example.
% If we consider two versions of this system, which at time $t$ is related by a rotation, then at all times these two systems will continue to evolve in the same manner, only distinguished by the rotational transformation.
% This is due to the underlying rotational symmetry of the physical system, which is an external symmetry.
We distinguish between internal and external symmetries.
An external symmetry is an active coordinate transformation, such as rotations or translations.
They relate degrees of freedom at different space-time points, while internal symmetry transforms degrees of freedom at each space-time point independently of what happens at other points.
A further distinction is between local and global symmetry transformations.
Local transformations have one rule for how to transform degrees of freedom at each point, which is applied everywhere, while local transformations might themselves be functions of space-time.

In classical field theory, symmetries are encoded in how the Lagrangian changes due to a transformation of the fields.
We will consider continuous transformations, which are can in general be written as
\begin{equation}
    \varphi(x) \longrightarrow \varphi'(x) = f_t[\varphi](x), \quad t \in [0, 1].
\end{equation}
Here, $f_t[\varphi]$ is a functional in $\varphi$, and a smooth function of $t$, with the constraint that $f_0[\varphi] = \varphi$.
This allows us to look at ``infinitesimal'' transformations,
\begin{equation}
    \label{infinitesimal transformation}
    \varphi'(x) = f_\epsilon[\varphi] \sim \varphi(x) + \epsilon g[\varphi](x), \quad \epsilon \rightarrow 0.
\end{equation}
Here, $g$ is a functional of $\varphi$.
We will consider internal, global transformations in which $g$ is linear in $\varphi$.
For $N$ fields, $\varphi_i$, this can be written
\begin{equation}
    \label{linear field transformation}
    \varphi_i'(x) = \varphi_i(x) + \epsilon \, i t_{ij} \varphi_j(x), \quad \epsilon \rightarrow 0.
\end{equation}
$t_{ij}$ is called the generator of the transformation.
A symmetry of the system is then one in which the Lagrangian is unchanged by the transformation, or at most is different by a divergence-term.
That is, a transformation $\varphi \rightarrow \varphi'$ is a symmetry if 
\begin{equation}
    \Ell[\varphi'] = \Ell[\varphi] + \partial_\mu K^\mu[\varphi],
\end{equation}
where $K^\mu[\varphi]$ is a functional of $\varphi$.\footnote{Terms of the form $\partial_\mu K^\mu$ does not affect the physics, as variational principle $\delta S = 0$ which gives the equations of motion do not vary the fields at infinity.}
This is a requirement for a symmetry in quantum field theory too.
However, as physical quantities are given by not just the action of a single state, but the path integral, the integration measure $\D \varphi_i$ has to be invariant as well.
If a classical symmetry fails due to the integration measure, it is called an anomaly.

We want to investigate what constraints a symmetry lies on the effective action.
To that end, assume 
\begin{equation}
    \D \varphi'(x) = \D \varphi(x), \quad
    S[\varphi'] = S[\varphi].
\end{equation}
In the generating functional, such a transformation corresponds to a change of integration variable.
Using the infinitesimal version of the transformation, we may write
\begin{align}
    Z[J] 
    = \int \D \varphi \, \exp{i S[\varphi] + i \int \dd^4 x J_i(x) \varphi_i(x)} 
    = \int \D \varphi' \, \exp{i S[\varphi'] + i \int \dd^4 x J_i(x) \varphi'_i(x)}
    \\
    = Z[J] -  \epsilon \int \dd^4 x J_i(x) \int \D \varphi \, e^{i S[\varphi]} [t_{ij} \varphi_j(x)],
\end{align}
Using \cref{effective equation of motion}, we can write this as
\begin{equation}
    \label{effective action symmetry requirement}
    \int \dd^4 x \, \fdiff{\Gamma[\varphi_J]}{\varphi_i(x)} \, t_{ij}\ex{\varphi_j(x)}_J = 0.
\end{equation}

\subsection*{Effective potential}
For a constant field configuration $\varphi(x) = \varphi_0$, the effective action, which is a functional, becomes a regular function.
We define the effective potential $\Veff$ by
\begin{equation}
    \label{definition effective potential}
    \Gamma[\varphi_0] = - V T \, \Ve_{\mathrm{eff}}(\varphi_0),
\end{equation}
$VT$ is the volume of space-time.
For a constant ground state, the effective potential will equal the energy of this state.
To calculate the effective potential, we can expand the action around this state to calculate the effective action,
by changing variables to $\varphi(x) = \varphi_0 + \eta(x)$.
$\eta(x)$ now parametrizes fluctuations around the ground state, and has by assumption a vanishing expectation value.
The generating functional becomes
\begin{align}
    Z[J] 
    = \int \D (\varphi_0 + \eta) \, 
    \exp{i S[\varphi_0 + \eta] + i \int \dd^4 x J (\varphi_0 + \eta) }
\end{align}
The notation 
\begin{equation}
    \fdiff{S[\varphi_0]}{\varphi(x)}
\end{equation}
indicates that the functional $S[\varphi]$ is differentiated with respect to $\varphi(x)$, then evaluated at $\varphi(x) = \varphi_0$.
The functional version of a Taylor expansion is
\begin{equation}
    S[\varphi_0 + \eta] = 
    S[\varphi_0]
    + \int \dd x \fdv{S[\varphi_0]}{\varphi(x)} \eta(x)
    + \frac{1}{2} \int \dd x \dd y\,  \frac{\delta^2 S[\varphi_0]}{\delta\varphi(x)\delta\varphi(y)} \eta(x) \eta(y)
    + \dots
\end{equation}
We will only consider this expansion up to second order in derivatives for now.
Inserting this into $Z[J]$ we get
\begin{align*}
    &Z[J] = \\ 
    &\int \D \eta  
    \exp{
        i \int \dd^4 x \left(  \Ell[\varphi_0] + J \varphi_0  \right)
        + i \int \dd^4x \left(  \fdv{S[\varphi_0]}{\varphi(x)} + J(x) \right) \eta(x)
        + i \frac{1}{2} \int \dd^4 x \dd^4 y \,  
        \frac{\delta^2 S[\varphi_0]}{\delta\varphi(x)\delta\varphi(y)} \eta(x) \eta(y)
        }
\end{align*}
The first term is constant with respect to $\eta$, and may therefore be taken outside the path integral.
The second term gives rise to tadpole diagrams, which alter the expectation value of $\eta(x)$.
For $J=0$, this expectation value should vanish, so this term can be ignored.
Furthermore, this means that the ground state must minimize the classical potential,
\begin{equation}
    \label{minimize classical potential}
    \diffp{\Ve(\varphi_0)}{\varphi} = 0.
\end{equation}

The one loop approximation to the effective potential is given by the Taylor-expansion up to second order.
This term is a Gaussian integral, and may be evaluated as described in \autoref{section:gaussian integrals},
\begin{equation}
    \int \D \eta \, 
    \exp(
        i \frac{1}{2} \int \dd^4x \dd^4y\,  
        \frac{\delta^2 S[\varphi_0]}{\varphi(x)\varphi(y)} \eta(x) \eta(y)
        )
        = C \det\left( - \fdiff{S[\varphi_0]}{\varphi(x), \varphi(y)} \right)^{-1/2}
\end{equation}
The generating functional for connected diagrams, as defined in \cref{generating functional of connected diagrams}, is therefore
\begin{align}
    \label{generating functional}
    W[J] 
    & = 
    \int\dd^4 x \, \left(\Ell[\varphi_0] + J \varphi_0\right)
    +i \frac{1}{2} \Tr{\ln\left( - \fdiff{S[\varphi_0]}{\varphi(x), \varphi(y)}  \right)}
    + \dots,
\end{align}
where we have used the identity $\ln \det M = \Tr \ln M$.
Using the definition of the effective action, \cref{1PI effective action}, and \cref{definition effective potential} we get an explicit formula for the effective potential to 1 loop order,
\begin{equation}
    \label{effective potential}
    \Veff(\varphi_0) = \Ve(\varphi_0) - \frac{i}{VT}  \frac{1}{2} \Tr{\ln\left( - \fdiff{S[\varphi_0]}{\varphi(x), \varphi(y)}  \right)}.
\end{equation}
