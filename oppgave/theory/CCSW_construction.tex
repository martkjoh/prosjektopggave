\section{CCSW construction}

As Goldstone bosons are massless, they play a crucial role in the low energy dynamics.
To best describe this limit, we seek a parametrization of the fields in which they are the degrees of freedom.
We saw that the Goldstone bosons corresponds to excitations within the vacuum manifold.
A generic field configuration of this kind can be written
\begin{equation}
    \varphi(x) = \exp(i \tilde\pi_a(x) t_a) \varphi_0 = \tilde\Sigma(x) \varphi_0
\end{equation}
where $\varphi_0$ is the vacuum, and $t_a$ are the generators of $G$.
The real functions $\tilde\pi_a(x)$ thus parametrizes fluctuations away from the ground state $\varphi_0$ in to the ground state manifold.
%TODO: Forklar hvorfor vi ikke bryr oss
% $\varphi(x)$ is not a completely generic field configurations, as it does not account for fluctuations in directions other than those in $G$, but we don't care. (?)
This parametrization, however, is highly redundant.
Two different functions, $\tilde\Sigma(x)$ and $\tilde\Sigma'(x)$, related by 
\begin{equation}
    \tilde \Sigma'(x) = \tilde\Sigma(x) h(x), \quad h(x) \in H
\end{equation}
results in the same $\varphi(x)$.
This is because $h(x)\varphi_0 = \varphi_0$, by assumption.
Let $\Sigma(x) = g \in G$.
To remove this redundancy, we create an equivalence class for all such $g$'s.
This is exactly the left coset, $gH = \setbuilder{gh}{ h \in H}$.
The set of cosets forms a new group, $G / H$, called the Goldstone manifold, and has dimension $\dim(G/H) = \dim(G) - \dim(H)$.
It is generated by the broken generators.
This means that the generic function $\varphi(x)$ can be written
\begin{equation*}
    \varphi(x) = \Sigma(x)\varphi_0, \quad \Sigma(x) \in G/H.
\end{equation*}
The coset is itself a Lie group, and thus also a set of generators $\{\tau_a\}$, which span the corresponding Lie algebra.
Finite transformations are written as the exponential elements of these,
\begin{equation}
    \Sigma(x) = \exp{i \pi_a(x) \tau_a}.
\end{equation}

In the linear sigma model, the original $O(N)$ symmetry is broken down to $O(N-1)$, which transforms the remaining $N-1$ fields with vanishing expectation value into each other.
However, $O(N)$ consists of two disconnected subsets, those matrices with determinant 1 and those with determinant -1.
There is no continuous path that takes an element of $O(N)$ with determinant of $-1$.\footnote{A simple proof of this is the fact that the determinant is a continuous function, while any path $\det A(t)$ such that $\det A(1) = -1,\, \det A(0) = 1$ must make a discontinuous jump.}
The set of symmetries that are connected to the identity is
\begin{eqnarray}
    G = SO(N) = \setbuilder{M \in O(N)}{ \det M = 1},
\end{eqnarray}
which is broken down to $SO(N-1)$.
The Goldstone manifold is thus $G/H = SO(N) / SO(N-1)$.

\subsection*{Transformation properties of Goldstone bosons}
We can deduce how the Goldstone bosons transforms under $G$ from how $\varphi$ transforms.
In general, 
\begin{equation}
    \varphi'(x) = g \varphi(x) = (g \Sigma(x)) \varphi_0 = \Sigma'(x) \sigma_0 \quad g \in G.
\end{equation}
Now, in general $g \Sigma(x)$ is not in the standard form $\Sigma'(x) = \exp{i \pi'_a(x)\tau^a}$, as they can differ by an element of the remaining symmetry group, $g\Sigma(x) = \Sigma'(x) h[g](x)$, $h[g](x) \in H$.
Therefore, we get the transformation rule
\begin{equation}
    \Sigma' = g \Sigma (h[g])^{-1}.
\end{equation}
