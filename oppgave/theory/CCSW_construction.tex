\section{CCSW construction}

Goldstone's theorem tells us that if a theory is invariant under the actions of a group $G$, while the ground state of that theory i.e. the symmetry is broken, then there will appear massless modes.
The low energy dynamics of the theory will be dominated by these modes, as they can be exited by arbitrarily small perturbation to the ground state.
If we want to treat the theory perturbatively, then, the original degrees of freedom might not be the best way to treat the theory.
For example, in QCD, the approximate chiral symmetries are apparent when describing the theory using the quark spinors, $\psi_{i, \alpha}$.
However, low energy QCD is notoriously non-perturbative, due to the strong coupling of the strong force.
Thus, we seek a way to be able to expand the theory in powers of the momenta of the Goldstone modes.

For concreteness, consider a theory consisting of $N$ real fields $\varphi_i(x)$.
The underlying fields of the theory might be complex, or Grassmann-number valued.
However, in the case of $N$ complex fields, they can be described as $2N$ real fields, and the transformation of real Grassmann-numbers, for our purposes, follows the same rules as real numbers.
We can then assume that $G$ is a real, compact Lie Group. 
The action of $g\in G$ can then be represented as a matrix $M_{ij}(g)$ acting on $\varphi_j$, and infinitesimal transformations has the form $M_{ij} = \delta_{ij} + i \epsilon n_{\alpha} t^\alpha_{ij}$.
Here, $t_{ij}^\alpha$ is the generators of $g$, and a basis for the Lie Group corresponding to $G$, and $n^{\alpha}$ is a normal vector.

Our goal is to remove the Goldstone-modes from the original degrees of freedom, $\varphi_i(x)$, by transforming them into a restricted set $\tilde \varphi_i(x)$.
The condition that $\tilde \varphi_i$ is 
\begin{equation}
    \tilde \varphi_i(x) t^\alpha_{ij} \varphi^*_{j} = 0,
\end{equation}
where $\varphi^*$ is the vacuum.
If $t^\alpha$ is not a broken generator, then this is trivially fulfilled, as $t^\alpha _{ij}\varphi_{j} = 0$.
Thus, this is one constraint per broken generator. 

Consider now the quantity
\begin{equation}
    V_{\varphi}(g) = \varphi_i g_{ij} \varphi^*_j.
\end{equation}
This is a continuous bounded function of $g$, as $G$ is compact.
This means that it has a maximum.
Given an arbitrary function $\varphi(x)$, there is a function $g(x)$ that maximizes $V$ for each $x$.
At this maximum, $V$ is stationary, and thus invariant under a small change in $g(x)$, 
$\delta g(x) = i \epsilon n_\alpha g(x) t^\alpha$.
Thus,
\begin{equation}
    \delta V_{\varphi(x)}(g(x)) = i n_\alpha \epsilon \varphi_i(x) g_{ij}(x) t^\alpha_{jk} \varphi_k = 0.
\end{equation}
As $n_\alpha$ is arbitrary, this gives us our transformed field,
\begin{equation}
    \tilde \varphi_i(x) = \varphi_j(x) g_{ij}(x)
\end{equation}

Let $H \subset G$ be the non-broken group left after the broken symmetry.
If this set is non-empty, then the choice of $g_{ij}(x)$ is highly non-unique.
This is becuse $h \in H$, $h_{ij} \varphi^*_j = \varphi^*_i$, by definition. 
Thus, $V_\varphi(gh) = V_\varphi(g)$, and if $g(x)$ maximizes $V_{\varphi(x)}$, so does $g(x)h$.
We therefore consider $g$ and $gh$ equivalent.
This is an equivalence relation, in the sens that it is reflective, symmetric and transiative.
This partitions $G$ into equivalence classes, where all the elements of the right coset,
\begin{equation}
    gH = \{g h| \forall h \in H  \}
\end{equation}
are equivalent to $g$.
The set of cosets, called the quotient group $G / H$, is a new group.
We only need one representative element for each coset, i.e. we have a bijective function from the equivalence classes of $g$ and $G / H$.

% The symmetry transformations $g \in G$ is then represented by a matrix $M(x)$ acting on $\varphi_i$, so a transformed field $\varphi'_i$ is related to the original field $\varphi_i$ through $\varphi_i(x) = M_{ij}(g)\varphi_{j}(x)$, where the summation is implied.
% An infinitesimal transformation matrix is represented by $M_{ij}(g) = \delta_{ij} + i \epsilon t_{ij}(g) $.
% $t_{ij}(g)$ is the generators of the transformation $M$.
% The generators of all elements $g \in G$ form a vector space, called a Lie Algebra.

We now insert $\varphi_i(x)$ into the Lagrangian.
The original theory was invariant under global transformations $g \in G$.
This means that any terms in the Lagrangian $f(\varphi)$ that does not depend on derivatives of $\varphi(x)$ only depend on $\tilde \varphi(x)$, as $f(\varphi(x)) = f(\tilde \varphi(x))$.
The derivative of $\varphi(x)$ is
\begin{equation}
    \partial_\mu \varphi(x) = [\partial_\mu \tilde \varphi_i(x) + \tilde \varphi(x) (\partial_\mu g^{-1}(x)) g(x) ]g^{-1}(x),
\end{equation}
As terms that depend on the derivatives of $\varphi(x)$ also are invariant under a global transformation, all terms that depend on $g$ will be proportional to at least one derivative of $g(x)$.
Thus, there will be no mass terms, and all terms can be ordered in terms of powers of the momenta of the Goldstone bosons.
The field $g(x)$ takes on values in $G / H$, and can therefore be parametrized as 
\begin{equation}
    g(x) = \exp{i \xi_a(x) t^a},
\end{equation}
where $t^a$ are the generators of $G / H$. 
These new fields $\xi_a$ are identified with the Goldstone bosons.
We see that there are one field per broken generator, as $|G/H| = |G| - |H|$.
These fields might in general transform non-linearly under G.
We may deduce heir new transformation rule by the fact that we might write any transformation $g' g(\xi(x))$ as an element in $G/H$, which we can write as $g(\xi'(x))$ for some $\xi$, and an element in $H$.
Thus, a transformation $\varphi(x) \rightarrow g' \varphi(x)$ induces a transfomration $\xi \rightarrow \xi'$ defined by 
\begin{equation}
    g' g(\xi) = g(\xi') h(\xi, g)
\end{equation}


