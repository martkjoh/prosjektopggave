In this section we will survey some general properties of quantum field theory that is necessary for chiral perturbation theory.
First, we will introduce the path integral and the 1-particle irreducible effective action, as well as the effective action.
We will derive Goldstone's theorem and present the CCWZ construction, which are the basis for \chpt.

\section{QFT via path integrals}
\label{section:path integral}

The theory in this section is based on \cite{Peskin:IntroQFT,weinberg_1995,weinberg_1996_vol2,Schwartz:QFT}.
Feynman diagrams are drawn using JaxoDraw~\cite{JaxoDraw}.

In the path integral formalism, one evaluates quantum observable by summing or integrating over all possible configuration in space and time.
If the system has specified initial and final configurations, this amounts to all possible paths the system might evolve between these, hence the name.
We assume the reader has some familiarity with this formalism, however if a refresher is needed, \autoref{section:imaginary-time formalism} contains a derivation of the closely related imaginary-time formalism, and compares it with the path integral approach.
In the path integral formalism, the vacuum-to-vacuum transition amplitude, i.e. the probability that that vacuum at $t = -\infty$ evolves to the vacuum at time $t = \infty$, is given by
\begin{equation}
    Z = \lim_{T\rightarrow \infty} \braket{\Omega, T/2}{-T/2, \Omega}
    = \lim_{T\rightarrow \infty} \inner{\Omega}{ e^{-iHT} }{\Omega}
    = \int \D \pi \D \varphi \, \exp{ i \int \dd^4 x \, \left(\pi \dot \varphi - \He[\pi, \varphi]\right) },
\end{equation}
where $\ket{\Omega}$ is the vacuum of the theory.
By introducing a source term into the Hamiltonian density, $\He \rightarrow \He - J(x)\varphi(x)$, we get the generating functional
\begin{equation}
    Z[J] = 
    \int \D \pi \D \varphi \, 
    \exp{ i \int \dd^4 x \, \left(\pi \dot \varphi - \He[\pi, \varphi]+ J\varphi\right) }.
\end{equation}
If $\He$ is quadratic in $\pi$, we can complete the square and integrate out $\pi$ to obtain
\begin{equation}
    Z[J] = C \int \D \varphi \, \exp{i \int \dd^4 x\, (\Ell[\varphi] + J \varphi)}.
\end{equation}
$C$ is infinite, but constant, and will drop out of physical quantities.
In scattering theory, the main objects of study are correlation functions $\ex{\varphi(x_1)\varphi(x_2)...} = \inner{\Omega}{T\left\{\varphi(x_1)\varphi(x_2)\dots\right\}}{\Omega}$, where $T$ is the time ordering operator.
These are given by functional derivatives of $Z[J]$, 
\begin{equation}
    \label{correlator from generating functional}
    \ex{\varphi(x_1)\varphi(x_2)...}
    = 
    \frac{\int \D \varphi(x)\,  (\varphi(x_1)\varphi(x_2)...) e^{i S[\varphi]}}
        {\int \D \varphi(x)\, e^{i S[\varphi]}}
    =
    \frac{1}{Z[0]} \prod_i\left( -i  \fdv{J(x_i)}\right) Z[J]\Big|_{J = 0},
\end{equation}
where 
\begin{equation}
    S[\varphi] = \int \dd^4 x \, \Ell[\varphi]
\end{equation}
is the action of the theory.
The functional derivative is described in \autoref{section:Functional derivative}.
In a free theory, we are able to write
\begin{equation}
    Z_0[J] = Z_0[0] \exp(i W_0[J]), \quad 
    W_0[J] = \frac{1}{2} \int \dd^4 x \dd^4 y \, J(x) D_0(x - y) J(y),
\end{equation}
where $D_0$ is the propagator of the free theory.
Using this form of the generating functional, \autoref{correlator from generating functional} becomes
\begin{align*}
    & \frac{1}{Z[0]}  (-i)^n\fdiff{}{J(x_1)} ... \fdiff{}{J(x_n)} Z_0[J]  \Big|_{J = 0}
    = (-i)^n \fdiff{}{J(x_1)} ... \fdiff{}{J(x_n)} e^{i W_0[J]} \Big|_{J = 0}\\
    & = (-i)^{n} \fdiff{}{J(x_1)} ... \fdiff{}{J(x_{n-1})} \left(i \fdiff{W_0[J]}{ J(x_{n}) } \right) e^{i W_0[J]} \Big|_{J = 0}\\
    & = (-i)^{n}\fdiff{}{J(x_2)} ... \fdiff{}{J(x_{n-1})}
    \left(
        i\fdiff{ W_0[J] }{ J(x_{n-1}), J(x_{n}) }
        + i^2 \fdiff{W_0[J]}{J(x_{n-1})} \fdiff{W_0[J]}{J(x_{n})}
    \right) 
    e^{i W_0[J]} \Big|_{J = 0}
    = \dots \\
    &= 
    (- i )^{n/2}\sum_{{(a, b)}} \prod_{i=1}^{n/2}
    \fdiff{ W_0[J] }{ J(x_{a(i)}), J(x_{b(i)}) } \Big|_{J = 0}.
\end{align*}
In the last line we have introduced the functions $a, \, b$ which define a way to pair up $n$ elements.
The domain of the functions are the integers between $1$ and $n/2$, the image a subset of the integers between $1$ and $n$ of size $n/2$.
A valid pairing is a set $\{(a(1), b(1)), \dots (a(n/2), b(n/2))\}$, where all elements $a(i)$ and $b(j)$ are different, such all integers up to and including $n$ are featured.
A pair is not directed, so $(a(i), b(i))$ is the same pair as $(b(i), a(i))$.
The sum is over the set ${\{(a, b)\}}$ of all possible, unique pairings.
If $n$ is odd, the expression is equal to $0$.
This is Wick's theorem, and it can more simply be stated as \emph{a correlation function is the sum of all possible pairings of 2-point functions},
\begin{equation}
    \ex{{\prod}_{i=1}^{n} \varphi(x_i)  }_0
    = \sum_{\{(a, b)\}}  \prod_{i=1}^{n/2}  \ex{\varphi(x_{a(i)}) \varphi(x_{b(i)})}_0.
\end{equation}
The subscript on the expectation value indicates that it is evaluated in the free theory.

If we have an interacting theory, that is a theory with an action $S = S_0 + S_I$, where $S_0$ is a free theory, the generating functional can be written
\begin{equation}
    \label{partition function of interacting theory.}
    Z[J] 
    = Z_0[0] \ex{\exp(iS_I + i\int \dd^4 x \, J(x) \varphi(x))}_0.
\end{equation}
We can expand the exponential in power series, which means the expectation in \autoref{partition function of interacting theory.} becomes
\begin{equation}
    \sum_{n, m} \frac{1}{n! m!} \ex{(iS_I)^n \left(i\int \dd^4 x \, J(x) \varphi(x)\right)^m}_0.
\end{equation}
The terms in this series are represented by Feynman-diagrams, which are constructed from the Feynman-rules, and can be read from the action.
We will not go into further details on how the Feynman-rules are derived, which can be found in any of the main sources for this section~\cite{Peskin:IntroQFT,weinberg_1995,weinberg_1996_vol2,Schwartz:QFT}.
The source terms gives rise to an additional vertex
\begin{equation}
    \includegraphics[width=0.25\textwidth, valign=c]{figurer/feynman-diagram/cnt_vertex.eps}.
\end{equation}
The generating functional $Z[J]$ equals $Z_0[0]$ times \emph{the sum of all diagrams with external sources $J(x)$}.

Consider a general diagram without external legs, built up of $N$ different connected subdiagrams, where subdiagram $i$ appears $n_i$ times.
As an illustration, a generic vacuum diagram in $\phi^4$-theory has the form
\begin{align}
    \label{Feinman diagrams}
    V = 
    \includegraphics[width=0.1\textwidth, valign=c]{figurer/feynman-diagram/phi-4_loop_notext.eps}
    \times
    \includegraphics[width=0.08\textwidth, valign=c]{figurer/feynman-diagram/phi-4_2_loop.eps}
    \times
    \includegraphics[width=0.14\textwidth, valign=c]{figurer/feynman-diagram/phi-4_2_loop2.eps}
    \times
    \includegraphics[width=0.1\textwidth, valign=c]{figurer/feynman-diagram/phi-4_loop_notext.eps}
    \times \dots.
\end{align}
If sub-diagram $i$ as a stand-alone diagram equals $V_i$, then each copy of that subdiagram contribute a factor $V_i$ to the total diagram.
However, due to the symmetry of permuting identical subdiagrams, one must divide by the extra symmetry factor $s = n_i !$, which is the total number of permutation of all the copies of diagram $i$.
The full diagram therefore equals
\begin{align}
    \label{Feynman diagrams}
    V
    = \prod_{i= 1}^N \frac{1}{n_i!} V_i^{n_i}.
\end{align}
$V$ is uniquely defined by a finite sequence of integers, $(n_1, n_2, \dots n_N, 0, 0, \dots)$, so the sum of all diagrams is the sum over the set $S$ of all finite sequences of integers.
This allows us to write the sum of all diagrams as
\begin{equation}
    \label{sum of all diagrams}
    \sum_{(n_1, ...)\in S} \prod_{i} \frac{1}{n_i!} V_i^{n_i}
    = \prod_{i = 1}^{\infty} \sum_{n_i=1}^{\infty} \frac{1}{n_i!} V_i^{n_i}
    = \exp({\sum}_i V_i).
\end{equation}
We showed that the generating functional $Z[J]$ were the $Z_0[0]$ times the sum of all diagrams due to external sources.
Using \autoref{sum of all diagrams}, we see that the sum of all \emph{connected} diagrams $W[J]$ is given by
\begin{equation}
    Z[J] = Z_0[0]\exp(i W[J]).
\end{equation}
We can see that this is trivially true for the free theory, the only connected diagram is
\begin{equation}
    \label{generating functional of connected diagrams}
    W_0[J] = 
    \includegraphics[valign=c, width=0.3\textwidth]{figurer/feynman-diagram/current-current.eps}.
\end{equation}
