\subsection{Free energy of the pions}
The action of \chpt is 
\begin{equation}
    S[\pi_a] =\int \dd x \Ell[\pi_a]
    = 
    \int \dd x \left(
        \Ell^{(0)}_2[\pi_a] +\Ell^{(1)}_2[\pi_a] + \Ell^{(2)}_2[\pi_a] + \Ell^{(2)}_4[\pi_a] 
        + \Ell_I[\pi_a] 
        \right),
\end{equation}
where $\Ell_I[\pi_a] $ is the higher order terms.
The action may be expanded around $\pi_a = 0$,
\begin{equation}
    S[\varphi]
    = 
    S^{(0)}[\pi_a = 0] 
    + \int \dd x \, \pi_a  \frac{\delta S}{\delta \pi_a(x)}\Big|_{\pi_a=0}
    + \int \dd x \dd y \, \pi_a(x) \pi_b(y)
    \frac{\delta^2 S}{\delta \pi_a(x) \delta \pi_a(y)}\Big|_{\pi_a=0}
    + \Oh[3]{(\pi/f)}.
\end{equation}
We showed that, when minimizing $\alpha$, $\frac{\delta S_2}{\delta \pi_a(x)}\Big|_{\pi_a=0} = 0$. 
Furthermore, $\Ell_2[\pi_a = 0] = \Ell_2^{(0)}$. 
% The lowest order contribution to the pion is given by the free propagator, \autoref{free pion propagator}, by the formula \autoref{free energy from propagator}.
% This is, however, evaluated in the imaginary-time formalism.
% In the zero-temperature limit, 
% This means that the time coordinate is replaced by $t \rightarrow - \tau$, and is restricted to $\tau \in [0, \beta]$.
% This result in a discrete set of energies, the Matsubara frequencies $\omega = 2 \pi n / \beta$.
% The result is
% \begin{equation}
%     \beta \Ef = \frac{1}{2V} \Tr{\ln[\beta^2 D_0^{-1}]} 
%     = \frac{1}{2} \int_{\tilde \Omega} \dd K \, \sum_a \ln[\beta^2 D_0^{-1}]_{aa}
% \end{equation}
In the limit $\beta = \infty$, the free energy density of the system is given by $ \beta \Ef
= \frac{i}{T V} \ln(Z)$ (HVORFOR?). 
Using the expansion as described in \autoref{Effective potential}, the free energy density to and including second order is given by 
\begin{equation}
    \beta \Ef
    = \frac{i}{T V} \ln(Z)
    = \int \dd x\, \Ell^{(0)}[\pi_a = 0]
    - \frac{1}{2} \Tr{\ln\left( - \frac{\delta^2 S[\pi_a=0]}{\delta \pi^2} \right)}.
    % = \frac{1}{2} \int \frac{\dd^4 k}{(2 \pi)^4} 
    % \left[\ln(p_0^2 - E_+^2) +  \ln(p_0^2 - E_-^2) + \ln(p_0^2 - E_0^2) \right]
\end{equation}
As the ground sate is $\pi_a = 0$, the only terms that will remain in the trace-terms are those that are second order in the pion-fields.
We can therefore evaluate
\begin{align}
    \frac{\delta^2 S_2}{\delta \varphi(x)\delta \varphi(y)}
    = \frac{\delta^2 }{\delta \varphi(x)\delta \varphi(y)} \left( \Ell^{2} \right)
    = D^{-1}(x - y),
\end{align}
where $D(x -y)$ is the propagator of the pion-fields.
As we found the eigenvalues of the inverse propagator, the trace part can be written as
(DETTE ER IKKE HELT RIKTIG)
\begin{equation}
    - \frac{1}{2} (VT) \int \frac{\dd^4 p}{(2 \pi )^4} \, 
    \left[ \ln(E_0^2) + \ln(E_+^2) + \ln(E_-^2) \right].
\end{equation}

