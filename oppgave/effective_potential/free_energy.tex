\subsection{Free energy of the pions}
(HAR JEG FORSTÅTT DETT RIKTIG?)
This section follows \cite{Andersen:two-flavor-chpt,mojahed}
In this section, we calculate the tree level contribution to the free energy of the pions.
Our approach is to start with a mean field approximation for the expectation value of the field, $\ex{\pi_a(x)}_0 = \pi^*_a = 0$, and then expand the action in a Taylor series around this point.
The free energy is 
\begin{equation}
    \Omega = i \ln(Z).
\end{equation}
With the constant expectation value $\pi^*_a$ we can use the result from \autoref{section: effective action} to write
\begin{equation}
    \Omega = -W[J=0] = V T \, \Veff(\pi^*_a).
\end{equation}
The first approximation of the effective potential, which is given by the classical action of a constant field.
In this case it is given by the static ($\pi_a$-independent) part of the Lagrangian, $S[\pi^*_a] = -V T \, \Ell_2^{(0)}$.
From \autoref{L0} we have
\begin{equation}
    - \Ell_2^{(0)} = 
    -f^2   
    \left(
        2B_0m \cos{\alpha}
        + \frac{1}{2} \mu^2 \sin^2{\alpha}
    \right).
\end{equation}
The value of $\alpha$ is found by minimizing the free energy,
\begin{align*}
    -&\dv{\alpha} \Ell_2^{(0)} 
    = - f^2\left(2B_0m - \mu_I^2\cos{\alpha}\right)\sin{\alpha}
    = 0.
\end{align*}
This gives the criterion
\begin{align}
    \alpha = \pi n, \, n \in \mathbb{Z} \quad
    \mathrm{or} \quad
    \cos{\alpha} = \frac{2B_0m}{\mu_I{}^2}.
\end{align}
In our discussion of effective action and effective potential, we saw that a constant expectation value, will fulfill the equation of motion
\begin{equation}
    \frac{\delta \Gamma[\pi^*_a]}{\delta \pi_a(x)}
    = - V T \pdv{\Ve_{\mathrm{eff}}(\pi^*_a)}{\pi_a} = 0.
\end{equation}
As the first approximation to the effective action is from the classical action, this equation becomes
\begin{equation}
    \pdv{\Ve^{(1)}(\pi^*_a)}{\pi_a} = 0,
\end{equation}
where $\Ve^{(1)}$ is the linear part of the potential. 
From \autoref{L1}, $\Ve^{(1)} = f(\mu_I{}^2\cos{\alpha} - 2B_0m)\sin{\alpha} \, \pi_1 $, which shows that the minimization criterion we found ensures that the equation of motion is obeyed.

Using the functional Taylor-expansion, as described in \autoref{section:Functional derivative}, we may expand the action of the pions around the ground state $\pi_a^* $
\begin{equation}
    S[\pi_a]
    = 
    S[\pi_a^*] 
    + \int \dd x \, \pi_a  \frac{\delta S[\pi_a^*]}{\delta \pi_a(x)}
    + \int \dd x \dd y \, \pi_a(x) \pi_b(y)
    \frac{\delta^2 S[\pi_a^*]}{\delta \pi_a(x) \delta \pi_a(y)}
    + \Oh[3]{(\pi/f)}.
\end{equation} 
We showed that
\begin{equation}
    \frac{\delta S[\pi_a^*]}{\delta \pi_a} = -VT\, \diffp{\Veff(\pi_a^*)}{\pi_a} = 0.
\end{equation}
Inserting this expansion into the partition function we get
\begin{equation}
    \ln(Z) = \ln\left[ \int \D \varphi_a e^{i S[\pi_a]} \right]
    = i S[\pi_a^*] + \ln\left[
        \int \D \varphi_a 
        \exp(
            i \int \dd^4 x \dd^4 y \, \pi_a(x) \pi_a(y)
            \frac{\delta^2 S[\pi_a^*]}{\delta \pi_a(x) \delta \pi_a(y)}
            ) 
    \right]
    + \dots,
\end{equation}
where $\dots$ represent loop corrections.
% The lowest order contribution to the pion is given by the free propagator, \autoref{free pion propagator}, by the formula \autoref{free energy from propagator}.
% This is, however, evaluated in the imaginary-time formalism.
% In the zero-temperature limit, 
% This means that the time coordinate is replaced by $t \rightarrow - \tau$, and is restricted to $\tau \in [0, \beta]$.
% This result in a discrete set of energies, the Matsubara frequencies $\omega = 2 \pi n / \beta$.
% The result is
% \begin{equation}
%     \beta \Ef = \frac{1}{2V} \Tr{\ln[\beta^2 D_0^{-1}]} 
%     = \frac{1}{2} \int_{\tilde \Omega} \dd K \, \sum_a \ln[\beta^2 D_0^{-1}]_{aa}
% \end{equation}
The functional integral in the second terms can be evaluated using the functional version of the Gaussian integral, as described in \autoref{section:gaussian integrals},
\begin{equation}
    \label{Free energy to second order}
    \Omega 
    = \Omega_0 + \Omega_ 1 + \dots, \quad 
    \Omega_1 = -i \frac{1}{2} \ln{\det\left( - \frac{\delta^2 S[\pi_a^*]}{\delta \pi^2} \right)}
\end{equation}
$\Omega_0 = -S[\pi_a^* ]$ is the static contribution we found using the mean field approximation.
The next term is the tree level contribution, $\Omega_1$.
The functional derivative can be evaluated using with the rules given in \autoref{section:Functional derivative}, which gives
\begin{align}
    \frac{\delta^2 S[\pi_a^*] }{\delta \varphi(x)\delta \varphi(y)}
    = \frac{\delta^2 }{\delta \varphi(x)\delta \varphi(y)} 
    \int \dd^4 x \, \Ell^{(2)}_2
    = D^{-1}_x \delta(x - y).
\end{align}
Here, $\Ell^{(2)}_2$ is the quadratic part of the Lagrangian, as given in \autoref{quadratic lagrangian}, and $D^{-1}_x$ is the corresponding inverse propagator of the pion fields,
\begin{equation}
    D_x^{-1} = 
    - \left[
        \delta_{ab}(\partial_x^\mu\partial_{x,\mu} + m^2_a)
        -  m_{12}(\delta_{a1} \delta_{b2} - \delta_{a2}\delta_{b1}) \partial_{x, 0}
    \right] 
\end{equation}
The functional determinant in \autoref{Free energy to second order} has a matrix part, due to the 3 pion indices, as well as a functional part.
In \autoref{section:propagator} we found the matrix part of  the determinant in momentum space, which we can write using the dispersion relations of the pion fields
\begin{equation}
    \det(- D^{-1}) = \det(p_0^2 - E_0^2) \det(p_0^2 - E_+^2) \det(p_0^2 - E_-^2).
\end{equation}
The functional determinant can therefore be evaluated as
\begin{align}
    \nonumber
    \ln{\det\left( - \frac{\delta^2 S[\pi_a^*]}{\delta \pi^2} \right)}
    & = \ln \det(p_0^2 - E_0^2) + \ln \det(p_0^2 - E_+^2) + \ln \det(p_0^2 - E_-^2) \\
    \nonumber
    & = \Tr{ \ln(p_0^2 - E_0^2) + \ln(p_0^2 - E_+^2)+  \ln(p_0^2 - E_-^2) } \\
    & = (VT) \int \frac{\dd^4 p}{(2 \pi)^4} 
    \left[ \ln(p_0^2 - E_0^2) + \ln(p_0^2 - E_+^2) + \ln(p_0^2 - E_-^2)  \right],
\end{align}
where we have used the identity $\ln\det(A) = \Tr \ln (A)$.
These terms all have the form
\begin{equation}
    I = \int \frac{\dd^4 p}{(2 \pi)^2} \ln(-p_0^2 + E^2),
\end{equation}
where $E$ is some function of the 3-momentum $\vec p$, but not $p_0$.
We use the trick
\begin{equation}
    \pdv{\alpha} \left(-p_0^2 + E^2\right)^{-\alpha} \Big|_{\alpha=0}
    = \pdv{\alpha} \exp\left[ -\alpha \ln\left(- p_0^2 + E^2\right)  \right] \Big|_{\alpha=0}
    = \ln\left(- p_0^2 + E^2\right),
\end{equation}
and then preform a Wick-rotation of the $p_0$-integral to write the integral on the form
\begin{equation}
    I = i \pdv{\alpha} \int \frac{\dd^4 p}{(2 \pi)^4} \left(p_0^2 + E^2\right)^{-\alpha} \Big|_{\alpha=0},
\end{equation}
where $p$ now is a Euclidean four-vector.
The $p_0$ integral equals $\Phi_1(E, 1, \alpha)$, as defined in \autoref{def dimreg integral}, 
\begin{equation}
    \int \frac{\dd p_0}{2 \pi} (p_0^2 + E)^{-\alpha} = \frac{E^{1-2\alpha}}{\sqrt{4 \pi}} \frac{\Gamma(\alpha-1/2)}{\Gamma(\alpha)}
\end{equation}
The derivative of the gamma function is $\Gamma(\alpha) = \psi(\alpha)\Gamma(\alpha)$, where $\psi(\alpha)$ is the digamma function.
Using
\begin{align}
    \diffp{}{\alpha} \left(\frac{\Gamma(\alpha - 1/2) }{\Gamma(\alpha)}\right) \Big|_{\alpha=0}
    = \Gamma(\alpha - 1 / 2) \frac{\psi(\alpha - 1/2) - \psi(\alpha)}{\Gamma(\alpha)} \Big|_{\alpha=0}
    = \sqrt{4 \pi}, \\
    \frac{\Gamma(\alpha - 1/2) }{\Gamma(\alpha)}\Big|_{\alpha=0} = 0,
\end{align}
we get
\begin{equation}
    I = i \int \frac{\dd^3 p}{(2 \pi)^3} E.
\end{equation}
We see that the is what we would expect physically, the total energy is the integral of the energy of each mode.
This results in 
\begin{equation}
    \Omega_1 = 
    V T \, \frac{1}{2} 
    \left[\int \frac{\dd^3 p}{(2\pi)^3} E_0 + \int  \frac{\dd^3 p}{(2\pi)^3} (E_+ + E_-)\right]
    = \Omega_{1, \pi_0} + \Omega_{1, \pi_\pm} + \Omega_{1, \pi_-}.
\end{equation}
The first integral is identical to what we find for the free scalar gas.
In the $\overline{\mathrm{MS}}$-scheme,
\begin{equation}
    \Omega_{1, \pi_0} 
    = 
    - VT \frac{1}{4} \frac{m_3^4}{(4\pi)^2} 
    \left[ \frac{1}{\epsilon} + \frac{3}{2} + \ln(\frac{\mu^2}{m_3^2}) \right] + \mathcal{O}(\epsilon),
\end{equation}
as calculated in \autoref{section: regualting free energy}.
$\mu$ is the renormalization scale, which is introduced to make dimensional regularization well-defined.
The contribution to the free energy from the $\pi_+$ and $\pi_-$ particles is more complicated.
By expanding $E_+ + E_-$, as given in \autoref{dispresion relation pi pm}, in powers of $1 / \abs{\vv{p}}$, we get 
\begin{equation}
    E_+ + E_-
    = 2  \abs{\vv{p}}
    + [m_{12} + 2(m_1^2 + m_2^2)]\frac{1}{4 \abs{\vv{p}}}
    - \left[ m_{12}^4 + 4 m_{12}^2(m_1^2 + m_2^2) + 8(m_1^4 + m_2^4) \right] \frac{1}{64 \abs{\vv{p}}^3}
    + \Oh[-5]{ \abs{\vv{p}}}.
\end{equation}
The higher order terms result in a convergent integral.
$E_+ + E_-$ has the same UV-behavior as $E_1 + E_2$, i.e. they differ by  $\Oh[-5]{|\vv p|}$, where $E_i = \sqrt{|\vv{p}|^2 + \tilde m_i^2}$, and $\tilde m_i^2 = m_i^2 + \frac{1}{4} m_{12}^2$. 
The divergent part of the free energy of the $\pi_\pm$-particles is therefore, using the $\mathrm{\overline{MS}}$-scheme, 
\begin{equation}
    \Omega^{\mathrm{div}}_{1, \pi_+} + \Omega^{\mathrm{div}}_{1, \pi_-}
    = 
    - VT \frac{1}{4}
    \left\{
        \frac{\tilde m_1^4}{(4\pi)^2} 
        \left[
            \frac{1}{\epsilon} + \frac{3}{2} + \ln(\frac{\mu^2}{\tilde m_1^2}) 
        \right]
        +
        \frac{\tilde m_2^4}{(4\pi)^2} 
        \left[
            \frac{1}{\epsilon} + \frac{3}{2} + \ln(\frac{\mu^2}{\tilde m_2^2})
        \right] 
    \right\} 
    + \mathcal{O}(\epsilon).
\end{equation}
We define
\begin{equation}
    \Omega^{\mathrm{fin}}_{1, \pi_+} + \Omega^{\mathrm{fin}}_{1, \pi_-}
    = 
    V T \, \frac{1}{2} \int \frac{\dd^3 p}{(2\pi)^3} (E_+ + E_- - E_1 + E_2),
\end{equation}
which is a finite integral.
The total free energy at tree level is thus
\begin{equation}
    \Omega = \Omega_0 
    + \Omega_{1, \pi_0} 
    + \Omega^{\mathrm{div}}_{1, \pi_+} + \Omega^{\mathrm{div}}_{1, \pi_-}
    + \Omega^{\mathrm{fin}}_{1, \pi_+} + \Omega^{\mathrm{fin}}_{1, \pi_-}.
\end{equation}
