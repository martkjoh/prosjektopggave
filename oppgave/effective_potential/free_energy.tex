\subsection{Free energy of the pions}
The action of \chpt is 
\begin{equation}
    S[\pi_a] =\int \dd x \Ell[\pi_a]
    = 
    \int \dd x \left(
        \Ell^{(0)}_2[\pi_a] +\Ell^{(1)}_2[\pi_a] + \Ell^{(2)}_2[\pi_a] + \Ell^{(2)}_4[\pi_a] 
        + \Ell_I[\pi_a] 
        \right),
\end{equation}
where $\Ell_I[\pi_a] $ is the higher order terms.
The action may be expanded around $\pi_a = 0$,
\begin{equation}
    S[\varphi]
    = 
    S[\pi_a = 0] 
    + \int \dd x \, \pi_a  \frac{\delta S}{\delta \pi_a(x)}\Big|_{\pi_a=0}
    + \int \dd x \dd y \, \pi_a(x) \pi_b(y)
    \frac{\delta^2 S}{\delta \pi_a(x) \delta \pi_a(y)}\Big|_{\pi_a=0}
    + \Oh[3]{(\pi/f)}.
\end{equation}
We showed that, when minimizing $\alpha$, $\frac{\delta S_2}{\delta \pi_a(x)}\Big|_{\pi_a=0} = 0$. 
Furthermore, $\Ell_2[\pi_a = 0] = \Ell_2^{(0)}$. 
% The lowest order contribution to the pion is given by the free propagator, \autoref{free pion propagator}, by the formula \autoref{free energy from propagator}.
% This is, however, evaluated in the imaginary-time formalism.
% In the zero-temperature limit, 
% This means that the time coordinate is replaced by $t \rightarrow - \tau$, and is restricted to $\tau \in [0, \beta]$.
% This result in a discrete set of energies, the Matsubara frequencies $\omega = 2 \pi n / \beta$.
% The result is
% \begin{equation}
%     \beta \Ef = \frac{1}{2V} \Tr{\ln[\beta^2 D_0^{-1}]} 
%     = \frac{1}{2} \int_{\tilde \Omega} \dd K \, \sum_a \ln[\beta^2 D_0^{-1}]_{aa}
% \end{equation}
In the limit $\beta = \infty$, the free energy density of the system is given by $ \beta \Ef
= \frac{i}{T V} \ln(Z)$ (HVORFOR?). 
Using the expansion as described in \autoref{Effective potential}, the free energy density to and including second order is given by 
\begin{equation}
    \label{Free energy to second order}
    \Omega = 
    \beta \Ef
    = \frac{i}{T V} \ln(Z)
    = S[\pi_a = 0]
    +\ln{\det\left( - \frac{\delta^2 S[\pi_a=0]}{\delta \pi^2} \right)}^{-1/2}.
\end{equation}
As the ground sate is $\pi_a = 0$, the only terms that will remain in the trace-terms are those that are second order in the pion-fields.
We can therefore evaluate the functional derivative with the rules given in \autoref{section:Functional derivative},
\begin{align}
    \frac{\delta^2 S }{\delta \varphi(x)\delta \varphi(y)}
    = \frac{\delta^2 }{\delta \varphi(x)\delta \varphi(y)} 
    \int \dd x \, \Ell^{(2)}_2
    = D^{-1}_x \delta(x - y),
\end{align}
where $\Ell^{(2)}_2$ is the quadratic part of the Lagrangian, as given in \autoref{quadratic lagrangian}, and $D^{-1}_x$ is the corresponding inverse propagator of the pion-fields,
\begin{equation}
    D_x^{-1} = 
    - \left[
        \delta_{ab}(\partial_x^\mu\partial_{x,\mu} + m^2_a)
        -  m_{12}(\delta_{a1} \delta_{b2} - \delta_{a2}\delta_{b1}) \partial_{x, 0}
    \right] 
\end{equation}
The first order contribution is
\begin{equation}
    \Omega_0 = \frac{1}{VT}\int \dd^4 x \, \Ell_2^{(0)} = \Ell_2^{(0)}.
\end{equation}
The leading order correction, $\Omega_2$, is given by the functional determinant.
The functional determinant in \autoref{Free energy to second order} has a matrix part, due to the 3 pion-indices, as well as a functional part.
In \autoref{section:propagator} we found the matrix part of  the determinant in momentum space, which we can write using the dispersion relations of the pion fields as
\begin{equation}
    \det(- D^{-1} \delta (x - y)) = \det(p_0^2 - E_0^2) \det(p_0^2 - E_+^2) \det(p_0^2 - E_-^2).
\end{equation}
The Dirac-delta is removed by using partial integration.
The functional determinant can therefore be evaluated as
\begin{align}
    -\frac{1}{2} \ln{\det\left( - \frac{\delta^2 S[\pi_a=0]}{\delta \pi^2} \right)}
    & = \ln \det(p_0^2 - E_0^2) + \ln \det(p_0^2 - E_+^2) + \ln \det(p_0^2 - E_-^2) \\
    & = \Tr{ \ln(p_0^2 - E_0^2) + \ln(p_0^2 - E_+^2)+  \ln(p_0^2 - E_-^2) } \\
    & = (VT) \int \frac{\dd^4 p}{(2 \pi)^4} 
    \left[ \ln(p_0^2 - E_0^2) + \ln(p_0^2 - E_+^2) + \ln(p_0^2 - E_-^2)  \right].
\end{align}
These all have the form
\begin{equation}
    I = \int \frac{\dd^4 p}{(2 \pi)^2} \ln(-p^2 + \omega(p)^2).
\end{equation}
We will here pereform a Wick-rotation, and use the trick
\begin{equation}
    \pdv{\alpha} (p^2 + \omega(p)^2)^{-\alpha} \Big|_{\alpha=0}
    = \pdv{\alpha} \exp{-\alpha \ln(p^2 + \omega(p)^2) } \Big|_{\alpha=0}
    = \ln(p^2 + \omega(p)^2),
\end{equation}
to write the integral on the form
\begin{equation}
    I = \pdv{\alpha} \int \frac{\dd^4 p_E}{(2 \pi)^4} (p_E^2 + \omega(p)^2)^{-\alpha} \Big|_{\alpha=0}
\end{equation}

