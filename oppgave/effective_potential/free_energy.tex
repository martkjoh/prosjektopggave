\subsection{Free energy of the pions}
The equation of state (EOS) dictates the thermodynamic properties of a system.
In this section, we will calculate the equation of state of the pions by calculating their free energy.
We use the effective Lagrangian found in \autoref{serction:effective_pion_lagrangian} to find the leading-order contribution to one loop, and the next-to-leading order contribution at tree level, following the procedure used in~\cite{mojahed, Andersen:two-flavor-chpt}.
\footnote{Leading order and next-to-leading order, in this context, refers to Weinberg's power counting scheme.}
The free energy density at zero temperature is given by
\begin{equation}
    \Ef = \frac{i}{VT} \ln Z,
\end{equation}
where $Z = Z[J=0]$ is the vacuum transition amplitude, as described in \autoref{section:path integral}, and $VT$ is the volume of space-time.
This equals the effective potential in the ground state, which we found an explicit formula for in \autoref{section: effective action}, \cref{effective potential}.
We write $\Ef = \Ef^{(0)} + \Ef^{(1)} + \dots$, where $\Ef^{(n)}$ refers to the n-loop contributions to the free energy density.
The tree-level contribution $\Ef^{(0)}$ is the classical potential, which is given by the static ($\pi$-independent) part of the Lagrangian.
From \cref{L0} we have the leading order contribution,
\begin{equation}
    \Ef_2^{(0)}
    = - \Ell_2^{(0)} 
    = 
    -f^2   
    \left(
        2B_0m \cos{\alpha}
        + \frac{1}{2} \mu^2 \sin^2{\alpha}
    \right).
\end{equation}
$\alpha$ parameterize the ground state, which means that it's value must minimize the free energy,
\begin{align*}
    &\dv{\alpha} \Ef_2^{(0)} 
    = f^2\left(2B_0m - \mu_I^2\cos{\alpha}\right)\sin{\alpha}
    = 0.
\end{align*}
This gives the criterion
\begin{align}
    \label{leading order minization}
    \alpha \in \{0, \pi\} \quad
    \mathrm{or} \quad
    \cos{\alpha} = \frac{2B_0m}{\mu_I{}^2}.
\end{align}
In our discussion of the effective potential we also found that the ground state should minimize the classical potential, as shown by \cref{minimize classical potential}.
This means that the linear part of the classical potential should vanish.
The linear part of the classical potential is given by \autoref{L1} to leading order, and reads $\Ve^{(1)} = f(\mu_I{}^2\cos{\alpha} - 2B_0m)\sin{\alpha} \, \pi_1 $, which vanishes given \cref{leading order minization}.

The one loop contribution to the free energy density is
\begin{equation}
    \label{one loop free energy}
    \Ef^{(1)}
    = - \frac{i}{V T} \frac{1}{2}
    \Tr{\ln\left( -\fdiff{S[\pi = 0]}{\pi_a(x), \pi_b(y)} \right)}.
\end{equation}
This can be evaluated using the rules for functional differentiation given in \autoref{section:Functional derivative}.
To leading order, 
\begin{align}
    \fdiff{S[\pi = 0]}{\pi_a(x), \pi_b(y)}
    = \fdiff{}{\pi_a(x), \pi_b(y)}
    \int \dd^4 x \, \Ell^{(2)}_2
    = D^{-1}_x \delta(x - y).
\end{align}
Here, $\Ell^{(2)}_2$ is the quadratic part of the Lagrangian, as given in \autoref{quadratic lagrangian}, and $D^{-1}_x$ is the corresponding inverse propagator of the pion fields,
\begin{equation}
    D_x^{-1} = 
    - \left[
        \delta_{ab}(\partial_x^\mu\partial_{x,\mu} + m^2_a)
        -  m_{12}(\delta_{a1} \delta_{b2} - \delta_{a2}\delta_{b1}) \partial_{x, 0}
    \right] 
\end{equation}
The functional determinant in \autoref{one loop free energy} has a matrix part, due to the three pion indices, as well as a functional part.
In \autoref{section:propagator} we found the matrix part of  the determinant in momentum space, which we can write using the dispersion relations of the pion fields
\begin{equation}
    \det(- D^{-1}) = \det(p_0^2 - E_0^2) \det(p_0^2 - E_+^2) \det(p_0^2 - E_-^2).
\end{equation}
These dispersion relations are functions of the three-momentum $\vec p$, and are given in \cref{dispresion relation pi 0,dispresion relation pi pm}.
The functional determinant can therefore be evaluated as
\begin{align}
    \nonumber
    \Tr{\ln\left( -\fdiff{S[\pi = 0]}{\pi_a(x), \pi_b(y)} \right)}
    & = \ln \det(p_0^2 - E_0^2) + \ln \det(p_0^2 - E_+^2) + \ln \det(p_0^2 - E_-^2) \\
    \nonumber
    & = \Tr{ \ln(p_0^2 - E_0^2) + \ln(p_0^2 - E_+^2)+  \ln(p_0^2 - E_-^2) } \\
    & = (VT) \int \frac{\dd^4 p}{(2 \pi)^4} 
    \left[ \ln(p_0^2 - E_0^2) + \ln(p_0^2 - E_+^2) + \ln(p_0^2 - E_-^2)  \right],
\end{align}
where we have used the identity $\ln\det M = \Tr \ln M $.
These terms all have the form
\begin{equation}
    I = \int \frac{\dd^4 p}{(2 \pi)^2} \ln(-p_0^2 + E^2),
\end{equation}
where $E$ is some function of the 3-momentum $\vec p$, but not $p_0$.
We use the trick
\begin{equation}
    \pdv{\alpha} \left(-p_0^2 + E^2\right)^{-\alpha} \Big|_{\alpha=0}
    = \pdv{\alpha} \exp\left[ -\alpha \ln\left(- p_0^2 + E^2\right)  \right] \Big|_{\alpha=0}
    = \ln\left(- p_0^2 + E^2\right),
\end{equation}
and then preform a Wick-rotation of the $p_0$-integral to write the integral on the form
\begin{equation}
    I = i \pdv{\alpha} \int \frac{\dd^4 p}{(2 \pi)^4} \left(p_0^2 + E^2\right)^{-\alpha} \Big|_{\alpha=0},
\end{equation}
where $p$ now is a Euclidean four-vector.
The $p_0$ integral equals $\Phi_1(E, 1, \alpha)$, as defined in \autoref{def dimreg integral}. 
The result is therefore given by \autoref{result dimreg},
\begin{equation}
    \int \frac{\dd p_0}{2 \pi} (p_0^2 + E)^{-\alpha} 
    = \frac{E^{1-2\alpha}}{\sqrt{4 \pi}} \frac{\Gamma(\alpha-\frac{1}{2})}{\Gamma(\alpha)}.
\end{equation}
The derivative of the Gamma function is $\Gamma'(\alpha) = \psi(\alpha)\Gamma(\alpha)$, where $\psi(\alpha)$ is the digamma function.
Using
\begin{align}
    \diffp{}{\alpha} & \frac{\Gamma(\alpha - \frac{1}{2}) }{\Gamma(\alpha)} \Big|_{\alpha=0}
    = \Gamma\left(\alpha - \frac{1}{2}\right) \frac{\psi(\alpha - \frac{1}{2}) - \psi(\alpha)}{\Gamma(\alpha)} \Big|_{\alpha=0}
    = \sqrt{4 \pi}, \\
    & \frac{\Gamma(\alpha - \frac{1}{2}) }{\Gamma(\alpha)}\Big|_{\alpha=0} = 0,
\end{align}
we get
\begin{equation}
    I = i \int \frac{\dd^3 p}{(2 \pi)^3} E.
\end{equation}
We see that the is what we would expect physically, the total energy is the integral of the energy of each mode.
This results in 
\begin{equation}
    \Ef^{(1)} = 
    \frac{1}{2} 
    \left[\int \frac{\dd^3 p}{(2\pi)^3} E_0 + \int  \frac{\dd^3 p}{(2\pi)^3} (E_+ + E_-)\right]
    = \Ef^{(1)}_{\pi_0} +\Ef^{(1)}_{\pi_+} + \Ef^{(1)}_{\pi_-}.
\end{equation}
The first integral is identical to what we find for a free field in \autoref{section:free scalar field}, in the low temperature limit $\beta \rightarrow \infty$.
These terms are all divergent, and must be regulated. 
We will use dimensional regularization, in which the integral is generalized to $d$ dimensions, and the $\overline{\mathrm{MS}}$-scheme, as described in \autoref{section: regualting free energy}.
Using the result for a free field \cref{free field regularized energy}, we get
\begin{equation}
    \Ef^{(1)}_{\pi_0} 
    = 
    - \frac{1}{4} \frac{m_3^4}{(4\pi)^2} 
    \left[ \frac{1}{\epsilon} + \frac{3}{2} + \ln(\frac{\mu^2}{m_3^2}) \right] + \mathcal{O}(\epsilon).
\end{equation}
$\mu$ is the renormalization scale, which is introduced ensure that the integral has the same physical dimension for $d \neq 3$.

The contribution to the free energy from the $\pi_+$ and $\pi_-$ particles is more complicated, as the dispersion relation is given by
\begin{equation}
    E_\pm
    = 
    \sqrt{
        |\vv p|^2 +
        \frac{1}{2}
        \left(
            m_1^2 + m_2^2 + m_{12}^2 
        \right)
        \pm 
        \frac{1}{2}
        \sqrt{
            4|\vv p|^2m_{12}^2 
            +
            \left(
                m_1^2 + m_2^2 + m_{12}^2
            \right)^2
            - 4 m_1^2 m_2^2
        }
    }.
\end{equation}
This is not an integral we can easily do in dimensional regularization.
Instead, we will seek a function $f(\vv p)$ with the same UV-behavior, that is behavior for large $\vv p$, as $E_+ + E_-$.
If we then add $0 = f(\vv p) - f(\vv p)$ to the integrand, we can isolate the divergent behavior
\begin{equation}
    \Ef_{\pi_\pm}^{(1)}
    = 
    \frac{1}{2} \int \frac{\dd^3 p}{(2\pi)^3} [E_+ + E_- + f(\vv p) - f(\vv p)]
    = \Ef^{(1)}_{\mathrm{fin}, \pi_\pm } + \Ef^{(1)}_{\mathrm{div}, \pi_\pm}.
\end{equation}
This results in a finite integral, 
\begin{equation}
    \Ef^{(1)}_{\mathrm{fin}, \pi_\pm } = \frac{1}{2} \int \frac{\dd^3 p}{(2\pi)^3} [E_+ + E_- - f(\vv p)],
\end{equation}
and isolate the divergence to
\begin{equation}
    \Ef^{(1)}_{\mathrm{div}, \pi_\pm }
    = 
    \frac{1}{2} \int \frac{\dd^3 p}{(2\pi)^3} f(\vv p),
\end{equation}
which we hopefully will be able to do in dimensional regularization.
We can explore the UV-behavior of $E_+ + E_-$ by expanding it in powers of $1 / \abs{\vv{p}}$,
\begin{align}
    \nonumber
    E_+ + E_-
    & = 
    2  \abs{\vv{p}}
    + \frac{m_{12} + 2(m_1^2 + m_2^2)}{4} \, {\abs{\vv{p}}}^{-1}
    - \frac{m_{12}^4 + 4 m_{12}^2(m_1^2 + m_2^2) + 8(m_1^4 + m_2^4)}{64}
    {\abs{\vv{p}}}^{-3}
    + \Oh[-5]{ \abs{\vv{p}}} 
    \\
    & = 
    a_1  \abs{\vv{p}}
    + a_2 \, {\abs{\vv{p}}}^{-1}
    + a_3
    {\abs{\vv{p}}}^{-3}
    + \Oh[-5]{ \abs{\vv{p}}}.
\end{align}
We have defined the new constants $a_i$ for brevity of notation.
This is the divergent part of the integral, so $f(\vv p)$ must match this expansion to and including $\Oh[-3]{|\vv{p}|}$ for $\Ef^{(1)}_{\mathrm{fin}, \pi_\pm }$ to be finite.
The most obvious choice for $f$ is
\begin{equation}
    f(\vv p) 
    = a_1  \abs{\vv{p}} + a_2 \, {\abs{\vv{p}}}^{-1} + a_3 \, {\abs{\vv{p}}}^{-3}.
\end{equation}
(TODO: HVORFOR IKKE BRUKE DETTE; IR DIVERGENS)

We will instead regularize this integral by defining $E_i = \sqrt{|\vv{p}|^2 + \tilde m_i^2}$, and $\tilde m_i^2 = m_i^2 + \frac{1}{4} m_{12}^2$.
$E_1 + E_2$ then has the same UV-behavior as $E_+ + E_-$, as they differ by $\Oh[-5]{|\vv p|}$.
This results in another integral of the same for as in the case of the free scalar.
The divergent part of the free energy density of the $\pi_\pm$-particles is therefore, in the $\mathrm{\overline{MS}}$-scheme, 
\begin{equation}
    \Ef^{(1)}_{\mathrm{div}, \pi_\pm }
    = 
    - \frac{1}{4}
    \left\{
        \frac{\tilde m_1^4}{(4\pi)^2} 
        \left[
            \frac{1}{\epsilon} + \frac{3}{2} + \ln(\frac{\mu^2}{\tilde m_1^2}) 
        \right]
        +
        \frac{\tilde m_2^4}{(4\pi)^2} 
        \left[
            \frac{1}{\epsilon} + \frac{3}{2} + \ln(\frac{\mu^2}{\tilde m_2^2})
        \right] 
    \right\} 
    + \mathcal{O}(\epsilon).
\end{equation}
We define
\begin{equation}
    \Ef^{(1)}_{\mathrm{fin}, \pi_\pm}
    = 
    \frac{1}{2} \int \frac{\dd^3 p}{(2\pi)^3} (E_+ + E_- - E_1 - E_2),
\end{equation}
which is a finite integral.
The total free energy density, to one loop and at next-to-leading order, is thus
\begin{equation}
    \Ef = 
    \Ef^{(0)}
    + \Ef^{(1)}_{\pi_0} 
    + \Ef^{(1)}_{\mathrm{fin}, \pi_\pm}
    + \Ef^{(1)}_{\mathrm{div}, \pi_\pm}
    + \dots.
\end{equation}
