\subsection{The 1PI effective action}
\label{section: effective action}
The generating functional for connected diagrams, $W[J]$, is dependent on the external current $J$.
Analogously to what is done in thermodynamics and Lagrangian and Hamiltonian mechanics, we can define a new quantity, with a different independent variable, using the Legendre transformation.
The new independent variable is 
\begin{equation}
    \frac{\delta W[J]}{\delta J(x)} = \ex{\varphi(x)}_J := \varphi_J(x).
\end{equation}
The subscript $J$ on the expectation value indicate that it is evaluated in the presence of a current.
The Legendre transformation of $W$ is then
\begin{equation}
    \Gamma[\varphi_J]
    = W[J] - \int \dd^4 x \, J(x) \varphi_J(x).
\end{equation}
Using the definition of $\varphi_J$, we have that
\begin{equation}
    \label{effective equation of motion}
    \fdv{\varphi_J(x)} \Gamma[\varphi_J]
    = \int \dd^4 y \, \fdv{J(y)}{\varphi_J(x)} \fdv{J(y)} W[J]
    - \int \dd^4 y \, \fdv{J(y)}{\varphi_J(x)} \varphi_J(y)
    - J(x)
    = - J(x).
\end{equation}
If we compare this to the classical equations of motion of a field $\varphi$ with the action $S$,
\begin{equation}
    \frac{\delta S[\varphi]}{\delta \varphi(x)} = -J(x),
\end{equation}
we see that $\Gamma$ is an action that gives the equation of motion for the expectation value of the field, given a source current $J(x)$.

To interpret $\Gamma$ further we define a new theory where it is the action, and with a coupling $g$, 
\begin{equation}
    \label{partition function with g}
    Z[J, g] = \int \D \varphi 
    \exp{ i g^{-1} \left( \Gamma[\varphi] + \int \dd^4x \varphi(x) J(x) \right) }
\end{equation}
The free propagator in this theory will be proportional to $g$, as it is given by the inverse of the equation of motion for the free theory.
All vertices in this theory, on the other hand, will be proportional to $g^{-1}$, as they are given by the higher order terms in the action $g^{-1}\Gamma$.
This means that a diagram with $V$ vertices and $I$ internal lines is proportional to $g^{I-V}$.
Regardless of what the Feynman-diagrams in this theory are, the number of loops of a connected diagram is $L = I - V + 1$.
\footnote{This is a consequence of the Euler characteristic $\chi = V - E + F$.}
To see this, we first observe that one single loop must have equally many internal lines as vertices, so the formula holds for $L = 1$.
If we add a new loop to a diagram with $n$ loops by joining two vertices, the formula still holds.
If we attach a new vertex with one line, the formula still holds, and as the diagram is connected, any more lines connecting the new vertex to the diagram will create additional loops.
This ensures that the formula holds, by induction.
As a consequence of this, any diagram is proportional to $g^{L-1}$.
This means that in the limit $g \rightarrow 0$, the theory is fully described at the tree-level diagrams, i.e. by only considering diagrams without loops.
In this limit, we may use the stationary phase approximation, as described in \autoref{section:gaussian integrals}, which gives
\begin{equation}
    Z \approx 
    C \det(- \frac{\delta^2 \Gamma[\varphi_J]}{\delta \varphi^2})
    \exp{i g^{-1} \left(\Gamma[\varphi_J] + \int \dd^4x J \varphi_J \right)  },
\end{equation}
This means that
\begin{equation}
    -i g \ln(Z[J]) 
    = g^{-1} W[J] 
    = \Gamma[\varphi_J] + \int \dd^4x\,  J(x) \varphi_J(x) + \mathcal{O}(g),
\end{equation}
which is exactly the Legendre transformation we started out with, modulo the factor $g$.
$\Gamma$ is therefore the action which describes the full theory at the three level.
For a free theory, the classical action $S$ equals the effective action, as there are no loop diagrams.

We showed in \autoref{correlation function} that the propagator $\ex{\varphi(x)\varphi(y)}_J$ is given by functional derivatives of $W$.
Using the chain rule, together with \autoref{effective equation of motion}, we get
\begin{align}
    \int \dd^4 z \frac{\delta^2 W[J]}{\delta J(x) \delta J(z)} 
    \frac{\delta^2 \Gamma[\varphi_J]}{\delta \varphi_J(z) \varphi_J(y)}
    =
    \int \dd^4 z \frac{\delta \varphi_J[z]}{\delta J(x)}
    \frac{\delta^2 \Gamma[\varphi_J]}{\delta \varphi_J(z) \varphi_J(y)}
    = \frac{\delta^2 \Gamma[\varphi_J]}{\delta J(x) \varphi_J(y)}
    = - \delta(x - y).
\end{align}
This shows that the inverse propagator is given by the effective action, which is the sum of all one-particle-irreducible (1PI) diagrams with two external vertices.
More generally, $\Gamma$ is the generating functional for 1PI diagrams.

The expectation value of the ground state field configuration, that is at $J=0$, is given by 
\begin{equation}
    \frac{\delta \Gamma[\varphi_0]}{\delta \varphi(x)} = 0.
\end{equation}
Assuming this configuration is independent of $x$, we can write
\begin{equation}
    \Gamma[\varphi_0] = - V T \Ve_{\mathrm{eff}}(\varphi_0),
\end{equation}
where $\Ve_{\mathrm{eff}}$ is the effective potential.
$V$ is the volume of space, while $T$ is the time duration of our space-time.
