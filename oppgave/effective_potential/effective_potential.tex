\subsection{Minimizing energy}
%%%%%%%%%%%%%%%%%
%%%% SECTION %%%%
%%%%%%%%%%%%%%%%%
The value of $\alpha$ is found by minimizing the free energy. 
The first approximation to the free energy in the ground state is the static part of the Hamiltonian density $\He^{(0)}$, which we get from \autoref{L0} through
\begin{equation}
    \He_2^{(0)} = - \Ell_2^{(0)} = 
    -f^2   
    \left(
        2B_0m \cos{\alpha}
        + \frac{1}{2} \mu^2 \sin^2{\alpha}
    \right),
\end{equation}
The minimum of this function is achieved when
\begin{align*}
    &\dv{\alpha} \He_2^{(0)} 
    = f^2\left(2B_0m - \mu_I^2\cos{\alpha}\right)\sin{\alpha}
    = 0.
\end{align*}
This gives the solution set and minimization criterion
\begin{align}
    \alpha = \pi n, \, n \in \mathbb{Z} \quad
    \mathrm{or} \quad
    \cos{\alpha} = \frac{2B_0m}{\mu_I{}^2}.
\end{align}
We see that the linear part of the potential from \autoref{L1}, $\Ve^{(1)} = f(\mu_I{}^2\cos{\alpha} - 2B_0m)\pi_1 \sin{\alpha}= 0$ if and only if the criterion for minimization is fulfilled.

\subsection{Propagator}
%%%%%%%%%%%%%%%%%
%%%% SECTION %%%%
%%%%%%%%%%%%%%%%%

We may write the quadratic part of the Lagrangian \autoref{L2} as \footnote{Summation over isospin index ($a,b,c$) will be explicit in this section.}
\begin{align}
    \Ell^{(2)}
    =
    \frac{1}{2} \sum_a \partial_\mu \pi_a \partial^\mu \pi_a
    + \frac{1}{2} m_{12} (\pi_1 \partial_0 \pi_2 - \pi_2 \partial_0 \pi_1)
    - \frac{1}{2} \sum_a m_a^2 \pi_a^2,
\end{align}
where
\begin{align}
    m_1^2 &= 2 B_0 m \cos{\alpha} - \mu_I^2 \cos{2\alpha}, \\
    m_2^2 &= 2 B_0 m \cos{\alpha} - \mu_I^2 \cos^2{\alpha}, \\
    m_3^2 &= 2 B_0 m \cos{\alpha} + \mu_I^2 \sin^2{\alpha}, \\
    m_{12} &= 2 \mu_I \cos{\alpha}.
\end{align}
The components of the Euler-Lagrange equations of this field are
\begin{equation*}
    \pdv{\Ell}{\pi_a} = 
    \frac{1}{2} m_{12} (\delta_{a1} \partial_0 \pi_2 - \delta_{a2}\partial_0 \pi_1) 
    - m^2_{a} \pi_a, \quad
    \pdv{\Ell}{(\partial_\mu \pi_a)} = 
    \partial^\mu \pi_a - \frac{1}{2} m_{12} \delta^\mu_0 (\delta_{a1}\pi_2  - \delta_{a2}\pi_1).
\end{equation*}
This gives the equation of motion for the field
\begin{equation}
    \partial^\mu \partial_\mu \pi_a + m_a^2 \pi_a
    =  m_{12}(\delta_{a1} \partial_0 \pi_2  - \delta_{a2} \partial_0 \pi_1).
\end{equation}
The propagator of the pion field is defined by
\begin{equation}
    \left[
        \delta_{ab}(\partial^\mu\partial_\mu + m^2_a)
        -  m_{12}(\delta_{a1} \delta_{b2} - \delta_{a2}\delta_{b1}) \partial_0
    \right] 
    D_{bc}(x, x') 
    = -i \delta(x - x') \delta_{ac}.
\end{equation}
The momentum space propagator, as defined in the \autoref{Conventions and notation}, fulfills
\begin{equation*}
    -\left[
        \delta_{ab}(p^2 - m_a^2)
        +  i p_0 m_{12}(\delta_{a1} \delta_{b2} - \delta_{a2}\delta_{b1}) 
    \right] 
    \tilde D_{bc}(p) 
    := A_{ab} \tilde D_{bc}(p) = -i \delta_{ac},
\end{equation*}
where
\begin{equation*}
    A = -
    \begin{pmatrix}
        p^2 - m^2_1             & i p_0 m_{12}     & 0             \\
        - i p_0 m_{12}            & p^2 - m^2_2       & 0             \\
        0                       & 0                 & p^2 - m^2_3
    \end{pmatrix}.
\end{equation*}
The spectrum of the particles is given by solving $\det(A) = 0$ for $p^0$. With $p = (p_0, P)$ as the four momentum, this gives
\begin{align*}
    \det(A) & = A_{33} \left(A_{11} A_{22} + A_{12}^2\right)
    = - \left(p^2 - m^2_3\right)
    \left[
        \left(p^2 - m^2_1\right)
        \left(p^2 - m^2_2\right)
        - p_0^2 m_{12}^2
    \right] = 0,
\end{align*}
This equation has the solutions
\begin{align}
    E_0^2 &= P^2 + m_2^2, \\
    E_\pm^2
    & = P^2 +
    \frac{1}{2}
    \left(
        m_1^2 + m_2^2 + m_{12}^2 
    \right)
    \pm 
    \frac{1}{2}
    \sqrt{
        4P^2m_{12}^2 
        +
        \left(
            m_1^2 + m_2^2 + m_{12}^2
        \right)^2
        - 4 m_1^2 m_2^2
    }.
\end{align}
This gives the effective masses
\begin{align}
    m_0^2 &= m_2^2, \\
    m_\pm^2
    & =  \frac{1}{2}
    \left[
        m_1^2 + m_2^2 + m_{12}^2 
    \right]
    \pm \frac{1}{2}
    \sqrt{
        \left(
            m_1^2 + m_2^2 + m_{12}^2
        \right)^2
        - 4 m_1^2 m_2^2
    }.
\end{align}
The propagator may then be obtained as described in \autoref{Conventions and notation},
\begin{align}
    \notag
    D & = i A^{-1} = \frac{i}{\det(A)}
    \begin{pmatrix}
        A_{22} A_{33}   & A_{12}A_{33}  & 0 \\
        -A_{12}A_{33}   & A_{11}A_{33}  & 0 \\
        0               & 0             & A_{11}A_{22} + A_{12}^2
    \end{pmatrix} \\
    & = i
    \begin{pmatrix}
        \frac{
            p^2 - m_2^2
        }
        {
            (p_0^2 - E_+^2)(p_0^2 - E_-^2)
        } 
        & \frac{
            - ip_0m_{12}
        }
        {
            (p_0^2 - E_+^2)(p_0^2 - E_-^2)
        } & 0 \\
        \frac{
            ip_0m_{12}
        }
        {
            (p_0^2 - E_+^2)(p_0^2 - E_-^2)
        }
        & \frac{
            p^2 - m_1^2
        }
        {
            (p_0^2 - E_+^2)(p_0^2 - E_-^2)
        } & 0 \\
        0 & 0 & 
        \frac{1}{p_0^2 - E_0^2}
    \end{pmatrix}.
\end{align}
