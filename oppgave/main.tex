\documentclass{book}

\usepackage[left=2.5cm, right=2.5cm, top=2cm, bottom=2cm]{geometry}

\usepackage{amsmath}
\usepackage{amsfonts}
\usepackage{amssymb}
\usepackage{physics}
\usepackage{slashed}
\usepackage{hyperref}
\usepackage[title]{appendix}
\usepackage[intlimits]{mathtools}
\usepackage[export]{adjustbox}
\usepackage{esvect}
\usepackage[capitalize]{cleveref}


\usepackage{pgfplots}
\pgfplotsset{compat=1.16}

\usepackage{tikz}
\usepackage[compat=1.1.0]{tikz-feynman}

\usepackage{caption}
\usepackage{subcaption}

\usepackage[ISO]{diffcoeff} % Get straight d's when differantiating

\usepackage{tocloft}    % Nice table of contents

\usepackage[section]{placeins} % Get floats right
\usepackage[super]{nth}


\setlength\cftparskip{1pt}
\setlength\cftbeforechapskip{0pt}

\setlength{\parindent}{0em}
\setlength{\parskip}{0.8em}

\def\equationautorefname~#1\null{Eq.~(#1)\null}

% Simple shortcuts
\newcommand{\Ell}{\mathcal{L}}      % Lagrangian L
\newcommand{\He}{\mathcal{H}}       % Hamiltonian H
\newcommand{\Ve}{\mathcal{V}}       % Potential V
\newcommand{\Em}{\mathcal{M}}       % Manifold M
\newcommand{\Ef}{\mathcal{F}}       % Fancy f
\newcommand{\R}{\mathbb{R}}         % Real numbers
\newcommand{\chpt}{$\chi$PT }       % Chiral pertubation theory
\newcommand{\SU}{\mathrm{SU}}       % SU(n)
\newcommand{\eps}{\varepsilon}      % nice epsilon
% \newcommand{\one}{\mathbb{1}}       % Identity
\newcommand{\hc}{\mathrm{h.c.}}     % Hermitian conjugate
\newcommand{\ex}[1]{\expectationvalue{#1}}
\newcommand{\D}{\mathcal{D}}

\newcommand{\one}{\text{\usefont{U}{bbold}{m}{n}1}}
\MakeRobust{\one}

% Big-O notation 
\newcommand{\Oh}[2][2]{\mathcal{O}\left(#2^{#1}\right)}

% (anti) commutator
\newcommand{\com}[2]{\left[#1, #2 \right]}
\newcommand{\acom}[2]{\left\{#1, #2 \right\}}

% Lie algebra
\newcommand{\liea}[2]{\mathfrak{#1}\left(#2\right)}
\newcommand{\lieg}[2]{\mathrm{#1}\left(#2\right)}

% Curly brackets
\newcommand{\curly}[1]{\left\{ #1 \right\}}

% Fourier Transform
\newcommand{\F}[1]{\mathcal{F}\curly{#1}}
\newcommand{\FInv}[1]{\mathcal{F}^{-1}\curly{#1}}

% operator in braket
\newcommand{\inner}[3]{\left\langle #1 {\left| #2 \right|} #3 \right\rangle}

\newcommand{\T}[1]{\textrm{T} \left\{ #1 \right\}}

% \usepackage{glossaries}

\makenoidxglossaries

\newglossaryentry{chi}
{
    name={$\chi$},
    description={Free parameter in the first order chiral Lagrangian. Related to the mass of the pion}
}


\title{Chiral Perturbation Theory}
\author{Martin Johnsrud}

%%%%%%%%%%%%%%%%%%%%
%%%%%%% TODO %%%%%%%
%%%%%%%%%%%%%%%%%%%%

%%%%%%%%%%%%%%%%%%%%%%%%%%%%%%%%%%%%
%%%%%%%%%%%%% SPØRSMÅL %%%%%%%%%%%%%
%%%%%%%%%%%%%%%%%%%%%%%%%%%%%%%%%%%%
% Hvor kommer EOM inn i bildet? Det ser ut som det bare er et navn...
% Hvorfor kan vi bruke EOM fra en annen parametrisering?


%%%%% INTRO
% Skrive abstrakt


%%%%% Teori
% Flytt opp diskusjon om Nöthers theorme fra appendix
% Fiks figur

%%%%% Pions
% QCD: rydd, forklar baryontall/ladning (Hvorfor er det ikke det samme?)
% CHPT: Konstruer Cartan-maurer, 
%       leading order lagrangian, 
%       finn transformasjonsegenskaper
%       Kjemisk potensiale
%       Forklar parametrisering
%       Nevne at vi ser bort fra WZW-termer
% LO: inkluder \delta m?
% EOM: Fiks
% Finn correspondansen mellom vår pioner og de ladde pionene

%%%%% Termisk
% pass på å bruk tilde \mu

%%%% Konklusjon
% Skriv konklusjon

%%%%% Thermal field theory
% Hva er vits å ha med?
% Flytte første kappittel til teori+thermo
% Ha \ex{...}_0 (subscript) over alt i interacting scalar

%%%%% Generelt
% Lingningsnummer --- når?
% Overskrifter --- Stor bokstav?
% være konsekevent på parantes {[()]}
% Alltid bruke cleverref



\begin{document}
\maketitle 

\tableofcontents

\chapter*{Abstract}
Pions are particles that describe the dynamics of QCD at low energies, and it has recently been proposed that they form compact stellar objects called pion stars.
We use two-flavor chiral perturbation theory to calculate the grand canonical free energy density to next-to-leading order, at $T = 0$ and with non-zero isospin chemical potential $\mu_I$.
At $\mu_I = m_\pi$, a pion condensate is formed and spontaneously breaks the isospin symmetry of the QCD Lagrangian.
We observe the resulting Goldstone mode.
The pion condensate phase is characterized by a non-zero isospin density.
We discuss the nature of the phase transition using Landua-theory of second-order phase transitions.
The free energy density is used to obtain the relationship between the pressure and the energy density of the system, the equation of state.
This, together with the Tolman-Oppenheimer-Volkoff equation, can be used to model pion stars, allowing for further investigation of these newly proposed objects.


% \mainmatter
\chapter{Introduction}
(KLADD, INNTRODUKSJON)

The standard model describes, together with Einstein's general theory of relativity, all experiments we as humans are able to perform.
It uses the language of quantum field theory to describe (X) particles, and their interactions.
Quantum electrodynamics, or QED, is the description of electromagnetic interactions.
It gives some of the most accurate predictions in science (EKSEMPEL).
This is calculated using the techniques of Feynman diagrams, which describes how quantum fields interact as the sum of all possible ways patterns of interaction.
When the interaction is weak, as is the case for QED, we get a sum that converges quickly, and can get highly accurate estimates by calculating a few orders of the sum.
The weakness of QED is quantified in the fine structure constant $\alpha \approx 0.00 7297$.(CITE PDG)
A Feynman diagram in QED is proportional to $\alpha^n$, where $n$ is the number of vertices in the diagram, which means the contributions from more complex diagrams with many vertices rapidly becomes insignificant


Quantum chromodynamics, or QCD, is the part of the standard model that describes quarks, the constituents of protons and neutrons, and how they interact via the strong nuclear force.
When dealing with the strong force, the fact that the strength of interaction depend on the energy scale of the interaction becomes apparent.
In high energy interactions, around $100\, \text{MeV}$ or more, the strong force equivalent to the fine structure constant is $\alpha_s \approx 0.1$. 
This makes it possible to do persuasive calculations using QCD.
However, the strong force has its name for a reason.
For scales around $1 \text{GeV}$ and below, the perturbation method breaks down, and we are no longer able to extract predictions of the theory using perturbation theory, at least not directly.

OUTLINE: 

- Standard model, QCD

- Pion stars: calculation of neutron star, not from first principle. Sign problem. Pion stars recently suggested. Equation of state

- Effective Lagrangians,  Weinberg's theorem, symmetry, chiral perturbation theory.

- Outline of text

\subsection*{Effective field theories}

In \autoref{section: effective action}, we studied the effective action, and found that it gave the equation of motion for the expectation value of the field in the full quantum theory.
Let $\varphi^*(x) = \ex{\varphi(x)}$, and $\varphi(x) = \varphi^*(x) + \eta(x)$.
We can write this as
\begin{equation}
    \exp{i \Gamma[\varphi^*]} = \int \D \eta \, \exp{i S[\varphi^* + \eta]}.
\end{equation}
As, by assumption, $\ex{\eta} = 0$, this only includes 1PI diagrams.
We say that the degree of freedom $\eta$ has been \emph{integrated out}.
More generally, we can integrate out some of the degrees of freedom of a system, to get an effective theory for what is left.
If we have two sets of fields, $\varphi$ and $\psi$, and a Lagrangian $\Ell[\varphi, \psi]$, then the effective theory of the $\varphi$ fields are defined by
\begin{equation}
    \int \D\varphi \D \psi \, \exp{i\int \dd x\, \Ell[\varphi, \psi]}
    = \int \D \varphi \exp{i S_\mathrm{eff}[\varphi]}.
\end{equation}
(EKSEMPLER? WILSON RENORMALISERING; FERMI TEORI)




\chapter{Theory}
\label{chapter:theory}

\label{Effective potential}
\subsection{QFT via path integrals}
This section is based on \cite{Peskin:IntroQFT,weinberg_1995,weinberg_1996_vol2} (Schwarz).
The ground state path integral is given by
\begin{equation}
    Z = \lim_{T\rightarrow \infty} \braket{\Omega, T}{-T, \Omega}
    = \lim_{T\rightarrow \infty} \inner{\Omega}{ e^{-2iHT} }{\Omega}
    = \int \D \pi \D \varphi \, \exp{ i \int \dd^4 x \, \left(\pi \dot \varphi - \He[\pi, \varphi]\right) },
\end{equation}
where $\ket{\Omega}$ is the vacuum of the theory~\cite{Peskin:IntroQFT,weinberg_1995}.
By introducing a source term into the Hamiltonian, $\He \rightarrow \He - J(x)\varphi(x)$, we get the generating functional
\begin{equation}
    Z[J] = 
    \int \D \pi \D \varphi \, 
    \exp{ i \int \dd^4 x \, \left(\pi \dot \varphi - \He[\pi, \varphi]+ J\varphi\right) }
\end{equation}
If $\He$ is quadratic in $\pi$, we can complete the square and integrate out $\pi$ to obtain
\begin{equation}
    Z[J] = C \int \D \varphi \, \exp{i \int \dd^4 x\, (\Ell[\varphi] + J \varphi)}.
\end{equation}
$C$ is infinite, but constant, and will drop out of physical quantities.
$Z[J]$ is called the generating functional as correlation functions $\ex{\varphi(X_1)\varphi(X_2)...} = \inner{\Omega}{T{\varphi(X_1)\varphi(X_2)\dots}}{\Omega}$ are given by functional derivatives of $Z[J]$, 
\begin{equation}
    \label{correlator from generating functional}
    \ex{\varphi(X_1)\varphi(X_2)...}
    = 
    \frac{\int \D \varphi(X)\,  (\varphi(X_1)\varphi(X_2)...) e^{i S[\varphi]}}
        {\int \D \varphi(X)\, e^{i S[\varphi]}}
    =
    \frac{1}{Z[0]} \prod_i\left( -i  \fdv{J(X_i)}\right) Z[J]\Big|_{J = 0},
\end{equation}
where 
\begin{equation}
    S[\varphi] = \int \dd^4 x \, \Ell[\varphi]
\end{equation}
is the action.
The functional derivative is described in \autoref{section:Functional derivative}.

% By Wick's theorem, an $n$-point correlated is given by the sum of all Feynman diagrams with $n$ external vertices.
% The factor $Z[0]^{-1}$ divides out all \emph{vacuum bubbles}, that is diagrams without external vertices.
% We can show this by considering 

In a free theory, we are able to write
\begin{equation}
    Z_0[J] = Z_0[0] \exp(i W_0[J]), \quad 
    W_0[J] = \frac{1}{2} \int \dd^4 x \dd^4 y \, J(x) D_0(x - y) J(y),
\end{equation}
where $D_0$ is the propagator of the free theory.
The exponential form of $Z[J]$ leads to Wick's theorem, which states that an expectation value of $2n$ fields is a sum of \emph{all possible, distinct} combination of $n$ propagators.
To write this in a formal way, we define the functions $a$ and $b$, which define a way to pair up $2m$ elements.
The domain of the functions are the integers between $1$ and $m$, the image a subset of the integers between $1$ and $2m$ of size $m$.
A valid pairing is a set $\{(a(1), b(1)), \dots (a(m), b(m))\}$, where all elements $a(i)$ and $b(j)$ are different, such all integers up to and including $2m$ are featured.
A pair is not directed, so $(a(i), b(i))$ is the same pair as $(b(i), a(i))$.
Wick theorem states that,
\begin{equation}
    \ex{{\prod}_{i=1}^{2m} \varphi(x_i)  }_0
    = \sum_{\{(a, b)\}} \ex{\varphi(x_{a(i)}) \varphi(x_{b(i)})}_0,
\end{equation}
where the sum is over all tuples $(a, b)$ that define a valid and unique pairing.
A generating functional in a general theory is 
\begin{equation}
    Z[J] 
    = Z[0] \ex{\exp(iS_I + i\int \dd^4 x \, J(x) \varphi(x))}_0,
\end{equation}
where the subscript indicates that the expectation value is taken in the free theory.
The expectation value can be expanded in a power series, 
\begin{equation}
    \sum_{n, m} \frac{1}{n! m!} \ex{(iS_I)^n \left(i\int \dd^4 x \, J(x) \varphi(x)\right)^m}_0.
\end{equation}
This equals \emph{the sum of all diagrams with external sources $J(x)$}.
The expectation value
\begin{equation}
    \ex{(iS_I)^n (\varphi(x)J(x))^m}_0
\end{equation}
are all diagrams with $n$ vertices given by the Feynman rules from the interacting action $S_I$, as well as $m$ m source vertices.

Consider a general diagram, built up of $N$ different connected sub-diagrams, where sub-diagram $i$ appears $n_i$ times.
As an illustration, a generic vacuum diagram in $\phi^4$-theory has the form
\begin{align}
    \label{Feinman diagrams}
    V = 
    \includegraphics[width=0.1\textwidth, valign=c]{figurer/feynman-diagram/phi-4_loop_notext.eps}
    \times
    \includegraphics[width=0.08\textwidth, valign=c]{figurer/feynman-diagram/phi-4_2_loop.eps}
    \times
    \includegraphics[width=0.14\textwidth, valign=c]{figurer/feynman-diagram/phi-4_2_loop2.eps}
    \times
    \includegraphics[width=0.1\textwidth, valign=c]{figurer/feynman-diagram/phi-4_loop_notext.eps}
    \times \dots.
\end{align}
If the value of sub diagram $i$ is $V_i$, then each copy of that sub diagram contribute a factor $V_i$ to the value of the total diagram.
However, due to the symmetry of permuting identical sub diagrams, one must divide by the extra symmetry factor $s = n_i !$, which is the total number of permutation of all the copies of diagram $i$.
The value of the total diagram is therefore
\begin{align}
    \label{Feynman diagrams}
    V
    = \prod_{i= 1}^N \frac{1}{n_i!} V_i^{n_i}.
\end{align}
$V$ is uniquely defined by a finite sequence of integers, $(n_1, n_2, \dots n_N, 0, 0, \dots)$, so the sum of all diagrams is the sum over the set $S$ of all finite integers.
This allows us to write the sum of all diagrams as
\begin{equation}
    \label{sum of all diagrams}
    \sum_{(n_1, ...)\in S} \prod_{i} \frac{1}{n_i!} V_i^{n_i}
    = \prod_{i = 1}^{\infty} \sum_{n_i=1}^{\infty} \frac{1}{n_i!} V_i^{n_i}
    = \exp(i {\sum}_i V_i).
\end{equation}
We showed that the generating functional $Z[J]$ were the sum of all diagrams due to external currents.
Using \autoref{sum of all diagrams}, we see that the sum of all \emph{connected} diagrams $W[J]$ is given by
\begin{equation}
    Z[J] = Z[0]\exp(i W[J]).
\end{equation}
We can see that this is trivially true for the free theory, the only connected diagram is
\begin{equation}
    W_0[J] = 
    \includegraphics[valign=c, width=0.3\textwidth]{figurer/feynman-diagram/current-current.eps}.
\end{equation}
Furthermore, correlation functions in the full interacting theory is
\begin{equation}
    \label{correlation function}
    \ex{\varphi(x_1)\dots} = \left(\prod_i \funcdv{J(x_i)}\right) W[J] \Big|_{J=0}.
\end{equation}

% Here we defined the \emph{generating functional for connected diagrams}, $W[J]$.
% The reason for the name will become apparent later. (HUSK Å REFFERE TILBAKE)
% The expectation value of some function of the field-configuration, $A = A[\varphi]$, in the precesense of the source $J$ is
% \begin{equation}
%     \ex{A}_J = \frac{1}{Z[J]} A\left( -i  \fdv{J}\right) Z[J].
% \end{equation}
% (DEFINE FUNCTIONAL DERIVATIVE)
% The expectation value of the field defines a functional,
% \begin{equation}
%     \label{calssical field functional}
%     \varphi[J](x) = \ex{\varphi(x)}_J = 
%     \frac{1}{Z[J]} \left( -i  \fdv{J}\right) Z[J]
%     = \fdv{J(x)} W[J],
% \end{equation}
% and is sometimes called the \emph{classical field}.
% The notation $\Ef[f](x)$ means that $\Ef$ is a functional which takes in a function $f$, and returns the new function $(\Ef[f])(x)$.
% One example is the Lagrangian density, which takes in a field, and returns a function which has a value for each point in space-time.
% We can reverse this relationship, by defining the functional $J[\varphi](x)$ as \emph{the current which causes the classical field $\varphi$}.
% That is, if $\varphi[J_0](x) = \varphi_0(x)$ for some source $J_0$, then $J[\varphi_0] = J_0$

\section{The 1PI effective action and the effective potential}

\label{section: effective action}
The generating functional for connected diagrams, $W[J]$, is dependent on the external source current $J$.
Analogously to what is done in thermodynamics and in Lagrangian and Hamiltonian mechanics, we can define a new quantity, with a different independent variable, using the Legendre transformation.
The new independent variable is 
\begin{equation}
    \varphi_J(x) := \frac{\delta W[J]}{\delta J(x)} = \ex{\varphi(x)}_J.
\end{equation}
The subscript $J$ on the expectation value indicate that it is evaluated in the presence of a source.
The Legendre transformation of $W$ is then
\begin{equation}
    \label{1PI effective action}
    \Gamma[\varphi_J]
    = W[J] - \int \dd^4 x \, J(x) \varphi_J(x).
\end{equation}
Using the definition of $\varphi_J$, we have that
\begin{equation}
    \label{effective equation of motion}
    \fdv{\varphi_J(x)} \Gamma[\varphi_J]
    = \int \dd^4 y \, \fdv{J(y)}{\varphi_J(x)} \fdv{J(y)} W[J]
    - \int \dd^4 y \, \fdv{J(y)}{\varphi_J(x)} \varphi_J(y)
    - J(x)
    = - J(x).
\end{equation}
If we compare this to the classical equations of motion of a field $\varphi$ with the action $S$,
\begin{equation}
    \frac{\delta S[\varphi]}{\delta \varphi(x)} = -J(x),
\end{equation}
we see that $\Gamma$ is an action that gives the equation of motion for the expectation value of the field, given a source current $J(x)$.

To interpret $\Gamma$ further we observe what happens if we treat $\Gamma[\varphi]$ as a classical action with a coupling $g$.
The generating functional in this new theory is
\begin{equation}
    \label{partition function with g}
    Z[J, g] = \int \D \varphi 
    \exp{ i g^{-1} \left( \Gamma[\varphi] + \int \dd^4x \varphi(x) J(x) \right) }
\end{equation}
The free propagator in this theory will be proportional to $g$, as it is given by the inverse of the equation of motion for the free theory.
All vertices in this theory, on the other hand, will be proportional to $g^{-1}$, as they are given by the higher order terms in the action $g^{-1}\Gamma$.
This means that a diagram with $V$ vertices and $I$ internal lines is proportional to $g^{I-V}$.
Regardless of what the Feynman-diagrams in this theory are, the number of loops of a connected diagram is $L = I - V + 1$.
\footnote{This is a consequence of the Euler characteristic $\chi = V - E + F$.}
To see this, we first observe that one single loop must have equally many internal lines as vertices, so the formula holds for $L = 1$.
If we add a new loop to a diagram with $n$ loops by joining two vertices, the formula still holds.
If we attach a new vertex with one line, the formula still holds, and as the diagram is connected, any more lines connecting the new vertex to the diagram will create additional loops.
This ensures that the formula holds, by induction.
As a consequence of this, any diagram is proportional to $g^{L-1}$.
This means that in the limit $g \rightarrow 0$, the theory is fully described at the tree-level, i.e. by only considering diagrams without loops.
In this limit, we may use the stationary phase approximation, as described in \autoref{section:gaussian integrals}, which gives
\begin{equation}
    Z[J, g\rightarrow 0] \approx 
    C \det(- \frac{\delta^2 \Gamma[\varphi_J]}{\delta \varphi^2})
    \exp{i g^{-1} \left(\Gamma[\varphi_J] + \int \dd^4x J \varphi_J \right)  }.
\end{equation}
This means that
\begin{equation}
    -i g \ln(Z[J, g]) 
    = g W[J, g] 
    = \Gamma[\varphi_J] + \int \dd^4x\,  J(x) \varphi_J(x) + \mathcal{O}(g),
\end{equation}
which is exactly the Legendre transformation we started out with, modulo the factor $g$.
$\Gamma$ is therefore the action which describes the full theory at the tree level.
For a free theory, the classical action $S$ equals the effective action, as there are no loop diagrams.

The propagator $D(x, y)$, which is the connected 2 point function $\ex{\varphi(x)\varphi(y)}_J$, is given by the second functional derivative of $W[J]$, times $-i$.
Using the chain rule, together with \autoref{effective equation of motion}, we get
\begin{align}
    \label{Effective action inverse propagator}
    (-i)\int \dd^4 z \frac{\delta^2 W[J]}{\delta J(x) \delta J(z)} 
    \frac{\delta^2 \Gamma[\varphi_J]}{\delta \varphi_J(z) \varphi_J(y)}
    =
    (-i)\int \dd^4 z \frac{\delta \varphi_J[z]}{\delta J(x)}
    \frac{\delta^2 \Gamma[\varphi_J]}{\delta \varphi_J(z) \varphi_J(y)}
    =
    \fdiff{}{J(x)}  \fdiff{\Gamma[\varphi_J]}{\varphi_J(y)}
    = \delta(x - y).
\end{align}
This shows that the second functional derivative of the effective action is $iD^{-1}$, where $D^{-1}$ is the inverse propagator.
The inverse propagator is the sum of all one-particle-irreducible (1PI) diagrams, with two external vertices.
More generally, $\Gamma$ is the generating functional for 1PI diagrams, which is why it is called the 1PI effective action.

\subsection*{The effective action and symmetries}
The symmetries of a theory are transformations of the physical state that leaves the governing equations unchanged.
A lot of physics is contained in the symmetries of a theory, such as the presence of conserved quantities and the systems low energy behavior.
% When we talk about symmetries of a theory, we mean that there is some sort of transformation of the physical state of the system that leaves it unchanged.
% If we take the system of the earth rotating around the sun as an example.
% If we consider two versions of this system, which at time $t$ is related by a rotation, then at all times these two systems will continue to evolve in the same manner, only distinguished by the rotational transformation.
% This is due to the underlying rotational symmetry of the physical system, which is an external symmetry.
We distinguish between internal and external symmetries.
An external symmetry is an active coordinate transformation, such as rotations or translations.
They relate degrees of freedom at different space-time points, while internal symmetry transforms degrees of freedom at each space-time point independently of what happens at other points.
A further distinction is between local and global symmetry transformations.
Local transformations have one rule for how to transform degrees of freedom at each point, which is applied everywhere, while local transformations might themselves be functions of space-time.

In classical field theory, symmetries are encoded in how the Lagrangian changes due to a transformation of the fields.
We will consider continuous transformations, which are can in general be written as
\begin{equation}
    \varphi(x) \longrightarrow \varphi'(x) = f_t[\varphi](x), \quad t \in [0, 1].
\end{equation}
Here, $f_t[\varphi]$ is a functional in $\varphi$, and a smooth function of $t$, with the constraint that $f_0[\varphi] = \varphi$.
This allows us to look at ``infinitesimal'' transformations,
\begin{equation}
    \label{infinitesimal transformation}
    \varphi'(x) = f_\epsilon[\varphi] \sim \varphi(x) + \epsilon g[\varphi](x), \quad \epsilon \rightarrow 0.
\end{equation}
Here, $g$ is a functional of $\varphi$.
We will consider internal, global transformations in which $g$ is linear in $\varphi$.
For $N$ fields, $\varphi_i$, this can be written
\begin{equation}
    \label{linear field transformation}
    \varphi_i'(x) = \varphi_i(x) + \epsilon \, i t_{ij} \varphi_j(x), \quad \epsilon \rightarrow 0.
\end{equation}
$t_{ij}$ is called the generator of the transformation.
A symmetry of the system is then one in which the Lagrangian is unchanged by the transformation, or at most is different by a divergence-term.
That is, a transformation $\varphi \rightarrow \varphi'$ is a symmetry if 
\begin{equation}
    \Ell[\varphi'] = \Ell[\varphi] + \partial_\mu K^\mu[\varphi],
\end{equation}
where $K^\mu[\varphi]$ is a functional of $\varphi$.\footnote{Terms of the form $\partial_\mu K^\mu$ does not affect the physics, as variational principle $\delta S = 0$ which gives the equations of motion do not vary the fields at infinity.}
This is a requirement for a symmetry in quantum field theory too.
However, as physical quantities are given by not just the action of a single state, but the path integral, the integration measure $\D \varphi_i$ has to be invariant as well.
If a classical symmetry fails due to the integration measure, it is called an anomaly.

We want to investigate what constraints a symmetry lies on the effective action.
To that end, assume 
\begin{equation}
    \D \varphi'(x) = \D \varphi(x), \quad
    S[\varphi'] = S[\varphi].
\end{equation}
In the generating functional, such a transformation corresponds to a change of integration variable.
Using the infinitesimal version of the transformation, we may write
\begin{align}
    Z[J] 
    = \int \D \varphi \, \exp{i S[\varphi] + i \int \dd^4 x J_i(x) \varphi_i(x)} 
    = \int \D \varphi' \, \exp{i S[\varphi'] + i \int \dd^4 x J_i(x) \varphi'_i(x)}
    \\
    = Z[J] -  \epsilon \int \dd^4 x J_i(x) \int \D \varphi \, e^{i S[\varphi]} [t_{ij} \varphi_j(x)],
\end{align}
Using \cref{effective equation of motion}, we can write this as
\begin{equation}
    \label{effective action symmetry requirement}
    \int \dd^4 x \, \fdiff{\Gamma[\varphi_J]}{\varphi_i(x)} \, t_{ij}\ex{\varphi_j(x)}_J = 0.
\end{equation}

\subsection*{Effective potential}

For a constant field configuration $\varphi(x) = \varphi_0$, the effective action, which is a functional, becomes a regular function.
We define the effective potential $\Veff$ by
\begin{equation}
    \label{definition effective potential}
    \Gamma[\varphi_0] = - V T \, \Ve_{\mathrm{eff}}(\varphi_0),
\end{equation}
$VT$ is the volume of space-time.
For a constant ground state, the effective potential will equal the energy of this state.
To calculate the effective potential, we can expand the action around this state to calculate the effective action,
by changing variables to $\varphi(x) = \varphi_0 + \eta(x)$.
$\eta(x)$ now parametrizes fluctuations around the ground state, and has by assumption a vanishing expectation value.
The generating functional becomes
\begin{align}
    Z[J] 
    = \int \D (\varphi_0 + \eta) \, 
    \exp{i S[\varphi_0 + \eta] + i \int \dd^4 x J (\varphi_0 + \eta) }
\end{align}
The notation 
\begin{equation}
    \fdiff{S[\varphi_0]}{\varphi(x)}
\end{equation}
indicates that the functional $S[\varphi]$ is differentiated with respect to $\varphi(x)$, then evaluated at $\varphi(x) = \varphi_0$.
The functional version of a Taylor expansion is
\begin{equation}
    S[\varphi_0 + \eta] = 
    S[\varphi_0]
    + \int \dd x \fdv{S[\varphi_0]}{\varphi(x)} \eta(x)
    + \frac{1}{2} \int \dd x \dd y\,  \frac{\delta^2 S[\varphi_0]}{\delta\varphi(x)\delta\varphi(y)} \eta(x) \eta(y)
    + \dots
\end{equation}
We will only consider this expansion up to second order in derivatives for now.
Inserting this into $Z[J]$ we get
\begin{align*}
    &Z[J] = \\ 
    &\int \D \eta  
    \exp{
        i \int \dd^4 x \left(  \Ell[\varphi_0] + J \varphi_0  \right)
        + i \int \dd^4x \left(  \fdv{S[\varphi_0]}{\varphi(x)} + J(x) \right) \eta(x)
        + i \frac{1}{2} \int \dd^4 x \dd^4 y \,  
        \frac{\delta^2 S[\varphi_0]}{\delta\varphi(x)\delta\varphi(y)} \eta(x) \eta(y)
        }
\end{align*}
The first term is constant with respect to $\eta$, and may therefore be taken outside the path integral.
The second term gives rise to tadpole diagrams, which alter the expectation value of $\eta(x)$.
For $J=0$, this expectation value should vanish, so this term can be ignored.
Furthermore, this means that the ground state must minimize the classical potential,
\begin{equation}
    \label{minimize classical potential}
    \diffp{\Ve(\varphi_0)}{\varphi} = 0.
\end{equation}

The one loop approximation to the effective potential is given by the Taylor-expansion up to second order.
This term is a Gaussian integral, and may be evaluated as described in \autoref{section:gaussian integrals},
\begin{equation}
    \int \D \eta \, 
    \exp(
        i \frac{1}{2} \int \dd^4x \dd^4y\,  
        \frac{\delta^2 S[\varphi_0]}{\varphi(x)\varphi(y)} \eta(x) \eta(y)
        )
        = C \det\left( - \fdiff{S[\varphi_0]}{\varphi(x), \varphi(y)} \right)^{-1/2}
\end{equation}
The generating functional for connected diagrams, as defined in \cref{generating functional of connected diagrams}, is therefore
\begin{align}
    \label{generating functional}
    W[J] 
    & = 
    \int\dd^4 x \, \left(\Ell[\varphi_0] + J \varphi_0\right)
    +i \frac{1}{2} \Tr{\ln\left( - \fdiff{S[\varphi_0]}{\varphi(x), \varphi(y)}  \right)}
    + \dots,
\end{align}
where we have used the identity $\ln \det M = \Tr \ln M$.
Using the definition of the effective action, \cref{1PI effective action}, and \cref{definition effective potential} we get an explicit formula for the effective potential to 1 loop order,
\begin{equation}
    \label{effective potential}
    \Veff(\varphi_0) = \Ve(\varphi_0) - \frac{i}{VT}  \frac{1}{2} \Tr{\ln\left( - \fdiff{S[\varphi_0]}{\varphi(x), \varphi(y)}  \right)}.
\end{equation}

\section{Symmetry and Goldstone's theorem}
\label{section:symmetry}

This section is based on \cite{Peskin:IntroQFT,weinberg_1995,weinberg_1996_vol2,smooth_manifolds}.

The symmetries of a theory are transformations of the physical state that leave the governing equations unchanged.
A lot of physics is contained in the symmetries of a theory, such as the presence of conserved quantities and the systems low energy behavior.
We distinguish between internal and external symmetries.
An external symmetry is an active coordinate transformation, such as rotations or translations.
They relate degrees of freedom at different space-time points, while internal symmetry transforms degrees of freedom at each space-time point independently of what happens at other points.
A further distinction is between local and global symmetry transformations.
Local transformations have one rule for how to transform degrees of freedom at each point, which is applied everywhere, while local transformations might themselves be functions of space-time.

In classical field theory, symmetries are encoded in how the Lagrangian changes due to a transformation of the fields.
We will consider continuous transformations, which are can in general be written as
\begin{equation}
    \varphi(x) \longrightarrow \varphi'(x) = f_t[\varphi](x), \quad t \in [0, 1].
\end{equation}
Here, $f_t[\varphi]$ is a functional in $\varphi$, and a smooth function of $t$, with the constraint that $f_0[\varphi] = \varphi$.
This allows us to look at ``infinitesimal'' transformations,
\begin{equation}
    \label{infinitesimal transformation}
    \varphi'(x) = f_\epsilon[\varphi] \sim \varphi(x) + \epsilon\diff{f[\varphi]}{t}\bigg |_{t=0}, \quad \epsilon \rightarrow 0.
\end{equation}
We will consider internal, global transformations which act as linearly on $\varphi$.
For $N$ fields, $\varphi_i$, this can be written
\begin{equation}
    \label{linear field transformation}
    \varphi_i'(x) = \varphi_i(x) + \epsilon \, i t_{ij} \varphi_j(x), \quad \epsilon \rightarrow 0.
\end{equation}
$t_{ij}$ is called the generator of the transformation.
A symmetry of the system is then one in which the Lagrangian is unchanged by the transformation, or at most is different by a divergence-term.
That is, a transformation $\varphi \rightarrow \varphi'$ is a symmetry if 
\begin{equation}
    \Ell[\varphi'] = \Ell[\varphi] + \partial_\mu K^\mu[\varphi],
\end{equation}
where $K^\mu[\varphi]$ is a functional of $\varphi$.\footnote{Terms of the form $\partial_\mu K^\mu$ does not affect the physics, as variational principle $\delta S = 0$ do not vary the fields at infinity.}
This is a requirement for a symmetry in quantum field theory too.
However, as physical quantities in quantum field theory are given not just by the action of a single state but the path integral, the integration measure $\D \varphi_i$ has to be invariant as well.
If a classical symmetry fails due to the integration measure, it is called an anomaly.

To investigate the symmetry properties of a quantum theory, we explore what constraints a symmetry lies on the effective action.
To that end, assume 
\begin{equation}
    \D \varphi'(x) = \D \varphi(x), \quad
    S[\varphi'] = S[\varphi].
\end{equation}
In the generating functional, such a transformation corresponds to a change of integration variable.
Using the infinitesimal version of the transformation, we may write
\begin{align}
    Z[J] 
    = \int \D \varphi \, \exp{i S[\varphi] + i \int \dd^4 x J_i(x) \varphi_i(x)} 
    = \int \D \varphi' \, \exp{i S[\varphi'] + i \int \dd^4 x J_i(x) \varphi'_i(x)}
    \\
    = Z[J] -  \epsilon \int \dd^4 x J_i(x) \int \D \varphi \, e^{i S[\varphi]} [t_{ij} \varphi_j(x)],
\end{align}
Using \cref{effective equation of motion}, we can write this as
\begin{equation}
    \label{effective action symmetry requirement}
    \int \dd^4 x \, \fdiff{\Gamma[\varphi_J]}{\varphi_i(x)} \, t_{ij}\ex{\varphi_j(x)}_J = 0.
\end{equation}
This constraint will allow us to deduce the properties of a theory close to the ground state, only using information about the symmetries of the theory.


The arch typical example of an internal, global and continuous symmetry is the linear sigma model, which we will use as an example throughout this section.
The linear sigma model is made up of $N$ real scalar fields $\varphi_i$, which are governed by the Lagrangian
\begin{equation}
    \Ell[\varphi] 
    = \frac{1}{2} \partial_\mu \varphi_i(x) \partial^\mu \varphi_i(x) - \Ve(\varphi),
    \quad \Ve(\varphi) = - \frac{1}{2} \mu^2 \varphi_i(x)\varphi_i(x)
    + \frac{1}{4} \lambda [\varphi_i(x) \varphi_i(x)]^2.
\end{equation}
This system is invariant under the rotation of the $N$ fields into each other,
\begin{equation}
    \varphi_i \longrightarrow \varphi_i' = O_{ij} \varphi_j,
    \quad O^{-1} = O^{T}.
\end{equation}
The set of all such transformations form the Lie group $O(N)$.
Lie groups will be discussed in the next section.
For $N = 2$, $\mu^2, \lambda > 0$ we get the famous Mexican hat potential, as illustrated in \autoref{fig:Mexican hat}.

\begin{figure}[ht]
    \centering
    \includegraphics[width=0.6\textwidth]{figurer/mexican_hat.pdf}
    \caption{The Mexican hat potential is the classical potential $\Ve$ for the $N=3$ linear sigma model.}
    \label{fig:Mexican hat}
\end{figure}

\subsection*{Lie groups}

Lie groups are a natural structure to capture the symmetries of a theory.
A Lie group is a smooth manifold, i.e. a space that is locally diffeomorphic to $\R^N$.
This means that we can locally parametrize the space by $N$ real numbers $\eta_\alpha$, using smooth invertible functions.
A Lie group is also equipped with group structure.
A group is a set, $G$, together with a map
\begin{align}
    (\cdot, \cdot):  G \times G &\longmapsto G ,\\
    (g_1, g_2) &\longmapsto g_3,
\end{align} 
called group multiplication. This map obeys the group axioms, which are the existence of an identity element $\one$, associativity and the existence of an inverse element $g^{-1}$ for all $g\in G$.
These can be written as
\begin{align*}
    (g, \one) &= g, \\
    (g_1, (g_2, g_3)) &= ((g_1, g_2), g_3), \\
    (g, g^{-1}) &= \one.
\end{align*}
In addition, we require that both the multiplication map and the inverse map, $g \mapsto g^{-1}$ are smooth.
We describe the set of continuous symmetry transformations, 
\begin{equation}
    G = \setbuilder{g}{g \varphi = \varphi', \, S[\varphi'] = S[\varphi], \D \varphi' = \D \varphi },
\end{equation}
as a Lie group, where the group multiplication is composition, i.e. performing transformations in succession.
This map is closed, as two symmetry transformations are another transformation.
The identity map is a symmetry transformation, and composition is associative.
This means that invertible symmetry transformations form a group.

We will focus on connected Lie groups, in which we all elements $g \in G$ is in the same connected piece as the identity map $\one \varphi = \varphi$.
This means that for each $g\in G$, one can find a continuous path $\gamma(t)$ in the manifold, such that $\gamma(0) = \one$ and $\gamma(1) = g$.
Given such a path, we can study transformations close to the identity.
As the Lie group is a smooth manifold, we can write\footnote{The factor $i$ is a physics convention, and differs from how mathematicians define generators of a lie group.}
\begin{equation}
    \gamma(\epsilon) = \one + i \epsilon V + \Oh{\epsilon}
\end{equation}
$V$ is a generator, and can be defined as
\begin{equation}
    iV = \diff{\gamma}{t}\Big|_{t=0}.
\end{equation}
We can define a path $\gamma$ by its path through parameter space, $\gamma(t) = g(\eta(t))$, which means that we can write the generator as
\begin{equation}
    V = \diff{\gamma}{t}\Big|_{t=0} = \diff{\eta_a}{t}\Big|_{t=0} \diffp{g}{\eta_a}\Big|_{\eta=0}
    = \diffp{g}{\eta_a}\Big|_{\eta=0} T_\alpha, \quad 
    T_\alpha = \diffp{g}{\eta_\alpha}.
\end{equation}
We see that the generators form a vector space, with the basis $T_\alpha$, induced by the coordinates $\eta_a$.
This vector space is called the tangent space of the identity element, $T_\one G$.
Infinitesimal transformations can therefore be written as
\begin{equation}
    \gamma(\epsilon) = \one + i \epsilon v_\alpha T_\alpha, \quad \epsilon \rightarrow 0.
\end{equation}
The tangent space, together with the additional operation
\begin{align}
    [T_\alpha, T_\beta] = iC_{\alpha\beta}^\gamma T_\gamma,
\end{align}
called the Lie bracket, forms a Lie algebra denoted $\mathfrak{g}$.
For matrix groups, which are what we deal with in this text, the Lie bracket is the commutator.
$C_{\alpha \beta}^\gamma$ are called structure constants.
They obey the Jacobi identity,
\begin{equation}
    \label{jacobi identity}
    C_{\alpha \beta}^\gamma + C_{\beta\gamma}^\alpha +  C_{\gamma\alpha}^\beta = 0,
\end{equation}
which mean that they are totally antisymmetric.
A subset of the original Lie group, $H \subset G$, which is closed under the group action is called a subgroup.
$H$ then has its own Lie algebra $\mathfrak{h}$, with a set of $m = \dim H$ generators, $x_i$, which is a subset of the original generators $T_\alpha$
We denote the remaining set of generators $t_a$, such that $x_i$ and $t_a$ together span $\mathfrak{g}$.

Of special importance are one parameter subgroups.
If a curve $\gamma(t)$ through $G$ obey
\begin{equation}
    \gamma(t)\gamma(s) = \gamma(t + s), \quad \gamma(0) = \one,
\end{equation}
then all the points on this curve from a one parameter subgroup of $G$.
This path is associated with a generator, 
\begin{equation}
    \diff{\gamma}{t} \Big|_{t=0} = i \eta_\alpha T_\alpha.
\end{equation}
This allows us to define the exponential map,
\begin{equation}
    \exp{i \eta_\alpha T_\alpha} := \gamma(1).
\end{equation}
For connected and compact Lie groups, all elements of the Lie group $g \in G$ can be written as an exponential of elements in the corresponding Lie algebra $\eta_\alpha T_\alpha \in \mathfrak g$.
For matrix groups, the exponential equals the familiar series expansion
\begin{equation}
    \exp{X} = \sum_n \frac{1}{n!} X^n.
\end{equation}

A subgroup $H \in G$ has its own Lie algebra $\mathfrak{h}$, with a set of $m = \dim H$ generators, $x_i$, which is a subset of the original generators $T_\alpha$.
We denote the remaining set of generators $t_a$, such that $x_i$ and $t_a$ together span $\mathfrak{g}$.
The commutators of $x_i$ must be closed, which means that we can write
\begin{align}
    [x_i, x_j] &= i C_{i j}^{k} x_k,\\
    [x_i, t_a] &= i C_{i a}^b t_b, \\
    [t_a, t_b] &= i C_{ab}^k x_k + i C_{ab}^c t_c,
\end{align}
where $ijk$ runs over the generators of $\mathfrak h$, and $abc$ runs over the rest.
The second formula can be derived using the Jacobi identity \cref{jacobi identity}, which implies that $C_{ia}^k = -C_{ik}^a = 0$.
This is called a Cartan decomposition.

\subsection*{Goldstone's theorem}

The fact that a theory is left invariant under some symmetry transformation does not imply that the ground state is invariant under this transformation.
The $N = 2$ linear sigma model illustrates this.
If we assume the ground state $\varphi_{0}$ is translationally invariant, then it is given by minimizing the effective potential, of which the classical potential, $\Ve$, is the leading order approximation.
This potential is illustrated in \autoref{fig:Mexican hat}.
The ground state is therefore given by any of the values along the brim of the potential.
If we, without loss of generality, choose $\varphi_0 = (0, v)$ as the ground state, then any symmetry transformation will change this state.
We say that the symmetry has been \emph{spontaneously broken}.

To explore this in a general context, assume a theory of $\varphi_i$ real scalar fields are invariant under the actions of some Lie group, $G$.
A symmetry $g \in G$ is broken if the vacuum expectation value of the original fields and the transformed fields differ.
That is, if
\begin{equation}
    \ex{\varphi}_0 \neq \ex{\varphi'}_0 = \ex{g \varphi}_0
\end{equation}
We can now exploit what we learned about Lie groups to write
\begin{equation}
    \ex{\varphi'}_0 = \ex{\varphi}_0 + i \epsilon \eta_\alpha T_\alpha \ex{\varphi}_0.
\end{equation}
Let $t_a$ be the set of generators corresponding to broken symmetries, i.e.
\begin{equation}
    t_\alpha \ex{\varphi}_0 \neq 0.
\end{equation}
These are called the \emph{broken generators}.
The remaining set of generators $x_i$, which obey
\begin{equation}
    x_i \ex{\varphi}_0 = 0,
\end{equation}
are called unbroken, and generate a subgroup $H \subset G$ as the set of symmetry transformations of the vacuum is a group.

In \cref{effective equation of motion} we found that the effective action obey
\begin{equation}
    \int \dd^4 x \fdiff{\Gamma[\varphi_J]}{\varphi_i} t_{ij} \ex{\varphi_j}_0 = 0.
\end{equation}
We now differentiate this expression with respect to $\varphi_k(y)$ and evaluate it in the vacuum, which gives
\begin{equation}
    \int \dd^4 x \, \fdiff{\Gamma[\varphi_0]}{\varphi_k(y), \varphi_i(x)}
    t_{ij} \ex{\varphi_j}_0 = 0.
\end{equation}
With the assumption that the ground state is constant, we get 
\begin{equation}
    \diffp{\Veff}{\varphi_k, \varphi_i} \, t_{i j} \ex{\varphi_j}_0 = 0.
\end{equation}
This is trival for unbroken symmetries, as $t_{ij}\ex{\varphi_j}_0 = 0$ by definition.
However, in the case of a broken symmetry, the second derivative of the effective potentail has an eigenvector $t^\alpha_{ij} \ex{\varphi_j}$ with a zero eigenvalue for each broken generator.
In \autoref{Effective action inverse propagator}, we found that the second derivative of the effective action is the inverse propagator,
\begin{equation}
    D^{-1}_{ij}(x,y) 
    = -i \fdiff{\Gamma[\varphi_0]}{\varphi_i(y), \varphi_j(x)}
    = \int \frac{\dd^4 p}{(2 \pi)^4} e^{-ip(x - y)} \, \tilde D^{-1}_{ij}(p).
\end{equation}
Using this, we can write
\begin{equation}
    \tilde D^{-1}_{i j}(p=0) \, t^\alpha_{j k} \ex{\varphi_k}_0 
    = 0.
\end{equation}
Zeros of the inverse propagator corresponds to the physical mass of particles.
In Lorentz-invariant systems, each zero-mass vector corresponds to a masses particles, called a Goldstone boson.\footnote{ The particles are bosons due to the bosonic nature of the transformations, $t$. If the generators are Grassmann numbers, the resulting particle, called a goldstinos, are fermions.}
This means there are $n_G = \dim G -\dim H$ zero mass modes.
In general, counting of massless modes is more complicated, and depends on the dispersion relation of the particles at low momenta.
Systems with Goldstone bosons with quadratic dispersion relation, that is $E = |\vv p|^2$ when $\vv p \rightarrow 0$, often exhibit a lower number of massless modes.
An example is ferromagnets, where the $\mathrm{SU}(2)$ rotational symmetry is broken down to $\mathrm{U}(1)$ when they align along one axis. 
This corresponds to two broken generators, yet the system only exhibits one massless mode~\cite{brauner_spont_sym}.

\begin{figure}[ht]
    \centering
    \begin{subfigure}{0.54\textwidth}
        \centering\captionsetup{width=.9\linewidth}
        \includegraphics[]{figurer/mexican_hat_zoom.pdf}
        \caption{Excitations along the brim does not cost any energy, as the potential is flat, unlike excitations in the radial direction.}
        \label{fig:Mexican hat zoom}
    \end{subfigure}
    \begin{subfigure}{0.45\textwidth}
        \centering\captionsetup{width=.9\linewidth}
        \includegraphics[]{figurer/sigma_ground_state.pdf}
        \caption{Excitations for the $N=3$ sigma model. Two of the symmetries are broken, while rotations around the groundstate leaves the system unchanged.}
        \label{fig:ground state manifold}
    \end{subfigure}
\end{figure}

The linear sigma model gives an intuition for the Goldstone mode.
In the case of $N = 2$, the symmetry of the Lagrangian are rotations in the plane.
As the ground state is a point along the ``brim'' of the hat, this rotational symmetry is broken.
Any excitations in the rotational direction, however, does not cost any energy, which is indicative of a massless mode.
This is illustrated in \autoref{fig:Mexican hat zoom}.
In this example, the original symmetry group is one dimensional, so there are no unbroken symmetries.
If we instead consider the $N=3$ linear sigma model, which has a three-dimensional symmetry group, rotations of the sphere, we see that the ground state is left invariant under subgroup of the original symmetry transformations.
The ground state manifold of this system, the set of all states that minimizes the effective potential, is then a sphere.
When the system chooses one single ground state, this symmetry is broken, but only for two of the generators. 
The generator for rotations around the ground state leaves the that point unchanged, and is thus an unbroken symmetry.
Any excitations in the direction of the broken symmetries does not cost energy, as it is in the ground state manifold.
The unbroken symmetry, on the other hand, does not correspond to an excitation.
This is illustrated in \autoref{fig:ground state manifold}.


\input{theory/CCWZ_construction.tex}

% % Effective Lagrangian
\chapter{The effective theory of pions}
\label{chapter:effective theory of pions}

\section{QCD}

Quantum chromodynamics, or QCD, is the theory of how quarks interact using the strong force.
It has its name due to the fact that the charges of QCD are called red, green and blue, which of course is only an analogy to colors.
The two main quantities are the quark spinors, $q_f$
if we have $N_f$ free quarks, with $N_c$ colors, then the Lagrangian is given by (HUSK i!)
\begin{equation}
    \Ell[q] = \sum_{f=1}^{N_f} \sum_{c = 1}^{N_c} i \bar q_{fc} (\gamma^\mu \partial_\mu - m_f )q_{fc}
    = i  \bar q (\slashed{\partial } - m)q.
\end{equation}
In the last equality, we have hidden the indices to reduce clutter.
Each element $q_{fc}$ is a spinor, and $\gamma^\mu$ are the gamma matrices, as described in \autoref{Conventions and notation}.
Furthermore, $\bar q = q^\dagger \gamma^0$.
This Lagrangian is invariant under rations of the quarks in the color indices,
i.e. the transformation
\begin{equation}
    q_c \rightarrow q'_c = U_{cc'} q_{c'},
    \quad 
    \bar q_c = \bar U_{cc'}^\dagger q_c'
\end{equation}
where $U_{cc'}$ is an $N_c \times N_c$ unitary matrix.
The set of all $N_c\times N_c$ unitary matrices form the Lie group $U(N_c)$.
All unitary matrices $U$ can be written as $e^{i\theta} U'$, where $\theta$ is a real number, and $U'$ is a matrix with determinant, which means we can decompose the Lie group into $U(1)\times SU(N_c)_c$.
We recognize $U(1)$ as the gauge group of the electromagnetic field, which we will ignore for the time being.
$SU(N_c)$ is the group of all complex $N_c\times N_c$ matrices with determinant 1, and the subscript $c$ specifies that this is the set of color transformation, and not just some abstract group.

\subsection*{The Yang-Mills Lagrangian}
$SU(N_c)$ is the gauge group for the strong force.
Given an element $U \in SU(N_c)$, we can write
\begin{equation}
    U = \exp{i \chi_\alpha \lambda_\alpha}, \quad
    \chi_\alpha \lambda_\alpha \in \liea{su}{N_c}_c,
\end{equation}
where $\liea{su}{N_c}_c$ is the Lie algebra of $SU(N_c)_c$.
We derive the full, interacting Lagrangian of QCD by demanding that it remain invariant under a \emph{local} $SU(N_c)_c$ transformation, i.e.
\begin{equation}
    q \rightarrow \exp{i \chi_\alpha(x) \lambda_\alpha} q, \quad
    \bar q \rightarrow \exp{-i \chi_\alpha(x) \lambda_\alpha} q'.
\end{equation}
This is no problem for the mass term, however, we need to modify the kinetic derivative term.
For the Lagrangian to be gauge-invariant, this has to obey
\begin{equation}
    D_\mu q \rightarrow (D_\mu q)' = U D_\mu q.
\end{equation}
As $q' = Uq$, this implies that
\begin{equation}
    D'_\mu = U D_\mu U^\dagger.
\end{equation}
If we posit $D_\mu = \one \partial_\mu + A_\mu^\alpha \lambda_\alpha$, then
\begin{equation}
    D_\mu' 
    =U D_\mu U^\dagger 
    = U(U^\dagger \partial_\mu +  i\partial_\mu \chi_\alpha(x)\lambda_\alpha U^\dagger)
    + U A_\mu U^\dagger
    = \partial_\mu + U(A_\mu + i\partial_\mu \chi_\alpha(x))U^\dagger,
\end{equation}
where $A_\mu = A_\mu^\alpha \lambda_\alpha$.
This means that if we demand that $A_\mu$ transforms as
\begin{equation}
    A_\mu \rightarrow U(A_\mu - i \partial_\mu \chi) U^\dagger,
\end{equation}
then our new derivative operator, called the \emph{covariant derivative}, follows the desired transformation rule.
The second derivative operator,
\begin{equation}
    D_\mu D_\nu = [\partial_\mu \partial_\nu - i(\partial_\mu A_\nu + A_\mu\partial_\nu + A_\nu\partial_\mu) - A_\mu A_\nu],
\end{equation}
transforms in the same way as the first derivative.
We see that the ``operator-part'' of this derivative is symmetric in the space-time indices, which means that the commutator will just a tensor, and not an operator.
We define
\begin{equation}
    F_{\mu\nu} := i[D_\mu, D_\nu] = (\partial_\mu A_\nu - \partial_\mu A_\nu) - i[A_\mu, A_\nu]
    = (\partial_\mu A_\nu^\alpha - \partial_\nu A_\mu^\alpha + C_{\beta \gamma }^\alpha A_{\mu}^\beta A_{\nu}^\gamma ) \lambda_\alpha.
\end{equation}
This transforms as
\begin{equation}
    F_{\mu\nu} \rightarrow U F_{\mu \nu} U^\dagger.
\end{equation}

We now need to include terms governing the gauge field $A_\mu$ in the Lagrangian.
The tensor $F_{\mu\nu}$ allows construct all gauge invariant terms to dimension 4, which are
\begin{equation}
    F_{\mu \nu}^a F_a^{\mu \nu}, 
    \quad 
    \varepsilon^{\mu\nu\rho\sigma} F_{\mu \nu}^a F_{\rho \sigma}^a.
\end{equation}
Here, $\varepsilon$ is the Levi-Civita symbol.
This allows us to write down the most general gauge-invariant Lagrangian for a $SU(N_c)$ gauge theory, the Yang-Mills Lagrangian
\begin{equation}
    \Ell = i  \bar q (\slashed{D} - m)q 
    + \frac{1}{4} F_{\mu \nu}^a F_a^{\mu \nu}
    + c \varepsilon^{\mu\nu\rho\sigma} F_{\mu \nu}^a F_{\rho \sigma}^a.
\end{equation}

\subsection*{Chiral symmetry}

In addition to the color and flavor indices $s$ and $f$, the quarks also have spinor indices, $i$, which are what the $\gamma$-matrices act on.
We can define the projection operators,
\begin{equation}
    P_\pm = \frac{1}{2}(1 \pm \gamma^5),
\end{equation}
which obey $P_\pm^2 = P_\pm$, $P_+P_- = P_-P_+ = 0$ and $P_+ + P_- = 1$, as good projection operators should.
Furthermore, $P^\dagger_\pm = P_\pm$.
These project spinors down to their chiral components, called left- and right-handed spinors,
\begin{equation}
    P_+ q = q_R, \quad P_- q = q_L.
\end{equation}
From \autoref{Conventions and notation}, we have 
\begin{equation}
    \acom{\gamma^\mu}{\gamma^5} = 0,
\end{equation}
which means that 
\begin{equation}
    \bar q P_\pm = (P_{\mp}q)^\dagger \gamma^0.
\end{equation}
This means that we can write the quark part of the Lagrangian as
\begin{align*}
    \bar q (i\slashed D - m) q
    & = 
    \bar q (P_+ + P_-) (P_+ + P_-) (i\slashed D - m) q
    = (q P_-)\gamma^0 P_+ (i \slashed D - m) q + (q P_+)\gamma^0 P_- (i \slashed D - m) q \\
    & = \bar q_L (i\slashed D) q_L + \bar q_R (i\slashed D) q_R
    - \bar q_L m q_R - \bar q_R m q_L.
\end{align*}
We see that the mass-term mixes the two chiral components.
In the limit $m \rightarrow 0$, called the \emph{chiral limit}, we gain two new symmetries,
\begin{equation}
    q_R \rightarrow U_R q_R, \quad q_L \rightarrow U_L q_L,
\end{equation}
where $U_L$ and $U_R$ are unitary matrices which act on the flavor indices.
The total set of such transformations form the Lie group $U(N_f)_L \times U(N_f)_R$.


\begin{equation}
    \label{Mass matrix}
    M =
    \begin{pmatrix}
        m_u & 0 \\
        0 & m_d
    \end{pmatrix}.
\end{equation}

\section{Chiral perturbation theory}

In this paper, we will consider the interaction of two quarks, the up and down quarks $u$ and $d$, which means that $N_f = 2$.
In this case, the approximate symmetry of the Lagrangian is $G = SU(2)_L \times SU(2)_R$, with the generators $T_\alpha = \tau_\alpha / 2$, where $\tau_\alpha$ are the Dirac matrices, as described in \autoref{Conventions and notation}.
The quarks are not massless, but have a non-zero mass matrix
\begin{equation}
    \label{Mass matrix}
    m =
    \begin{pmatrix}
        m_u & 0 \\
        0 & m_d
    \end{pmatrix}
    = \frac{1}{2} (m_u + m_d) \one + \frac{1}{2}(m_u - m_d) \tau_3,
\end{equation}
where $m_u \approx 2.16 \, \text{MeV}$, and $m_d \approx 4.67 \, \text{MeV}$~\cite{PDG}.
This means that the $G$ symmetry is \emph{explicitly} broken.
In what is called the isospin limit, $m_u = m_d$, the subgroup $SU(2)_V$ remains intact.
The difference between these masses is small, which is why isospin is a good quantum number.
However, even though the underlying symmetry is only approximate, we can still apply the formalism from Goldstone's theorem and the CCWZ construction by including a small mass term for the Goldstone bosons, which in this case are called \emph{Pseudo Goldstone bosons}.

The approximate $G = SU(2)_V\times SU(2)_A$ symmetry of the two-flavor QCD-Lagrangian is spontaneously broken by the ground state.
As quarks $q$ are spinors, a non-zero expectation value of the quark field would break Lorentz-invariance.
Instead, the spontaneous symmetry breaking is characterized by the \emph{scalar quark condensate}, $\ex{\bar q q}$.
% In the isospin-limit, one can show that $\ex{\bar u u } = \ex{\bar d d} = \ex{\bar q q} / 2$.
The scalar quantity $\bar q q$ is invariant under isospin transformations $H = SU(2)_V$, but not under $SU(2)_A$.
The Goldstone manifold is therefore $G/H = SU(2)_L\times\SU(2)_R/SU(2)_V = SU(2)_A$.
To model the low energy dynamics of QCD, we start with the massless QCD Lagrangian,
\begin{align*}
    \Ell^0_\mathrm{QCD}[q, \bar q, A_\mu] 
    = i \bar q \slashed{D} q - \frac{1}{4}G_{\mu \nu}^\alpha G^{\mu \nu}_\alpha
\end{align*}
Following~\cite{Gasser-Leutwyler:chiral,Scherer2002IntroductionTC}, we couple quarks to external currents.
These can be used to model external fields, or to capture the symmetry breaking mass term.
As found in \autoref{section:QCD}, the conserved currents are
\begin{equation}
    V_a^\mu = \frac{1}{2} \bar q \gamma^\mu \tau_a q, \quad
    A_a^\mu = \frac{1}{2} \bar q \gamma^\mu \gamma^5 \tau_a q, \quad
    J^\mu = \bar q \gamma^\mu q.
\end{equation}
In addition, external currents can couple to the scalar and pseudo-scalar quark bilinears
\begin{equation}
    \bar q q, \quad \bar q \gamma^5 q, 
    \quad \bar q \tau_a q, \quad \bar q \gamma^5 \tau_a q.
\end{equation}
The external Lagrangian is thus
\begin{align}
    \Ell_\text{ext} 
    & = 
    - (\bar q q )\, s_0 + (i \bar q \gamma^5 q) \, p_0
    - (\bar q \tau_a q)\, s_a + (i \bar q \gamma^5 \tau_a q) \, p_a
    + \frac{1}{3} J_\mu v_{(s)}^\mu 
    + V_\mu^a v_a^\mu + A_\mu^a a_a^\mu \\
    & = 
    - \bar q \left(s + i \gamma^5 p \right) q
    + \bar q \left(\frac{1}{3} v^\mu_{(s)} + v^\mu + a^\mu \gamma^5\right) \gamma_\mu q.
\end{align}
Here, $v, v_{(s)}, a, s$ and $p$ are the external currents, where we denote
\begin{equation}
    s = s_0 \one + s_a \tau_a, \quad
    p = p_0 \one + p_a \tau_a, \quad
    v^\mu = \frac{1}{2} v_a^\mu \tau_a, \quad
    a^\mu = \frac{1}{2} a_a^\mu \tau_a.
\end{equation}
Setting $v = v_{(s)} = a = s = p = 0$ recover the chiral limit.
To include the effect of the quark masses, we set $s = m$.
We denote all the external currents as $j = (v, v_{(s)}, a, s, p )$.
The generating functional is then
\begin{equation}
    Z[j] 
    = 
    \int \D \bar q \D q \D A_\mu \, 
    \exp{
        i \int \dd^4 x 
        \left( 
            \Ell^0_\text{QCD}[q, \bar q, A_\mu] + \Ell_\text{ext}[q, \bar q, j]
        \right)
    }
\end{equation}
As with the conserved currents, we define
\begin{equation}
    v_\mu = \frac{1}{2}(r_\mu + l_\mu),
    \quad
    a_\mu = \frac{1}{2}(r_\mu - l_\mu).
\end{equation}
With this, as well as 
\begin{equation}
    \bar q (s - i \gamma^5 p) q
    = \bar q_R (s - i p) q_L + \bar q_L (s + i p) q_R,
\end{equation}
we can write the external Lagrangian as
\begin{equation}
    \Ell_\text{ext} 
    = - \bar q_R (s + i p) q_L - \bar q_L (s - i p) q_R
    + \frac{1}{3} (J_R^\mu + J_V^\mu)(v_{(s)})_\mu
    + R_\mu^a r^\mu_a + L_\mu^a l^\mu_a
\end{equation}
We will now use the CCWZ construction and effective field theory to construct the effective Lagrangian, which obeys
\begin{equation}
    Z[j] = \int \D \pi \, \exp{i \int \dd^4 x \, \Ell_\text{eff}[\pi]},
\end{equation}
where $\pi$ are the Goldstone bosons.

\subsection*{Non-linear realization}
% We now use the formalism we developed in \autoref{section:symmetry}.
% The full symmetry is $G = SU(2)_V\times SU(2)_A$, and the subgroup of the vacuum is $H = SU(2)_V$.
The Goldstone manifold is $G/H = SU(2)_A$, which means that we parametrize the Goldstone modes as
\begin{equation}
    \Sigma(x) = \exp{i \frac{\pi_a(x) \tau_a}{f} }.
\end{equation}
Here, $f$ is the bare pion decay constant.

We start by fining a representative element of the Goldstone manifold $G/H = SU(2)_A$.
Let $g\in G$. 
We write $g = (L, R)$, where $R \in SU(2)_R$, $L \in SU(2)_L$.
Elements $h \in H$ are then of the form $h = (V, V)$.
A general element g can be written as
\begin{equation}
    g = (L, R) = (1, R L^\dagger) (L, L).
\end{equation}
Since $(L, L) \in H$, this means that we can write the coset $g H$ as $(1, R L^\dagger)H$, which gives a way to choose a representative element for each coset.
We identify
\begin{equation}
    \Sigma = R L^\dagger. 
\end{equation}
This is our standard form, and therefore makes it possible to obtain the transformation properties, which is given by the function $h(g, \xi)$.
We have 
\begin{equation}
    \tilde g (1, \Sigma)
    = (\tilde L, \tilde R) (1, R L^\dagger)
    = (1, \tilde R (R L^\dagger) \tilde L^\dagger) (\tilde L, \tilde L)
    = (1, \tilde R \Sigma \tilde L) h.
\end{equation}
This gives the transformation rule
\begin{equation}
    \Sigma \rightarrow \Sigma' = R \Sigma L^\dagger.
\end{equation}
Under transformations by $h = (V, V^\dagger) \in H$, we have
\begin{equation}
    \Sigma \rightarrow \Sigma' = V \Sigma V^\dagger.
\end{equation}
As $\partial_\mu  (1, \Sigma) = (0, \partial_\mu \Sigma)$, the constituents of the Mauer-Cartan form are
\begin{equation}
    d_\mu = i \Sigma(x)^\dagger \partial_\mu \Sigma(x),\quad
    e_\mu = 0.
\end{equation}
Using $\partial_\mu [\Sigma(x)^\dagger\Sigma(x)] = 0 $, we can write
\begin{equation}
    d_\mu d^\mu = 
    - \Sigma(x)^\dagger [\partial_\mu \Sigma(x)] \Sigma(x)^\dagger [\partial^\mu \Sigma(x)]
    =\Sigma(x)^\dagger [\partial_\mu \Sigma(x)] [\partial^\mu \Sigma(x)^\dagger] \Sigma(x).
\end{equation}
Inserting this into (REF. TIL LO TERM I CCWZ), we get
\begin{equation}
    \Tr{d_\mu d^\mu} = \Tr{\partial_\mu \Sigma (\partial^\mu \Sigma)},
\end{equation}
where we have used the cyclic property of the trace.

\subsection*{External currents and explicit symmetry breaking}
\label{Covarinat derivative}

Using $d_\mu$ we are able to construct any terms in the effective Lagrangian corresponding to $j=0$.
To construct the effective Lagrangian when $j \neq 0$, which is needed to capture the effects of nonzero quark masses and external currents, we treat $SU(2)_L\times SU(2)_R\times U(1)$ as a gauge group, and the source currents as the corresponding gauge fields.
% This as gauge field, the  and they can thus be included into the effective Lagrangian.
The effective Lagrangian is then constructed as the most general Lagrangian that is invariant under \emph{local} $SU(2)_L\times SU(2)_R\times U(1)$ transformations.
The Ward-identities corresponding to the local symmetries of the Lagrangian are equivalent with the statement that $Z[j]$ is invariant under a gauge transformation of the external fields, in the absence of anomalies~\cite{Leutwyler:on_the_fundations}.

Following~\cite{Scherer2002IntroductionTC}, we write the gauge transformation as
\begin{equation}
    q_L \rightarrow e^{i\theta(x)/3} U_L(x) q_L, \quad
    q_R \rightarrow e^{i\theta(x)/3} U_R(x) q_R.
\end{equation}
First, we consider the $U(1)_V$ transformation.
The massless QCD Lagrangian then transforms as
\begin{equation}
    \Ell_\text{QCD}^0 = i \bar q \slashed{D} q
    \rightarrow
    i \bar q e^{-i\theta(x)/3} \slashed{D} e^{i\theta/3} q
    = i \bar q \slashed{D} q - \frac{1}{3} \bar q \gamma^\mu q \partial_\mu \theta(x).
\end{equation}
This gives us the transformation rule
\begin{equation}
    v_{(s)}^\mu \rightarrow v_{(s)}^\mu - \partial^\mu \theta(x).
\end{equation}
Then, applying the $SU(2)_R$ transformation, we get
\begin{equation}
    \bar q_R \slashed{D} q_R + (i \bar q_R \gamma^\mu \tau_a  q_R) \, r^a_\mu
    \rightarrow
    i \bar q_R \slashed{D} q_R + 
    (i \bar q_R \gamma^\mu q_R ) \, U_R^\dagger \partial_\mu U_R  
    + [\bar q_R \gamma^\mu (U_R^\dagger \tau_a U_r)  q_R ] \, r^a_\mu
\end{equation}
This, and the similar expression for $U_L$, gives the gauge transformation rules
\begin{align}
    r_\mu^a \tau_a & \rightarrow U_R^\dagger (r_\mu^a\tau_a + i\one \partial_\mu) U_R, \\
    l_\mu^a \tau_a & \rightarrow U_L^\dagger (l_\mu^a\tau_a + i\one \partial_\mu) U_L, \\
    s + i p & \rightarrow U_R^\dagger (s + i p) U_L, \\
    s - i p & \rightarrow U_L^\dagger (s - i p) U_R.
\end{align}
 Goldstone fields now transform as
\begin{equation}
    \Sigma(x) \rightarrow U_L(x) \Sigma(x) U_R(x)^\dagger,
\end{equation}
and the derivative as
\begin{align}
    \nonumber
    \partial_\mu \Sigma \rightarrow & \, \partial_\mu (U_L \Sigma U_R) \\
    &= 
    U_L (\partial_\mu \Sigma )U_R^\dagger
    + (\partial_\mu  U_L) \Sigma U_R^\dagger
    + U_L \Sigma (\partial_\mu U_R^\dagger)
    \nonumber
    \\
    & = 
    U_L
    \left[
        \partial_\mu \Sigma
        + U_L^\dagger (\partial_\mu U_L) \Sigma
        + \Sigma(x) (\partial_\mu U_R(x)^\dagger) U_R
    \right]
    U_R^\dagger.
    \label{Sigma partial derivative}
\end{align}
The covariant derivative is defined to transform in the same way as the object it is acting upon.
This means that the definition of the covariant derivative depends on what it acts on.
Assume $A$, $B$ and $\Sigma$ transforms as $A \rightarrow U_R A U_R^\dagger$, $B \rightarrow U_L A U_L^\dagger$, and $\Sigma \rightarrow U_L \Sigma U_R^\dagger$.
In each of these cases, we define the covariant derivative as
\begin{align}
    \label{covariant derivative general}
    \nabla_\mu A = \partial_\mu A - i [r_\mu, A], \\
    \nabla_\mu B = \partial_\mu B - i [r_\mu, B], \\
    \nabla_\mu \Sigma = \partial_\mu \Sigma - i r_\mu \Sigma + i \Sigma l_\mu.
\end{align}
It follows from \cref{Sigma partial derivative} that $\nabla_\mu \Sigma$ transforms as $\Sigma$,and the proof for the other two are special cases.
For quantities that do not transform under $SU(2)_L\times SU(2)_R$, the covariant derivative is the ordinary partial derivative.
If $A$, $B$ and $C = AB$ are all quantities with a well-defined covariant derivative, then
\begin{equation}
    \nabla_\mu (AB) = (\nabla_\mu A) B + A (\nabla_\mu B).
\end{equation}
We prove the special case of $a_\mu = 0$, which is what we use in this text, in which case
\begin{equation}
    \nabla_\mu \Sigma = \partial_\mu \Sigma - i [v_\mu, \Sigma].
\end{equation}
Assume $A, B$ transforms as $\Sigma$. 
Then,
\begin{align*}
    \nabla_\mu (A B) 
    & = (\partial_\mu A) B + A (\partial_\mu B) - i \com{v_\mu}{AB}
    = (\partial_\mu A - i \com{v_\mu}{A})B + A(\partial_\mu B- i \com{v_\mu}{B})\\
    & = (\nabla_\mu A)B + A (\nabla_\mu B).
\end{align*}
The more general theorem follows by applying the definition of the various covariant derivatives~\cite{Scherer:PhysRevD.53.315}.
Decomposing a 2-by-2 matrix $M$, as described in \autoref{Conventions and notation}, shows that the trace of the commutator of $\tau_b$ and $M$ is zero:
\begin{equation*}
    \Tr{\com{\tau_a}{M}]} = M_b\Tr{ \com{\tau_a}{\tau_b}} = 0.
\end{equation*}
Together with the fact that $\Tr{\partial_\mu A} = \partial_\mu \Tr{A}$, this gives the product rule for invariant traces:
\begin{equation*}
    \Tr{A \nabla_\mu B} = \partial_\mu \Tr{AB} - \Tr{(\nabla_\mu A) B}.
\end{equation*}
This allows for the use of the divergence theorem when doing partial integration.
Assume $A$, $K^\mu$, and $A K^\mu$ have a well-defined covariant derivative, and that $\Tr{A K^\mu}$ is invariant under transformations by the gauge group.
Let $\Tr{K^\mu}$ be a space-time vector, and $\Tr{A}$ scalar. 
Let $\Omega$ be the domain of integration, with coordinates $x$ and $\partial \Omega$ its boundary, with coordinates $y$. Then, 
\begin{align*}
    \int_\Omega \dd x \, \Tr{A \nabla_\mu K^\mu} 
    = 
    - \int_\Omega \dd x \, \Tr{(\nabla_\mu A) K^\mu}
    + \int_{\partial\Omega} \dd y\, n_\mu \Tr{A K^\mu},
\end{align*}
where $n_\mu$ is the normal vector of $\partial \Omega$~\cite{Carroll:spacetime}.
By the assumption of no variation on the boundary, the last term is constant when varying the action, and may therefore be discarded.

We define the scalar current
\begin{equation}
    \chi = 2 B_0 (s + ip),
\end{equation}
where $B_0$ is defined by the up quark condensate in the chiral limit by
\begin{equation}
    \ex{\bar u u}_{j = 0} = - f^2 B_0.
\end{equation}
Here, $f$ is the bear pion decay constant.
The pseudoscalar current is thus $\chi^\dagger = 2 B_0 (s - ip)$.
These can be combined to give more gauge-invariant terms, such as
\begin{equation}
    \Tr{\chi^\dagger \Sigma}, \quad \Tr{\Sigma^\dagger \chi}.
\end{equation}

In the grand canonical ensemble, we modify the Lagrangian as
\begin{equation}
    \Ell \rightarrow \Ell + \mu Q,
\end{equation}
where $Q$ is a conserved charge, and $\mu$ is the corresponding chemical component.
We are interested systems with non-zero isospin, 
\begin{equation}
    Q_I = \int \dd^3 x \, V^0_3 = \int \dd^3 x \, \frac{1}{2}  \bar q \gamma^0 \tau_3 q,
\end{equation}
and the corresponding isospin chemical potential $\mu_I$.
We do this by considering the system with a external current
\begin{equation}
    v_\mu = \frac{1}{2} \mu_I \delta_\mu^0 \tau_3.
\end{equation}

The most straight forward parametrization of the Goldstone manifold is 
\begin{equation}
    U(x) = \exp{i \frac{\pi_a\tau_a}{2 f}},
\end{equation}
where $f$ is the bare pion decay constant, $\pi_a$ are the three Goldstone bosons.
They are real functions, and are related to the pions $\pi_0, \pi_+$ and $\pi_-$ by
\begin{equation}
    \pi_a\tau_a
    = 
    \begin{pmatrix}
        \pi_3 & \pi_1 - i \pi_2 \\
        \pi_1 + i \pi_2 & - \pi_3
    \end{pmatrix}
    = 
    \begin{pmatrix}
        \pi_0 & \sqrt{2} \pi_+ \\
        \sqrt 2 \pi_- & - \pi_0
    \end{pmatrix}.
\end{equation}
For $\mu_I = 0$, the ground state of the system will be the vacuum, i.e. at $\pi_a = 0$.
For non-zero isospin chemical potential, however, we expect that the ground state may be rotated away from the vacuum.
Following~\cite{Andersen:two-flavor-chpt}, we parametrize the Goldstone manifold as
\begin{align}
\label{sigma}
    \Sigma(x) = A_\alpha [U(x) \Sigma_0 U(x)] A_\alpha,
\end{align}
where
\begin{align}
    \Sigma_0 = \one,\, 
    A_\alpha = \exp{\frac{i \alpha}{2} \tau_1},\, 
    U(x) = \exp{i \frac{\tau_a\pi_a(x)}{2f}}.
\end{align}
Here, $\alpha$ is a real number between $0$ and $2 \pi$, that parametrizes the ground state of the system.


\section{Leading order Lagrangian}

The leading order \chpt Lagrangian is made up of the terms \cref{leading order chi sigma term} and \cref{leading order term sigma}, and reads
\begin{equation}
    \label{chpt lagrangian}
    \Ell_2 = 
    \frac{1}{4} f^2 \Tr{\nabla_\mu \Sigma (\nabla^\mu \Sigma)^\dagger}
    + \frac{1}{4} f^2 \Tr{\chi^\dagger \Sigma + \Sigma^\dagger \chi}.
\end{equation}
In the ground state, we set the external scalar current $s = m$, where $m$ is the mass matrix \autoref{Mass matrix}, so
\begin{equation}
    \chi = 2 B_0 m = \bar m^2 \one + \Delta m^2 \tau_3,
\end{equation}
where we have defined
\begin{equation}
    \bar m^2 = 2B_0(m_u + m_d), \quad \Delta m^2 = 2 B_0 (m_u - m_d).
\end{equation}
In this section, we will expand this Lagrangian in $\pi/f$, which we will use to calculate the free energy density.
To get the series expansion of $\Sigma$ in powers of $\pi/f$, we start by using the fact that $\tau_a^2 = \one$ to write
\begin{equation}
    \label{A}
    A_\alpha 
    = \sum_n^\infty \frac{1}{n!} \left(\frac{i \alpha}{2} \tau_1 \right)^n 
    = \sum_n^\infty 
    \left[
        \frac{\one}{(2n)!} \left(\frac{i \alpha}{2}\right)^{(2n)} 
        + \frac{\tau_1}{(2n + 1)!} \left(\frac{i\alpha}{2}\right)^{(2n + 1)}
    \right] 
    = \one \cos{\frac{\alpha}{2}} + i \tau_1 \sin{\frac{\alpha}{2}}.
\end{equation}
The series expansion of $U$ is
\begin{align*}
    U = \exp(\frac{i \pi_a \tau_a}{2f}) = 
    1
    + \frac{i \pi_a \tau_a}{2f} 
    + \frac{1}{2}\left(\frac{i \pi_a \tau_a}{2f}\right)^2 
    + \frac{1}{6}\left(\frac{i \pi_a \tau_a}{2f}\right)^3 
    + \frac{1}{24}\left(\frac{i \pi_a \tau_a}{2f}\right)^4 
    + \Oh[5]{(\pi/f)},
\end{align*}
which we use to calculate the expansion of the inner part of $\Sigma$, as given in \autoref{sigma},
\begin{align*}
    U\Sigma_0U & = 
    \left(
        1
        + \frac{i \pi_a \tau_a}{2f} 
        + \frac{1}{2}\left(\frac{i \pi_a \tau_a}{2f}\right)^2 
        + \frac{1}{6}\left(\frac{i \pi_a \tau_a}{2f}\right)^3 
        + \frac{1}{24}\left(\frac{i \pi_a \tau_a}{2f}\right)^4 
    \right)\\
    & \times
    \left(
        1
        + \frac{i \pi_a \tau_a}{2f} 
        + \frac{1}{2}\left(\frac{i \pi_a \tau_a}{2f}\right)^2 
        + \frac{1}{6}\left(\frac{i \pi_a \tau_a}{2f}\right)^3 
        + \frac{1}{24}\left(\frac{i \pi_a \tau_a}{2f}\right)^4 
    \right)
    + \Oh[5]{(\pi/f)}\\
    &=
    1 + \frac{i \pi_a \tau_a}{f}
    + 2 \left( \frac{i \pi_a \tau_a}{2f} \right)^2
    + \frac{4}{3} \left( \frac{i \pi_a \tau_a}{2f} \right)^3
    + \frac{2}{3} \left( \frac{i \pi_a \tau_a}{2f} \right)^4
    + \Oh[5]{(\pi/f)}.
\end{align*}
The symmetry of $\pi_a\pi_b$ means that
\begin{align*}
% Identitites
    & (\pi_a \tau_a)^2
    = 
    \pi_a \pi_b \frac{1}{2} \acom{\tau_a}{\tau_b} 
    =
    \pi_a \pi_a, \quad
    (\pi_a \tau_a)^3
    =
    \pi_a \pi_a \pi_b \tau_b,\quad
    (\pi_a \tau_a)^4
    =
    \pi_a \pi_a \pi_b \pi_b.
\end{align*}
This gives us the expression
\begin{align*}
% Final expression
    & U\Sigma_0U 
    =
    1
    + i \frac{\pi_a \tau_a}{f} 
    - \frac{\pi_a^2}{2f^2}
    - i \frac{\pi_a^2\pi_b \tau_b}{6f^3}
    + \frac{\pi_a^2\pi_b^2}{24f^4}
    + \Oh[5]{(\pi/f)}.
\end{align*}
We combine this result with \autoref{A} to get an expression for $\Sigma$ up to $\Oh[5]{(\pi/f)}$
\begin{align*}
    \Sigma 
    % & =   
    % \Big( \cos{\frac{\alpha}{2}} + i \tau_1 \sin{\frac{\alpha}{2}} \Big) 
    % \left(
    %     1
    %     + i \frac{\pi_a \tau_a}{f} 
    %     - \frac{\pi_a^2}{2f^2}
    %     - i \frac{\pi_a^2\pi_b \tau_b}{6f^3}
    %     + \frac{\pi_a^2\pi_b^2}{24f^4}    
    % \right)
    % \Big( \cos{\frac{\alpha}{2}} + i \tau_1 \sin{\frac{\alpha}{2}} \Big) \\
    & =
    \left(
        1
        + i \frac{\pi_a \tau_a}{f} 
        - \frac{\pi_a^2}{2f^2}
        - i \frac{\pi_a^2\pi_b \tau_b}{6f^3}
        + \frac{\pi_a^2\pi_b^2}{24f^4}    
    \right)
    \cos^2{\frac{\alpha}{2}} \\
    & -
    \left(
        1
        + i \frac{\pi_a}{f} \tau_1\tau_a\tau_1
        - \frac{\pi_a^2}{2f^2}
        - i \frac{\pi_a^2\pi_b}{6f^3} \tau_1\tau_b\tau_1
        + \frac{\pi_a^2\pi_b^2}{24f^4}
    \right)
    \sin^2{\frac{\alpha}{2}}\\
    & + i
    \left(
        2 \tau_1
        + i \frac{\pi_a}{f} \acom{\tau_1}{\tau_a}
        - 2\tau_1 \frac{\pi_a^2}{2f^2}
        - i \frac{\pi_a^2\pi_b}{6f^3} \acom{\tau_1}{\tau_b}
        + 2\tau_1 \frac{\pi_a^2\pi_b^2}{24f^4}
    \right)
    \sin{\frac{\alpha}{2}}\cos{\frac{\alpha}{2}}.
\end{align*}
Using trigonometric identities and the commutator,
\begin{align*}
    \cos^2{\frac{\alpha}{2}} - \sin^2{\frac{\alpha}{2}} = \cos{\alpha}, \quad 
    2 \cos{\frac{\alpha}{2}} \sin{\frac{\alpha}{2}} = \sin{\frac{\alpha}{2}}, \quad
    \tau_1 \tau_a \tau_1
    = -\tau_a + 2 \delta_{1a}\tau_1,
\end{align*}
the final expression of $\Sigma$ to $\Oh[5]{(\pi/f)}$ is
\begin{align}
    \Sigma =
     \left(
        1 
        - \frac{\pi_a^2}{2f^2}
        + \frac{\pi_a^2\pi_b^2}{24f^4}
    \right)
    (\cos{\alpha} + i \tau_1 \sin{\alpha})
    +
    \left(
        \frac{\pi_a}{f} 
        - \frac{\pi_b^2\pi_a}{6f^3} 
    \right)
    \left(
        i\tau_a - 2i \delta_{a1}\tau_1\sin^2{\frac{\alpha}{2}} - \delta_{a1} \sin{\alpha}
    \right).
    \label{expansion of sigma}
\end{align}

The kinetic term in the \chpt Lagrangian is
\begin{equation}
    \nabla_\mu \Sigma (\nabla^\mu \Sigma)^\dagger 
    = \partial_\mu \Sigma \partial^\mu \Sigma^\dagger 
    - i \left(\partial_\mu \Sigma \com{v^\mu}{\Sigma^\dagger} - \hc \right)
    - \com{v_\mu}{\Sigma}\com{v_\mu}{\Sigma^\dagger}.
    \label{kinetic term}
\end{equation}
Using \autoref{expansion of sigma} we find the expansion of the constitutive parts of the kinetic term to be
\begin{align}
    \notag
    \partial_\mu \Sigma 
    = &
    % \left(
    %     \frac{-1}{f^2}
    %     + \frac{\pi_b^2}{6f^4}
    % \right)
    % (\cos{\alpha} + i \tau_1 \sin{\alpha}) (\pi_a \partial_\mu \pi_a)\\\notag
    % +&
    % \left(
    %     \frac{\partial_\mu \pi_a}{f} 
    %     - \frac{\pi_b^2 \partial_\mu\pi_a 
    %     + 2 \pi_a \pi_b \partial_\mu\pi_b}{6f^3} 
    % \right)
    % \left(
    %     i\tau_a - 2i \delta_{a1}\tau_1\sin^2{\frac{\alpha}{2}} - \delta_{1a} \sin{\alpha}
    % \right)
    % \\ \notag
    % =& 
    \left[
        \left(
            \frac{-1}{f^2}
            + \frac{\pi_b^2}{6f^4}
        \right)
        (\pi_a \partial_\mu \pi_a)
        \cos{\alpha}
        - 
        \left(
            \frac{\partial_\mu \pi_1}{f} 
            - \frac{\pi_b^2 \partial_\mu\pi_1
            + 2 \pi_1 \pi_b \partial_\mu\pi_b}{6f^3} 
        \right)
        \sin{\alpha}
    \right]
    \\ \notag 
    - &
    \left[
        \left(
            \frac{-1}{f^2}
            + \frac{\pi_b^2}{6f^4}
        \right)
        (\pi_a \partial_\mu \pi_a)
        \sin{\alpha}
        - \left(
        \frac{\partial_\mu \pi_1}{f} 
        - \frac{\pi_b^2 \partial_\mu\pi_1
        + 2 \pi_1 \pi_b \partial_\mu\pi_b}{6f^3}
        \right)
        2 \sin^2{\frac{\alpha}{2}}
    \right]
    i \tau_1 \\ \label{Sigma derivative}
    +& 
    \left(
        \frac{\partial_\mu \pi_a}{f} 
        - \frac{\pi_b^2 \partial_\mu\pi_a 
        + 2 \pi_a \pi_b \partial_\mu\pi_b}{6f^3} 
    \right)
    i \tau_a,
\end{align}
and
\begin{align}
    % \notag
    \com{v_\mu}{\Sigma} 
    % & = 
    % \frac{1}{2} \mu_I \delta^0_\mu
    % \left[
    %     \left(
    %         1 
    %         - \frac{\pi_a^2}{2f^2}
    %         + \frac{\pi_a^2\pi_b^2}{24f^4}
    %     \right)
    %     i \sin{\alpha} \com{\tau_3}{\tau_1}
    %     + 
    %     \left(
    %         \frac{\pi_a}{f} 
    %         - \frac{\pi_b^2\pi_a}{6f^3} 
    %     \right)
    %     \left(
    %         i\com{\tau_a}{\tau_3} 
    %         - 2i\delta_{a1}\sin^2{\frac{\alpha}{2}}\com{\tau_3}{\tau_1}
    %     \right)
    % \right] \\
    % \notag
    % & =
    % -\mu_I \delta^0_\mu
    % \left\{
    %     \left(
    %         1 
    %         - \frac{\pi_a^2}{2f^2}
    %         + \frac{\pi_a^2\pi_b^2}{24f^4}
    %     \right)
    %     \tau_2 \sin{\alpha}
    %     + 
    %     \left(
    %         \frac{\pi_a}{f} 
    %         - \frac{\pi_b^2\pi_a}{6f^3} 
    %     \right)
    %     \left[
    %         (\delta_{a1} \tau_2 - \delta_{a2} \tau_1)
    %         - 2\delta_{a1} \tau_2 \sin^2{\frac{\alpha}{2}}
    %     \right]
    % \right\} \\
    & =
    -\mu_I \delta^0_\mu
    \left\{
        \left[
        \left(
            1 
            - \frac{\pi_a^2}{2f^2}
            + \frac{\pi_a^2\pi_b^2}{24f^4}
        \right)
        \sin{\alpha}
        + 
        \left(
            \frac{\pi_1}{f} 
            - \frac{\pi_b^2\pi_1}{6f^3} 
        \right) \cos{\alpha}
        \right]
         \tau_2
        -
        \left(
            \frac{\pi_2}{f} 
            - \frac{\pi_b^2\pi_2}{6f^3} 
        \right)
        \tau_1
    \right\}.
    \label{sigma commutator}
\end{align}
Combining \autoref{Sigma derivative} and \autoref{sigma commutator} gives the following terms \footnote{The scripts used to aid the calculation of the Lagrangian is available at \url{https://github.com/martkjoh/prosjektopggave}}
\begin{align*}
    % Term 1
    & \Tr{\partial_\mu \Sigma \partial^\mu \Sigma^\dagger}
    = \frac{2}{f^2} \partial_\mu \pi_a \partial^\mu \pi_a
    + \frac{2}{3f^4}
    \left[
        (\pi_a\partial_\mu \pi_a)(\pi_b\partial^\mu \pi_b)
        -        
        (\pi_a\partial_\mu \pi_b)(\pi_b\partial^\mu \pi_a)
    \right], \\
    % Term 2
    -i  &\Tr{\partial^\mu\Sigma\com{v_\mu}{\Sigma^\dagger} - \hc}
    =
    4 \mu_I \frac{\partial_0\pi_2}{f}
    + 8 \mu_I \frac{\pi_3}{3f^3}\sin{\alpha}(
        \pi_2 \partial_0 \pi_3 - \pi_3 \partial_0 \pi_2
        ) \sin{\alpha}
    \\ & \quad \quad \quad \quad \quad \quad \quad \quad \quad \quad \quad
    +
    \left(
        \frac{4\mu_I}{f^2} \cos{\alpha}
        - \frac{8 \mu_I\pi_1}{3f^3} \sin{\alpha}
        - \frac{4 \mu_I \pi_a \pi_a} {3f^4}\cos{\alpha} 
    \right) 
    (\pi_1\partial_0 \pi_2 - \pi_2 \partial_0 \pi_1), \\
    % Term 3
    - & \Tr{\com{v_\mu}{\Sigma}\com{v^\mu}{\Sigma^\dagger}}
    = \mu_I{}^2
    \bigg[
        2 \sin^2{\alpha}
        +
        \left(
            \frac{2}{f} 
            - \frac{4\pi_a \pi_a}{3 f^3} 
        \right)
        \pi_1  \sin{2\alpha}
        + \left(
            \frac{2}{f^2}
            - \frac{2 \pi_a \pi_a}{3 f^4} 
        \right)
        \pi_a \pi_b k_{ab}
    \bigg], 
    \\
    % Mass Term
    & \Tr{\chi^\dagger \Sigma + \Sigma^\dagger\chi}
    = 
    \bar m^2 
    \left(
        4 \cos{\alpha} 
        - \frac{4 \pi_1}{f} \sin{\alpha} 
        - \frac{2 \pi_a \pi_a}{f^2} \cos{\alpha}
        + \frac{2 \pi_1 \pi_a \pi_a}{3 f^3} \sin{\alpha}
        + \frac{(\pi_a \pi_a)^2}{6 f^4}\cos{\alpha}
    \right), 
    \end{align*}
where $k_{ab} =\delta_{a1} \delta_{b1} \cos{2\alpha}  + \delta_{a2}\delta_{b2}\cos^2{\alpha} - \delta_{a3}\delta_{b3} \sin^2{\alpha}$.
Notice that the mass term is independent of the difference in quark masses, $\Delta m$.
If we write the Lagrangian as show in \autoref{chpt lagrangian} as $\Ell_2 = \Ell_2^{(0)} + \Ell_2^{(1)} + \Ell_2^{(2)} +...$, where $\Ell_2^{(n)}$ contains all terms of order $\Oh[n]{(\pi/f)}$, then the result of the series expansion is
\begin{align}
%%%%%%%%%%%%%%%%%%
%% zeroth order %%
%%%%%%%%%%%%%%%%%%
\Ell_2^{(0)}
&  =
    f^2   
    \left(
        \bar m^2 \cos{\alpha}
        + \frac{1}{2} \mu^2 \sin^2{\alpha}
    \right),
    \label{L0}
\\
%%%%%%%%%%%%%%%%%%
%% first order %%
%%%%%%%%%%%%%%%%%%
\label{L1}
\Ell_2^{(1)}
& =
    f 
    (
        \mu_I^2\cos{\alpha}
        - \bar m^2
    ) \pi_1 \sin{\alpha}
    + f \mu_I \partial_0\pi_2 \sin{\alpha},
\\
%%%%%%%%%%%%%%%%%%
%% second order %%
%%%%%%%%%%%%%%%%%%
\Ell_2^{(2)}
& =
    \frac{1}{2} \partial_\mu\pi_a\partial^\mu\pi_a
    + \mu_I \cos{\alpha} \left( \pi_1 \partial_0\pi_2 - \pi_2\partial_0\pi_1 \right)
    - \frac{1}{2} \bar m^2 \pi_a \pi_a \cos{\alpha}
    + \frac{1}{2} \mu_I ^2 \pi_a \pi_b k_{ab},
\label{L2}
\\
%%%%%%%%%%%%%%%%%%
%% third order %%
%%%%%%%%%%%%%%%%%%
\notag
\Ell_2^{(3)}
& =
    \frac{\pi_a\pi_a \pi_1}{6f}
    (\bar m^2 \sin{\alpha}-2\mu_I{}^2 \sin{2\alpha})\\ \label{L3}
    &
    -
    \frac{2 \mu_I}{3 f}
    \left[
        \pi_1(\pi_1 \partial_0\pi_2 - \pi_2\partial_0\pi_1)
        +
        \pi_3(\pi_3\partial_0\pi_2-\pi_2 \partial_0\pi_3)
    \right]
    \sin{\alpha},
\\
%%%%%%%%%%%%%%%%%%
%% fourth order %%
%%%%%%%%%%%%%%%%%%
\notag
\Ell_2^{(4)}
& =
\frac{1}{6f^2}
\curly{    
    \frac{1}{4} \bar m^2 (\pi_a\pi_a)^2 \cos{\alpha}
    -
    \left[
        (\pi_a \pi_a) (\partial_\mu \pi_b \partial^\mu \pi_b )
        - (\pi_a \partial_\mu \pi_a)(\pi_b \partial^\mu \pi_b )
    \right]
}
\\
&
- \frac{\mu_I \pi_a\pi_a}{3f^2}
\left[
    \left(\pi_1\partial_0 \pi_2 - \pi_2 \partial_0 \pi_1\right)
    \cos{\alpha}
    + \frac{1}{2} \mu_I \pi_a \pi_b k_{ab}
\right].
\label{L4}
\end{align}

\section{Equation of motion and redundant terms}

Changing the field parametrization that appear in the Lagrangian does not affect any of the physics, as it corresponds to a change of variables in the path integral~\cite{Scherer2002IntroductionTC,Chisholm:changeOfVar,Kamefuchi:changeOfVar}.
However, a change of variables can result in new terms in the Lagrangian.
As a result of this, terms that on the face of it appear independent may be redundant.
These terms can be eliminated by using the classical equation of motion.
In this section we show first the derivation of the equation of motion, then use this result to identify redundant terms which need not be included in the most general Lagrangian.

We derive the equation of motion for the leading order Lagrangian using the principle of least action.
Choosing the parametrization $\Sigma = \exp(i \pi_a \tau_a)$, a variation $\pi_a \rightarrow \pi_a + \delta \pi_a$ results in a variation in $\Sigma$, $\delta \Sigma = i \tau_a \delta \pi_a \Sigma $.
The variation of the leading order action,
\begin{equation}
    S_2 = \int_\Omega \dd^4x \, \Ell_2,
\end{equation}
when varying $\pi_a$ is 
\begin{align*}
    \delta S = \int_\Omega \dd x \, \frac{f^2}{4}
    \Tr{
        (\nabla_\mu \delta \Sigma) (\nabla^\mu \Sigma)^\dagger
        + (\nabla_\mu \Sigma) (\nabla^\mu \delta \Sigma)^\dagger
        + \chi \delta \Sigma^\dagger + \delta \Sigma \chi^\dagger
    }.
\end{align*}
Using the properties of the covariant derivative to do partial integration, as show in \autoref{Covarinat derivative}, as well as $\delta(\Sigma \Sigma^\dagger) = (\delta\Sigma)\Sigma^\dagger + \Sigma (\delta \Sigma)^\dagger = 0$, the variation of the action can be written
\begin{align*}
    \delta S 
    & = \frac{f^2}{4} \int_\Omega \dd x\, 
    \Tr{
        - \delta \Sigma \nabla^2 \Sigma^\dagger
        + (\nabla^2 \Sigma) (\Sigma^\dagger \delta \Sigma \Sigma^\dagger)
        - \chi (\Sigma^\dagger \delta \Sigma \Sigma^\dagger)
        + \delta \Sigma \chi^\dagger
    } \\
    & = 
    \frac{f^2}{4} \int_\Omega \dd x\, 
    \Tr{
        \delta \Sigma \Sigma^\dagger 
        \left[
            (\nabla^2 \Sigma)\Sigma^\dagger
            - \Sigma \nabla^2 \Sigma^\dagger
            - \chi \Sigma^\dagger
            + \Sigma \chi^\dagger
        \right]
        } \\
    & = 
    i \frac{f^2}{4} \int_\Omega \dd x\, 
    \Tr{\tau_a 
    \left[
         (\nabla^2 \Sigma)\Sigma^\dagger
        - \Sigma \nabla^2 \Sigma^\dagger
        - \chi \Sigma^\dagger
        + \Sigma \chi^\dagger
    \right]
    } 
    \delta \pi_a = 0.
\end{align*}  
As the variation is arbitrary, the equation of motion to leading order is
\begin{equation}
    \Tr{
        \tau_a 
        \left[
            (\nabla^2 \Sigma)\Sigma^\dagger
            - \Sigma \nabla^2 \Sigma^\dagger
            - \chi \Sigma^\dagger
            + \Sigma \chi^\dagger
        \right]
    } = 0.
\end{equation}
This may be rewritten as a matrix equation. 
Using that 
\begin{align*}
    \Tr{(\nabla_\mu \Sigma)\Sigma^\dagger}
    = 
    \Tr{i \tau_a (\partial_\mu \pi_a )\Sigma \Sigma^\dagger}
    - i\Tr{[v_\mu, \Sigma]\Sigma^\dagger}
    = 0,
\end{align*}
we can see that $\Tr{(\nabla^2 \Sigma)\Sigma^\dagger - \Sigma \nabla^2 \Sigma^\dagger} = 0$, and the equation of motion may be written as
\begin{equation}
    \label{EOM matrix form}
    \mathcal{O}_\mathrm{EOM}^{(2)}(\Sigma) 
    = 
    (\nabla^2 \Sigma)\Sigma^\dagger
    - \Sigma \nabla^2 \Sigma^\dagger
    - \chi \Sigma^\dagger
    + \Sigma \chi^\dagger
    - \frac{1}{2}
    \Tr{ \chi \Sigma^\dagger - \Sigma \chi^\dagger} = 0.
\end{equation}

The next step in eliminating redundant terms is to change the parametrization of $\Sigma$ by $\Sigma(x) \rightarrow \Sigma'(x)$. 
Here, $ \Sigma(x) = e^{iS(x)} \Sigma'(x), \, S(x) \in \liea{su}{2}$. This change leads to a new Lagrange density, $\Ell[\Sigma] = \Ell[\Sigma'] + \Delta \Ell[\Sigma']$.
We are free to choose $S(x)$, as long $\Sigma'$ still obeys the required transformation properties.
Any terms in the Lagrangian $\Delta \Ell$ due to a reparametrization can be neglected, as argued earlier.
When demanding that $\Sigma'$ obey the same symmetries as $\Sigma$,
the most general transformation to second order in Weinberg's power counting scheme  is~\cite{Scherer2002IntroductionTC}
\begin{equation}
    \label{S reparametrization}
    S_{2} = 
    i \alpha_2 
    \left[
        (\nabla^2 \Sigma') \Sigma^\dagger - \Sigma' (\nabla^2 {\Sigma'})^\dagger
    \right]
    + i \alpha_2
    \left[
        \chi \Sigma'^\dagger - \Sigma' \chi^\dagger 
        - \frac{1}{2} \Tr{\chi \Sigma'^\dagger - \Sigma' \chi^\dagger}
    \right].
\end{equation}
$\alpha_1$ and $\alpha_2$ are arbitrary real numbers. As \autoref{S reparametrization} is to second order, $\Delta \Ell$ is fourth order in Weinberg's power counting scheme.
To leading order is given by
\begin{align*}
    \Ell_2\left[e^{i S_2}\Sigma '\right]
    & =
    \frac{f^2}{4}\Tr{[\nabla_\mu (1 +i S_2)\Sigma'][\nabla^\mu \Sigma'^\dagger  (1 - i S_2)]}
    + \frac{f^2}{4} \Tr{\chi\Sigma'^\dagger (1 - i S_2) + (1 +i S_2)\Sigma' \chi^\dagger} \\
    & = \Ell[\Sigma'] + 
    i \frac{f^2}{4}
    \Tr{[\nabla_\mu (S_2\Sigma')][\nabla^\mu\Sigma']^\dagger 
    -  [\nabla_\mu\Sigma'][\nabla^\mu (\Sigma'^\dagger  S_2) ]}
    - i \frac{f^2}{4} \Tr{\chi \Sigma'^\dagger S_2 - S_2 \Sigma' \chi^\dagger}
\end{align*}
Using the properties of the covariant derivative, as described in \autoref{Covarinat derivative}, we may use the product rule and partial integration to write the difference between the two Lagrangians to fourth order as
\begin{align*}
    \Delta \Ell[\Sigma'] 
    & = 
    i \frac{f^2}{4}
    \Tr{
        (\nabla_\mu S_2)
        (\Sigma' \nabla^\mu \Sigma'^\dagger - (\nabla^\mu \Sigma') \Sigma'^\dagger) 
    }
    - i \frac{f^2}{4} \Tr{\chi \Sigma'^\dagger  S_2 - S_2 \Sigma' \chi^\dagger} \\
    & = 
    i \frac{f^2}{4}
    \Tr{
        S_2
        \left[
            \Sigma'^\dagger\nabla^2 \Sigma' - ( \nabla^2 \Sigma') \Sigma'^\dagger 
            - \chi\Sigma'^\dagger + \Sigma' \chi^\dagger
        \right]
    }.
\end{align*}
Using the equation of motion \autoref{EOM matrix form}, and the fact that $\Tr{S_2} = 0$, this difference can be written as
\begin{align}
    \label{Delta reparametrization}
    \Delta \Ell[\Sigma'] = \frac{f^2}{4} \Tr{i S_2 \mathcal{O}_\mathrm{EOM}^{(2)}(\Sigma')}.
\end{align}
Any term that can be written in the form of \autoref{Delta reparametrization} for arbitrary $\alpha_1, \alpha_2 \in \R$ is redundant, as we argued earlier, and may therefore be discarded.
$\Delta \Ell_2$ is of fourth order, and it can thus be used to remove terms from $\Ell_4$ or higher order.

\subsection{Next to leading order Lagrangian}

The next to leading order Lagrangian density is, assuming no external fields
\begin{align}
    \notag
    \Ell_4 
    & = 
    \frac{l_1}{4} \Tr{\nabla_\mu \Sigma (\nabla^\mu \Sigma)^\dagger}^2
    + \frac{l_2}{4} \Tr{\nabla_\mu \Sigma (\nabla_\nu \Sigma)^\dagger} 
    \Tr{\nabla^\mu \Sigma (\nabla^\nu \Sigma)^\dagger} 
    +
    \frac{l_3 + h_1 - h_3 }{16} \Tr{\chi \Sigma^\dagger + \Sigma \chi^\dagger}^2
    \\ \notag
    &
    + \frac{l_4}{8}\Tr{\nabla_\mu \Sigma (\nabla^\mu \Sigma)^\dagger} \Tr{\chi \Sigma^\dagger + \Sigma \chi^\dagger}
    + \frac{h_1 - h_3 - l_4-l_7}{16} \Tr{\chi \Sigma^\dagger - \Sigma \chi^\dagger}^2
    + \frac{h_1 + h_3 - l_4}{4} \Tr{\chi \chi^\dagger} \\
    & -
    \frac{h_1 - h_3 - l_4}{8} 
        \Tr{\left(\chi \Sigma^\dagger\right)^2 + \left( \Sigma \chi^\dagger\right)^2}
    \label{NLO Lagrangian}
\end{align}
 
To $\Ell_4$ to $\Oh[3]{(\pi/f)}$, we use the result from \autoref{Sigma derivative} and \autoref{sigma commutator},
% \begin{align*}
%     \Sigma & =
%     \left(
%        1 
%        - \frac{\pi_a^2}{2f^2}
%    \right)
%    (\cos{\alpha} + i \tau_1 \sin{\alpha})
%    +  \frac{\pi_a}{f}
%    \left(
%        i\tau_a 
%        - 2i \delta_{a1} \tau_1 \sin^2{\frac{\alpha}{2}}
%        - \delta_{a1} \sin{\alpha}
%    \right), \\
%     \partial_\mu \Sigma 
%     & = 
%     - \frac{\pi_a \partial_\mu \pi_a}{f^2}
%     \left(\cos{\alpha} + i \tau_1 \sin{\alpha}\right)
%     + \frac{\partial_\mu \pi_a}{f}
%     \left(
%         i\tau_a 
%         -2 i \delta_{a1} \tau_1 \sin^2{\frac{\alpha}{2}}
%         - \delta_{a1}\sin{\alpha}\right), \\
%     [v_\mu, \Sigma^\dagger] 
%     & =
%     \mu_I \delta^0_\mu
%     \left[
%         \left( 1 - \frac{\pi_a^2}{2f^2} \right) \tau_2 \sin{\alpha}
%         + \frac{\pi_a}{f}
%         \left(
%             \delta_{a1} \cos{\alpha} \tau_2 - \delta_{a2} \tau_1
%         \right)
%     \right].
%     \end{align*}
up to and including $\Oh{(\pi/f)}$, which gives
\begin{align*}
    %%%%%%%%%%%%%%%%%%
    % parital square %
    %%%%%%%%%%%%%%%%%%
    \Tr{\partial_\mu \Sigma \partial_\nu \Sigma^\dagger}
    & = 2 \frac{\partial_\mu \pi_a \partial_\nu \pi_a}{f^2} \\
    %%%%%%%%%%%%%%
    % Cross term %
    %%%%%%%%%%%%%
    -i \Tr{\partial_\mu \Sigma \com{v_\nu}{\Sigma^\dagger} - \hc}
    & = 
    \frac{2\mu_I\pi_2}{f}(\delta_\mu^0\partial_\nu + \delta_\nu^0\partial_\mu)\sin{\alpha} + 
    \frac{2\mu_I}{f^2}
    [
        \pi_1 (\delta_\mu^0\partial_\nu + \delta_\nu^0\partial_\mu)\pi_2 
        - \pi_2 (\delta_\mu^0\partial_\nu + \delta_\nu^0\partial_\mu)\pi_1
    ]\cos{\alpha}
    \\
    %%%%%%%%%%%%%%%%%
    % Comm. squared %
    %%%%%%%%%%%%%%%%%
    - \Tr{\com{v_\nu}{\Sigma} \com{v_\nu}{\Sigma^\dagger}}
    & = 2 \mu_I{}^2 \delta_\mu^0 \delta_\nu^0 
    \left[
        \sin^2{\alpha} + \frac{\pi_1}{f} \sin{2\alpha} 
        + \frac{\pi_a \pi_b}{f^2} 
        k_{ab}
    \right].
\end{align*}
Using the form of the Pauli matrices, we can write $\chi$ as 
\begin{align*}
    \chi = 2 B_0 M = 2 B_0 (\bar m \one + \Delta m \tau_3),
\end{align*}
where $\bar m  = (m_u + m_d)/2, \, \Delta m = (m_u - m_d)/2$, which gives 
\begin{align*}
    \chi \Sigma^\dagger + \Sigma \chi^\dagger
    & = 4 B_0 \Bigg\{
         (\bar m + \Delta m \tau_3)
        \left[
            \left(
                1 
                - \frac{\pi_a^2}{2f^2}
            \right)
            \cos{\alpha}
            - \frac{\pi_1}{f}    
            \sin{\alpha}
        \right]\\
    & \quad \quad \quad
    + \Delta m 
    \left[
        \left(
            1 
            - \frac{\pi_a^2}{2f^2}
        \right)
        \tau_2 \sin{\alpha}
        +  \frac{\pi_a}{f}
        \left(
            \delta_{a1} \tau_2 \cos{\alpha} - \delta_{a2} \tau_1
        \right)
    \right]
    \Bigg\}, \\
    %%%%%%%%%%%%%%
    % Difference %
    %%%%%%%%%%%%%%
    \chi \Sigma^\dagger  - \Sigma \chi^\dagger
    & = - 4 i B_0 \Bigg\{
        \bar m 
        \left[
            \left(
                1 - \frac{\pi_a^2}{2f^2}
            \right)
            \tau_1 \sin{\alpha}
            +  \frac{\pi_a}{f}    \left(
                \tau_a 
                - 2 \delta_{1a} \tau_1 \sin^2{\frac{\alpha}{2}}
            \right)        
        \right]
        + \Delta m 
            \frac{\pi_3}{f}
    \Bigg\}.
\end{align*}
Combining these results gives all the terms in $\Ell_4$, to $\Oh[3]{(\pi/f)}$:
%%%%%%%%%%%%%%%%%
% Parts for L_4 %
%%%%%%%%%%%%%%%%%
\begingroup
\allowdisplaybreaks % Make page break possible
\begin{align}
    %%%%%%%
    % l_1 %
    %%%%%%%
    \notag
    & \Tr{\nabla_\mu \Sigma (\nabla^\mu \Sigma)^\dagger}^2 
    =
    \Tr{\partial_\mu \Sigma \partial^\mu \Sigma ^\dagger
    - i \left( \partial_\mu \Sigma \com{v^\mu}{\Sigma^\dagger} - \hc \right)
    - \com{v_\mu}{\Sigma}\com{v^\mu}{\Sigma^\dagger} 
    }^2 \\\notag
    &\quad  =
    \frac{8 \mu_I^2}{f^2} 
    (\partial_\mu \pi_a \partial^\mu \pi_a + 2 \partial_\mu \pi_2 \partial^\mu \pi_2)
    \sin^2{\alpha} \\\notag
    &\quad  + 16 \mu_I^3 \left[
        \frac{\partial_0 \pi_2}{f}
            \sin^3{\alpha}
        + \frac{1}{f^2} \left(3 \pi_1 \partial_0 \pi_2 - \pi_2 \partial_0 \pi_1\right)
            \cos{\alpha} \sin^2{\alpha}
    \right] \\ \label{l1}
    & \quad + 4 \mu_I^4 
    \left\{
        \sin^4{\alpha}
        + 2 \sin^2{\alpha}
        \left[
            \frac{\pi_1}{f}\sin{2\alpha}
            + \frac{\pi_a \pi_b}{f^2}        
            \left(k_{ab} + 2\delta_{a1}\delta_{a2}\cos^2{\alpha} \right)
        \right]
    \right\}, \\
    %%%%%%%
    % l_2 %
    %%%%%%%
    \notag
    & \Tr{\nabla_\mu \Sigma (\nabla_\nu \Sigma)^\dagger} \Tr{\nabla^\mu \Sigma (\nabla^\nu \Sigma)^\dagger}\\ \notag
    & \quad = \frac{4 \mu_I{}^2}{f^2}
    \left(
        \partial_0 \pi_a\partial_0 \pi_a + \partial_0 \pi_2\partial_0 \pi_2 + \partial_\mu \pi_2\partial^\mu \pi_2
    \right) \sin^2{\alpha} \\\notag
    &\quad  + 16 \mu_I^3 \left[
        \frac{\partial_0 \pi_2}{f}\
            \sin^3{\alpha}
        + \frac{1}{f^2} \left(3 \pi_1 \partial_0 \pi_2 - \pi_2 \partial_0 \pi_1\right)
        \cos{\alpha} \sin^2{\alpha}
    \right] \\\label{l2}
    & \quad + 4 \mu_I^4 
    \left\{
        \sin^4{\alpha}
        + 2 \sin^2{\alpha}
        \left[
            \frac{\pi_1}{f}\sin{2\alpha} 
            + \frac{\pi_a \pi_b}{f^2}        
            \left(k_{ab} + 2 \delta_{a1}\delta_{a2} \cos^2{\alpha} \right)
        \right]
    \right\}, \\
    %%%%%%%
    % l_4 %
    %%%%%%%
    \notag
    & \Tr{\nabla_\mu \Sigma (\nabla^\mu \Sigma)^\dagger} 
    \Tr{\chi \Sigma^\dagger + \Sigma \chi^\dagger } \\\label{l4}
    & \quad =
    8 B_0 \bar m 
    \Bigg\{
        2 \frac{\partial_\mu \pi_a \partial^\mu \pi_a}{f^2} \cos{\alpha}
        + 4 \mu_I 
        \left[
            \frac{\partial_0 \pi_2}{2 f} \sin{2\alpha}
            + \frac{1}{f^2}
            \left(
                \pi_1 \partial_0 \pi_2 \cos{2\alpha}
                - \pi_2 \partial_0 \pi_1\cos^2{\alpha}
            \right)
        \right]\\\notag
        & \quad + \mu_I{}^2
        \left[
            2\cos{\alpha}\sin^2{\alpha} 
            - 2 \frac{\pi_1}{f} \sin{\alpha}
            \left(2 - 3\sin^2{\alpha}\right)
            + \frac{1}{f^2}
            \left(                
                \pi_1^2[2 - 9 \sin^2{\alpha}]
                + \pi_2^2 [2 - 3 \sin^2{\alpha}]
                - 3\pi_3^2\sin^2{\alpha}
            \right)
            \cos{\alpha}
        \right]
    \Bigg\}, \\
    %%%%%%%
    % l_3 %
    %%%%%%%
    \label{l3}
    & \Tr{\chi \Sigma^\dagger + \Sigma \chi^\dagger }^2
    = (8 B_0 \bar m)^2 
    \left[
        \cos^2{\alpha} 
        - \frac{\pi_1}{f} \sin{2\alpha}
        + \frac{1}{f^2}\left(\pi_1^2 \sin^2{\alpha} - \pi_a \pi_a \cos^2{\alpha}\right)
    \right], \\
    %%%%%%%
    % l_7 %
    %%%%%%%
    \label{l7}
    & \Tr{\chi \Sigma^\dagger - \Sigma \chi^\dagger }^2
     = - 16 \left( \frac{2 \Delta m B_0 \pi_3}{f} \right)^2, \\
    %%%%%%%
    % h_3 %
    %%%%%%%
    \notag
    & \Tr{\left(\chi \Sigma^\dagger\right)^2 + \left(\Sigma \chi^\dagger \right)^2}
    \\\label{h3}
    & \quad \quad = 16 B_0^2 \bar m^2
    \left(
        \cos{2\alpha} 
        - 2\frac{\pi_1}{f} \sin{2\alpha}
        - 2\frac{\pi_a \pi_a}{f^2} \cos^2{\alpha}
        + 2\frac{\pi_1^2}{f^2} \sin^2{\alpha}
    \right)
    + 16 B_0^2 \Delta m^2
    \left(
        1- 2\frac{ \pi_3^2}{f^2}
    \right)
    , \\
    %%%%%%%
    % h_2 %
    %%%%%%%
    \label{h2}
    & \Tr{\chi^\dagger \chi} = 8B_0^2\left( {\bar m}^2 + {\Delta m}^2\right).
\end{align}

\endgroup

The different terms of the NLO Lagrangian is
\begin{align*}
    \Ell_4^{(0)} & =
    %
    %
    % zeroth order
    %
    %
    (l_1 + l_2)\mu_I^4 \sin^4{\alpha}
    + (l_3 + l_4)(2 B_0 \bar m)^2 \cos^2{\alpha}
    + l_4 (2 B_0 \bar m ) \mu_I{}^2 \cos{\alpha} \sin^2{\alpha}
    - l_4 (2B_0 \bar m )^2
    + h_1 (2B_0 \bar m)^2
    + h_3 (2B_0 \Delta m)^2
    \\
    %
    %
    % first order
    %
    %
    \Ell_4^{(1)} & =
    \frac{l_1 + l_2}{f}
    \left(
        4 \mu_I^3 
        \partial_0 \pi_2 \sin^3{\alpha}
        % term 5 pi
        + \mu_I^4 
        2 \sin^2{\alpha} \pi_1 \sin{2\alpha}
    \right)
    % term 13 pi
    -
    \frac{l_3 + l_4}{f}
    (2 B_0 \bar m)^2
    \pi_1 \sin{2\alpha}
    %%%%%%%% 
    \\
    & 
    +
    2 B_0 \bar m
    \frac{l_4}{f}
    \left[
        % term 16 pi
        2 \mu_I 
        \partial_0 \pi_2 \sin{2\alpha}
        - 2 \mu_I{}^2
        % term 19 pi
        \pi_1 \sin{\alpha}
        \left(3\sin^2{\alpha} - 2\right)
    \right]
    \\
    %
    %
    % second order
    %
    %
    \Ell_4^{(2)} & = 
    l_1
    \frac{2 \mu_I^2}{f^2}
        (\partial_\mu \pi_a \partial^\mu \pi_a + 2 \partial_\mu \pi_2 \partial^\mu \pi_2)
        \sin^2{\alpha}
    + l_2 
    \frac{4 \mu_I{}^2}{f^2}
    % term 7 pi^2
    \left(
        \partial_0 \pi_a\partial_0 \pi_a 
        + \partial_0 \pi_2\partial_0 \pi_2
        + \partial_\mu \pi_2 \partial^\mu \pi_2 
    \right) 
    \sin^2{\alpha}
    \\
    & + 
    \frac{l_1 + l_2}{f^2}
    \left[
        % term 3 pi^2
        4 \mu_I^3 
        \left( 3 \pi_1 \partial_0 \pi_2 - \pi_2 \partial_0 \pi_1 \right)
        \cos{\alpha} \sin^2{\alpha}
        % term 6 pi^2
        + 2 \mu_I^4  \sin^2{\alpha} \, \pi_a \pi_b 
            \left(k_{ab} + 2\delta_{a1}\delta_{a2}\cos^2{\alpha} \right)
    \right]
    % 
    \\
    & +
    \frac{l_3 + l_4}{f^2}
    (2 B_0 \bar m)^2
    % term 14 pi^2
    \left(\pi_1^2 \sin^2{\alpha} - \pi_a \pi_a \cos^2{\alpha}\right)
    % \\
    % &
    +  \frac{l_4}{f^2}
    2 B_0 \bar m
    \bigg[
    % term 15 pi^2
    2 \partial_\mu \pi_a \partial^\mu \pi_a \cos{\alpha}
    + 4 \mu_I 
    % term 17 pi^2
    \left(
        \pi_1 \partial_0 \pi_2 \cos{2\alpha}
        - \pi_2 \partial_0 \pi_1\cos^2{\alpha}
    \right)
    \\
    & 
    + \mu_I{}^2
    % term 20 pi^2
    \left(                
        \pi_1^2[2 - 9 \sin^2{\alpha}]
        + \pi_2^2 [2 - 3 \sin^2{\alpha}]
        - 3\pi_3^2\sin^2{\alpha}
    \right)
    \cos{\alpha}
    \bigg]
    % \\
    % & 
    % term 21 pi^2
    + \frac{l_7}{f^2}
    \left( 2 \Delta m B_0 \right)^2 \pi_3^2
\end{align*}


\subsection{Propagator}
%%%%%%%%%%%%%%%%%
%%%% SECTION %%%%
%%%%%%%%%%%%%%%%%
\label{section:propagator}
We may write the quadratic part of the Lagrangian \autoref{L2} as \footnote{Summation over isospin index ($a,b,c$) will be explicit in this section.}
\begin{align}
    \label{quadratic lagrangian}
    \Ell_2^{(2)}
    =
    \frac{1}{2} \sum_a \partial_\mu \pi_a \partial^\mu \pi_a
    + \frac{1}{2} m_{12} (\pi_1 \partial_0 \pi_2 - \pi_2 \partial_0 \pi_1)
    - \frac{1}{2} \sum_a m_a^2 \pi_a^2,
\end{align}
where
\begin{align}
    m_1^2 &= 2 B_0 m \cos{\alpha} - \mu_I^2 \cos{2\alpha}, \\
    m_2^2 &= 2 B_0 m \cos{\alpha} - \mu_I^2 \cos^2{\alpha}, \\
    m_3^2 &= 2 B_0 m \cos{\alpha} + \mu_I^2 \sin^2{\alpha}, \\
    m_{12} &= 2 \mu_I \cos{\alpha}.
\end{align}
The components of the Euler-Lagrange equations of this field are
\begin{equation*}
    \pdv{\Ell}{\pi_a} = 
    \frac{1}{2} m_{12} (\delta_{a1} \partial_0 \pi_2 - \delta_{a2}\partial_0 \pi_1) 
    - m^2_{a} \pi_a, \quad
    \pdv{\Ell}{(\partial_\mu \pi_a)} = 
    \partial^\mu \pi_a - \frac{1}{2} m_{12} \delta^\mu_0 (\delta_{a1}\pi_2  - \delta_{a2}\pi_1).
\end{equation*}
This gives the equation of motion for the field
\begin{equation}
    \partial^\mu \partial_\mu \pi_a + m_a^2 \pi_a
    =  m_{12}(\delta_{a1} \partial_0 \pi_2  - \delta_{a2} \partial_0 \pi_1).
\end{equation}
The propagator of the pion field is defined by
\begin{equation}
    \left[
        \delta_{ab}(\partial^\mu\partial_\mu + m^2_a)
        -  m_{12}(\delta_{a1} \delta_{b2} - \delta_{a2}\delta_{b1}) \partial_0
    \right] 
    D_{bc}(x, x') 
    = -i \delta(x - x') \delta_{ac}.
\end{equation}
The momentum space propagator, as defined in the \autoref{Conventions and notation}, fulfills
\begin{equation*}
    -\left[
        \delta_{ab}(p^2 - m_a^2)
        +  i p_0 m_{12}(\delta_{a1} \delta_{b2} - \delta_{a2}\delta_{b1}) 
    \right] 
     D_{bc}(p) 
    := D^{-1}_{ab}  D_{bc}(p) = -i \delta_{ac},
\end{equation*}
where
\begin{equation*}
    D^{-1} = -
    \begin{pmatrix}
        p^2 - m^2_1             & i p_0 m_{12}     & 0             \\
        - i p_0 m_{12}            & p^2 - m^2_2       & 0             \\
        0                       & 0                 & p^2 - m^2_3
    \end{pmatrix}.
\end{equation*}
The spectrum of the particles is given by solving $\det(D^{-1}) = 0$ for $p^0$. With $p = (p_0, P)$ as the four momentum, this gives
\begin{align*}
    \det(D^{-1}) & = D^{-1}_{33} \left(D^{-1}_{11} D^{-1}_{22} + (D^{-1}_{12})^2\right)
    = - \left(p^2 - m^2_3\right)
    \left[
        \left(p^2 - m^2_1\right)
        \left(p^2 - m^2_2\right)
        - p_0^2 m_{12}^2
    \right] = 0,
\end{align*}
This equation has the solutions
\begin{align}
    E_0^2 &= P^2 + m_2^2, \\
    E_\pm^2
    & = P^2 +
    \frac{1}{2}
    \left(
        m_1^2 + m_2^2 + m_{12}^2 
    \right)
    \pm 
    \frac{1}{2}
    \sqrt{
        4P^2m_{12}^2 
        +
        \left(
            m_1^2 + m_2^2 + m_{12}^2
        \right)^2
        - 4 m_1^2 m_2^2
    }.
\end{align}
The (tree-level) masses are found by setting $P = 0$, i.e. the rest-frame energy, and are
\begin{align}
    m_0^2 &= m_2^2, \\
    m_\pm^2
    & =  \frac{1}{2}
    \left[
        m_1^2 + m_2^2 + m_{12}^2 
    \right]
    \pm \frac{1}{2}
    \sqrt{
        \left(
            m_1^2 + m_2^2 + m_{12}^2
        \right)^2
        - 4 m_1^2 m_2^2
    }.
\end{align}
The propagator may then be obtained as described in \autoref{Conventions and notation},
\begin{align}
    \notag
    D & = i (D^{-1})^{-1} = \frac{i}{\det(D^{-1})}
    \begin{pmatrix}
        D^{-1}_{22} D^{-1}_{33}   & D^{-1}_{12}D^{-1}_{33}  & 0 \\
        -D^{-1}_{12}D^{-1}_{33}   & D^{-1}_{11}D^{-1}_{33}  & 0 \\
        0               & 0             & D^{-1}_{11}D^{-1}_{22} + (D^{-1}_{12})^2
    \end{pmatrix} \\
    \label{free pion propagator}
    & = i
    \begin{pmatrix}
        \frac{
            p^2 - m_2^2
        }
        {
            (p_0^2 - E_+^2)(p_0^2 - E_-^2)
        } 
        & \frac{
            - ip_0m_{12}
        }
        {
            (p_0^2 - E_+^2)(p_0^2 - E_-^2)
        } & 0 \\
        \frac{
            ip_0m_{12}
        }
        {
            (p_0^2 - E_+^2)(p_0^2 - E_-^2)
        }
        & \frac{
            p^2 - m_1^2
        }
        {
            (p_0^2 - E_+^2)(p_0^2 - E_-^2)
        } & 0 \\
        0 & 0 & 
        \frac{1}{p_0^2 - E_0^2}
    \end{pmatrix}.
\end{align}



% Effective potential
\chapter{Thermodynamics}
\label{chapter:thermodynamics}

\section{Free energy at lowest order}
\label{section: free energy at lowest order}

The equation of state (EOS) relates the thermodynamic variables of as system.
In this section, we will obtain the equation of state of the pions by calculating their free energy.
We use the effective Lagrangian found in \autoref{serction:effective_pion_lagrangian} to find the leading-order contribution to one loop, and the next-to-leading order contribution at the tree-level, following the procedure used in~\cite{mojahed, Andersen:two-flavor-chpt}.\footnote{Leading order and next-to-leading order, in this context, refers to Weinberg's power counting scheme.}
The free energy density of a homogenous system is
\begin{equation}
    \Ef = - \frac{1}{V \beta} \ln Z.
\end{equation}
Here, $Z$ is the partition function, and $V$ the volume of space.
Using imaginary time formalism for thermal field theory, which is described in \autoref{section:thermal field theory}, we find that the partition function is given by the path integral of the \emph{Euclidean} Lagrange density, as shown in equation \cref{free scalar result 2}.
In the zero temperature limit $\beta \rightarrow \infty$, the partition function is related to vacuum transition amplitude $Z_0 = Z[J=0]$, as described in \autoref{section:path integral}, by a Wick rotation.
The free energy density at zero temperature is therefore 
\begin{equation}
    \Ef = \frac{i}{VT} \ln Z_0,
\end{equation}
where $VT$ is the volume of space-time.
This equals the effective potential in the ground state, which we found an explicit formula for in \autoref{section: effective action}, \cref{effective potential}.
We write $\Ef = \Ef^{(0)} + \Ef^{(1)} + \dots$, where $\Ef^{(n)}$ refers to the $n$-loop contributions to the free energy density.

\subsection*{Tree-level contribution}
The tree-level contribution $\Ef^{(0)}$ is the classical potential, which is given by the static ($\pi$-independent) part of the Lagrangian.
From \cref{L0} we have the leading order contribution,
\begin{equation}
    \Ef_2^{(0)}
    = - \Ell_2^{(0)} 
    = 
    -f^2   
    \left(
        \bar m^2 \cos{\alpha}
        + \frac{1}{2} \mu^2 \sin^2{\alpha}
    \right),
\end{equation}
where $\alpha$ parameterizes the ground state, which means that its value must minimize the free energy.
\begin{align*}
    &\diffp{}{\alpha} \Ef_2^{(0)} 
    = f^2\left(\bar m^2 - \mu_I^2\cos{\alpha}\right)\sin{\alpha}
    = 0.
\end{align*}
This equation defines the relationship between the chemical potential $\mu_I$, and the ground state parameter $\alpha$, as illustrated in \autoref{fig:free energy surface}.
This gives the criterion
\begin{align}
    \label{leading order minization}
    \alpha \in \{0, \pi\} \quad
    \mathrm{or} \quad
    \cos{\alpha} = \frac{\bar m^2}{\mu_I{}^2}.
\end{align}
As we see in the figure, $\alpha = \pi$ is a maximum, and thus unstable.
This means that for all values $\mu_I^2 \leq \bar m^2$, we will have $\alpha = 0$, and the system will remain in its ground state.

In our discussion of the effective potential we also found that the ground state should minimize the classical potential, as shown by \cref{minimize classical potential}.
This means that the linear part of the classical potential should vanish.
The linear part of the classical potential is given by \autoref{L1} to leading order, and reads $\Ve^{(1)} = f(\mu_I{}^2\cos{\alpha} - \bar m^2)\sin{\alpha} \, \pi_1 $, which vanishes given \cref{leading order minization}.
\begin{figure}[ht]
    \centering
    \includegraphics{figurer/numerics/free_energy_surface.pdf}
    \caption{The surface gives free energy as a function of $\mu_I$ and $\alpha$. $\alpha$ is the found by minimizing $\Ef$ for a given $\mu_I$. This leads to a curve across the free energy surface, as show in the plot.}
    \label{fig:free energy surface}
\end{figure}

\subsection*{One-loop contribution}
The one loop contribution to the free energy density is
\begin{equation}
    \label{one loop free energy}
    \Ef^{(1)}
    = - \frac{i}{V T} \frac{1}{2}
    \Tr{\ln\left( -\fdiff{S[\pi = 0]}{\pi_a(x), \pi_b(y)} \right)}.
\end{equation}
This can be evaluated using the rules for functional differentiation given in \autoref{section:Functional derivative}.
To leading order, 
\begin{align}
    \fdiff{S[\pi = 0]}{\pi_a(x), \pi_b(y)}
    = \fdiff{}{\pi_a(x), \pi_b(y)}
    \int \dd^4 x \, \Ell^{(2)}_2
    = D^{-1}_x \delta(x - y).
\end{align}
Here, $\Ell^{(2)}_2$ is the quadratic part of the Lagrangian, as given in \autoref{quadratic lagrangian}, and $D^{-1}_x$ is the corresponding inverse propagator of the pion fields,
\begin{equation}
    D_x^{-1} = 
    - \left[
        \delta_{ab}(\partial_x^\mu\partial_{x,\mu} + m^2_a)
        -  m_{12}(\delta_{a1} \delta_{b2} - \delta_{a2}\delta_{b1}) \partial_{x, 0}
    \right] 
\end{equation}
The inverse propagator is a matrix, which means that the determinant in \autoref{one loop free energy} i both a matrix determinant, over the three pion indices, as well as a functional determinant.
In \autoref{section:propagator} we found the matrix part of the determinant in momentum space, which we can write using the dispersion relations of the pion fields
\begin{equation}
    \det(- D^{-1}) = \det(-p_0^2 + E_0^2) \det(-p_0^2 + E_+^2) \det(-p_0^2 + E_-^2).
\end{equation}
These dispersion relations are functions of the three-momentum $\vec p$, and are given in \cref{dispresion relation pi 0,dispresion relation pi pm}.
The functional determinant can therefore be evaluated as
\begin{align}
    \nonumber
    \Tr{\ln\left( -\fdiff{S[\pi = 0]}{\pi_a(x), \pi_b(y)} \right)}
    & = \ln \det(-p_0^2 + E_0^2) + \ln \det(-p_0^2 + E_+^2) + \ln \det(-p_0^2 + E_-^2) \\
    \nonumber
    & = \Tr{ \ln(-p_0^2 + E_0^2) + \ln(-p_0^2 + E_+^2)+  \ln(-p_0^2 + E_-^2) } \\
    & = (VT) \int \frac{\dd^4 p}{(2 \pi)^4} 
    \left[ \ln(-p_0^2 + E_0^2) + \ln(-p_0^2 + E_+^2) + \ln(-p_0^2 + E_-^2)  \right],
\end{align}
where we have used the identity $\ln\det M = \Tr \ln M $.
These terms all have the form
\begin{equation}
    I = \int \frac{\dd^4 p}{(2 \pi)^2} \ln(-p_0^2 + E^2),
\end{equation}
where $E$ is some function of the 3-momentum $\vec p$, but not $p_0$.
We use the trick
\begin{equation}
    \pdv{\alpha} \left(-p_0^2 + E^2\right)^{-\alpha} \Big|_{\alpha=0}
    = \pdv{\alpha} \exp\left[ -\alpha \ln\left(- p_0^2 + E^2\right)  \right] \Big|_{\alpha=0}
    = \ln\left(- p_0^2 + E^2\right),
\end{equation}
and then perform a Wick-rotation of the $p_0$-integral to write the integral on the form 
\begin{equation}
    I = i \pdv{\alpha} \int \frac{\dd^4 p}{(2 \pi)^4} \left(p_0^2 + E^2\right)^{-\alpha} \Big|_{\alpha=0},
\end{equation}
where $p$ now is a Euclidean four-vector.
The $p_0$ integral equals $\Phi_1(E, 1, \alpha)$, as defined in \autoref{def dimreg integral}. 
The result is therefore given by \autoref{result dimreg},
\begin{equation}
    \int \frac{\dd p_0}{2 \pi} (p_0^2 + E)^{-\alpha} 
    = \frac{E^{1-2\alpha}}{\sqrt{4 \pi}} \frac{\Gamma(\alpha-\frac{1}{2})}{\Gamma(\alpha)}.
\end{equation}
The derivative of the Gamma function is $\Gamma'(\alpha) = \psi(\alpha)\Gamma(\alpha)$, where $\psi(\alpha)$ is the digamma function.
Using
\begin{align}
    \diffp{}{\alpha} & \frac{\Gamma(\alpha - \frac{1}{2}) }{\Gamma(\alpha)} \Big|_{\alpha=0}
    = \Gamma\left(\alpha - \frac{1}{2}\right) \frac{\psi(\alpha - \frac{1}{2}) - \psi(\alpha)}{\Gamma(\alpha)} \Big|_{\alpha=0}
    = \sqrt{4 \pi}, \\
    & \frac{\Gamma(\alpha - \frac{1}{2}) }{\Gamma(\alpha)}\Big|_{\alpha=0} = 0,
\end{align}
we get
\begin{equation}
    I = i \int \frac{\dd^3 p}{(2 \pi)^3} E.
\end{equation}
We see that the result is what we would expect physically, the total energy is the integral of the energy of each mode.
This also agrees with the result from \autoref{section:thermal field theory} in low temperature limit $\beta \rightarrow \infty$.
This results in 
\begin{equation}
    \Ef^{(1)} = 
    \frac{1}{2} 
    \left[\int \frac{\dd^3 p}{(2\pi)^3} E_0 + \int  \frac{\dd^3 p}{(2\pi)^3} (E_+ + E_-)\right]
    = \Ef^{(1)}_{\pi_0} +\Ef^{(1)}_{\pi_\pm}.
\end{equation}
The first integral is identical to what we find for a free field in \autoref{section:free scalar field}, in the zero temperature limit $\beta \rightarrow \infty$.
These terms are all divergent, and must be regularized. 
We will use dimensional regularization, in which the integral is generalized to $d$ dimensions, and the $\overline{\mathrm{MS}}$-scheme, as described in \autoref{section: regualting free energy}.
Using the result for a free field \cref{free field regularized energy}, we get
\begin{equation}
    \Ef^{(1)}_{\pi_0} 
    = 
    - \frac{1}{4} \frac{m_3^4}{(4\pi)^2} 
    \left[ \frac{1}{\epsilon} + \frac{3}{2} + \ln(\frac{\mu^2}{m_3^2}) \right] + \mathcal{O}(\epsilon),
\end{equation}
where $\mu$ is the renormalization scale, which is introduced ensure that the integral has the same engineering dimension for $d \neq 3$.

The contribution to the free energy from the $\pi_+$ and $\pi_-$ particles is more complicated, as the dispersion relation is given by
\begin{equation}
    E_\pm
    = 
    \sqrt{
        |\vv p|^2 +
        \frac{1}{2}
        \left(
            m_1^2 + m_2^2 + m_{12}^2 
        \right)
        \pm 
        \frac{1}{2}
        \sqrt{
            4|\vv p|^2m_{12}^2 
            +
            \left(
                m_1^2 + m_2^2 + m_{12}^2
            \right)^2
            - 4 m_1^2 m_2^2
        }
    }.
\end{equation}
This is not an integral we can easily do in dimensional regularization.
Instead, we will seek a function $f(|\vv p|)$ with the same UV-behavior, that is behavior for large $\vv p$, as $E_+ + E_-$.
If we then add $0 = f(|\vv p|) - f(|\vv p|)$ to the integrand, we can isolate the divergent behavior
\begin{equation}
    \Ef_{\pi_\pm}^{(1)}
    = 
    \frac{1}{2} \int \frac{\dd^3 p}{(2\pi)^3} [E_+ + E_- + f(|\vv p|) - f(|\vv p|)]
    = \Ef^{(1)}_{\mathrm{fin}, \pi_\pm } + \Ef^{(1)}_{\mathrm{div}, \pi_\pm}.
\end{equation}
This results in a finite integral, 
\begin{equation}
    \Ef^{(1)}_{\mathrm{fin}, \pi_\pm } = \frac{1}{2} \int \frac{\dd^3 p}{(2\pi)^3} [E_+ + E_- - f(|\vv p|)],
\end{equation}
which we can evaluate numerically, and isolate the divergence to
\begin{equation}
    \Ef^{(1)}_{\mathrm{div}, \pi_\pm }
    = 
    \frac{1}{2} \int \frac{\dd^3 p}{(2\pi)^3} f(|\vv p|),
\end{equation}
which we hopefully will be able to do in dimensional regularization.
We can explore the UV-behavior of $E_+ + E_-$ by expanding it in powers of $1 / \abs{\vv{p}}$,
\begin{align}
    \nonumber
    E_+ + E_-
    & = 
    2  \abs{\vv{p}}
    + \frac{m_{12} + 2(m_1^2 + m_2^2)}{4} \, {\abs{\vv{p}}}^{-1}
    - \frac{m_{12}^4 + 4 m_{12}^2(m_1^2 + m_2^2) + 8(m_1^4 + m_2^4)}{64}
    {\abs{\vv{p}}}^{-3}
    + \Oh[-5]{ \abs{\vv{p}}} 
    \\
    & = 
    a_1  \abs{\vv{p}}
    + a_2 \, {\abs{\vv{p}}}^{-1}
    + a_3
    {\abs{\vv{p}}}^{-3}
    + \Oh[-5]{ \abs{\vv{p}}}.
\end{align}
We have defined the new constants $a_i$ for brevity of notation.
As
\begin{equation}
    \int_{\R^3} \frac{\dd^3 p}{(2 \pi)^3} |\vv p|^{n}
    = C \int_{0}^\infty \dd p \, p^{2 + n}
\end{equation}
is UV divergent for $n \geq -3$, $f$ need to match the expansion of $E_+ + E_-$ up to and including $\Oh[-3]{|\vv{p}|}$ for $\Ef^{(1)}_{\mathrm{fin}, \pi_\pm }$ to be finite.
The most obvious choice for $f$ is
\begin{equation}
    f(|\vv p|) 
    = a_1  \abs{\vv{p}} + a_2 \, {\abs{\vv{p}}}^{-1} + a_3 \, {\abs{\vv{p}}}^{-3}.
\end{equation}
However, this introduces a new problem.
$f$ has the same UV-behavior as $E_+ + E_-$, but the last term diverges in the IR, that is for low $|\vv p|$.
This can be amended by introducing a mass term.
Let
\begin{equation}
    |\vv p|^{-3} 
    = 
    \left(\frac{1}{\sqrt{|\vv p|^2}}\right)^3 
    \longrightarrow 
    \left(\frac{1}{\sqrt{|\vv p|^2 + m^2}}\right)^3.
\end{equation}
For $|\vv p|^2 \rightarrow \infty$, this term is asymptotic to $|\vv p|^{-3}$, so it retains its UV behavior.
However, for $|\vv p| \rightarrow 0$, it now approaches $m^{-3}$, so the IR-divergence is gone.
The cost of this technique is that we have introduced an arbitrary mass parameter.
Any final result must thus be independent of the value of $m$ to be acceptable.

We will instead regularize the integral by defining $E_i = \sqrt{|\vv{p}|^2 + \tilde m_i^2}$, and $\tilde m_i^2 = m_i^2 + \frac{1}{4} m_{12}^2$.
Using the definition of the mass parameters, \cref{m1,m2,m3,m12}, we get
\begin{align}
    m_3^2 & = \bar m^2 \cos \alpha + \mu_ I^2 \sin^2 \alpha, \\
    \label{tilde m1}
    \tilde m_1^2 
    & 
    = m_1^2 + \mu^2 \cos\alpha^2
    = \bar m^2 \cos \alpha + \mu_I^2 \sin^2 \alpha
    = m_3^2 \\
    \label{tilde m2}
    \tilde m_2^2 
    & = m_2^2 + \mu^2 \cos\alpha^2
    = \bar m^2 \cos \alpha.
\end{align}
Finally, we define $f(|\vv p|) = E_1 + E_2$ which differ from $E_+ + E_-$ by $\Oh[-5]{|\vv p|}$, and is well-behaved in the IR.
This leads to a divergent integral the same form as in the case of a free scalar.
Thus, in the $\mathrm{\overline{MS}}$-scheme, 
\begin{equation}
    \Ef^{(1)}_{\mathrm{div}, \pi_\pm }
    =
    - \frac{1}{4} \frac{\tilde m_1^4}{(4\pi)^2} 
    \left[
        \frac{1}{\epsilon} + \frac{3}{2} + \ln(\frac{\mu^2}{\tilde m_1^2}) 
    \right] 
    - \frac{1}{4} \frac{\tilde m_2^4}{(4\pi)^2} 
    \left[
        \frac{1}{\epsilon} + \frac{3}{2} + \ln(\frac{\mu^2}{\tilde m_2^2})
    \right] 
    + \mathcal{O}(\epsilon).
\end{equation}
We define
\begin{equation}
    \Ef^{(1)}_{\mathrm{fin}, \pi_\pm}
    = 
    \frac{1}{2} \int \frac{\dd^3 p}{(2\pi)^3} (E_+ + E_- - E_1 - E_2),
\end{equation}
which is a finite integral.
The total one-loop contribution is then, using \cref{tilde m1,tilde m2},
\begin{align}
    \Ef^{(2)}
    & = 
    \Ef^{(1)}_{\mathrm{fin}, \pi_\pm} 
    - \frac{1}{2}\frac{1}{(4\pi)^2}
    \left[
        \left( \frac{1}{\epsilon} + \frac{3}{2} + \ln \frac{\mu^2}{m_3^2 } \right)
        m_3^4
        +
        \frac{1}{2}
        \left( \frac{1}{\epsilon} + \frac{3}{2} + \ln \frac{\mu^2}{\tilde m_2^2} \right)
        \tilde m_2^4
    \right]
\end{align}

\section{Next-to-leading order and renormalization}
We have now regularized the divergences, so they can be handled in a well-defined way.
However, they are still there.
To get rid of them, we need to use renormalization.
The power counting scheme used when constructing the Effective Lagrangians ensures that all terms in $\Ell_{2n}$ is of order $p^{2n}$ in the pion momenta.\footnote{Remember that the bare pion mass $\bar m = B_0(m_u + m_d)$ is considered to be of order $p^2$.}
The tree level free energy from $\Ell_{2n}$ is thus of order $p^{2n}$.
The n-loop correction to the tree level result is then suppressed by $p^{2n}$~\cite{Gasser-Leutwyler:chiral,WeinbergPhenom}.
Our one-loop calculation using $\Ell_2$ therefore contains divergences of order $p^{4}$. 
Since $\Ell_4$ is, by construction, the most general possible Lagrangian at order $p^4$, it contains coupling constants which can be renormalized to absorb all these divergences.

The renormalized coupling constants in $\Ell_2$ have been calculated for $\mu_I = 0$~\cite{Gasser-Leutwyler:chiral}.
They are independent of $\mu_I$, and we can therefore use them in this calculation.
The renormalized coupling constants in the $\overline{\mathrm{MS}}$-scheme are related to the bare couplings through
\begin{align}
    l_i 
    & = 
    l_i^r \, \,
    - \frac{1}{2} \mu^{-2\epsilon} \frac{\gamma_i }{(4 \pi)^2} 
    \left(\frac{1}{\epsilon} + 1 \right),
    \quad \quad
    i \in \{1, ... 7\} 
    \\
    h_i 
    & = 
    h_i^r
    -\frac{1}{2} \mu^{-2\epsilon}  \frac{\delta_i }{(4 \pi)^2} 
    \left(\frac{1}{\epsilon} + 1 \right), 
    \quad \quad
    i \in \{1, ... 3\}.
\end{align}
Here, $\gamma_i$ and $\delta_i$ are numerical constants that are used to match the divergences.
The relevant terms are\footnote{Some authors~\cite{Andersen:two-flavor-chpt,GERBER1989387} instead use $h_1' = h_1 - l_4$, with a corresponding $\delta_1' = \delta_1 - \gamma_1 = 0$.}
\begin{gather}
    \gamma_1 = \frac{1}{3}, \quad
    \gamma_2 = \frac{2}{3}, \quad
    \gamma_3 = - \frac{1}{2}, \quad
    \gamma_4 = 2, \\
    \delta_1 = 2, \quad
    \delta_3 = 0.
\end{gather}
The bare coupling constants, though infinite, are independent of our renormalization scale $\mu$.
From this we obtain the renormalization group equations for the running coupling constants,
\begin{equation}
    \mu \diff{l_i^r}{\mu } = - \frac{\gamma_i}{(4 \pi)^2} + \Oh{\epsilon}, \quad
    \mu \diff{h_i^r}{\mu } = - \frac{\delta_i}{(4 \pi)^2} + \Oh{\epsilon}.
\end{equation}
These have the general solutions
\begin{equation}
    l_i^r 
    = \frac{1}{2} \frac{\gamma_i}{(2 \pi)^2} 
    \left( \bar l_i - \ln{\frac{\mu^2}{M^2}} \right),
    \quad
    h_i^r 
    = \frac{1}{2} \frac{\gamma_i}{(2 \pi)^2} 
    \left( \bar h_i - \ln{\frac{\mu^2}{M^2}} \right),
\end{equation}
where $\bar l_i$ and $\bar h_i$ are the values of the coupling constants (times a constant) measured at the energy $M$.
This only applies if the numerical constants $\gamma_i$/$\delta_i$ are non-zero.
If they are zero, then the renormalized constant is not running, and instead equal to its measured value at all energies.
The bare couplings are thus given by
\begin{equation}
    l_i = \frac{1}{2} \frac{\gamma_i}{(4 \pi)^2}
    \left[\bar l_i - \mu^{-2 \epsilon}\left(1+ \frac{1}{\epsilon}\right) - \ln\frac{\mu^2}{M^2}\right], \quad
    h_i = \frac{1}{2} \frac{\delta_i}{(4 \pi)^2}
    \left[\bar h_i - \mu^{-2 \epsilon}\left(1+ \frac{1}{\epsilon}\right)- \ln\frac{\mu^2}{M^2}\right]
\end{equation}
(HVOR BLE DET AV $\ln \mu$?)
The free energy at tree-level, at next-to-leading order is, according to \cref{NLO-L0}, 
\begin{align*}
    \Ef^{(0)}_4
    & 
    = - \Ell_4^{(0)} 
    \\
    & = 
    - (l_1 + l_2)\mu_I^4 \sin^4{\alpha}
    - (l_3 + l_4)\bar m^4 \cos^2{\alpha}
    - l_4 \bar m^2 \mu_I{}^2 \cos{\alpha} \sin^2{\alpha}
    -(h_1- l_4) \bar m^2
    - h_3 \Delta m^2
    \\
    & = 
    - \frac{1}{2} \frac{1}{(4 \pi)^2}
    \bigg[
        \frac{1}{3}
        \left( 
            \bar l_1 + 2 \bar l_2 - 3
        \right) \mu_I^4 \sin^4 \alpha
        +
        \frac{1}{2}
        \left(
            - \bar l_3 + 4 \bar l_4 - 3
        \right) \bar m^4 \cos^2\alpha
        \\
        & \quad \quad \quad \quad \quad
        + 2 \left(\bar l_4 - 1\right)
        \bar m^2 \mu_I^2 \cos\alpha \sin^2 \alpha
        + 2 (\bar l_4 - \bar h_1) \bar m^2
        + \bar h_3 \Delta m^2
        \\
        & \quad \quad \quad \quad \quad
        - 
        \left(\frac{1}{\epsilon} + \ln \frac{\mu^2}{M^2}\right) 
        \left(
            \mu_I^4\sin^4\alpha + \frac{3}{2} \bar m^4 \cos^2 \alpha
            + 2 \bar m^2 \mu_I^2 \cos\alpha \sin^2 \alpha
        \right) 
    \bigg] 
\end{align*}
Adding all the contribution to the free energy density, we get the next-to-leading order free energy,
\begin{align}
    \nonumber
    \Ef_{\mathrm{NLO}} &=
    - f^2 \left(\bar m^2 \cos \alpha + \frac{1}{2}\mu_I^2 \sin^2 \alpha\right)
    + \Ef^{(1)}_{\mathrm{fin}, \pi_\pm}
    - \frac{1}{2}\frac{1}{(4 \pi)^2}
    \bigg[
        \frac{1}{3}
        \left( 
            \bar l_1 + 2 \bar l_2 + \frac{3}{2} + 3 \ln \frac{M^2}{m_3^2}
        \right) \mu_I^4 \sin^4 \alpha
        \\ %\nonumber
        &
        +
        \frac{1}{2}
        \left(
            - \bar l_3 + 4 \bar l_4 + \frac{3}{2} + 2\ln \frac{M^2}{m_3^2}
            + \ln \frac{M^2}{\tilde m^2_2}
        \right) \bar m^4 \cos^2\alpha 
        + 2 \left(\bar l_4 - \frac{1}{2} + \ln \frac{M^2}{m_3^2}\right)
        \bar m^2 \mu_I^2 \cos\alpha \sin^2 \alpha
        \label{NLO free energy}
        % \\
        % & 
        % + 2 (\bar l_4 - \bar h_1) \bar m^2
        % + \bar h_3 \Delta m^2
    \bigg].
\end{align}
We have dropped the terms proportional to $\bar l_4 - \bar h_1$ and $\bar h_3$, as they only add an unobservable constant value to the free energy.

\subsection*{Parameters, and keeping the expansion consistent}
The coupling constants are free parameters, and can therefore not be calculated from first principles, but must be measured.
The values for the pion mass and pion decay constants are (HVORFOR?)
\begin{equation}
    m_\pi = 131 \, \mathrm{MeV}, \quad f_\pi = \frac{1}{\sqrt 2} 128 \, \mathrm{MeV}.
\end{equation}
This is the physical mass, $m_\pi$, is defined as the pole of the propagator and thus the zero of the inverse propagator,
\begin{equation}
    D^{-1}(p^2 = m_\pi^2) = 0.
\end{equation}
This relates it to the which relates it to the bare mass $\bar m$.
We found, in \cref{m1}, that $m_\pi = m_3^2(\mu_I = 0) = \bar m^2$ to leading order.
Similarly, $f_\pi = f$ to leading order. (HVORFOR)
However, in any NLO results we need $\bar m^2$ and $f$ to NLO.
This is given by~\cite{Andersen:two-flavor-chpt} (UTLEDE?)
\begin{align}
    \label{equation bare mass}
    m_\pi^2 & = \bar m^2 + \frac{\bar l_3}{2 (4\pi)^2} \frac{\bar m^4}{f^2}, \\
    \label{equation bare decay constant}
    f_\pi^2 & = f^2 + \frac{2\bar l_4}{(4\pi)^2} \frac{\bar m^2}{f^2}
\end{align}
All results in this text are given in units of $m_\pi$.

\begin{table}[h]
    \centering
    \caption{The measured values and corresponding uncertainties of the relevant coupling constants.}
    \begin{tabular}{c c c c}
        \hline \hline
        & value & uncertainty & source \\
        \hline \\[-1em]
        $\bar l_1$ & -0.4   & \pm 0.6   & \cite{pipi_scattering}    \\
        $\bar l_2$ & 4.3    & \pm 0.1   & \cite{pipi_scattering}    \\
        $\bar l_3$ & 2.9    & \pm 2.4   & \cite{Gasser-Leutwyler:chiral} \\    
        $\bar l_4$ & 4.4    & \pm 0.2   & \cite{pipi_scattering}    \\
        \hline
    \end{tabular}
    \label{table:coupling constants}
\end{table}
The values for the coupling constants used in this text are given in \autoref{table:coupling constants}.
The constants $\bar l_1$, $\bar l_2$ and $\bar l_3$ are estimated using data from $\pi \pi$-scattering~\cite{pipi_scattering}, while $\bar l_3$ is estimated using three flavor chiral perturbation theory~\cite{Gasser-Leutwyler:chiral}.
These are the same values as those used in~\cite{Andersen:two-flavor-chpt}.
Together with \cref{equation bare mass,equation bare decay constant}, the NLO results for the bare mass and decay constant are
\begin{align}
    \label{NLO m}
    \bar m / m_\pi = 1.01136, \\
    \label{NLO f}
    f / m_\pi = 0.64835.
\end{align}

In \autoref{section: free energy at lowest order}, we found a relationship between $\alpha$ and $\mu_i$, using the lowest order estimate of $\Ef$, given in \cref{leading order minization}.
To calculate any thermodynamic quantities to leading order, we must thus use this result.
If we are to calculate any other quantities to next-to-leading order, we also have to use the next-to-leading order criterion for $\alpha$ as a function of $\mu_I$, which is given by
\begin{eqnarray}
    \diffp{\Ef_{\mathrm{NLO}}}{\alpha} = 0,
\end{eqnarray}
using the result \cref{NLO free energy}.
In \autoref{fig:alpha}, the NLO result is compared with the LO result, \cref{leading order minization}.
\begin{figure}
    \centering
    \includegraphics[width=0.7\textwidth]{figurer/numerics/alpha.pdf}
    \caption{The leading order and next-to-leading order results for $\alpha$ as a function of $\mu_I$.}
    \label{fig:alpha}
\end{figure}

\section{Equation of state}
The methods and analysis of this section is based on~\cite{Peskin:IntroQFT,Andersen:two-flavor-chpt,andersen2012introduction}.

The free energy\footnote{As we are in the grand canonical ensemble, this is the grand canonical, or Landau, free energy, and not Helmholtz' free energy.}
is defined as
\begin{equation}
    \label{thermodynamic free energy}
    F = U - TS - \mu_I Q_I, \quad \dd F = - S \dd T - P \dd V - Q_I \dd \mu_I ,
\end{equation}
where $Q_I$ is the isospin charge, and $U$ is the energy.
As we have seen earlier, our system is homogenous.
This means that the free energy density is independent of volume, and thus $F = V \Ef$.
From  \cref{thermodynamic free energy} we see that the pressure is given by
\begin{equation}
    P = - \left(\diffp{F}{V}\right)_{T, \mu_I} = - \Ef.
\end{equation}
We are interested in the pressure relative to the state in which $\mu_I$ = 0. We therefore normalize $P(\mu_I=0) = 0$, which gives  
\begin{equation}
    P(\mu_I) = -[\Ef(\mu_I) - \Ef(\mu_I = 0)]
\end{equation}
This is illustrated in \autoref{fig:pressure}.
\begin{figure}[h]
    \centering
    \vspace{-0.2cm}
    \includegraphics[width=0.7\textwidth]{figurer/numerics/pressure.pdf}
    \caption{The NLO and LO result for the pressure, as a function of $\mu_I$.}
    \label{fig:pressure}
\end{figure}

Likewise, the total isospin is proportional to volume, which means that the isospin density is
\begin{equation}
    n_I = \frac{Q_I}{V} = - \frac{1}{V} \left(\diffp{F}{\mu_I}\right)_{T, V}
    = - \diffp{\Ef}{\mu_I}.
\end{equation}
Using \cref{NLO free energy}, this equals
\begin{align}
    \nonumber
    n_I & = 
    f^2 \mu_I \sin^2 \alpha
    - \diffp{\Ef_\mathrm{fin}}{\mu_I} \\
    & + \frac{1}{(4 \pi)^2}
    \left[
            \left(2\bar l_4+\ln\frac{M^2}{m_3^2}\right)\bar m^2 \mu_I \cos\alpha \sin^2 \alpha
            +\frac{1}{3}\left(2\bar l_1 + 4 \bar l_2 + 3\ln\frac{M^2}{m_3^2}\right)\mu_I^3 \sin^4 \alpha
    \right]
\end{align}
The isospin density, as a function of $\mu_I$, is shown in \autoref{fig:isospin_density}.
\begin{figure}[h]
    \centering
    \vspace{-0.2cm}
    \includegraphics[width=0.7\textwidth]{figurer/numerics/isospin_density.pdf}
    \caption{The NLO and LO result for the isospin density, as a function of $\mu_I$.}
    \label{fig:isospin_density}
\end{figure}

From \cref{thermodynamic free energy} we get the energy density, $u = U/V$, at $T = 0$ is given by
\begin{equation}
    u(\mu_I) = -P(\mu_I) + \mu_I n_I(\mu_I),
\end{equation}
where we again have normalized so that $u(\mu_I = 0) = 0$.
Now that we have both the dependence of pressure and energy density on the isospin chemical potential, we can trace out the line in the pressure-energy density plane, parametrized by $\mu_I$.
This is the equation of state of the system, and is shown in \autoref{fig:equation of state}.


\begin{figure}[h]
    \centering
    \vspace{-0.2cm}
    \includegraphics[width=0.7\textwidth]{figurer/numerics/energy_density.pdf}
    \caption{The leading and next-to-leading order equation of state. Both the pressure and energy density are given in units of $m_\pi^4$.}
    \label{fig:equation of state}
\end{figure}

\FloatBarrier


\input{eos/phase_transition.tex}

\chapter{Conclusion and outlook}
\label{chpater:conclusion and outlook}
The thermodynamic behavior of chiral perturbation theory at non-zero isospin chemical potential serves as a fruitful avenue for exploration of QCD at low temperatures, as it can be compared to calculations from first principles using lattice QCD.
The recent proposals that pions can form compact stellar objects called pion stars~\cite{new_clas_of_compact_stars,andersen:bose_einstein} increase the importance of better understanding of this regime.
It is speculated that lepton asymetires in the early universe could result in pion condensation~\cite{new_clas_of_compact_stars,abduki:Pion_condensation_in_a_dense_neutrino_gas,Wygas:Cosmic_QCD_Epoch_at_Nonvanishing_Lepton_Asymmetry,Schwarz_2009:Lepton_asymmetry_and_the_cosmic_QCD_transition}.
In this case, pion stars might have left observable traces in the form of neutrino and photon spectra from their evaporation or in the form of gravitational waves and can have thus played a role in forming the universe we see today~\cite{new_clas_of_compact_stars}.

In this specialization project, we have discussed the theoretical foundations for chiral perturbation theory and derived the building blocks of the effective Lagrangian governing pions.
Using the most general Lagrangian up to next-to-leading order in Weinberg's power counting scheme, we calculated the grand canonical free energy density in the case of a non-zero isospin chemical potential to next to leading order.
From this, we calculated the equation of state, or EOS.
We find the EOS to remain trivial for isospin chemical potential $\mu_I$ less than some critical value $\mu_I^c$ and showed that this value equals the pion mass $m_\pi$ to next-to-leading order, as expected.
Furthermore, at $\mu_I = \mu_I^c$, we showed that the system undergoes a phase transition, breaking the isospin symmetry.
In this new phase, it becomes energetically favorable for the system to move to an excited state, leading to a pion condensate.
The pion condensate is characterized by a non-zero isospin density $n_I$, caused by the ground state being rotated away from the vacuum.
As we expect from Goldstone's theorem, we observe a massless mode in the spectrum.


\subsection*{Outlook}

The equation of state of a material is used in conjunction with the Tolman–Oppenheimer–Volkoff (TOV) equations to model the internal dynamics of stars.
This enables calculation of the relationship between the mass and radius of stars~\cite{Carroll:spacetime}.
Stellar objects, such as stars, display electric charge neutrality on macroscopic scales.
In the case of pion stars, one has to include leptons in the model to ensure electric charge neutrality.
Isospin density is related to the up- and down-quark density $n_u$ and $n_d$ by
\begin{equation}
    n_I 
    = \frac{1}{TV} Q_I 
    = \frac{1}{TV} \left(\ex{\bar u \gamma^0 u} - \ex{\bar d \gamma^0 d} \right) 
    = n_u - n_d,
\end{equation}
At zero baryon chemical potential, $\mu_B = 0$, the isospin chemical potential is related to the up- and down-quark chemical potential by $\mu_I/2 = \mu_u = -\mu_d$.
We thus get the relationship $n_u = -n_d$~\cite{new_clas_of_compact_stars}.
As the up quark has electric charge $\frac{2}{3}e$, and the down quark $-\frac{1}{3}e$, the pion condensate alone is electrically charged.
A realistic stellar object thus must include a lepton-density $n_l$ to remain neutrally charged.
This gives the criterion for charge density,
\begin{equation}
    n_Q = \frac{2}{3}n_u - \frac{1}{3} n_d - n_l = n_I - n_l = 0.
\end{equation}
Together with the equation of state and the TOV-equation, this equation allows for the investigation of charge-neutral pion stars.

Comparisons between the free energy density and other thermodynamic quantities obtained from chiral perturbation theory and lattice QCD at $T = 0$ are in good agreement~\cite{Andersen:two-flavor-chpt,mojahed}.
Using thermal field thoery, \chpt can be extended to finite temperature.
Recent 
studies of \chpt at non-zero temperature find that the theory remains in good agreement with lattice QCD as well as other models for temperatures below $20 \, \text{MeV}$~\cite{andersen_mojahed:condensates_and_pressure}.
A good understanding of the thermal properties of pion condensates is critical to account for the non-zero temperature of real stellar objects.

Our results can be improved by using three-flavor chiral perturbation theory, taking into account the strange quark.
The strange quark $s$ has a mass $m_s \approx 93 \, \text{MeV}$~\cite{PDG}, which is considerably larger than the up and down quark, but still within the strong-interaction regime, which suggests that it can play an important role in the equation of state.
Chiral perturbation theory can be extended to include the strange quark by considering the larger group $\SU(3)_L \times \SU(3)_R$, consisting of rotations of all three quarks into each other.


\appendix

% Thermal field theory
\chapter{Thermal Field Theory}
\label{appendix:thermal field theory}
\subsection{Statistichal Mechanics}
In classical mechanics, a thermal system at temperature $T = 1 / \beta$ is described as an ensemble state, which have a probability $P_n$ of being in state $n$, with energy $E_n$.
In the canonical ensemble, the probability is proportional to $e^{-\beta E_n}$.
The expectation value of some quantity $A$, with value $A_n$ in state $n$ is
\begin{equation*}
    \ex{A} 
    = \sum_n A_n P_n = \frac{1}{Z} \sum_n A_n e^{- \beta E_n}, \quad 
    Z  = \sum_n e^{-\beta A_n}.
\end{equation*}
$Z$ is the partition function. In quantum mechanics, an ensemble configuration is described by a non-pure density operator,
\begin{equation*}
    \hat \rho = C \sum_n P_n \ketbra{n}{n},
\end{equation*}
where $\ket{n}$ is some basis for the relevant Hilbert space and $C$ is a constant. Assuming $\ket{n}$ are energy eigenvectors, i.e. $\hat H \ket{n} = E_n \ket{n}$, the density operator for the canonical ensemble is
\begin{equation*}
    \hat \rho 
    = C \sum_n e^{-\beta E_n} \ketbra{n}{n} 
    = C e^{-\beta \hat H} \sum_n \ketbra{n}{n} 
    = C e^{-\beta \hat H}.
\end{equation*}
The expectation value in the ensemble state of a quantity corresponding to the operator $\hat A$ is given by
\begin{align}
    \ex{A} = \frac{ \Tr{\hat \rho \hat A} }{\Tr{\hat \rho }}
    = \frac{1}{Z} \Tr{\hat A e^{-\beta \hat H}}
\end{align}
The partition function $Z$ ensures that the probabilities adds up to 1, and is defined as
\begin{equation}
    Z = \Tr{e^{-\beta \hat H}}.
\end{equation}

The grand canonical ensemble takes into account the conserved charges of the system.
Conserved charges are a result of Nöther's theorem.
Assume we have a set of fields $\varphi_\alpha$. Nöther's theorem tells us that if the Lagrangian $\Ell[\varphi_\alpha]$ has a \emph{continuous symmetry}, then there is a corresponding conserved current~\cite{Peskin:IntroQFT,Carroll:spacetime}.
To define a continuous symmetry of the Lagrangian, we need a one-parameter family of transformations,
\begin{equation*}
    \varphi_\alpha(x) \longrightarrow \varphi_\alpha'(x; \epsilon)
    \sim \varphi_\alpha(x) + \varepsilon \eta_\alpha(x), \, \varepsilon \rightarrow 0.
\end{equation*}
Here, $\eta_\alpha(x)$ is some arbitrary function which define the transformation as $\varepsilon \rightarrow 0$.
Applying this transformation to the Lagrangian will in general change its form,
\begin{equation*}
    \Ell[\varphi_\alpha] \rightarrow \Ell[\varphi_\alpha']
    \sim \Ell[\varphi_\alpha] + \varepsilon \delta \Ell, \,
    \varepsilon \rightarrow 0.
\end{equation*}
If the change in the Lagrangian can be written as a total derivative, i.e.
\begin{equation*}
    \delta \Ell = \partial_\mu K^\mu(x),
\end{equation*}
we say that the Lagrangian has a continuous symmetry.
This is because a term of this form will result in a boundary term in the action integral, which does not contribute to the variation of the action.
Nöther's theorem states more precisely that the current
\begin{equation}
    j^\mu = \pdv{\Ell}{(\partial_\mu \varphi_\alpha)} \eta_\alpha - K^\mu
\end{equation}
obeys the conservation law
\begin{equation}
    \partial_\mu j^\mu = 0.
\end{equation}
The flux of current through some space-like surface $V$, i.e. a surface with a time-like normal vector, defines a conserved charge. This surface is most commonly a surface of constant time in some reference frame. 
The charge is defined as
\begin{equation}
    Q = \int_V \dd^3 x\, n_\mu j^\mu,
\end{equation}
where $n^\mu$ is the normal vector of $V$.
Using the divergence theorem again, and assuming the current falls of quickly enough, we can show that the total charge is conserved,
\begin{equation*}
    \pdv{t} Q = \int_V \dd^3 x \, \nabla \cdot \vec j = \int_{\partial V} \dd^2 x\, n'_\mu j^\mu = 0.
\end{equation*}
For a surface of constant time we may choose $n^\mu = (1, 0, 0, 0)$, and the conserved charged is
\begin{equation*}
    Q = \int_V \dd x \, j^0.
\end{equation*}
In the grand canonical ensemble, a system with $n$ conserved charges $Q_i$ has probability proportional to $e^{-\beta (H - \mu_i Q_i)}$.
$\mu_i$ are the chemical potentials corresponding to conserved charge $Q_i$.
This leads to the partition function
\begin{equation}
    Z = \Tr{e^{-\beta(\hat H - \mu_i \hat Q_i)}}.
\end{equation}

\subsection{Imaginary-time formalism}
\label{imaginary-time formalism}
The partition function may be calculated in a similar way to the path integral approach, in what is called the imaginary-time formalism. 
This formalism is restricted to time independent problems, and is used to study fields in a volume $V$.
This volume is taken to infinity in the thermodynamic limit.
As an example, take a scalar quantum field theory with the Hamiltonian
\begin{equation}
    \hat H
    = \int_V \dd^3 x \, \hat \He\left[\hat \varphi(\vec x), \hat \pi(\vec x)\right],
\end{equation}
where $\hat \varphi(\vec x)$ is the field operator, and $\hat \pi(\vec x)$ is the corresponding canonical momentum operator.
These field operators have time independent eigenvectors, $\ket{\varphi}$ and $\ket \pi$, defined by
\begin{equation}
    \hat \varphi(\vec x) \ket{\varphi} = \varphi(\vec x) \ket{\varphi}, \quad
    \hat \pi(\vec x) \ket{\pi} = \pi(\vec x) \ket{\pi}.
\end{equation}
In analogy with regular quantum mechanics, they obey the relations~\cite{Kapusta:finiteTemp}
\footnote{Some authors write $\D\pi/2 \pi$. This extra factor $2\pi$ is a convention which in this text is left out for notational clarity.}
\begin{gather}
    \label{functional completness}
    \one
    = \int \D \varphi(\vec x) \ketbra{\varphi}{\varphi} 
    = \int \D\pi(\vec x) \ketbra{\pi}{\pi}, \\
     \braket{\varphi}{\pi} 
    = \exp(i \int_V \dd x \, \varphi(\vec x) \pi(\vec x)), \\
    \braket{\pi_a}{\pi_b}
    =  \delta(\phi_a - \phi_b), \quad
    \braket{\varphi_a}{\varphi_b} 
    = \delta(\varphi_a - \varphi_b).
\end{gather}
The functional integral is defined by starting with $M$ degrees of freedom, $\{\varphi_m\}_{m=1}^M$ located at a finite grid $\{\vec x_m\}_{m=1}^M \subset V$.
The integral is then the limit of the integral over all degrees of freedom, as $M \rightarrow \infty$:
\begin{equation*}
    \int \D \varphi(\vec x) = \lim_{M \rightarrow \infty} \int \left(\prod_{m=1}^M \dd \varphi_m\right).
\end{equation*}
The functional Dirac-delta $\delta(f) = \prod_x\delta(f(x))$ is generalization of the familiar Dirac delta function.
Given a functional $\mathcal{F}[f]$, it is defined by the relation
\begin{equation}
    \int \D f(x)\, \mathcal{F}[f] \delta(f - g) = \mathcal{F}[g].
\end{equation}
The Hamiltonian is the limit of a sum of Hamiltonians $\hat H_m$ for each point $\vec x_m$
\begin{equation*}
    \hat H
    = \lim_{M \rightarrow \infty} \sum_{m=1}^M 
    \frac{V}{M} \hat H_m(\{\hat \varphi_m\}, \{\hat \pi_m\}).
\end{equation*}
$H_m$ may depend on the local degrees of freedom $\hat \varphi_m, \, \hat \pi_m$ as well as those at neighboring points.
By inserting the completeness relations \autoref{functional completness} $N$ times into the definition of the partition function, it may be written as
\begin{align*}
    Z& 
    = \int \D\varphi(\vec x) \, \inner{\varphi}{e^{- \beta \hat H}}{\varphi}
    = 
    \prod_{n=1}^N
    \left(
        \int \D \varphi_n (\vec x) \int \D \pi_n(\vec x)
    \right) 
    \prod_{n=1}^N  \braket{\varphi_n}{\pi_n}
    \inner{\pi_n}{e^{- \epsilon \hat H}}{\varphi_{n+1}} \braket{\varphi_{1}}{\varphi_{n+1}},
\end{align*}
where $\epsilon = \beta / N$. The last term ensures that $\varphi_1 = \varphi_{N+1}$.
Strictly speaking, we only need to require $\varphi_1 = e^{i\theta}\varphi_{N+1}$, as the partition function is only defined up to a constant.
As will be shown later, bosons such as the scalar field $\varphi$, follow the periodic boundary condition $\varphi(0, \vec x) = \varphi(\beta, \vec x)$, i.e. $e^{i\theta} = 1$, while fermions follow the anti-periodic boundary condition $\psi(0, \vec x) = -\psi(\beta, \vec x)$, i.e. $e^{i\theta} = -1$.
We now want to exploit the fact that $\ket{\pi}$ and $\ket{\varphi}$ are the eigenvectors of the operators that define the Hamiltonian.
In our case, as the Hamiltonian density $\He$ can be written as a sum of functions of $\varphi$ and $\pi$ separately, $\He[\varphi(\vec x), \pi(\vec x)] = \mathcal{F}_1[\varphi(\vec x)] + \mathcal{F}_2[\pi(\vec x)]$ we may evaluate it as $\inner{\pi_n}{\He[\hat \varphi(x), \hat \pi(x)]}{\varphi_{n+1}} = \He[\varphi_{n+1}(x), \pi_n(x)] \braket{\pi_n}{\varphi_{n+1}}$.
This relationship does not, however, hold for more general functions of the field operators.
In that case, one has to be more careful about the ordering of the operators, for example by using \emph{Weyl ordering}~\cite{Peskin:IntroQFT}.
By series expanding $e^{-\epsilon \hat H}$ and exploiting this relationship, the partition function can be written as, to second order in $\epsilon$,
\begin{align*}
    Z = 
    \prod_{n=1}^N  
    \left(
        \int \D \varphi_n (\vec x) \int \D \pi_n(\vec x)
    \right)
    \exp[-\epsilon \sum_{n=1}^N \int_V \dd^3x \,
    \left(
        \He(\varphi_n(\vec x), \pi_n(\vec x)) - i \pi_n(\vec x) \frac{\varphi_n(\vec x) - \varphi_{n+1}(\vec x)}{\epsilon}
    \right)
    ].
\end{align*}
We denote $\varphi_n(\vec x) = \varphi(\tau_n, \vec x) $, $\tau \in [0, \beta]$ and likewise with $\pi_n(\vec x)$. 
In the limit $N \rightarrow \infty$, the expression for the partition function becomes
\begin{align}
    \label{Thermal partition function}
    Z = \int_S \D \varphi(\tau, \vec x)
    \int \D \pi(\tau, \vec x)
    \exp{
        - \int_0^\beta \dd \tau \int_V \dd \vec x \, 
        \left\{
            \He[\varphi(\tau, \vec x), \pi(\tau, \vec x)]
            - i \pi(\tau, \vec x) \dot \varphi(\tau, \vec x)
        \right\}
        },
\end{align}
where $S$ is the set of field configurations $\varphi$ such that  $\varphi(\beta, \vec x) = \varphi(0, \vec x)$.
With a Hamiltonian density of the form $\He = \frac{1}{2} \pi^2 + \frac{1}{2} (\nabla \varphi)^2 + \Ve(\varphi)$, we can evaluate the integral over the canonical momentum $\pi$ by discretizing $\pi(\tau_n, \vec x_m) = \pi_{n,m}$,
\begin{align*}
    & \int \D \pi \exp{-  \int_0^\beta\dd \tau \int_V \dd^3x 
    \left(
        \frac{1}{2} \pi^2 - i \pi \dot \varphi 
    \right)} \\
    & = \lim_{M,N \rightarrow \infty} \int \left(\prod_{m, n = 1}^{M, N} \frac{\dd \pi_{m, n}}{2 \pi}\right)
    \exp{
        - \sum_{m, n} \frac{V\beta}{MN}
        \left[
            \frac{1}{2}  (\pi_{m, n} - i \dot \varphi_{m, n})^2
            + \frac{1}{2} \dot \varphi_{m, n}^2
        \right]
    } \\
    & = \lim_{M,N \rightarrow \infty} \left( \frac{M N }{2 \pi V \beta} \right)^{MN/2}
    \exp{- \int_0^\beta\dd \tau \int_V \dd^3x \, \frac{1}{2}\dot \varphi^2},
\end{align*}
where $\dot \varphi_{m, n} = (\varphi_{m, n+1} - \varphi_{m, n})/\epsilon$.
The partition function is then, 
\begin{equation}
    Z = C \int \D \varphi
    \exp{
        - \int_0^\beta \dd \tau \int_V \dd^3 x
        \left[
            \frac{1}{2} \left(\dot \varphi^2 + \nabla \varphi^2\right) 
            + \Ve(\varphi)
        \right]
    }.
\end{equation}
Here, $C$ is the divergent constant that results from the $\pi$-integral.
In the last line, we exploited the fact that the variable of integration $\pi_{n,m}$ may be shifted by a constant without changing the integral, and used the Gaussian integral
\begin{equation*}
    \int_{-\infty}^{\infty} \dd x \, e^{-a x^2/2} = \sqrt{\frac{2 \pi}{a}}.
\end{equation*}


The partition function resulting from this procedure may also be obtained by starting with the ground state path integral
\begin{equation*}
    Z_g
    =\int \D\varphi\D\pi
    \exp{i \int_{\Omega'} \dd^4 x \, \left(\pi\dot\varphi - \He[\varphi, \pi]\right)}
    = C' \int \D \varphi(x)
    \exp{i \int_{\Omega'} \dd^4 x \, \Ell[\varphi, \partial_\mu \varphi]},
\end{equation*}
and follow a formal procedure.
First, the action integral is modified by performing a Wick-rotation of the time coordinate $t$.
This involves changing the domain of $t$ from the real line to the imaginary line by closing the contour at infinity, and making the change of variable $it \rightarrow \tau$.
The new variable is then restricted to the interval $\tau\in [0, \beta]$, and the domain of the functional integral $\int \D \varphi$ is restricted from \emph{all} (smooth enough) field configurations $\varphi(t, \vec x)$, to only those that obey $\varphi(\beta, \vec x) = e^{i\theta} \varphi(0, \vec x) $, which is denoted $S$.
This procedure motivates the introduction of  the Euclidean Lagrange density, 
$\Ell_E(\tau, \vec x) = -\Ell(-i \tau, \vec x)$, as well as the name ``imaginary-time formalism''.
The result is the same partition function as before,
\begin{align}
    Z & = C \int_S \D \varphi \int \D \pi
    \exp{
        - \int_0^\beta \dd \tau \int_V \dd^3x \, 
        \left[
            - i\dot \varphi \pi
            + \He(\varphi, \pi)
        \right]
    } \nonumber \\ \label{free scalar result 2}
    & =
    C' \int_S \D \varphi
    \exp{- \int_0^\beta \dd \tau \int_V \dd^3x \, \Ell_E(\varphi, \pi)}.
\end{align}

\subsection{Free scalar field}

This section uses notation as described in \autoref{Conventions and notation}.
The procedure for obtaining the thermal properties of an interacting scalar field is similar to that used in scattering theory.
One starts with a free theory, which can be solved exactly.
Then an interaction term is added, which is accounted for peturbatively by using Feynman diagrams.
The Euclidean Lagrangian for a free scalar gas is, after integrating by parts,
\begin{equation*}
    \Ell_E = \frac{1}{2} \varphi(X) \left( -\partial_E^2 + m^2 \right) \varphi(X)
\end{equation*}
Here, $X = (\tau, \vec x)$ is the Euclidean coordinate which is a result of the Wick-rotation.
We have introduced the Euclidean Laplace operator, $\partial_E^2 = \partial_\tau^2 + \nabla^2$.
Following the procedure as described in \autoref{imaginary-time formalism} to obtain the thermal partition function yields
\begin{equation*}
    Z = C \int_S \D \varphi(X) 
    \exp{
        - \int_\Omega \dd X \frac{1}{2} 
        \varphi(X) \left( -\partial_E^2 + m^2 \right) \varphi(X)
    }.
\end{equation*}
Here, $\Omega$ is the domain $[0, \beta] \times V$.
We then insert the Fourier expansion of $\varphi$, and change the functional integration variable to the Fourier components.
The integration measures are related by
\begin{equation*}
    \D \varphi(X) = \det(\frac{\delta \varphi(X)}{\delta \tilde \varphi(K)}) \D\tilde \varphi(K),
\end{equation*}
where $K = (\omega_n, \vec k)$ is the Euclidean Fourier-space coordinate.
The determinant factor which appears may be absorbed into the constant $C$, as the integration variables are related by a linear transform.
The action becomes 
\begin{align*}
    S & = - \int_\Omega \dd X\, \Ell_e 
    = - \frac{1}{2} V \beta \int_\Omega \dd X \int_{\tilde \Omega} \dd K \int_{\tilde \Omega} \dd K' \,
    \tilde \varphi(K') 
    \left(
        \omega_n^2 + \vec k^2 + m^2
    \right)
    \tilde \varphi(K)
    e^{iX\cdot(K - K')} \\
    & = - \frac{1}{2} V \beta^2 \, \int_{\tilde \Omega} \dd K \,
    \tilde \varphi(K)^*
    \left(
        \omega_n^2 + \omega_k^2
    \right)
    \tilde \varphi(K),
\end{align*}
where $\omega_k^2 = \vec k^2 + m^2$.
$\tilde \Omega$ is the reciprocal space corresponding to $\Omega$, as described in \autoref{Conventions and notation}.
We used the fact that $\varphi$ is real, which implies that $\tilde \varphi(-K) = \tilde \varphi(K)^*$, as well as the identity \autoref{thermal delta}.
This gives the partition function 
\begin{equation*}
    Z = C \int_{\tilde S} \D \tilde \varphi(K) 
    \exp{
        -  \frac{1}{2} V \int_{\tilde \Omega} \dd K \, 
        \tilde \varphi(K)^* \left[\beta^2 (\omega_n^2+ \omega_k^2)\right] \tilde \varphi(K)
    },
\end{equation*}
Going back to before the continuum limit, this integral can be written as a product of Gaussian integrals, and may therefore be evaluated
\begin{align*}
    Z = C \prod_{n=-\infty}^\infty \prod_{k \in \tilde V}
    \left(
        \int \dd \tilde \varphi_{n, \vec k} \,
        \exp{
            - \frac{1}{2} \tilde \varphi_{n, \vec k}^*
            \left[\beta^2 (\omega_n^2+ \omega_k^2)\right] 
            \tilde \varphi_{n, \vec k}
            }
    \right)
    = 
    C \prod_{n=-\infty}^\infty \prod_{k \in \tilde V} 
    \sqrt{\frac{2 \pi}{\beta^2 (\omega_n^2 + \omega_k^2)}}.
\end{align*}
The partition function is related to Helmholtz free energy $F$ through
\begin{equation*}
    \frac{F}{T V}= - \frac{\ln(Z)}{V} = \frac{1}{2} \int_{\tilde \Omega} \dd K \frac{1}{2} \ln[\beta^2(\omega_n^2 + \omega_k^2)] + \frac{F_0}{TV},
\end{equation*}
where $F_0$ is a constant.

A faster and more formal way to get to this result is to compare the partition function to the multidimensional version of the Gaussian integral~\cite{Kapusta:finiteTemp, Peskin:IntroQFT}.
The partition function is on the form 
\begin{equation*}
    I_n = \int_{\R^n} \dd^n x\, \exp{- \frac{1}{2} \langle x, D x\rangle },
\end{equation*}
where $D$ is a linear operator, and $\langle \cdot , \cdot \rangle$ an inner product on the corresponding vector space.
By diagonalizing $D$, we get the result
\begin{equation*}
    I_n = \sqrt{\frac{(2 \pi)^n}{\det(D)}}.
\end{equation*}
We may then use the identity
\begin{equation}
    \det(D) = \prod_i \lambda_i = \exp{\Tr[\ln(D)]},
\end{equation}
where $\lambda_i$ are the eigenvalues of $D$.
The trace in this context is defined by the vector space $D$ acts on.
For given an orthonormal basis $x_n$, such that $\langle x_n, x_n'\rangle = \delta_{nn'}$ the trace can be evaluated as $\Tr{D} = \sum_n \langle x_n, D x_n \rangle$.
Identifying 
\begin{equation*}
    \langle x, D x\rangle = \int_\Omega \dd X \varphi(X)\left(-\partial_E^2+m^2\right)\varphi(X),
\end{equation*}
we get the formal result
\begin{equation*}
    Z = \det(-\partial_E^2 + m^2)^{-1/2},
\end{equation*}
and 
\begin{equation*}
    \beta F = \frac{1}{2}\Tr{\ln(-\partial_E^2 + m^2)}.
\end{equation*}
The logarithm may then be evaluated by using the eigenvalues of the linear operator.
This is found by diagonalizing the operator,
\begin{equation*}
    \langle x, D x \rangle 
    = \int_\Omega \dd X \varphi(X)\left(-\partial_E^2+m^2\right)\varphi(X)
    = V  \int_{\tilde \Omega} \dd K 
    \tilde \varphi(K)^* [\beta^2(\omega_k^2 +\omega_n^2)] \tilde \varphi(K),
\end{equation*}
leaving us with the same result
\begin{equation*}
    \beta F 
    = \frac{1}{2} \Tr{\ln(-\partial_E^2 + m^2)} 
    = \frac{1}{2} V \int_{\tilde \Omega} \dd K \ln[\beta^2(\omega_n^2 + \omega_k^2)].
\end{equation*}
How to work this sum is shown in \autoref{thermal sum}
This gives the free energy density, after dropping the additive constant,
\begin{equation}
    \label{free scalar free enrgy}
    \beta \Ef = \frac{F}{VT}
    = \frac{1}{2} \int_{\tilde V} \frac{\dd^3 k}{(2 \pi)^3}
    \left[
        \beta \omega_k + 2\ln(1 - e^{-\beta\omega_k})
    \right].
\end{equation}

\subsection{Thermal sum}
\label{section:thermal sum}

When evaluating thermal integral, we will often encounter sums of the form
\begin{equation}
    \label{j func}
    j(\omega, \mu) = \frac{1}{2\beta} \sum_{\omega_n} 
    \ln\{\beta^2 [(\omega_n + i \mu) + \omega^2] \} + g(\beta),
\end{equation}
where the sum is over either the bosonic Matsubara frequencies $\omega_n = 2n \pi / \beta,\, n \in \mathbb{Z}$, or the fermionic ones, $\omega_n = (2n + 1) \pi /\beta ,\, n \in \mathbb{Z}$.
$\mu \in \R$ is the chemical potential.
$g$ may be a function of $\beta$, but we assume it is independent of $\omega$.
Thus, the factor $\beta^2$ could strictly be dropped, but it is kept to make the argument within the logarithm dimensionless.
We define the function
\begin{equation}
    \label{i func}
    i(\omega, \mu) = \frac{1}{\omega} \dv{\omega} j(\omega, \mu) 
    = \frac{1}{\beta} \sum_{\omega_n} \frac{1}{(\omega_n + i\mu)^2 + \omega^2}. 
\end{equation}

\begin{figure}
    \centering
    \begin{subfigure}{0.4\textwidth}
        \centering
        \includegraphics{thermal_field_theory/plots/integral_cont.pdf}        
    \end{subfigure}
    \begin{subfigure}{0.18\textwidth}
        \centering
        \begin{tikzpicture}
            \draw[-stealth] (0, 0) -- (1, 0);
        \end{tikzpicture}
        \end{subfigure}
    \begin{subfigure}{0.4\textwidth}
        \centering
        \includegraphics{thermal_field_theory/plots/integral_cont2.pdf}
    \end{subfigure} 
    \caption{The integral contour $\gamma$, and the result of deforming it into to contours close to the real line.
    The red crosses illustrate the poles of $n_B$.}
    \label{fig:integral contours}
\end{figure}

We will first work with the sum over bosonic Matsubara frequencies.
Assume $f(z)$ is an analytic function, except perhaps on a set of isolated poles $\{z_i\}$ located outside the real line. 
By exploiting the properties of the Bose-distribution $n_B(z)$, as described in \autoref{Conventions and notation}, we can rewrite the sum over Matsubara frequencies as a contour integral
\begin{equation*}
    \frac{1}{\beta} \sum_{\omega_n} f(\omega_n) 
    = \oint_\gamma \frac{\dd z}{2 \pi i} f(z) i n_B(i z),
\end{equation*}
where $\gamma$ is a contour that goes from $- \infty - i \epsilon$ to $+ \infty - i \epsilon$, crosses the real line at $\infty$, goes from $+ \infty - i \epsilon$ to $- \infty + i \epsilon$ before closing the curve.
The contour $\gamma$, and the change of integral contours is illustrated in \autoref{fig:integral contours}
This result exploits Cauchy's integral formula, by letting the poles of $in_B(iz)$ at the Matsubara frequencies ``pick out'' the necessary residues.
The integral over $\gamma$ is equivalent to two integrals along $\R \pm i \epsilon$,
\begin{align}
    \nonumber
    \frac{1}{\beta} \sum_{\omega_n} f(\omega_n) 
    &= \left(
        \int_{\infty + i \epsilon}^{-\infty + i \epsilon} \frac{\dd z}{2 \pi} 
        + \int_{-\infty - i \epsilon}^{\infty - i \epsilon}\frac{\dd z}{2 \pi}
    \right) 
    f(z) n_B(i z),
    \\
    \nonumber
    & = \int_{-\infty - i \epsilon}^{\infty - i \epsilon}\frac{\dd z}{2 \pi}
    \left\{
        f(-z) + \left[f(z) + f(-z)\right] n_B(iz)
    \right\} 
    \\
    \label{bosonic sum to integral}
    & = \int_{-\infty}^{\infty} \frac{\dd z}{2 \pi} f(z)
    +
    \int_{-\infty - i \epsilon}^{\infty - i \epsilon}\frac{\dd z}{2 \pi}
    \left[
        f(z) - f(-z)
    \right]
    n_B(iz).
\end{align}

In the second line, we have changed variables $z \rightarrow -z$ in the first integral, and exploited the property $n_B(-i z) = -1 - n_B(iz)$.
In the last line, we use the assumption that $f(z)$ is analytic on the real line, and therefore also in a neigbourhood of it. 
This allows us to shift the first integral back to the real line.
As $n_B(iz)$ is analytic outside the real line, the result of the second integral is the sum of residues of $f(z) + f(-z)$ in the lower half plane.
The function
\begin{equation}
    f(z) = \frac{1}{(z + i \mu)^2 + \omega^2} 
    = \frac{i}{2 \omega } 
    \left(
        \frac{1}{z + i(\mu + \omega)} - \frac{1}{z + i(\mu - \omega)}
    \right)
\end{equation}
obeys the assumed properties, as it has poles at
$z = - i (\mu \pm \omega)$, with residue $1 / 2 \omega$, so the function defined in \autoref{i func} may be written \footnote{Assuming $\omega>\mu$.}
\begin{equation}
    i(\omega, \mu) 
    % = \left(\frac{-2 \pi i^2}{2 \pi \omega}\right)
    % [1 + n_B(\omega + \mu) + n_B(\omega - \mu)].
    = \frac{1}{2\omega}
    [1 + n_B(\omega - \mu) + n_B(\omega + \mu)].
\end{equation}

Using the anti-derivative of the Bose distribution, we get the final form of \autoref{j func}
\begin{equation}
    j(\omega, \mu) = \int \dd \omega'\, \omega' i(\omega', \mu)
    =  
    \frac{1}{2}\omega + \frac{1}{2\beta} 
    \left[
        \ln\left(1 - e^{-\beta(\omega - \mu)}\right)
        + \ln\left(1 - e^{-\beta(\omega + \mu)}\right)
    \right]
    + g'(\beta).
\end{equation}
The extra $\omega$-independent term $g'(\beta)$ is an integration constant.
We see there are temperature dependent terms, one due to the particle and one due to the anti-particle, and one due to the antiparticle, as they have opposite chemical potentials.

We now consider the sum over fermionic frequencies, which we for clarity denote $\tilde \omega_n$ in this chapter.
The procedure in this case is the same, except that we have to use a function to with poles at the fermionic Matsubara frequencies.
This is done by the Fermi distribution, $n_F(z)$, as described in \autoref{Conventions and notation}.
The result is 
\begin{equation}
    \frac{1}{\beta} \sum_{\tilde \omega_n} f(\tilde \omega_n) 
    = 
    -\int_{-\infty}^{\infty} \frac{\dd z}{2 \pi} f(z)
    +
    \int_{-\infty - i \epsilon}^{\infty - i \epsilon}\frac{\dd z}{2 \pi}
    \left[
        f(z) - f(-z)
    \right]
    n_F(iz),
\end{equation}
and 
\begin{equation}
    i(\omega, \mu) = \frac{1}{2 \omega} [-1 + n_F(\omega - \mu) + n_F(\omega + \mu)].
\end{equation}
Using the antiderivative of the Fermi-distribution, we get
\begin{equation}
    j(\omega, \mu) 
    = - \frac{1}{2} \omega 
    - \frac{1}{2 \beta }
    \left[
        \ln\left(1 + e^{-\beta(\omega - \mu)}\right)
        + \ln\left(1 + e^{-\beta(\omega + \mu)}\right)
    \right].
\end{equation}

% If we denote the two different sums
% \begin{equation}
%     s_b(\beta) = \frac{1}{\beta} \sum_{\omega_n} f(\omega_n), \quad 
%     s_f(\beta) = \frac{1}{\beta} \sum_{\tilde\omega_n} f(\tilde \omega_n),
% \end{equation}
% then they are related by
% \begin{equation}
%     s_f(\beta) 
%     = \frac{1}{\beta} \sum_n f([2 \pi n + 1]/\beta)
%     = \frac{1}{\beta} \sum_n \left\{ f(\pi n /\beta) - f(2 \pi n /\beta) \right\}
%     = 2 s_b(2\beta) - s_b(\beta).
% \end{equation}

\subsection{Regulating the free energy}
The free energy integral we obtained from the energy for the free scalar \autoref{free scalar free enrgy}, has two parts,
Noticing that the integral is spherically symmetric, we may write
\begin{equation}
    J_0 = \frac{1}{2} \int_{\tilde \Omega} \frac{\dd^3 k}{(2 \pi)^3} \sqrt{k^2 + m^2}, \quad
    J_T = \frac{1}{\beta} \int_{\tilde \Omega} \frac{\dd^3 k}{(2 \pi)^3} 
    \ln\left[1 - \exp(- \beta \sqrt{k^2 + m^2})\right], 
\end{equation}
The temperature-independent part, $J_0$, is clearly divergent, and we must therefore impose a regulator, and then add counter-terms.
This differs from the zero-temperature formalism, where there were no need to renormalize the free theory.
The second part of the integral, is convergent. 
To see this, we first exploit that it is spherically symmetric to write
\begin{equation}
    J_T = \frac{T^4}{2 \pi^2}\int_\R \dd x \, x^2  \ln(1 - e^{-\sqrt{x^2 + (\beta m)^2}}).
\end{equation}
Using the series expansion $\ln(1 + \epsilon) \sim \epsilon + \Oh{\epsilon}$, we can find the leading part of the  integrand for large $k$'s, 
\begin{equation}
    x^2 \ln(1 - e^{-\sqrt{x^2 + (\beta m)^2}}) \sim - x^2 e^{-x}, 
\end{equation}
which is suppressed exponentially, making the integral convergent.
In the limit of $T \rightarrow \infty$, we get
\begin{equation}
    J_\infty \sim \frac{T^4}{2 \pi^2} \int_\R \dd x \, x^2 \ln(1 - e^{-x})
    = - T^4 \frac{\pi^2}{90} = -\frac{3}{2} \sigma T^4,
\end{equation}
where $\sigma$ is the Stefan-Boltzmann constant. 

We use dimensional regularization on the temperature independent term. 
To that end, we define
\begin{equation}
    \Phi(m, d, A) = \int_{\tilde \Omega} \frac{\dd^d k}{(2 \pi)^d} (k^2 + m^2)^{-A},
\end{equation}
so that $J_0 = \Phi(m, 3, 1/2) / 2$.
Dimensional regularization takes uses the formula for integration of spherically symmetric function in $d$-dimensions,
\begin{equation}
    \int_{\R^d} \dd^d x \, f(r) 
    = \frac{2 \pi^{d/2}}{\Gamma(d/2)} \int_\R \dd r \, r^{d-1}f(r),
\end{equation}
where $r = \sqrt{x_i x_i}$ is the radial distance, and $\Gamma$ is the gamma-function.
The factor in the front of the integral comes from the solid-angle.
By extending this formula from integer-valued $d$ to real numbers, the function we defined becomes
\begin{equation}
    \Phi 
    = \frac{2 \pi^{d/2}}{\Gamma(d/2)} \int_\R \dd k \, 
    \frac{k^{d-1}}{(k^2 + m^2)^A}
    = \frac{m^{d - 2A}}{(4 \pi)^{d / 2}\Gamma(d/2)} 
    2 \int_\R \dd z \, \frac{z^{d - 1}}{(1 + z)^A}, 
\end{equation}
where we have made the change of variables $m z = k$.
We make one more change of variable to the integral,
\begin{equation}
    I = 2 \int_\R \dd z \, \frac{z^{d - 1}}{(1 + z)^A}
\end{equation}
Let
\begin{equation}
    z^2 = \frac{1}{s} - 1 \implies 2 z \dd z = - \frac{\dd s}{s^2}
\end{equation}
Thus,
\begin{equation}
    I = \int_0^a \dd s \, s^{A - d/2 - 1} (1 - z)^{d/2 - 1}.
\end{equation}
This is the beta-function, which can be written in terms of gamma-funcitons~\cite{Peskin:IntroQFT}
\begin{equation}
    I = B\left(A - \frac{d}{2}, \frac{d}{2}\right) 
    = \frac{\Gamma\left(A - \frac{d}{2}\right) \Gamma\left(\frac{d}{2}\right)}{\Gamma(A)}.
\end{equation}
Putting it all together, this gives
\begin{equation}
    \Phi = 
    \frac{
        (m^2)^{d/2 - A} \Gamma \left(A - \frac{d}{2} \right) 
    }
    {
        (4 \pi)^{d / 2}\Gamma(A)
    }.
\end{equation}
Inserting $d = 3 - 2\epsilon$ and $A = -1/2$, we get
\begin{equation}
    \frac{
        (m^2)^{3/2 - \epsilon - A} \Gamma \left(-2 + \epsilon \right) 
    }
    {
        (4 \pi)^{3/2 - \epsilon}\Gamma(-1/2)
    }
    = 
    \left(\frac{m^2}{4 \pi}\right)^{3/2} 
    \frac{m}{- 2 \pi^{1/2}}
    \mu^{-2\epsilon}
    \left(\frac{m^2}{4 \pi \mu^2}\right)^{- \epsilon}
    \frac{\Gamma(\epsilon)}{(\epsilon - 2)(\epsilon - 1)},
\end{equation}
where we have used the defining property $\Gamma(z + 1) = z\Gamma(z)$, and inserted a parameter $\mu$ with the dimensions of $m$.
Expanding around $\epsilon = 0$ gives
\begin{align}
    \left(\frac{m^2}{4 \pi \mu^2}\right)^{- \epsilon}
    &\sim 1 + \epsilon \ln\left(4 \pi \frac{\mu^2}{m^2}\right),\\
    \Gamma(\epsilon) 
    & \sim \frac{1}{\epsilon} - \gamma, \\
    \frac{1}{(\epsilon - 2)(\epsilon - 1)}
    &\sim \frac{1}{2}\left(1 + \frac{3}{2} \epsilon\right).
\end{align}
Inserting this back into the function gives
\begin{align}
    \Phi = 
    - \frac{m^4}{2^5 \pi^2} \mu^{-2 \epsilon}
    \left[1 + \epsilon \ln\left(4 \pi \frac{\mu^2}{m^2}\right)\right]
    \left(\frac{1}{\epsilon} - \gamma\right) \left(1 + \frac{3}{2} \epsilon\right)
    \sim
    - \frac{m^4}{2^5 \pi^2} \mu^{-2 \epsilon}
    \left[
        \frac{1}{\epsilon} 
        - \gamma + \frac{3}{2}
        + \ln\left(4 \pi \frac{\mu^2}{m^2}\right)
    \right].
\end{align}

\subsection*{Interacting scalar}

We now study a scalar field with a $\lambda \varphi^4$ interaction term.
We write the Lagrangian in the form
\begin{equation*}
    \Ell = \Ell^{(0)} + \Ell^{(I)}, \quad 
    \Ell^{(0)} = 
    \frac{1}{2} \partial_\mu \varphi \partial^\mu \varphi  + m^2 \varphi^2 , \quad
    \Ell^{(I)} = - \frac{\lambda}{4!} \varphi^4
\end{equation*}
$\Ell^{(I)}$ is called the interaction term, and makes it impossible to exactly solve for the partition function.
Instead, we turn to perturbation theory.
The grand canonical partition function in this theory
\begin{equation}
    Z = \Tr{e^{- \beta \hat H}}
    = \int_S \D \varphi \, \exp{
        - \int_\Omega \dd X \left(\Ell_E^{(0)} + \Ell_E^{(I)}\right)
    }
    = \int_S \D \varphi \, e^{S_0} e^{S_I}.
\end{equation}
Here, $S_0$ and $S_I$ denote the Euclidean action due to the free and interacting Lagrangian, respectively.
The domain of integration $S$ is again periodic field configurations $\varphi(\beta, \vec x) = \varphi(0, \vec x)$.
We may write the free energy as
\begin{equation*}
    - \beta F = \ln
    \left[
        \int_S \D \varphi \, e^{S_0} \sum_n \frac{1}{n!} {S_I}^n
    \right]
    = \ln[Z_0] 
    + \ln
    \left[
        Z_I
        % \sum_n \frac{1}{n!}  
        % \frac{
        %     \int_S \D \varphi \, e^{S_0} {S_I}^n}
        % {\int_S \D \varphi \, e^{S_0}}
    \right],
\end{equation*}
where $Z_0$ is the partition function from the free theory.
The correction to the partition function is thus given by
\begin{equation}
    Z_1 = \sum_{n=0}^\infty \frac{1}{n!} \ex{{S_I}^n}_0,
\end{equation}
where
\begin{equation}
    \ex{A}_0 = \frac{
        \int_S \D \varphi \, A \, e^{S_0} }
    {\int_S \D \varphi \, e^{S_0}}.
\end{equation}
% Notice that the constant factor from the Jacobian due to the change of variable $\varphi \rightarrow \tilde \varphi$ does not affect the expectation value, as the same factor is in both the numerator and denominator.
% If the quantity $A$ is a function of the momentum-space fields, $A = A[\tilde \varphi(K)]$, then this expectation value takes the form
% \begin{equation}
%     \ex{A}_0 = 
%     \frac{
%         \int_{\tilde S} \D \tilde \varphi(K) \, f[\tilde\varphi(K)] \, 
%         \exp{- \frac{1}{2} \langle \tilde \varphi^*, D \tilde \varphi \rangle}
%         }
%     {
%         \int_{\tilde S} \D \tilde \varphi(K) \,
%         \exp{
%             - \frac{1}{2} \langle \tilde \varphi^*, D \tilde \varphi \rangle
%             }
%     }.
% \end{equation}
% where, as before, 
% \begin{equation}
%     \langle \tilde \varphi^*, D \tilde \varphi \rangle
%     = 
%     \int_\Omega 
%             \tilde \dd K\, [\beta^2(\omega_n^2 + \omega_n^2)] |\tilde \varphi(K)|^2
% \end{equation}

To evaluate expectation values of the form $\ex{\varphi(X_1) ... }_0$, we introduce the the partition function with a source term
\begin{align}
    Z[J] = \int_S \D \varphi \, \exp{
        - \frac{1}{2} \int_\Omega \dd X \, \varphi (-\partial_E^2 + m^2) \varphi
        + \int_\Omega \dd X \, J \varphi
    }.
\end{align}
Using the thermal Greens function $D(X, Y)$, as defined in \autoref{Conventions and notation}, we may complete the square to write
\begin{align}
    Z[J] = Z[0]\exp{\frac{1}{2} \int_{\Omega} \dd X \dd Y J(X) D_0(X, Y) J(Y)}
    = Z[0] \exp(W[J])
\end{align}
We can now write
\begin{equation}
    \ex{\varphi(X)\varphi(Y)}_0 
    = \frac{1}{Z[0]}
    \frac{\delta}{\delta J(X)} \frac{\delta}{\delta J(Y)} 
    Z[J] \Big|_{J=0} 
    = D(X, Y),
\end{equation}
where $D(X, Y)$ is the thermal propagator, as defined in \autoref{Conventions and notation}
This generalizes to higher order expectation values,
\begin{equation}
    \ex{\varphi(X_i) \dots \varphi(X_n)}_0
    = \left(\prod_{i=1}^n \frac{\delta}{\delta J(X_i)}\right) 
    Z[J] \Big|_{J=0},
\end{equation}
The exponential form of $Z[J]$ leads straight forwardly to Wick's theorem, which states that an expectation value of $2n$ fields is a sum of \emph{all possible, distinct} combination of $n$ propagators.
To write this in a formal way, we define the functions $a$ and $b$, which define a way to pair up $2m$ elements.
The domain of the functions are the integers between $1$ and $m$, the image a subset of the integers between $1$ and $2m$ of size $m$.
A valid pairing is a set $\{(a(1), b(1)), \dots (a(m), b(m))\}$, where all elements $a(i)$ and $b(j)$ are different, such all integers up to and including $2m$ are featured.
A pair is not directed, so $(a(i), b(i))$ is the same pair as $(b(i), a(i))$.
Wick theorem states that,
\begin{equation}
    \ex{\prod_{i=1}^{2m} \varphi(X_i)  }_0
    = \sum_{\{(a, b)\}} \ex{\varphi(X_{a(i)}) \varphi(X_{b(i)})}.
\end{equation}
where the  sum is over all tuples $(a, b)$ that define a valid and unique pairing.
Using Wick's theorem, the expectation values we are evaluating can be written
\begin{align*}
    \ex{{S_I}^m} & 
    = \left(- \frac{\lambda }{4!}\right)^m 
    \int_{\Omega} \dd X_1 \dots \dd X_m
    \ex{\varphi(X_1)^4 \dots \varphi(X_m)^4} \\ 
    & \quad
    = \left(- \frac{\lambda }{4!}\right)^m 
    \int_{\Omega} \dd X_1 \dots \dd X_m \sum_{\{a, b\}}
    \ex{\varphi(X_{a(1)}) \varphi(X_{b(1)})} 
    \dots
    \ex{\varphi(X_{a(2m)}) \varphi(X_{b(2m)})}
\end{align*}
where $X_i$ for $i>m$ is defined as $X_j$, where $j = i \mod m$.
Inserting the fourier expandions of the field gives
\begin{align*}
    & \ex{{S_I}^m} \\ 
    &\quad 
    = \left(-\frac{\lambda }{4!}\right)^m 
    \int_{\Omega} \dd X_1 \dots \dd X_m
    (V \beta)^2 \int_{\tilde \Omega} \dd K_1 ... \dd K_{2m} \sum_{\{a, b\}} \\
    & \quad \quad \quad \quad\quad \quad \quad
    \ex{\varphi(K_{a(1)}) \varphi(K_{b(1)})} 
    \dots
    \ex{\varphi(K_{a(2m)}) \varphi(K_{b(2m)})}     
    \exp(i {\sum}_{i=1}^{m} X_i \cdot K_i)\\ 
    & \quad  
    = \left(-\frac{\lambda }{4!}\right)^m 
    \frac{(V \beta)^{2m} \beta^m}{(V \beta^2)^{2m}}
    \int_{\tilde \Omega} \dd K_1 ... \dd K_{2m} \sum_{\{a, b\}} \\
    & \quad \quad \quad \quad \quad \quad \quad \quad \quad
    \tilde D(K_{a(1)}) \delta(K_{a(1)} + K_{b(1)}) \dots 
    \tilde D(K_{a(2m)}) \delta(K_{a(2m)} + K_{b(2m)})
    \prod_{i=1}^m \delta\left(X_i \cdot {\sum}_{j=0}^3 K_{i + jm}\right) \\
    & \quad 
    = \left(-\frac{\lambda \beta}{4!}\right)^m 
    \prod_{i=1}^{2m} \int_{\tilde \Omega} 
    \left( \dd K_i \frac{1}{\beta} \tilde D(K_i)  \right) 
    \prod_{i=1}^m \delta\left(X_i \cdot {\sum}_{j=0}^3 K_{i + jm}\right)
    \sum_{\{a, b\}} 
    \prod_{n=1}^{2m}\delta(K_{a(k)} + K_{b(k)})
\end{align*}
Here we have used that $V \beta^2 \tilde D_0(K, P) = \tilde D_0(K) \delta(P + K)$, and $\tilde D(K)$ is the thermal propagator for the free field, as defined in \autoref{Conventions and notation}.
In this case, the thermal propagator of the free field is
\begin{equation}
    D_0(K) = D_0(\omega_n, \vec k) = \frac{1}{\omega_k^2 + \omega_n^2}.
\end{equation}
This expectation value can be represented graphically using Feynman diagrams.
The thermal $\lambda \varphi^2$-theory gets the prescription

\begin{align}
    \feynmandiagram [inline=(a.base), small, horizontal=i1 to f2]
    {
    {i1} -- [fermion, edge label'=$K_1$] a[dot] 
    -- [anti fermion, edge label'=$K_3$] {f1},
    {i2} -- [fermion, edge label'=$K_2$] a -- [anti fermion, edge label'=$K_4$] {f2},
    };
    & = -\lambda \beta
    \delta \left({\sum}_i K_i \right), \\ \nonumber \\
    \feynmandiagram[horizontal= i to f]{
        i[particle=$K$] -- [fermion] f,
    };
    & = \frac{1}{\beta} \int_{\tilde \Omega} \dd K \,  D_0(K).
\end{align}
The factor $1/4!$ is removed as a general Feynman diagram represent all diagrams with the same form, but different pairing of the momenta.
Some diagrams are more symmetric, such that an exchange of momenta still gives \emph{the same pairing}. 
This is dealt with by dividing with a symmetry factor $s$, which is described in detail in~\cite{Peskin:IntroQFT}.

Calculating $\ex{{S_I}^n}_0$ boils down to the sum of all possible Feynman diagrams with $m$ vertices.
The first example is 
\begin{align}
    \ex{S_I} = 
    \feynmandiagram[small, horizontal=a to b, inline=(b.base)]
    {
        b[dot] --[fermion, half left, looseness=1.5, edge label'=$K_1$] a 
        --[fermion, half left, looseness=1.5, edge label'=$K_2$] b,
        b --[fermion, half right, looseness=1.5, edge label'=$K_3$] c 
        --[fermion, half right, looseness=1.5, edge label'=$K_4$] b,
    }; 
    +
    \feynmandiagram[small, horizontal=a to b, inline=(b.base)]
    {
        b[dot] --[fermion, half left, looseness=1.5, edge label'=$K_1$] a 
        --[fermion, half left, looseness=1.5, edge label'=$K_4$] b,
        b --[fermion, half right, looseness=1.5, edge label'=$K_2$] c 
        --[fermion, half right, looseness=1.5, edge label'=$K_4$] b,
    };
    +
    \feynmandiagram[small, horizontal=a to b, inline=(b.base)]
    {
        b[dot] --[fermion, half left, looseness=1.5, edge label'=$K_1$] a 
        --[fermion, half left, looseness=1.5, edge label'=$K_4$] b,
        b --[fermion, half right, looseness=1.5, edge label'=$K_2$] c 
        --[fermion, half right, looseness=1.5, edge label'=$K_3$] b,
    };
    = 
    \frac{3}{4!} \times 
    \feynmandiagram[small, horizontal=a to b, inline=(b.base)]
    {
        b[dot] --[fermion, half left, looseness=1.5] a 
        --[fermion, half left, looseness=1.5] b,
        b --[fermion, half right] c 
        --[fermion, half right] b,
    };.
\end{align}

For higher order, one gets both connected and disconnected diagrams.
Let $\ex{{S_I}^n}_{0, c}$ be only the connected diagrams, i.e. those diagrams in which it is possible to move between all vertices along a series of edges.

A general diagram contain $n_i$ copies of a connected diagram with the value $V_i$.
The value of the total diagram is then the product of the value of all its disconnected pieces, but with the caveat that each diagram has a symmetry factor of $n_i!$.
The sum of all diagrams is thus 
\begin{equation}
    Z_I = \sum_n \frac{1}{n!} \ex{{S_I}^n} 
    = \sum_{ all\,sets\,\{n_i\}} \prod_i \frac{1}{n_i!}V_i^{n_i}
    = \prod_i \sum_{n_i} \frac{1}{n_i!}V_i^{n_i} = \exp({\sum}_i V_i).
\end{equation}
Thus, the correction to the free energy is given by the sum of all connected diagrams,
\begin{equation}
    - \beta F = \ln(Z_0) + \sum_n \ex{{S_I}^n}_{0, c}.
\end{equation}

\subsection*{Rester}


Expectation values of field configurations with different momenta vanish,
\begin{equation}
    K' \neq K \implies \ex{\varphi(K) \varphi(K')}_0 = 0.
\end{equation}
This is due to the symmetry of the functional integrand.
This can be seen by going back to the discrete version of the integral.
The integral can then be factored into integrals over each degree of freedom, $\varphi_{n,m}$.
The integrall over the degrees of freedom corresponding to $\varphi(K)$ and $\varphi(K')$ are of the form
\begin{equation}
    \int \dd \varphi_{n, m} \, \varphi_{n, m} \exp(- S^{(0)}_{n, m}) = 0,
\end{equation}
as the integral is over the entier real line, and the integrand is anti-symmetric in $\varphi_{n, m}$.
The generalization of this rule is straight forward. 
Assuming $f[\tilde \varphi]$ \emph{does not} depend on $\tilde \varphi(K)$, but possibly on $\tilde \varphi(K'), \, K \neq K'$, then
\begin{equation}
    \ex{f[\tilde \varphi](\tilde \varphi(K))^{(2n+1)}}_0 = 0, \, n \in \mathbb{Z}.
\end{equation}
Furthermore, as the field at different values of $K$ are independent, we get the familiar formula for expectation values of independent variables, that is
\begin{equation}
    \ex{f[\varphi(K)]g[\varphi(K')]}_0 = \ex{f[\varphi(K)]}_0 \ex{g[\varphi(K')]}_0, \quad
    K \neq K'.
\end{equation}
This leads to a version of Wick's theorem for thermal systems, as it lays constrains on integrals of the form
\begin{equation}
    \int_{\tilde \Omega} \dd K_1 \dots \dd K_n \,
    \ex{\tilde \varphi(K_1) ... \tilde \varphi(K_n)}_0 \delta\left({\sum}_i K_i\right)
    = 
\end{equation}

\subsection*{Fermions}
The phase factor $e^{i\theta}$ that was introduced in (REF IMAG FORM.) can be decided by studying the properties of the thermal greens function, which are defined as
\begin{equation*}
    G(X_1, X_2) = G(\vec x, \vec y, \tau_1, \tau_2) 
    = \ex{e^{-\beta \hat H} \T{ \varphi(X_1) \varphi(X_2) } }.
\end{equation*}
$\T{...}$ is the temperature ordering operator, analogous to the time ordering operator from scattering theory.
It is defined as
\begin{equation*}
    \T{\varphi(\tau_1)\varphi(\tau_2)}
    = \theta(\tau_1 - \tau_2) \varphi(\tau_1)\varphi(\tau_2)
    + \nu \theta(\tau_2 - \tau_1) \varphi(\tau_2)\varphi(\tau_1),
\end{equation*}
where $\nu = \pm 1$ for respectively bosons and fermions, and $\theta(\tau)$ is the Heaviside step function.
In the same way that $i \hat H$ generates the time-dependence of a quantum field operator through $\hat\varphi(x) = \hat\varphi(t, \vec x) = e^{it\hat H} \hat \varphi(\vec x) e^{-it\hat H} $, the imaginary-time formalism implies the relation
\begin{equation}
    \hat\varphi(X) = \hat\varphi(\tau, \vec x) 
    = e^{\tau\hat H} \hat \varphi(\vec x) e^{-\tau \hat H}.
\end{equation}
Using $\one = e^{\tau \hat H} e^{-\tau \hat H}$ and the cyclic property of the trace, we show that, assuming $\beta>\tau>0$,
\begin{align*}
    G(\vec x, \vec y, \tau, 0)
    & = \ex{e^{-\beta \hat H} \T{\varphi(\tau, \vec x) \varphi(0, \vec y)}} \\
    & = \frac{1}{Z} \Tr{
        e^{-\beta \hat H} \varphi(\tau, \vec x) \varphi(0, \vec y)
    } \\
    & = \frac{1}{Z} \Tr{
        \varphi(0, \vec y) e^{-\beta \hat H} \varphi(\tau, \vec x)
    } \\
    & = \frac{1}{Z} \Tr{
        e^{-\beta \hat H} e^{\beta \hat H} \varphi(0, \vec y) 
        e^{-\beta \hat H} \varphi(\tau, \vec x)
    } \\
    & = \frac{1}{Z} \Tr{
        e^{-\beta \hat H} \varphi(\vec y, \beta) \varphi(\tau, \vec x)
    } \\
    & = \nu \ex{
        e^{-\beta \hat H} \T{ \varphi(\tau, \vec x) \varphi(\vec y, \beta) }
    }.
\end{align*}
This implies that $\varphi(0, x) = \nu \varphi(\beta, \varphi)$, which show that bosons are periodic in time, as stated earlier, while fermions are anti-periodic.

The Lagrangian density of a free fermion is
\begin{equation}
    \Ell = \bar \psi \left( i \slashed{\partial} - m \right) \psi.
\end{equation}
This Lagrangian is invariant under the transformation $\psi \rightarrow e^{-i \alpha} \psi$, which by Nöther's theorem results in a conserved current
\begin{equation}
    j^\mu = \pdv{\Ell}{(\partial_\mu \psi)} \delta \psi=  \bar \psi \gamma^\mu \psi.
\end{equation}
The corresponding charge is 
\begin{equation}
    Q = \int_V \dd^3 x\, j^0 = \int_V \dd^3 x \, \psi^\dagger \psi.
\end{equation}
We can now use our earlier result for the thermal partition function, \autoref{Thermal partition function}, only with the substitution $\He \rightarrow \He - \mu \psi^\dagger \psi$, and integrate over anti-periodic $\psi$'s:
\begin{equation*}
    Z = \Tr{e^{-\beta(\hat H - \mu \hat Q)}}
    = \prod_{a b}\int \D \psi_a\D \pi_b \exp{
        \int_{\Omega} \dd X \, 
        \left(
            i\dot \psi \pi - \He(\psi, \pi) + \mu \psi^\dagger \psi
        \right)
    },
\end{equation*}
where $a, b$ are the spinor indices.
The canonical momentum corresponding to $\psi$ is
\begin{equation}
    \pi = \pdv{\Ell}{(\partial_0 \psi)} = i \psi^\dagger,
\end{equation}
and the Hamiltonian density is 
\begin{equation}
    \He = \pi \dot\psi - \Ell
    = \bar \psi ( - \gamma^i\partial_i + m) \psi
\end{equation}
which gives
\begin{equation}
    i \dot\psi \pi - \He(\psi, \pi) + \mu \psi^\dagger \psi
    = \bar\psi[-\gamma^0 (\partial_\tau - \mu) + i\gamma^i \partial_i - m] \psi,
\end{equation}
By using the Grassman-version of the Gaussian integral formula, the partition function can be written
\begin{align*}
    Z = \prod_{a b}\int \D \psi_a\D i\psi^\dagger_b 
    \exp{
        \int_\omega \dd X \, \bar \psi
        \left[
            -\gamma^0 (\partial_\tau -\mu)+ i \gamma^i \partial_i - m
        \right]
        \psi
    }
    = \prod_{a b}\int \D \psi_a\D i\psi^\dagger_b  e^{\langle i\psi^\dagger, D \psi\rangle} 
    = C \det(D),
\end{align*}
where 
\begin{equation}
    D = -i \gamma^0  [
        (\mu-\partial_\tau) + i \gamma^i \partial_i - m
        ].
\end{equation}
This results in
\begin{equation}
    -\beta F = \ln(Z) = \Tr[\ln(D)] + C.
\end{equation}
Assuming $\psi_a(X)$ is a orthonormal set, we can calculate the trace
\begin{align*}
    \Tr[\ln(D)] 
    & = \int_\Omega \dd X \, (i \psi_a(X)^\dagger) 
    \ln\{ -i \gamma^0 [\gamma^0 (\mu-\partial_\tau) + i \gamma^i\partial_i -  m]\}_{ab} 
    \psi_b(X) \\
    & =
    V \int_{\tilde \Omega} \dd K \,
    \ln\{ -i \beta\gamma^0  [\gamma^0 (\mu - i\omega_n) - \gamma^i p_i - m]\}_{aa} \\
    & =
    V \int_{\tilde \Omega} \dd K \,
    \ln\{ i \beta \gamma^0 [i\gamma^0 (\omega_n + i\mu) + i(-i\gamma^i) p_i + m]\}_{aa}\\
    & =
    V \int_{\tilde \Omega} \dd K \,
    \ln\{ i \beta \gamma^0 [i \tilde \gamma_a p_{n;a} + m]\}_{aa}.
\end{align*}
In the last line, we have introduced the notation $p_{n;a} = (\omega_n + i \mu, p_i)$ and use the Euclidean gamma matrices, as defined in \autoref{Conventions and notation}.
(JEG ER IKKE SIKKER PÅ DETTE) 
Furthermore, we use the fact that
\begin{equation*}
    \ln[i\tilde\gamma_a p_a - m]
    = \ln[i\tilde\gamma_a p_a - m] + \ln(\tilde\gamma_5 \tilde\gamma_5) 
    = \ln[\tilde\gamma_5(i\tilde\gamma_a p_a - m)\tilde\gamma_5]
    = \ln[-i\tilde\gamma_a p_a - m],
\end{equation*}
which allows us to calculate
\begin{align*}
    &\ln\{ i \beta \gamma^0[i\tilde\gamma^0 (\omega_n + i\mu) + i(-i\tilde\gamma^i) p_i + m]\}\\
    & = \ln[(i \tilde\gamma_a p_a + m)] + \frac{1}{4} \ln[(i \beta \gamma^0)^4]\\
    & = \ln[\beta] + \frac{1}{2} 
    \left\{
        \ln[i \tilde\gamma^a p^a + m] + \ln[- i \tilde\gamma^a p^a + m]
    \right\} \\
    & = \frac{1}{2} 
    \left\{
        \ln[\beta^2 (i \tilde\gamma^a p^a + m)( - i \tilde\gamma^a p^a + m)]
    \right\} \\
    & = \frac{1}{2} \one \ln\{ \beta^2[(\omega_n + i\mu)^2 + \omega^2]\} 
\end{align*}
which gives
\begin{equation}
    \Ef
    = -\frac{2}{\beta} \int_{\tilde \Omega} \dd X \,  \ln\{ \beta^2[(\omega_n + i\mu)^2 + \omega^2]\} .
\end{equation}
Using the fermionic version of the thermal sum, this gives the answer
\begin{equation}
    \Ef = - \frac{2}{\beta} \int\frac{\dd^3 p}{(2\pi)^3} \, 
    \left[
        \beta \omega 
        + \ln\left(1 + e^{-\beta(\omega+\mu)}\right)
        + \ln\left(1 + e^{-\beta(\omega-\mu)}\right)
    \right].
\end{equation}

\chapter{Appendices}
\label{appendix:conventions and notaition}

Throughout this text, \emph{natural units} are used.
These units are defined so that
\begin{equation}
    \hbar = c = k_B = 1,
\end{equation}
where $\hbar$ is the Planck reduced constant, $k_B$ is the Boltzmann constant, and $c$ is the speed of light.
These constants will therefore be dropped from all expressions.
They can be reintroduced using dimensional analysis.
In natural units, \emph{mass dimension} is the only engineering dimension.
Dimensionfull results are given in units of electronvolt (eV), or pion-masses, 
\begin{equation}
    m_\pi = 131 \, \text{MeV}.
\end{equation}

The Minkowski metric convention used is the ``mostly minus'',
\begin{equation}
    g_{\mu \nu} = \mathrm{diag}(1, -1, -1, -1).
\end{equation}
The Fourier transform used in this text is defined by
\begin{align*}
    \F{f(x)}(p) = \tilde f(p) = \int \dd x\, e^{i p x}f(x), \quad 
    \FInv{\tilde f(p)}(x) = f(x) = \int \frac{\dd p}{2 \pi}\, e^{- i p x} \tilde f(p).
\end{align*}
We employ the \emph{Einstiein summation convention}, in which pairwise matching indices are summed.
That is,
\begin{equation}
    a_i b_i = \sum_i a_i b_i = a_1 b_1 + \dots.
\end{equation}
For Minkowski-space indices, $\mu$, $\nu$, $\rho$ and $\sigma$, the metric raises and lower indices, and summation should always be over one raised and one lowered index,
\begin{equation}
    a_\mu b^\mu = g_{\mu\nu} a^\mu b^\nu 
    = a^0 b^0 - a^1 b^1 - \dots.
\end{equation}

\chapter{Functional Derivatives}
\label{section:Functional derivative}
Functional derivatives generalize the notion of a gradient and the directional derivative.
A function $f(p)$, where $p$ is point with coordinates $x_i = x_i(p)$, has a gradient
\begin{equation}
    \dd f_p = \pdv{f(p)}{x_i} \dd x_i.
\end{equation}
The derivative in a particular direction $v = v^i \partial_i$ is 
\begin{equation}
    \dv{\epsilon} f(x_i + \epsilon v_i) = f(x) + \dd f_x (v) = f(x) + \pdv{f}{x^i}v_i.
\end{equation}
This is generalized to functionals through the definition of the functional derivative and the variation of a functional.
Let $F[f]$ be a functional, i.e., a machine that takes in a function and returns a number.
The obvious example in our case is the action, which takes in one or more field configurations, and returns a single real number.
We will assume here that the functions have the domain $\Omega$, with coordinates $x$.
The functional derivative is defined as
\begin{equation}
    \delta F[f]
    =
    \dv{\epsilon} F[f + \epsilon \eta] \Big|_{\epsilon = 0}
    = \int_\Omega \dd x \, \frac{\delta F[f]}{\delta f(x)} \eta(x).
\end{equation}
$\eta(x)$ is here an arbitrary function, but we will make the important assumption that it as well as all its derivatives are zero at the boundary of its domain $\Omega$.
This allows us to discard surface terms stemming from partial integration, which we will use frequently.
We may use the definition to derive one of the fundamental relations of functional derivation.
Take the functional $F[f] = f(x)$. 
Then,
\begin{equation}
    \label{Functional derivative delta identity}
    \delta F[f] = \dv{\epsilon} [f(x) + \epsilon \eta(x)] = \eta(x) = \int \dd y \, \delta(x - y) \eta(y)
\end{equation}
This leads to the identity
\begin{equation}
    \frac{\delta f(x)}{\delta f(y)} = \delta(x - y),
\end{equation}
for any function $f$.
Higher functional derivatives are defined similarly, by applying functional variation repeatedly
\begin{equation}
    \delta^n F[f] = \dv{\epsilon} \delta^{n-1}F[f + \epsilon \eta_n] \big|_{\epsilon=0}
    = \int \left(\prod_{i=1}^n \dd x_i\right)
    \frac{\delta^n F[f]}{ \delta f(x_n)\dots\delta f(x_n)} \left(\prod_{i=1}^n \eta_i(x_i)\right).
\end{equation}
A functional may be expanded in a generalization of the Fourier series, which has the form
\begin{equation}
    F[f_0 + f] = F[f_0] + \int_\Omega \dd x \, f(x) \frac{\delta F[f_0]}{\delta f(x)}\bigg|_{f = f_0}
    + \frac{1}{2!}\int_\Omega \dd x \dd y \, f(x) f(y) \frac{\delta^2 F [f_0]}{\delta f(x) \delta f(y)}
    + \dots
\end{equation}
As an example, the Klein-Gorodn action
\begin{equation}
    S[\varphi] = - \frac{1}{2}\int_\Omega \dd x \, \varphi (\partial^2 + m^2) \varphi(x)
\end{equation}
can be evaluated quickly by using \autoref{Functional derivative delta identity} and partial integration
\begin{align}
    \nonumber
    \funcdv{\varphi(x)} S[\varphi] 
    & = 
    - \frac{1}{2} \int_\Omega \dd y \, 
    [\delta(x - y)(\partial_y^2 + m^2)\varphi(y) + \varphi(y) (\partial_y^2 + m^2)\delta(x - y)] \\
    & = 
    - \int_\Omega \dd y \, 
    \delta(x - y)(\partial_y^2 + m^2)\varphi(y) 
    = (\partial_x^2 + m^2)\varphi(x)
\end{align}
The second derivative is
\begin{equation}
    \frac{\delta^2S[\varphi]}{\delta \varphi(x)\delta \varphi(y)}
    =
    \funcdv{\varphi(x)} (\partial_y^2 + m^2)\varphi(y)
    = 
    (\partial_y^2 + m^2) \delta(x - y).
\end{equation}


\subsection*{Gaussian integrals}
\label{section:gaussian integrals}

\begin{figure}[ht]
    \centering
    \begin{tikzpicture}
        \draw (-2, 0) -- (2, 0) node[right] {$\mathrm{Re}(x)$};
        \draw (0, -2) -- (0, 2) node[above] {$\mathrm{Im}(x)$};
        \draw[->, thick] (-1.75, 0.1) -- (1.8, 0.1);
        \draw[->, thick] (1.8, 0.15) arc (10:45:1.8);
        \draw[->, thick] ({1.8/sqrt(2)}, {1.8/sqrt(2)}) -- ({-1.8/sqrt(2)}, {-1.8/sqrt(2)});
        \draw[->, thick] ({-1.8/sqrt(2)}, {-1.8/sqrt(2)}) arc (225:180:1.8);
    \end{tikzpicture}
    \caption{Wick rotation}
    \label{Wick rotation}
\end{figure}


A useful integral is the Gaussian integral,
\begin{equation}
    \int_\R \dd z \, \exp(- \frac{1}{2} a z^2) = \sqrt{\frac{2 \pi}{a}},
\end{equation}
for $a \in \R$. The imaginary version,
\begin{equation}
    \int_R \dd z \, \exp(i \frac{1}{2} a z^2 )
\end{equation}
does not converge. However, if we change $a \rightarrow a + i\epsilon$, the integrand is exponentially suppressed.
\begin{equation}
    f(x) = \exp(i \frac{1}{2}a x^2) \rightarrow
    \exp(i\frac{1}{2}a x^2 - \frac{1}{2} \epsilon  x^2).
\end{equation}
As the integrand falls exponentially for $x\rightarrow \infty$ and contains no poles in the upper right nor lower left quarter of the complex plane, we may perform a wick rotation by closing the contour as shown in \autoref{Wick rotation}.
This gives the result
\begin{equation}
    \label{complex gauss 1D}
    \int_\R \dd x \, \exp(i \frac{1}{2}(a + i\epsilon) x^2) 
    = \int_{\sqrt{i}\R} \dd x \, \exp(i\frac{1}{2} ax^2)
    = \sqrt{i} \int_\R \dd y\, \exp(-\frac{1}{2} (-a) y^2) = \sqrt{\frac{2 \pi i}{(-a)}}
\end{equation}
where we have made the change of variable $y = (1+i)/\sqrt{2} x = \sqrt{i} x$.
In $n$ dimensions, the Gaussian integral formula generalizes to
\begin{equation}
    \int_{\R^n} \dd^n x \, \exp{-\frac{1}{2} x_n A_{nm} x_m } =\sqrt{\frac{(2 \pi)^n}{\det(A)}},
\end{equation}
where $A$ is a matrix with $n$ real, positive eigenvalues.
We may also generalize \autoref{complex gauss 1D},
\begin{align}
    \int_{\R^n} \dd^n x \, \exp{i\frac{1}{2} x_n( A_{nm} + i \epsilon \delta_{nm}) x_m } =\sqrt{\frac{(2 \pi i )^n}{\det(-A)}}.
\end{align}
The final generalization is to functional integrals,
\begin{align}
    \int \D \varphi \, \exp(- \frac{1}{2} \int \dd x \, \varphi(x) A \varphi(x) )
    = C (\det(A))^{-1/2},
    \int \D \varphi \, \exp(i\frac{1}{2} \int \dd x \, \varphi(x) A \varphi(x) )
    = C (\det(-A))^{-1/2}.
\end{align}
$C$ is here a divergent constant, but will either fall away as we are only looking at the logarithm of $I_\infty$ and are able to throw away additive constants, or ratios between quantities which are both multiplied by $C$.

The Gaussian integral can be used for the stationary phase approximation.
In one dimension, it is
\begin{equation}
    \int \dd x \, \exp(i \alpha f(x)) 
    \approx \sqrt{\frac{2 \pi }{f''(x_0)}}\exp( f(x_0)), 
    \, f'(x) = 0, \, \alpha\rightarrow \infty
\end{equation}
The functional generalization of this is
\begin{equation}
    \int \D \varphi \exp{i S[\varphi]}
    \approx 
    C \det(- \frac{\delta^2 S[\varphi_0]}{\delta \varphi^2})
    \exp{i \alpha S[\varphi_0]  }, \quad
    \frac{\delta S[\varphi_0]}{\delta \varphi} = 0…
\end{equation}
Here, $S[\varphi]$ is a general functional of $\varphi$, and we have used the Taylor expansion, and $\varphi_0$ fulfills
\begin{equation}
    \funcdv{\varphi(x)}{S[\varphi_0]} = 0,
\end{equation}


\section{Integrals}
\subsection{Gaussian integrals}
\label{section:gaussian integrals}
A useful integral is the Gaussian integral,
\begin{equation}
    \int_\R \dd z \, \exp(- \frac{1}{2} a z^2) = \sqrt{\frac{2 \pi}{a}},
\end{equation}
for $a \in \R$. The imaginary version,
\begin{equation}
    \int_R \dd z \, \exp(i \frac{1}{2} a z^2 )
\end{equation}
does not converge. However, if we change $a \rightarrow a + i\epsilon$, 
% contour of integration slightly, by rotating it clockwise to $C = \R(1 + i\epsilon)$,
% \begin{figure}
%     \begin{subfigure}{0.4\textwidth}
%         \begin{tikzpicture}
%             \draw (-2, 0) -- (2, 0) node[right] {$\mathrm{Re}(x)$};
%             \draw (0, -2) -- (0, 2) node[above] {$\mathrm{Im}(x)$};
%             \draw[->, thick] (-1.8, 0) -- (1.8, 0);
%         \end{tikzpicture}    
%     \end{subfigure}
%     \begin{subfigure}{0.18\textwidth}
%         \begin{tikzpicture}
%             \draw[->] (-1, 0) -- (1, 0);
%         \end{tikzpicture}
%     \end{subfigure}
%     \begin{subfigure}{0.4\textwidth}
%         \begin{tikzpicture}
%             \draw (-2, 0) -- (2, 0) node[right] {$\mathrm{Re}(x)$};
%             \draw (0, -2) -- (0, 2) node[above] {$\mathrm{Im}(x)$};
%             \draw[->, thick] (-1.8, -0.1) -- (1.8, 0.1);
%         \end{tikzpicture}    
%     \end{subfigure}
% \end{figure}
then the integrand is exponentially supressed.
\begin{equation}
    f(x) = \exp(i \frac{1}{2}a x^2) \rightarrow
    \exp(i\frac{1}{2}a x^2 - \frac{1}{2} \epsilon  x^2),
    % \exp(i \frac{1}{2}a z(t)^2) = \exp(i\frac{1}{2}a t^2(1 + i \epsilon)^2) \sim \exp(-\frac{1}{2}a \epsilon t^2 + i \frac{1}{2}at^2).
\end{equation}
As the integrand falls of exponentially for $x\rightarrow \infty$, and contains no poles in the upper right nor lower left quarter of the complex plane, we may perform a wick rotation by closing the contour as shown in \autoref{Wick rotation}.
\begin{figure}
    \centering
    \begin{tikzpicture}
        \draw (-2, 0) -- (2, 0) node[right] {$\mathrm{Re}(x)$};
        \draw (0, -2) -- (0, 2) node[above] {$\mathrm{Im}(x)$};
        \draw[->, thick] (-1.75, 0.1) -- (1.8, 0.1);
        \draw[->, thick] (1.8, 0.15) arc (10:45:1.8);
        \draw[->, thick] ({1.8/sqrt(2)}, {1.8/sqrt(2)}) -- ({-1.8/sqrt(2)}, {-1.8/sqrt(2)});
        \draw[->, thick] ({-1.8/sqrt(2)}, {-1.8/sqrt(2)}) arc (225:180:1.8);
    \end{tikzpicture}
    \caption{Wick rotation}
    \label{Wick rotation}
\end{figure}
This gives the result
\begin{equation}
    \label{complex gauss 1D}
    \int_\R \dd x \, \exp(i \frac{1}{2}(a + i\epsilon) x^2) 
    = \int_{\sqrt{i}\R} \dd x \, \exp(i\frac{1}{2} ax^2)
    = \sqrt{i} \int_\R \dd y\, \exp(-\frac{1}{2} (-a) y^2) = \sqrt{\frac{2 \pi i}{(-a)}}
\end{equation}
where we have made the change of variable $y = (1+i)/\sqrt{2} x = \sqrt{i} x$.
In $n$ dimensions, the Gaussian integral formula generalizes to
\begin{equation}
    \int_{\R^n} \dd^n x \, \exp{-\frac{1}{2} x_n A_{nm} x_m } =\sqrt{\frac{(2 \pi)^n}{\det(A)}},
\end{equation}
where $A$ is a matrix with $n$ real, positive eigenvalues.
We may also generalize \autoref{complex gauss 1D},
\begin{align}
    \int_{\R^n} \dd^n x \, \exp{i\frac{1}{2} x_n( A_{nm} + i \epsilon \delta_{nm}) x_m } =\sqrt{\frac{(2 \pi i )^n}{\det(-A)}}.
\end{align}
The final generalization is to functional integrals,
% The bilinear becomes
% \begin{equation}
%     x_n A_{nm} x_m \rightarrow \int \dd x \, \varphi(x) A \varphi(x),
% \end{equation}
% where $A$ is some operator.
% The first Gaussian integral becomes
\begin{align}
    \int \D \varphi \, \exp(- \frac{1}{2} \int \dd x \, \varphi(x) A \varphi(x) )
    = C (\det(A))^{-1/2},
    \int \D \varphi \, \exp(i\frac{1}{2} \int \dd x \, \varphi(x) A \varphi(x) )
    = C (\det(-A))^{-1/2}.
\end{align}
$C$ is here a divergent constant, but will either fall away as we are only looking at the logarithm of $I_\infty$ and are able to throw away additive constants, or ratios between quantities which are both multiplied by $C$.

The Gaussian integral can be used for the stationary phase approximation.
In one dimension, it is
\begin{equation}
    \int \dd x \, \exp(i \alpha f(x)) 
    \approx \sqrt{\frac{2 \pi }{f''(x_0)}}\exp( f(x_0)), 
    \, f'(x) = 0, \, \alpha\rightarrow \infty
\end{equation}
The functional generalization of this is
\begin{equation}
    \int \D \varphi \exp{i S[\varphi]}
    \approx 
    C \det(- \frac{\delta^2 S[\varphi_0]}{\delta \varphi^2})
    \exp{i \alpha S[\varphi_0]  }, \quad
    \frac{\delta S[\varphi_0]}{\delta \varphi} = 0…
\end{equation}
Here, $S[\varphi]$ is a general functional of $\varphi$, and we have used the Taylor expansion, as described in \autoref{section:Functional derivative}, and $\varphi_0$ fulfills
\begin{equation}
    = \funcdv{\varphi(x)}{\Gamma} + J(x) = 0,
\end{equation}


\section{Derivations}

\subsection*{Rewriting NLO Lagrangian}
\label{subsection:rewriting NLO Lagrangian}

The NLO Lagrangian used in this text is given in \cref{NLO Lagrangian}, and is
\begin{align}
    \notag
    \Ell_4 
    & = 
    \frac{l_1}{4} \Tr{\nabla_\mu \Sigma (\nabla^\mu \Sigma)^\dagger}^2
    + \frac{l_2}{4} \Tr{\nabla_\mu \Sigma (\nabla_\nu \Sigma)^\dagger} 
    \Tr{\nabla^\mu \Sigma (\nabla^\nu \Sigma)^\dagger} 
    +
    \frac{l_3 + l_4}{16} \Tr{\chi \Sigma^\dagger + \Sigma \chi^\dagger}^2
    \\ \notag
    &
    + \frac{l_4}{8}\Tr{\nabla_\mu \Sigma (\nabla^\mu \Sigma)^\dagger} \Tr{\chi \Sigma^\dagger + \Sigma \chi^\dagger}
    - \frac{l_7}{16} \Tr{\chi \Sigma^\dagger - \Sigma \chi^\dagger}^2
    + \frac{h_1 + h_3 - l_4}{4} \Tr{\chi \chi^\dagger} \\
    & -
    \frac{h_1 - h_3 - l_4}{8} 
    \left[
        \Tr{\chi \Sigma^\dagger + \Sigma \chi^\dagger}^2
        + \Tr{\chi \Sigma^\dagger - \Sigma \chi^\dagger}^2
        -2 \Tr{\left(\chi \Sigma^\dagger\right)^2 + \left( \Sigma \chi^\dagger\right)^2}
    \right]
    \label{NLO Lagrangian}.
\end{align}
We can rewrite it to match the one used in~\cite{Andersen:two-flavor-chpt,mojahed}, We

We can rewirte
\begin{align}
    & \frac{h_1 - h_3 - l_4}{16}
    \left(
        \Tr{\chi \Sigma^\dagger - \Sigma \chi^\dagger}^2
        - 2 \Tr{(\chi \Sigma^\dagger)^2 + (\Sigma \chi^\dagger)^2}
    \right) \\
    & = 
    \frac{h_1 - h_3 - l_4}{16}
    \left(
        \Tr{\chi \Sigma^\dagger}^2 - 2\Tr{\chi \Sigma^\dagger}\Tr{\Sigma \chi^\dagger}
        + \Tr{\Sigma \chi^\dagger}^2
        - 2 \Tr{(\chi \Sigma^\dagger)^2} -2 \Tr{(\Sigma \chi^\dagger)^2}
    \right)
\end{align}
Using $\Tr{A^2} = \Tr{A}^2 - \det(A)\Tr{\one}$, we get
\begin{align}
    &= - \frac{h_1 - h_3 - l_4}{16}
    \left(
        \Tr{\chi \Sigma^\dagger}^2 + 2\Tr{\chi \Sigma^\dagger}\Tr{\Sigma \chi^\dagger}
        + \Tr{\Sigma \chi^\dagger}^2
        - 4 \det(\chi \Sigma^\dagger)
        - 4 \det(\Sigma\chi^\dagger)
    \right) \\
    &= - \frac{h_1 - h_3 - l_4}{16}
    \left(
        \Tr{\chi \Sigma^\dagger + \Sigma \chi^\dagger}^2
        - 4 \det(\chi \Sigma^\dagger)
        - 4 \det(\Sigma\chi^\dagger)
    \right)
\end{align}
Furthermore, as $\det(\Sigma)= 1$, 
% and $\chi = a_i \tau_i$, we have $\Tr{\chi \chi^\dagger} = a_i a_i^*$, $\det(\chi) + \det(\chi^\dagger) = a_i a_i + a_i^* a_i^*$
\begin{align}
    \notag
    \Ell_4 
    & = 
    \frac{l_1}{4} \Tr{\nabla_\mu \Sigma (\nabla^\mu \Sigma)^\dagger}^2
    + \frac{l_2}{4} \Tr{\nabla_\mu \Sigma (\nabla_\nu \Sigma)^\dagger} 
    \Tr{\nabla^\mu \Sigma (\nabla^\nu \Sigma)^\dagger} 
    +
    \frac{l_3 + l_4 }{16} \Tr{\chi \Sigma^\dagger + \Sigma \chi^\dagger}^2
    \\\notag
    &
    + \frac{l_4}{8}\Tr{\nabla_\mu \Sigma (\nabla^\mu \Sigma)^\dagger} \Tr{\chi \Sigma^\dagger + \Sigma \chi^\dagger}
    - \frac{l_7}{16} \Tr{\chi \Sigma^\dagger - \Sigma \chi^\dagger}^2
    + \frac{h_1 + h_3 -l_4}{4} \Tr{\chi \chi^\dagger} \\
    & +\frac{h_1 - h_3 - l_4}{4} (\det{\chi} + \det{\chi^\dagger})
\end{align}
For real $\chi$, we have $\Tr{\chi \chi^\dagger} = \det(\chi) + \det(\chi^\dagger)$, so
\begin{align}
    \notag
    \Ell_4 
    & = 
    \frac{l_1}{4} \Tr{\nabla_\mu \Sigma (\nabla^\mu \Sigma)^\dagger}^2
    + \frac{l_2}{4} \Tr{\nabla_\mu \Sigma (\nabla_\nu \Sigma)^\dagger} 
    \Tr{\nabla^\mu \Sigma (\nabla^\nu \Sigma)^\dagger} 
    +
    \frac{l_3 + l_4 }{16} \Tr{\chi \Sigma^\dagger + \Sigma \chi^\dagger}^2
    \\\notag
    &
    + \frac{l_4}{8}\Tr{\nabla_\mu \Sigma (\nabla^\mu \Sigma)^\dagger} \Tr{\chi \Sigma^\dagger + \Sigma \chi^\dagger}
    - \frac{l_7}{16} \Tr{\chi \Sigma^\dagger - \Sigma \chi^\dagger}^2
    + \frac{h_1 - l_4}{2} \Tr{\chi \chi^\dagger} \\
\end{align}



\bibliographystyle{unsrt}
\bibliography{referanser}


\end{document}