\section{Conventions and notation}
\label{Conventions and notation}
Throughout this text, natural units are employed, in which
\begin{equation}
    \hbar = c = k_B = 1,
\end{equation}
where $\hbar$ is the Planck reduced constant, $k_B$ is the Boltzmann constant and $c$ is the speed of light.
The Minkowski metric convention used is the ``mostly minus'', $g_{\mu \nu} = \mathrm{diag}(1, -1, -1, -1)$.

The $\liea{su}{2}$ basis used is the Pauli matrices,
\begin{align*}
    \tau_1 = 
    \begin{pmatrix}
        0 & 1 \\
        1 & 0 \\
    \end{pmatrix}
    , \quad 
    \tau_2 = 
    \begin{pmatrix}
        0 & -i \\
        i & 0 \\
    \end{pmatrix}, \quad 
    \tau_3 = 
    \begin{pmatrix}
        1 & 0 \\
        0 & -1 \\
    \end{pmatrix}.
\end{align*}
They obey
\begin{align*}
    \com{\tau_a}{\tau_b} = 2 i \eps_{abc}\tau_c, \quad 
    \acom{\tau_a}{\tau_b} = 2\delta_{ab} \one,
    \quad \Tr[\tau_a] = 0,
    \quad \Tr[\tau_a \tau_b] = 2 \delta_{ab} \one.
\end{align*}
Together with the identity matrix $\one$, the Pauli matrices form a basis for the vector space of all 2-by-2 matrices.
An arbitrary 2-by-2 matrix $M$ may be written
\begin{equation}
    \label{2-by-2 matrix decomp}
    M = M_0 \one + M_a \tau_a, \quad 
    M_0 = \frac{1}{2} \Tr{M}, \,\, M_a = \frac{1}{2} \Tr{\tau_a M}.
\end{equation}

The gamma matrices $\gamma^\mu$, $\mu \in \{0, 1, 2, 3\}$, obey
\begin{equation}
    \acom{\gamma^\mu}{\gamma^\nu} = 2 g^{\mu \nu} \one.
\end{equation}
The ``fifth $\gamma$-matrix'' is defined by
\begin{equation}
    \gamma^5 
    = \frac{i}{4!}\epsilon_{\mu \nu \rho \sigma} \gamma^{\mu}\gamma^{\nu}\gamma^{\rho}\gamma^{\sigma}
    = i \gamma^0\gamma^1\gamma^2\gamma^3.
\end{equation}
The $\gamma^5$-matrix obey
\begin{equation}
    \acom{\gamma^5}{\gamma^\mu} = 0, \quad (\gamma^5)^2 = \one, \quad
    {\gamma^0}^\dagger = \gamma^0, \, {\gamma^i}^\dagger = - \gamma^i
\end{equation}
Their Euclidean counterpart obey
\begin{equation}
    \acom{\tilde \gamma_a}{\tilde \gamma_b} = 2 \delta_{ab}\one, \quad
    {\tilde\gamma_a}^\dagger = \tilde\gamma_a,
\end{equation}
and they are related by $\tilde \gamma_0 = \gamma^0$, and $\tilde \gamma_j = -i\gamma^j$.
The Euclidean $\tilde \gamma_5$ is defined as
\begin{equation}
    \tilde \gamma_5 = \gamma_0\gamma_1\gamma_2\gamma_3 = i \gamma^0\gamma^1\gamma^2\gamma^3 = \gamma^5.
\end{equation}
It thus also anti-commutes with the Euclidean $\gamma$-matrices.


\subsection{Fourier transform}
The Fourier transform used in this text is defined by
\begin{align*}
    \F{f(x)}(p) = \tilde f(p) = \int \dd x\, e^{i p x}f(x), \quad 
    \FInv{\tilde f(p)}(x) = f(x) = \int \frac{\dd p}{2 \pi}\, e^{- i p x} \tilde f(p).
\end{align*}

\subsection{Fourier series}
Imaginary-time formalism is set in a Euclidean space $\Omega = [0, \beta] \times V$,
where  $V = L_xL_yL_z$ is a space-like volume. The possible momenta in this space are
\begin{equation*}
    \tilde V = \left\{ \vec k \in \R^3 \,\Big|\, \vec k = 
    \left(
        \frac{2 \pi n_x}{L_x}, 
        \frac{2 \pi n_y}{L_y},
        \frac{2 \pi n_z}{L_z} 
        \right)
    \right\}
\end{equation*}
$\omega_n$ are the Matsubara-frequencies, with $\omega_n = 2 n \pi / \beta$ for bosons and $\omega_n = (2n + 1) \pi / \beta$ for fermions.
They together form the reciprocal space $\tilde \Omega = \{\omega_n\}\times \tilde V$.
The Euclidean coordinates are denoted $X = (\tau, \vec x)$ and $K = (\omega_n, \vec K)$, and have the dot product $X\cdot P = \omega_n \tau + \vec k \cdot \vec x$.
In the limit $V\rightarrow \infty$, we follow the prescription
\begin{equation*}
    \frac{1}{V} \sum_{\vec p \in \tilde V} \rightarrow \int_{\R^3} 
    \frac{\dd^3 p}{(2 \pi)^3}.
\end{equation*}
The sum over all degrees of freedom, and the corresponding integrals for the thermodynamic limit are
\begin{align*}
     \frac{\beta V}{NM}\sum_{n=1}^N \sum_{\vec x_m \in V} 
    & \xrightarrow{N,M\rightarrow \infty} \int_{0}^\beta \dd \tau \int_{\R^3} \dd^3 x
    = \int_\Omega \dd X, \\
     \frac{1}{V} \sum_{n=-\infty}^\infty \sum_{\vec k \in \tilde V}
    & \xrightarrow{V \rightarrow \infty} \sum_{n=-\infty}^\infty \int_{R^3} \frac{\dd^3 p}{(2 \pi)^3}
    = \int_{\tilde \Omega} \dd K.
\end{align*}
The convention used for the Fourier expansion of thermal fields is in accordance with \cite{Kapusta:finiteTemp}. 
The prefactor is chosen to make the Fourier components of the field dimensionless, which makes it easier to evaluate the trace correctly.
For bosons, the Fourier expansion is
\begin{align*}
    \varphi(X)
    = &
    \sqrt{V \beta} \int_{\tilde \Omega} \dd K \,  \tilde \varphi(K) e^{i X\cdot K}
    =
    \sqrt{\frac{\beta}{V}} \sum_{n=-\infty}^\infty \sum_{\vec k \in \tilde V}
    \tilde \varphi_n(\vec p) \exp{i(\omega_n \tau + \vec x \cdot \vec k)}, \\
    \tilde \varphi(K)
    = &
    \sqrt{\frac{1}{V \beta^3}} \int_{\tilde \Omega} \dd X \,  \tilde \varphi(X) e^{ - i X\cdot K}
\end{align*}
while for Fermions it is
\begin{equation}
    \psi(X) 
    = \sqrt{V} \int_{\tilde \Omega} \dd K \, \tilde \psi(K) e^{i X\cdot K} 
    = \frac{1}{\sqrt{V}} \sum_{n = - \infty}^\infty \sum_{\vec k \in \tilde V}
    \psi(\omega_n, \vec k) \exp{i(\omega_n \tau + \vec x \cdot \vec k)}
\end{equation}
A often used identity is 
\begin{align}
    \label{thermal delta}
    \int_{\Omega} \dd X e^{i X\cdot(K - K')} 
    & = \beta \delta_{nn'} (2 \pi)^3 \delta^3(\vec k - \vec k') := \beta \delta(K - K'), \\\
    \int_{\tilde \Omega} \dd K \, e^{i K(X - X')} 
    & = \beta \delta (\tau - \tau') \delta^3(\vec x - \vec x') 
    := \beta \delta(X - X').
\end{align}

\subsection{Particle distributions}
The Bose distribution is defined as 
\begin{equation}
    n_B(\omega) = \frac{1}{e^{\beta \omega} - 1}.
\end{equation}
This function obeys
\begin{equation}
    n_B(- i \omega) = -1 - n_B(i \omega).
\end{equation}
We can expand it around the Bose Matsubara frequencies on the imaginary line:
\begin{equation}
    i n_B[i (\omega_n + \epsilon)] = \frac{i}{e^{i\beta \epsilon + 2 \pi i n} - 1}
    = i [i\beta \epsilon + \Oh{\epsilon} ]^{-1} \sim  \frac{1}{\epsilon \beta}.
\end{equation}
This means that $in_B(i\omega)$ has a pole on all Matsubara-frequencies, with residue $1/\beta$.
Furthermore, we have
\begin{equation}
    \dv{\omega} \ln(1 - e^{-\beta \omega}) = \beta n_B(\omega).
\end{equation}
The Fermi distribution is
\begin{equation}
    n_F(\omega) = \frac{1}{e^{\beta \omega} + 1}.
\end{equation}
It obeys
\begin{align}
    &\diff{}{\omega} \ln(1 - e^{-\beta \omega}) = - \beta n_F(\omega), \\
    & n_F(- i\omega) = 1 - n_F(i\omega).
\end{align}
The two distributions are related by
\begin{equation}
    2 n_B(i \omega; 2\beta) - n_B(i \omega; \beta)
    = - n_F(i \omega; \beta ).
\end{equation}

\subsection{Propagators}
If $D^{-1}[f(x)] = 0$ is the equation of motion for some field $f$, where $D^{-1}$ in general is a differential operator, then the propagator $D(x, x')$ for this field is defined by
\begin{align*}
    D^{-1}[D(x, x')] = - i \delta(x - x') \one.
\end{align*}
Assuming $A$ is linear and independent of space, we may redefine $D(x - x', 0) \rightarrow D(x - x')$, and the Fourier transform with respect to both $x$ and $x'$ to obtain
\begin{align*}
    \F{D^{-1}[D(x, x')]}(p, p') 
    = \tilde D^{-1}(p) \, \tilde D(p) \, \delta(p + p')
    = -i \delta(p + p'),
\end{align*}
meaning the momentum space propagator $\tilde D(p) = \F{D(x)}(p)$ is given by $\tilde D = - i (\tilde  D^{-1})^{-1}.$

For some differential operator $D^{-1}$, the thermal propagator is defined as
\begin{equation}
    D^{-1} D(X, Y) = \beta \delta(X - Y).
\end{equation}
The Fourier transformed propagator is, assuming $D(X, Y) = D(X-Y, 0)$,
\begin{align}
    \tilde D(K, K') 
    & = \frac{1}{V \beta^3} \int_{\Omega} \dd X \dd Y \, 
    D(X, Y) \exp(- i [X\cdot K + Y\cdot K']) \\
    & = \frac{1}{V \beta^3} \int_{\Omega} \dd X' \dd Y' \, D(X', 0) 
    \exp(- i [X'\cdot \frac{1}{2} (K - K') + Y\cdot (K + K')]) \\
    & = \frac{1}{V \beta^2} \tilde D(K) \delta(K + K'),
\end{align}
where
\begin{equation}
    \tilde D(K) = \int \dd X e^{iK\cdot X} D(X, 0).
\end{equation}

% Thus, 
% \begin{equation}
%     D^{-1} D(X, Y)
%     =  \int \dd K \, \tilde D(K) D^{-1} e^{i K(X - Y)}
%     = \beta \delta(X - Y)
%     =  \int \dd K e^{iK(X - Y) }.
% \end{equation}
% The thermal propagator in momentum space, $\tilde D(K)$, is given by 
% \begin{equation}
%     \tilde D^{-1} \tilde D(K) = 1.
% \end{equation}

