\section{Functional Derivatives}
\label{section:Functional derivative}
Functional derivatives generalize the notion of a gradient and the directional derivative.
A function $f(p)$, where $p$ is point with coordinates $x_i = x_i(p)$, has a gradient
\begin{equation}
    \dd f_p = \pdv{f(p)}{x_i} \dd x_i.
\end{equation}
The derivative in a particular direction $v = v^i \partial_i$ is 
\begin{equation}
    \dv{\epsilon} f(x_i + \epsilon v_i) = f(x) + \dd f_x (v) = f(x) + \pdv{f}{x^i}v_i.
\end{equation}
This is generalized to functionals through the definition of functional derivative, and the variation of a functional.
Let $F[f]$ be a functional, i.e. a machine that takes in a function, and returns a number.
The obvious example in our case is the action, which takes in one or more field-configurations, and returns a single real number.
We will assume here that the functions have the domain $\Omega$, with coordinates $x$.
The functional derivative is defined as
\begin{equation}
    \delta F[f]
    =
    \dv{\epsilon} F[f + \epsilon \eta] \Big|_{\epsilon = 0}
    = \int_\Omega \dd x \, \frac{\delta F[f]}{\delta f(x)} \eta(x).
\end{equation}
$\eta(x)$ is here an arbitrary function, but we will make the important assumption that it as well as all its derivatives are zero at the boundary of its domain $\Omega$.
This allows us to discard surface terms stemming from partial integration, which we will use frequently.
We may use the definition to derive one of the fundamental relations of functional derivation.
Take the functional $F[f] = f(x)$. 
Then,
\begin{equation}
    \label{Functional derivative delta identity}
    \delta F[f] = \dv{\epsilon} [f(x) + \epsilon \eta(x)] = \eta(x) = \int \dd y \, \delta(x - y) \eta(y)
\end{equation}
This leads to the identity
\begin{equation}
    \frac{\delta f(x)}{\delta f(y)} = \delta(x - y),
\end{equation}
for any function $f$.
Higher functional derivatives are defined similarly, by applying functional variation repeatedly
\begin{equation}
    \delta^n F[f] = \dv{\epsilon} \delta^{n-1}F[f + \epsilon \eta_n] \big|_{\epsilon=0}
    = \int \left(\prod_{i=1}^n \dd x_i\right)
    \frac{\delta^n F[f]}{ \delta f(x_n)\dots\delta f(x_n)} \left(\prod_{i=1}^n \eta_i(x_i)\right).
\end{equation}
A functional may be expanded in a generalization of the Fourier series, which has the form
\begin{equation}
    F[f_0 + f] = F[f_0] + \int_\Omega \dd x \, f(x) \frac{\delta F[f_0]}{\delta f(x)}\bigg|_{f = f_0}
    + \frac{1}{2!}\int_\Omega \dd x \dd y \, f(x) f(y) \frac{\delta^2 F [f_0]}{\delta f(x) \delta f(y)}
    + \dots
\end{equation}
As an example, the Klein-Gorodn action,
\begin{equation}
    S[\varphi] = - \frac{1}{2}\int_\Omega \dd x \, \varphi (\partial^2 + m^2) \varphi(x).
\end{equation}
It can be evaluated quickly by using \autoref{Functional derivative delta identity} and partial integration
\begin{equation}
    \funcdv{\varphi(x)} S[\varphi] 
    = 
    - \frac{1}{2} \int_\Omega \dd y \, 
    [\delta(x - y)(\partial_y^2 + m^2)\varphi(y) + \varphi(y) (\partial_y^2 + m^2)\delta(x - y)]
    = 
    - \int_\Omega \dd y \, 
    \delta(x - y)(\partial_y^2 + m^2)\varphi(y)
    = (\partial_y^2 + m^2)\varphi(y)
\end{equation}
The second derivative is
\begin{equation}
    \frac{\delta^2S[\varphi]}{\delta \varphi(x)\delta \varphi(y)}
    =
    \funcdv{f(x)} (\partial_y^2 + m^2)\varphi(y)
    = 
    (\partial_y^2 + m^2) \delta(x - y).
\end{equation}


